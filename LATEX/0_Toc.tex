
% Default to the notebook output style

    


% Inherit from the specified cell style.




    
\documentclass{article}

    
    
    \usepackage{graphicx} % Used to insert images
    \usepackage{adjustbox} % Used to constrain images to a maximum size 
    \usepackage{color} % Allow colors to be defined
    \usepackage{enumerate} % Needed for markdown enumerations to work
    \usepackage{geometry} % Used to adjust the document margins
    \usepackage{amsmath} % Equations
    \usepackage{amssymb} % Equations
    \usepackage{eurosym} % defines \euro
    \usepackage[mathletters]{ucs} % Extended unicode (utf-8) support
    \usepackage[utf8x]{inputenc} % Allow utf-8 characters in the tex document
    \usepackage{fancyvrb} % verbatim replacement that allows latex
    \usepackage{grffile} % extends the file name processing of package graphics 
                         % to support a larger range 
    % The hyperref package gives us a pdf with properly built
    % internal navigation ('pdf bookmarks' for the table of contents,
    % internal cross-reference links, web links for URLs, etc.)
    \usepackage{hyperref}
    \usepackage{longtable} % longtable support required by pandoc >1.10
    \usepackage{booktabs}  % table support for pandoc > 1.12.2
    \usepackage{ulem} % ulem is needed to support strikethroughs (\sout)
    

    
    
    \definecolor{orange}{cmyk}{0,0.4,0.8,0.2}
    \definecolor{darkorange}{rgb}{.71,0.21,0.01}
    \definecolor{darkgreen}{rgb}{.12,.54,.11}
    \definecolor{myteal}{rgb}{.26, .44, .56}
    \definecolor{gray}{gray}{0.45}
    \definecolor{lightgray}{gray}{.95}
    \definecolor{mediumgray}{gray}{.8}
    \definecolor{inputbackground}{rgb}{.95, .95, .85}
    \definecolor{outputbackground}{rgb}{.95, .95, .95}
    \definecolor{traceback}{rgb}{1, .95, .95}
    % ansi colors
    \definecolor{red}{rgb}{.6,0,0}
    \definecolor{green}{rgb}{0,.65,0}
    \definecolor{brown}{rgb}{0.6,0.6,0}
    \definecolor{blue}{rgb}{0,.145,.698}
    \definecolor{purple}{rgb}{.698,.145,.698}
    \definecolor{cyan}{rgb}{0,.698,.698}
    \definecolor{lightgray}{gray}{0.5}
    
    % bright ansi colors
    \definecolor{darkgray}{gray}{0.25}
    \definecolor{lightred}{rgb}{1.0,0.39,0.28}
    \definecolor{lightgreen}{rgb}{0.48,0.99,0.0}
    \definecolor{lightblue}{rgb}{0.53,0.81,0.92}
    \definecolor{lightpurple}{rgb}{0.87,0.63,0.87}
    \definecolor{lightcyan}{rgb}{0.5,1.0,0.83}
    
    % commands and environments needed by pandoc snippets
    % extracted from the output of `pandoc -s`
    \providecommand{\tightlist}{%
      \setlength{\itemsep}{0pt}\setlength{\parskip}{0pt}}
    \DefineVerbatimEnvironment{Highlighting}{Verbatim}{commandchars=\\\{\}}
    % Add ',fontsize=\small' for more characters per line
    \newenvironment{Shaded}{}{}
    \newcommand{\KeywordTok}[1]{\textcolor[rgb]{0.00,0.44,0.13}{\textbf{{#1}}}}
    \newcommand{\DataTypeTok}[1]{\textcolor[rgb]{0.56,0.13,0.00}{{#1}}}
    \newcommand{\DecValTok}[1]{\textcolor[rgb]{0.25,0.63,0.44}{{#1}}}
    \newcommand{\BaseNTok}[1]{\textcolor[rgb]{0.25,0.63,0.44}{{#1}}}
    \newcommand{\FloatTok}[1]{\textcolor[rgb]{0.25,0.63,0.44}{{#1}}}
    \newcommand{\CharTok}[1]{\textcolor[rgb]{0.25,0.44,0.63}{{#1}}}
    \newcommand{\StringTok}[1]{\textcolor[rgb]{0.25,0.44,0.63}{{#1}}}
    \newcommand{\CommentTok}[1]{\textcolor[rgb]{0.38,0.63,0.69}{\textit{{#1}}}}
    \newcommand{\OtherTok}[1]{\textcolor[rgb]{0.00,0.44,0.13}{{#1}}}
    \newcommand{\AlertTok}[1]{\textcolor[rgb]{1.00,0.00,0.00}{\textbf{{#1}}}}
    \newcommand{\FunctionTok}[1]{\textcolor[rgb]{0.02,0.16,0.49}{{#1}}}
    \newcommand{\RegionMarkerTok}[1]{{#1}}
    \newcommand{\ErrorTok}[1]{\textcolor[rgb]{1.00,0.00,0.00}{\textbf{{#1}}}}
    \newcommand{\NormalTok}[1]{{#1}}
    
    % Additional commands for more recent versions of Pandoc
    \newcommand{\ConstantTok}[1]{\textcolor[rgb]{0.53,0.00,0.00}{{#1}}}
    \newcommand{\SpecialCharTok}[1]{\textcolor[rgb]{0.25,0.44,0.63}{{#1}}}
    \newcommand{\VerbatimStringTok}[1]{\textcolor[rgb]{0.25,0.44,0.63}{{#1}}}
    \newcommand{\SpecialStringTok}[1]{\textcolor[rgb]{0.73,0.40,0.53}{{#1}}}
    \newcommand{\ImportTok}[1]{{#1}}
    \newcommand{\DocumentationTok}[1]{\textcolor[rgb]{0.73,0.13,0.13}{\textit{{#1}}}}
    \newcommand{\AnnotationTok}[1]{\textcolor[rgb]{0.38,0.63,0.69}{\textbf{\textit{{#1}}}}}
    \newcommand{\CommentVarTok}[1]{\textcolor[rgb]{0.38,0.63,0.69}{\textbf{\textit{{#1}}}}}
    \newcommand{\VariableTok}[1]{\textcolor[rgb]{0.10,0.09,0.49}{{#1}}}
    \newcommand{\ControlFlowTok}[1]{\textcolor[rgb]{0.00,0.44,0.13}{\textbf{{#1}}}}
    \newcommand{\OperatorTok}[1]{\textcolor[rgb]{0.40,0.40,0.40}{{#1}}}
    \newcommand{\BuiltInTok}[1]{{#1}}
    \newcommand{\ExtensionTok}[1]{{#1}}
    \newcommand{\PreprocessorTok}[1]{\textcolor[rgb]{0.74,0.48,0.00}{{#1}}}
    \newcommand{\AttributeTok}[1]{\textcolor[rgb]{0.49,0.56,0.16}{{#1}}}
    \newcommand{\InformationTok}[1]{\textcolor[rgb]{0.38,0.63,0.69}{\textbf{\textit{{#1}}}}}
    \newcommand{\WarningTok}[1]{\textcolor[rgb]{0.38,0.63,0.69}{\textbf{\textit{{#1}}}}}
    
    
    % Define a nice break command that doesn't care if a line doesn't already
    % exist.
    \def\br{\hspace*{\fill} \\* }
    % Math Jax compatability definitions
    \def\gt{>}
    \def\lt{<}
    % Document parameters
    \title{0\_Toc}
    
    
    

    % Pygments definitions
    
\makeatletter
\def\PY@reset{\let\PY@it=\relax \let\PY@bf=\relax%
    \let\PY@ul=\relax \let\PY@tc=\relax%
    \let\PY@bc=\relax \let\PY@ff=\relax}
\def\PY@tok#1{\csname PY@tok@#1\endcsname}
\def\PY@toks#1+{\ifx\relax#1\empty\else%
    \PY@tok{#1}\expandafter\PY@toks\fi}
\def\PY@do#1{\PY@bc{\PY@tc{\PY@ul{%
    \PY@it{\PY@bf{\PY@ff{#1}}}}}}}
\def\PY#1#2{\PY@reset\PY@toks#1+\relax+\PY@do{#2}}

\expandafter\def\csname PY@tok@nd\endcsname{\def\PY@tc##1{\textcolor[rgb]{0.67,0.13,1.00}{##1}}}
\expandafter\def\csname PY@tok@mb\endcsname{\def\PY@tc##1{\textcolor[rgb]{0.40,0.40,0.40}{##1}}}
\expandafter\def\csname PY@tok@gs\endcsname{\let\PY@bf=\textbf}
\expandafter\def\csname PY@tok@nb\endcsname{\def\PY@tc##1{\textcolor[rgb]{0.00,0.50,0.00}{##1}}}
\expandafter\def\csname PY@tok@mf\endcsname{\def\PY@tc##1{\textcolor[rgb]{0.40,0.40,0.40}{##1}}}
\expandafter\def\csname PY@tok@bp\endcsname{\def\PY@tc##1{\textcolor[rgb]{0.00,0.50,0.00}{##1}}}
\expandafter\def\csname PY@tok@gh\endcsname{\let\PY@bf=\textbf\def\PY@tc##1{\textcolor[rgb]{0.00,0.00,0.50}{##1}}}
\expandafter\def\csname PY@tok@si\endcsname{\let\PY@bf=\textbf\def\PY@tc##1{\textcolor[rgb]{0.73,0.40,0.53}{##1}}}
\expandafter\def\csname PY@tok@gt\endcsname{\def\PY@tc##1{\textcolor[rgb]{0.00,0.27,0.87}{##1}}}
\expandafter\def\csname PY@tok@s\endcsname{\def\PY@tc##1{\textcolor[rgb]{0.73,0.13,0.13}{##1}}}
\expandafter\def\csname PY@tok@gu\endcsname{\let\PY@bf=\textbf\def\PY@tc##1{\textcolor[rgb]{0.50,0.00,0.50}{##1}}}
\expandafter\def\csname PY@tok@ge\endcsname{\let\PY@it=\textit}
\expandafter\def\csname PY@tok@nt\endcsname{\let\PY@bf=\textbf\def\PY@tc##1{\textcolor[rgb]{0.00,0.50,0.00}{##1}}}
\expandafter\def\csname PY@tok@kr\endcsname{\let\PY@bf=\textbf\def\PY@tc##1{\textcolor[rgb]{0.00,0.50,0.00}{##1}}}
\expandafter\def\csname PY@tok@cpf\endcsname{\let\PY@it=\textit\def\PY@tc##1{\textcolor[rgb]{0.25,0.50,0.50}{##1}}}
\expandafter\def\csname PY@tok@vi\endcsname{\def\PY@tc##1{\textcolor[rgb]{0.10,0.09,0.49}{##1}}}
\expandafter\def\csname PY@tok@sx\endcsname{\def\PY@tc##1{\textcolor[rgb]{0.00,0.50,0.00}{##1}}}
\expandafter\def\csname PY@tok@nc\endcsname{\let\PY@bf=\textbf\def\PY@tc##1{\textcolor[rgb]{0.00,0.00,1.00}{##1}}}
\expandafter\def\csname PY@tok@s1\endcsname{\def\PY@tc##1{\textcolor[rgb]{0.73,0.13,0.13}{##1}}}
\expandafter\def\csname PY@tok@sc\endcsname{\def\PY@tc##1{\textcolor[rgb]{0.73,0.13,0.13}{##1}}}
\expandafter\def\csname PY@tok@sr\endcsname{\def\PY@tc##1{\textcolor[rgb]{0.73,0.40,0.53}{##1}}}
\expandafter\def\csname PY@tok@nn\endcsname{\let\PY@bf=\textbf\def\PY@tc##1{\textcolor[rgb]{0.00,0.00,1.00}{##1}}}
\expandafter\def\csname PY@tok@gp\endcsname{\let\PY@bf=\textbf\def\PY@tc##1{\textcolor[rgb]{0.00,0.00,0.50}{##1}}}
\expandafter\def\csname PY@tok@cm\endcsname{\let\PY@it=\textit\def\PY@tc##1{\textcolor[rgb]{0.25,0.50,0.50}{##1}}}
\expandafter\def\csname PY@tok@kn\endcsname{\let\PY@bf=\textbf\def\PY@tc##1{\textcolor[rgb]{0.00,0.50,0.00}{##1}}}
\expandafter\def\csname PY@tok@kc\endcsname{\let\PY@bf=\textbf\def\PY@tc##1{\textcolor[rgb]{0.00,0.50,0.00}{##1}}}
\expandafter\def\csname PY@tok@mo\endcsname{\def\PY@tc##1{\textcolor[rgb]{0.40,0.40,0.40}{##1}}}
\expandafter\def\csname PY@tok@cs\endcsname{\let\PY@it=\textit\def\PY@tc##1{\textcolor[rgb]{0.25,0.50,0.50}{##1}}}
\expandafter\def\csname PY@tok@na\endcsname{\def\PY@tc##1{\textcolor[rgb]{0.49,0.56,0.16}{##1}}}
\expandafter\def\csname PY@tok@vc\endcsname{\def\PY@tc##1{\textcolor[rgb]{0.10,0.09,0.49}{##1}}}
\expandafter\def\csname PY@tok@nl\endcsname{\def\PY@tc##1{\textcolor[rgb]{0.63,0.63,0.00}{##1}}}
\expandafter\def\csname PY@tok@ow\endcsname{\let\PY@bf=\textbf\def\PY@tc##1{\textcolor[rgb]{0.67,0.13,1.00}{##1}}}
\expandafter\def\csname PY@tok@sd\endcsname{\let\PY@it=\textit\def\PY@tc##1{\textcolor[rgb]{0.73,0.13,0.13}{##1}}}
\expandafter\def\csname PY@tok@gd\endcsname{\def\PY@tc##1{\textcolor[rgb]{0.63,0.00,0.00}{##1}}}
\expandafter\def\csname PY@tok@c1\endcsname{\let\PY@it=\textit\def\PY@tc##1{\textcolor[rgb]{0.25,0.50,0.50}{##1}}}
\expandafter\def\csname PY@tok@kp\endcsname{\def\PY@tc##1{\textcolor[rgb]{0.00,0.50,0.00}{##1}}}
\expandafter\def\csname PY@tok@il\endcsname{\def\PY@tc##1{\textcolor[rgb]{0.40,0.40,0.40}{##1}}}
\expandafter\def\csname PY@tok@ni\endcsname{\let\PY@bf=\textbf\def\PY@tc##1{\textcolor[rgb]{0.60,0.60,0.60}{##1}}}
\expandafter\def\csname PY@tok@ss\endcsname{\def\PY@tc##1{\textcolor[rgb]{0.10,0.09,0.49}{##1}}}
\expandafter\def\csname PY@tok@c\endcsname{\let\PY@it=\textit\def\PY@tc##1{\textcolor[rgb]{0.25,0.50,0.50}{##1}}}
\expandafter\def\csname PY@tok@cp\endcsname{\def\PY@tc##1{\textcolor[rgb]{0.74,0.48,0.00}{##1}}}
\expandafter\def\csname PY@tok@o\endcsname{\def\PY@tc##1{\textcolor[rgb]{0.40,0.40,0.40}{##1}}}
\expandafter\def\csname PY@tok@kd\endcsname{\let\PY@bf=\textbf\def\PY@tc##1{\textcolor[rgb]{0.00,0.50,0.00}{##1}}}
\expandafter\def\csname PY@tok@go\endcsname{\def\PY@tc##1{\textcolor[rgb]{0.53,0.53,0.53}{##1}}}
\expandafter\def\csname PY@tok@kt\endcsname{\def\PY@tc##1{\textcolor[rgb]{0.69,0.00,0.25}{##1}}}
\expandafter\def\csname PY@tok@mi\endcsname{\def\PY@tc##1{\textcolor[rgb]{0.40,0.40,0.40}{##1}}}
\expandafter\def\csname PY@tok@no\endcsname{\def\PY@tc##1{\textcolor[rgb]{0.53,0.00,0.00}{##1}}}
\expandafter\def\csname PY@tok@ch\endcsname{\let\PY@it=\textit\def\PY@tc##1{\textcolor[rgb]{0.25,0.50,0.50}{##1}}}
\expandafter\def\csname PY@tok@ne\endcsname{\let\PY@bf=\textbf\def\PY@tc##1{\textcolor[rgb]{0.82,0.25,0.23}{##1}}}
\expandafter\def\csname PY@tok@gi\endcsname{\def\PY@tc##1{\textcolor[rgb]{0.00,0.63,0.00}{##1}}}
\expandafter\def\csname PY@tok@w\endcsname{\def\PY@tc##1{\textcolor[rgb]{0.73,0.73,0.73}{##1}}}
\expandafter\def\csname PY@tok@se\endcsname{\let\PY@bf=\textbf\def\PY@tc##1{\textcolor[rgb]{0.73,0.40,0.13}{##1}}}
\expandafter\def\csname PY@tok@s2\endcsname{\def\PY@tc##1{\textcolor[rgb]{0.73,0.13,0.13}{##1}}}
\expandafter\def\csname PY@tok@nv\endcsname{\def\PY@tc##1{\textcolor[rgb]{0.10,0.09,0.49}{##1}}}
\expandafter\def\csname PY@tok@m\endcsname{\def\PY@tc##1{\textcolor[rgb]{0.40,0.40,0.40}{##1}}}
\expandafter\def\csname PY@tok@k\endcsname{\let\PY@bf=\textbf\def\PY@tc##1{\textcolor[rgb]{0.00,0.50,0.00}{##1}}}
\expandafter\def\csname PY@tok@mh\endcsname{\def\PY@tc##1{\textcolor[rgb]{0.40,0.40,0.40}{##1}}}
\expandafter\def\csname PY@tok@gr\endcsname{\def\PY@tc##1{\textcolor[rgb]{1.00,0.00,0.00}{##1}}}
\expandafter\def\csname PY@tok@sb\endcsname{\def\PY@tc##1{\textcolor[rgb]{0.73,0.13,0.13}{##1}}}
\expandafter\def\csname PY@tok@sh\endcsname{\def\PY@tc##1{\textcolor[rgb]{0.73,0.13,0.13}{##1}}}
\expandafter\def\csname PY@tok@vg\endcsname{\def\PY@tc##1{\textcolor[rgb]{0.10,0.09,0.49}{##1}}}
\expandafter\def\csname PY@tok@nf\endcsname{\def\PY@tc##1{\textcolor[rgb]{0.00,0.00,1.00}{##1}}}
\expandafter\def\csname PY@tok@err\endcsname{\def\PY@bc##1{\setlength{\fboxsep}{0pt}\fcolorbox[rgb]{1.00,0.00,0.00}{1,1,1}{\strut ##1}}}

\def\PYZbs{\char`\\}
\def\PYZus{\char`\_}
\def\PYZob{\char`\{}
\def\PYZcb{\char`\}}
\def\PYZca{\char`\^}
\def\PYZam{\char`\&}
\def\PYZlt{\char`\<}
\def\PYZgt{\char`\>}
\def\PYZsh{\char`\#}
\def\PYZpc{\char`\%}
\def\PYZdl{\char`\$}
\def\PYZhy{\char`\-}
\def\PYZsq{\char`\'}
\def\PYZdq{\char`\"}
\def\PYZti{\char`\~}
% for compatibility with earlier versions
\def\PYZat{@}
\def\PYZlb{[}
\def\PYZrb{]}
\makeatother


    % Exact colors from NB
    \definecolor{incolor}{rgb}{0.0, 0.0, 0.5}
    \definecolor{outcolor}{rgb}{0.545, 0.0, 0.0}



    
    % Prevent overflowing lines due to hard-to-break entities
    \sloppy 
    % Setup hyperref package
    \hypersetup{
      breaklinks=true,  % so long urls are correctly broken across lines
      colorlinks=true,
      urlcolor=blue,
      linkcolor=darkorange,
      citecolor=darkgreen,
      }
    % Slightly bigger margins than the latex defaults
    
    \geometry{verbose,tmargin=1in,bmargin=1in,lmargin=1in,rmargin=1in}
    
    

    \begin{document}
    
    
    \maketitle
    
    

    
    \begin{enumerate}
\def\labelenumi{\arabic{enumi}.}
\tightlist
\item
  \href{/1-Motivation-and-Introduction.html}{Motivation and
  Introduction}
\item
  \href{/2-ROOT-Basics.html}{ROOT Basics}

  \begin{itemize}
  \tightlist
  \item
    2.1 \href{/2-ROOT-Basics.html\#2.1-ROOT-as-calculator}{ROOT as
    calculator}
  \item
    2.2
    \href{/2-ROOT-Basics.html\#2.2-Learn-C++-at-the-ROOT-prompt}{Learn
    C++ at the ROOT prompt}
  \item
    2.3
    \href{/notebooks/notebooks/2-ROOT-Basics.html\#2.3-ROOT-as-function-plotter}{ROOT
    as function plotter}
  \item
    2.4 \href{/2-ROOT-Basics.html\#2.4Controlling-ROOT}{Controlling
    ROOT}
  \item
    2.5 \href{/2-ROOT-Basics.html\#2.5-Plotting-Measurements}{Plotting
    Measurements}
  \item
    2.6 \href{/2-ROOT-Basics.html\#2.6-Histograms-in-ROOT}{Histograms in
    ROOT}
  \item
    2.7 \href{/2-ROOT-Basics.html\#2.7-Interactive-ROOT}{Interactive
    ROOT}
  \item
    2.8 \href{/2-ROOT-Basics.html\#2.8-ROOT-Beginners’-FAQ}{ROOT
    Beginners' FAQ}

    \begin{itemize}
    \tightlist
    \item
      2.8.1
      \href{/2-ROOT-Basics.html\#2.8.1-ROOT-type-declarations-for-basic-data-types}{ROOT
      type declarations for basic data types}
    \item
      2.8.2
      \href{/2-ROOT-Basics.html\#2.8.2-Configure-ROOT-at-start-up}{Configure
      ROOT at start-up}
    \item
      2.8.2 \href{/2-ROOT-Basics.html\#2.8.3-ROOT-command-history}{ROOT
      command history}
    \item
      2.8.3 \href{/2-ROOT-Basics.html\#2.8.4-ROOT-Global-Pointers}{ROOT
      Global Pointers}
    \end{itemize}
  \end{itemize}
\item
  \href{/3-ROOT-Macros.html}{Root Macros}

  \begin{itemize}
  \item
    3.1
    \href{/3-ROOT-Macros.html\#3.1-General-Remarks-on-ROOT-macros}{General
    Remarks on ROOT macros}
  \item
    \begin{itemize}
    \tightlist
    \item
      3.2 \href{/3-ROOT-Macros.html\#3.2-A-more-complete-example}{A more
      complete example}
    \end{itemize}
  \item
    3.3
    \href{/3-ROOT-Macros.html\#3.3-Summary-of-Visual-effects}{Summary of
    Visual effects}

    \begin{itemize}
    \item
      \begin{itemize}
      \item
        \begin{itemize}
        \tightlist
        \item
          3.3.1
          \href{/3-ROOT-Macros.html\#3.3.1-Colours-and-Graph-Markers}{Colours
          and Graph Markers}
        \end{itemize}
      \end{itemize}
    \item
      \begin{itemize}
      \tightlist
      \item
        3.3.2 \href{/3-ROOT-Macros.html\#3.3.2-Arrows-and-Lines}{Arrows
        and Lines}
      \end{itemize}
    \item
      3.3.3 \href{/3-ROOT-Macros.html\#3.3.3-Text}{Text}
    \end{itemize}
  \item
    3.4
    \href{/3-ROOT-Macros.html\#3.4-Interpretation-and-Compilation}{Interpretation
    and Compilation}

    \begin{itemize}
    \item
      \begin{itemize}
      \item
        \begin{itemize}
        \tightlist
        \item
          3.4.1
          \href{/3-ROOT-Macros.html\#3.4.1-Compile-a-Macro-with-ACLiC}{Compile
          a Macro with ACLiC}
        \end{itemize}
      \end{itemize}
    \item
      \begin{itemize}
      \tightlist
      \item
        3.4.2
        \href{/3-ROOT-Macros.html\#3.4.2-Compile-a-Macro-with-the-Compiler}{Compile
        a Macro with the Compiler}
      \end{itemize}
    \end{itemize}
  \end{itemize}
\item
  \href{/4-Graphs.html}{Graphs}

  \begin{itemize}
  \tightlist
  \item
    4.1 \href{/4-Graphs.html\#4.1-Read-Graph-Points-from-File}{Read
    Graph Points from File}
  \item
    4.2 \href{/4-Graphs.html\#4.2-Polar-Graphs}{Polar Graphs}
  \item
    4.3 \href{/4-Graphs.html\#4.3-2D-Graphs}{2D Graphs}
  \item
    4.4 \href{/4-Graphs.html\#4.4-Multiple-graphs}{Multiple graphs}
  \end{itemize}
\item
  \href{/5-Histograms.html}{Histograms}

  \begin{itemize}
  \tightlist
  \item
    5.1 \href{/5-Histograms.html\#5.1-Your-First-Histogram}{Your First
    Histogram}
  \item
    5.2 \href{/5-Histograms.html\#5.2-Add-and-Divide-Histograms}{Add and
    Divide Histograms}
  \item
    5.3
    \href{/5-Histograms.html\#5.3-Two-dimensional-Histograms}{Two-dimensional
    Histograms}
  \item
    5.4 \href{/5-Histograms.html\#5.4-Multiple-histograms}{Multiple
    histograms}
  \end{itemize}
\item
  \href{/6-Functions-and-Parameter-Estimation.html}{Functions and
  Parameter Estimation}

  \begin{itemize}
  \tightlist
  \item
    6.1
    \href{/6-Functions-and-Parameter-Estimation.html\#6.1-Fitting-Functions-to-Pseudo-Data}{Fitting
    Functions to Pseudo Data}
  \item
    6.2
    \href{/6-Functions-and-Parameter-Estimation.html\#6.2-Toy-Monte-Carlo-Experiments}{Toy
    Monte Carlo Experiments}
  \end{itemize}
\item
  \href{/7-File-IO-and-Parallel-Analysis.html}{File IO and Parallel
  Analysis}

  \begin{itemize}
  \tightlist
  \item
    7.1
    \href{/7-File-IO-and-Parallel-Analysis.html\#7.1-Storing-ROOT-Objects}{Storing
    ROOT Objects}
  \item
    7.2
    \href{/7-File-IO-and-Parallel-Analysis.html\#7.2-N-tuples-in-ROOT}{N-tuples
    in ROOT}

    \begin{itemize}
    \tightlist
    \item
      7.2.1
      \href{/7-File-IO-and-Parallel-Analysis.html\#7.2.1-Storing-simple-N-tuples}{Storing
      simple N-tuples}
    \item
      7.2.2
      \href{/7-File-IO-and-Parallel-Analysis.html\#7.2.2-Reading-N-tuples}{Reading
      N-tuples}
    \item
      7.2.3
      \href{/7-File-IO-and-Parallel-Analysis.html\#7.2.3-Storing-Arbitrary-N-tuples}{Storing
      Arbitrary N-tuples}
    \item
      7.2.4
      \href{/7-File-IO-and-Parallel-Analysis.html\#7.2.4-Processing-N-tuples-Spanning-over-Several-Files}{Processing
      N-tuples Spanning over Several Files}
    \item
      7.2.5
      \href{/7-File-IO-and-Parallel-Analysis.html\#7.2.5-For-the-advanced-user:-Processing-trees-with-a-selector-script}{For
      the advanced user: Processing trees with a selector script}
    \item
      7.2.6
      \href{/7-File-IO-and-Parallel-Analysis.html\#7.2.6-For-power-users:-Multi-core-processing-with-PROOF-lite}{For
      power-users: Multi-core processing with PROOF lite}
    \item
      7.2.7
      \href{/7-File-IO-and-Parallel-Analysis.html\#7.2.7-Optimisation-Regarding-N-tuples}{Optimisation
      Regarding N-tuples}
    \end{itemize}
  \end{itemize}
\item
  \href{/8-ROOT-in-Python.html}{ROOT in Python}

  \begin{itemize}
  \tightlist
  \item
    8.1 \href{/8-ROOT-in-Python.html\#8.1-PyROOT}{PyROOT}

    \begin{itemize}
    \tightlist
    \item
      8.1.1
      \href{/8-ROOT-in-Python.html\#8.1.1-More-Python--less-C++}{More
      Python- less C++}

      \begin{itemize}
      \tightlist
      \item
        8.1.1.1
        \href{/8-ROOT-in-Python.html\#8.1.1.1-Customised-Binning}{Customised
        Binning}
      \end{itemize}
    \end{itemize}
  \item
    8.2
    \href{/8-ROOT-in-Python.html\#8.2-Custom-code:-from-C++-to-Python}{Custom
    code: from C++ to Python}
  \end{itemize}
\item
  \href{/9-Concluding-Remarks.html}{Concluding Remarks}

  \begin{itemize}
  \tightlist
  \item
    9.1 \href{/9-Concluding-Remarks.html\#9-References}{References}
  \end{itemize}
\end{enumerate}


    % Add a bibliography block to the postdoc
    
    
    
    \end{document}
