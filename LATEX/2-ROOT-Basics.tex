
% Default to the notebook output style

    


% Inherit from the specified cell style.




    
\documentclass{article}

    
    
    \usepackage{graphicx} % Used to insert images
    \usepackage{adjustbox} % Used to constrain images to a maximum size 
    \usepackage{color} % Allow colors to be defined
    \usepackage{enumerate} % Needed for markdown enumerations to work
    \usepackage{geometry} % Used to adjust the document margins
    \usepackage{amsmath} % Equations
    \usepackage{amssymb} % Equations
    \usepackage{eurosym} % defines \euro
    \usepackage[mathletters]{ucs} % Extended unicode (utf-8) support
    \usepackage[utf8x]{inputenc} % Allow utf-8 characters in the tex document
    \usepackage{fancyvrb} % verbatim replacement that allows latex
    \usepackage{grffile} % extends the file name processing of package graphics 
                         % to support a larger range 
    % The hyperref package gives us a pdf with properly built
    % internal navigation ('pdf bookmarks' for the table of contents,
    % internal cross-reference links, web links for URLs, etc.)
    \usepackage{hyperref}
    \usepackage{longtable} % longtable support required by pandoc >1.10
    \usepackage{booktabs}  % table support for pandoc > 1.12.2
    \usepackage{ulem} % ulem is needed to support strikethroughs (\sout)
    

    
    
    \definecolor{orange}{cmyk}{0,0.4,0.8,0.2}
    \definecolor{darkorange}{rgb}{.71,0.21,0.01}
    \definecolor{darkgreen}{rgb}{.12,.54,.11}
    \definecolor{myteal}{rgb}{.26, .44, .56}
    \definecolor{gray}{gray}{0.45}
    \definecolor{lightgray}{gray}{.95}
    \definecolor{mediumgray}{gray}{.8}
    \definecolor{inputbackground}{rgb}{.95, .95, .85}
    \definecolor{outputbackground}{rgb}{.95, .95, .95}
    \definecolor{traceback}{rgb}{1, .95, .95}
    % ansi colors
    \definecolor{red}{rgb}{.6,0,0}
    \definecolor{green}{rgb}{0,.65,0}
    \definecolor{brown}{rgb}{0.6,0.6,0}
    \definecolor{blue}{rgb}{0,.145,.698}
    \definecolor{purple}{rgb}{.698,.145,.698}
    \definecolor{cyan}{rgb}{0,.698,.698}
    \definecolor{lightgray}{gray}{0.5}
    
    % bright ansi colors
    \definecolor{darkgray}{gray}{0.25}
    \definecolor{lightred}{rgb}{1.0,0.39,0.28}
    \definecolor{lightgreen}{rgb}{0.48,0.99,0.0}
    \definecolor{lightblue}{rgb}{0.53,0.81,0.92}
    \definecolor{lightpurple}{rgb}{0.87,0.63,0.87}
    \definecolor{lightcyan}{rgb}{0.5,1.0,0.83}
    
    % commands and environments needed by pandoc snippets
    % extracted from the output of `pandoc -s`
    \providecommand{\tightlist}{%
      \setlength{\itemsep}{0pt}\setlength{\parskip}{0pt}}
    \DefineVerbatimEnvironment{Highlighting}{Verbatim}{commandchars=\\\{\}}
    % Add ',fontsize=\small' for more characters per line
    \newenvironment{Shaded}{}{}
    \newcommand{\KeywordTok}[1]{\textcolor[rgb]{0.00,0.44,0.13}{\textbf{{#1}}}}
    \newcommand{\DataTypeTok}[1]{\textcolor[rgb]{0.56,0.13,0.00}{{#1}}}
    \newcommand{\DecValTok}[1]{\textcolor[rgb]{0.25,0.63,0.44}{{#1}}}
    \newcommand{\BaseNTok}[1]{\textcolor[rgb]{0.25,0.63,0.44}{{#1}}}
    \newcommand{\FloatTok}[1]{\textcolor[rgb]{0.25,0.63,0.44}{{#1}}}
    \newcommand{\CharTok}[1]{\textcolor[rgb]{0.25,0.44,0.63}{{#1}}}
    \newcommand{\StringTok}[1]{\textcolor[rgb]{0.25,0.44,0.63}{{#1}}}
    \newcommand{\CommentTok}[1]{\textcolor[rgb]{0.38,0.63,0.69}{\textit{{#1}}}}
    \newcommand{\OtherTok}[1]{\textcolor[rgb]{0.00,0.44,0.13}{{#1}}}
    \newcommand{\AlertTok}[1]{\textcolor[rgb]{1.00,0.00,0.00}{\textbf{{#1}}}}
    \newcommand{\FunctionTok}[1]{\textcolor[rgb]{0.02,0.16,0.49}{{#1}}}
    \newcommand{\RegionMarkerTok}[1]{{#1}}
    \newcommand{\ErrorTok}[1]{\textcolor[rgb]{1.00,0.00,0.00}{\textbf{{#1}}}}
    \newcommand{\NormalTok}[1]{{#1}}
    
    % Additional commands for more recent versions of Pandoc
    \newcommand{\ConstantTok}[1]{\textcolor[rgb]{0.53,0.00,0.00}{{#1}}}
    \newcommand{\SpecialCharTok}[1]{\textcolor[rgb]{0.25,0.44,0.63}{{#1}}}
    \newcommand{\VerbatimStringTok}[1]{\textcolor[rgb]{0.25,0.44,0.63}{{#1}}}
    \newcommand{\SpecialStringTok}[1]{\textcolor[rgb]{0.73,0.40,0.53}{{#1}}}
    \newcommand{\ImportTok}[1]{{#1}}
    \newcommand{\DocumentationTok}[1]{\textcolor[rgb]{0.73,0.13,0.13}{\textit{{#1}}}}
    \newcommand{\AnnotationTok}[1]{\textcolor[rgb]{0.38,0.63,0.69}{\textbf{\textit{{#1}}}}}
    \newcommand{\CommentVarTok}[1]{\textcolor[rgb]{0.38,0.63,0.69}{\textbf{\textit{{#1}}}}}
    \newcommand{\VariableTok}[1]{\textcolor[rgb]{0.10,0.09,0.49}{{#1}}}
    \newcommand{\ControlFlowTok}[1]{\textcolor[rgb]{0.00,0.44,0.13}{\textbf{{#1}}}}
    \newcommand{\OperatorTok}[1]{\textcolor[rgb]{0.40,0.40,0.40}{{#1}}}
    \newcommand{\BuiltInTok}[1]{{#1}}
    \newcommand{\ExtensionTok}[1]{{#1}}
    \newcommand{\PreprocessorTok}[1]{\textcolor[rgb]{0.74,0.48,0.00}{{#1}}}
    \newcommand{\AttributeTok}[1]{\textcolor[rgb]{0.49,0.56,0.16}{{#1}}}
    \newcommand{\InformationTok}[1]{\textcolor[rgb]{0.38,0.63,0.69}{\textbf{\textit{{#1}}}}}
    \newcommand{\WarningTok}[1]{\textcolor[rgb]{0.38,0.63,0.69}{\textbf{\textit{{#1}}}}}
    
    
    % Define a nice break command that doesn't care if a line doesn't already
    % exist.
    \def\br{\hspace*{\fill} \\* }
    % Math Jax compatability definitions
    \def\gt{>}
    \def\lt{<}
    % Document parameters
    \title{2-ROOT-Basics}
    
    
    

    % Pygments definitions
    
\makeatletter
\def\PY@reset{\let\PY@it=\relax \let\PY@bf=\relax%
    \let\PY@ul=\relax \let\PY@tc=\relax%
    \let\PY@bc=\relax \let\PY@ff=\relax}
\def\PY@tok#1{\csname PY@tok@#1\endcsname}
\def\PY@toks#1+{\ifx\relax#1\empty\else%
    \PY@tok{#1}\expandafter\PY@toks\fi}
\def\PY@do#1{\PY@bc{\PY@tc{\PY@ul{%
    \PY@it{\PY@bf{\PY@ff{#1}}}}}}}
\def\PY#1#2{\PY@reset\PY@toks#1+\relax+\PY@do{#2}}

\expandafter\def\csname PY@tok@nd\endcsname{\def\PY@tc##1{\textcolor[rgb]{0.67,0.13,1.00}{##1}}}
\expandafter\def\csname PY@tok@mb\endcsname{\def\PY@tc##1{\textcolor[rgb]{0.40,0.40,0.40}{##1}}}
\expandafter\def\csname PY@tok@gs\endcsname{\let\PY@bf=\textbf}
\expandafter\def\csname PY@tok@nb\endcsname{\def\PY@tc##1{\textcolor[rgb]{0.00,0.50,0.00}{##1}}}
\expandafter\def\csname PY@tok@mf\endcsname{\def\PY@tc##1{\textcolor[rgb]{0.40,0.40,0.40}{##1}}}
\expandafter\def\csname PY@tok@bp\endcsname{\def\PY@tc##1{\textcolor[rgb]{0.00,0.50,0.00}{##1}}}
\expandafter\def\csname PY@tok@gh\endcsname{\let\PY@bf=\textbf\def\PY@tc##1{\textcolor[rgb]{0.00,0.00,0.50}{##1}}}
\expandafter\def\csname PY@tok@si\endcsname{\let\PY@bf=\textbf\def\PY@tc##1{\textcolor[rgb]{0.73,0.40,0.53}{##1}}}
\expandafter\def\csname PY@tok@gt\endcsname{\def\PY@tc##1{\textcolor[rgb]{0.00,0.27,0.87}{##1}}}
\expandafter\def\csname PY@tok@s\endcsname{\def\PY@tc##1{\textcolor[rgb]{0.73,0.13,0.13}{##1}}}
\expandafter\def\csname PY@tok@gu\endcsname{\let\PY@bf=\textbf\def\PY@tc##1{\textcolor[rgb]{0.50,0.00,0.50}{##1}}}
\expandafter\def\csname PY@tok@ge\endcsname{\let\PY@it=\textit}
\expandafter\def\csname PY@tok@nt\endcsname{\let\PY@bf=\textbf\def\PY@tc##1{\textcolor[rgb]{0.00,0.50,0.00}{##1}}}
\expandafter\def\csname PY@tok@kr\endcsname{\let\PY@bf=\textbf\def\PY@tc##1{\textcolor[rgb]{0.00,0.50,0.00}{##1}}}
\expandafter\def\csname PY@tok@cpf\endcsname{\let\PY@it=\textit\def\PY@tc##1{\textcolor[rgb]{0.25,0.50,0.50}{##1}}}
\expandafter\def\csname PY@tok@vi\endcsname{\def\PY@tc##1{\textcolor[rgb]{0.10,0.09,0.49}{##1}}}
\expandafter\def\csname PY@tok@sx\endcsname{\def\PY@tc##1{\textcolor[rgb]{0.00,0.50,0.00}{##1}}}
\expandafter\def\csname PY@tok@nc\endcsname{\let\PY@bf=\textbf\def\PY@tc##1{\textcolor[rgb]{0.00,0.00,1.00}{##1}}}
\expandafter\def\csname PY@tok@s1\endcsname{\def\PY@tc##1{\textcolor[rgb]{0.73,0.13,0.13}{##1}}}
\expandafter\def\csname PY@tok@sc\endcsname{\def\PY@tc##1{\textcolor[rgb]{0.73,0.13,0.13}{##1}}}
\expandafter\def\csname PY@tok@sr\endcsname{\def\PY@tc##1{\textcolor[rgb]{0.73,0.40,0.53}{##1}}}
\expandafter\def\csname PY@tok@nn\endcsname{\let\PY@bf=\textbf\def\PY@tc##1{\textcolor[rgb]{0.00,0.00,1.00}{##1}}}
\expandafter\def\csname PY@tok@gp\endcsname{\let\PY@bf=\textbf\def\PY@tc##1{\textcolor[rgb]{0.00,0.00,0.50}{##1}}}
\expandafter\def\csname PY@tok@cm\endcsname{\let\PY@it=\textit\def\PY@tc##1{\textcolor[rgb]{0.25,0.50,0.50}{##1}}}
\expandafter\def\csname PY@tok@kn\endcsname{\let\PY@bf=\textbf\def\PY@tc##1{\textcolor[rgb]{0.00,0.50,0.00}{##1}}}
\expandafter\def\csname PY@tok@kc\endcsname{\let\PY@bf=\textbf\def\PY@tc##1{\textcolor[rgb]{0.00,0.50,0.00}{##1}}}
\expandafter\def\csname PY@tok@mo\endcsname{\def\PY@tc##1{\textcolor[rgb]{0.40,0.40,0.40}{##1}}}
\expandafter\def\csname PY@tok@cs\endcsname{\let\PY@it=\textit\def\PY@tc##1{\textcolor[rgb]{0.25,0.50,0.50}{##1}}}
\expandafter\def\csname PY@tok@na\endcsname{\def\PY@tc##1{\textcolor[rgb]{0.49,0.56,0.16}{##1}}}
\expandafter\def\csname PY@tok@vc\endcsname{\def\PY@tc##1{\textcolor[rgb]{0.10,0.09,0.49}{##1}}}
\expandafter\def\csname PY@tok@nl\endcsname{\def\PY@tc##1{\textcolor[rgb]{0.63,0.63,0.00}{##1}}}
\expandafter\def\csname PY@tok@ow\endcsname{\let\PY@bf=\textbf\def\PY@tc##1{\textcolor[rgb]{0.67,0.13,1.00}{##1}}}
\expandafter\def\csname PY@tok@sd\endcsname{\let\PY@it=\textit\def\PY@tc##1{\textcolor[rgb]{0.73,0.13,0.13}{##1}}}
\expandafter\def\csname PY@tok@gd\endcsname{\def\PY@tc##1{\textcolor[rgb]{0.63,0.00,0.00}{##1}}}
\expandafter\def\csname PY@tok@c1\endcsname{\let\PY@it=\textit\def\PY@tc##1{\textcolor[rgb]{0.25,0.50,0.50}{##1}}}
\expandafter\def\csname PY@tok@kp\endcsname{\def\PY@tc##1{\textcolor[rgb]{0.00,0.50,0.00}{##1}}}
\expandafter\def\csname PY@tok@il\endcsname{\def\PY@tc##1{\textcolor[rgb]{0.40,0.40,0.40}{##1}}}
\expandafter\def\csname PY@tok@ni\endcsname{\let\PY@bf=\textbf\def\PY@tc##1{\textcolor[rgb]{0.60,0.60,0.60}{##1}}}
\expandafter\def\csname PY@tok@ss\endcsname{\def\PY@tc##1{\textcolor[rgb]{0.10,0.09,0.49}{##1}}}
\expandafter\def\csname PY@tok@c\endcsname{\let\PY@it=\textit\def\PY@tc##1{\textcolor[rgb]{0.25,0.50,0.50}{##1}}}
\expandafter\def\csname PY@tok@cp\endcsname{\def\PY@tc##1{\textcolor[rgb]{0.74,0.48,0.00}{##1}}}
\expandafter\def\csname PY@tok@o\endcsname{\def\PY@tc##1{\textcolor[rgb]{0.40,0.40,0.40}{##1}}}
\expandafter\def\csname PY@tok@kd\endcsname{\let\PY@bf=\textbf\def\PY@tc##1{\textcolor[rgb]{0.00,0.50,0.00}{##1}}}
\expandafter\def\csname PY@tok@go\endcsname{\def\PY@tc##1{\textcolor[rgb]{0.53,0.53,0.53}{##1}}}
\expandafter\def\csname PY@tok@kt\endcsname{\def\PY@tc##1{\textcolor[rgb]{0.69,0.00,0.25}{##1}}}
\expandafter\def\csname PY@tok@mi\endcsname{\def\PY@tc##1{\textcolor[rgb]{0.40,0.40,0.40}{##1}}}
\expandafter\def\csname PY@tok@no\endcsname{\def\PY@tc##1{\textcolor[rgb]{0.53,0.00,0.00}{##1}}}
\expandafter\def\csname PY@tok@ch\endcsname{\let\PY@it=\textit\def\PY@tc##1{\textcolor[rgb]{0.25,0.50,0.50}{##1}}}
\expandafter\def\csname PY@tok@ne\endcsname{\let\PY@bf=\textbf\def\PY@tc##1{\textcolor[rgb]{0.82,0.25,0.23}{##1}}}
\expandafter\def\csname PY@tok@gi\endcsname{\def\PY@tc##1{\textcolor[rgb]{0.00,0.63,0.00}{##1}}}
\expandafter\def\csname PY@tok@w\endcsname{\def\PY@tc##1{\textcolor[rgb]{0.73,0.73,0.73}{##1}}}
\expandafter\def\csname PY@tok@se\endcsname{\let\PY@bf=\textbf\def\PY@tc##1{\textcolor[rgb]{0.73,0.40,0.13}{##1}}}
\expandafter\def\csname PY@tok@s2\endcsname{\def\PY@tc##1{\textcolor[rgb]{0.73,0.13,0.13}{##1}}}
\expandafter\def\csname PY@tok@nv\endcsname{\def\PY@tc##1{\textcolor[rgb]{0.10,0.09,0.49}{##1}}}
\expandafter\def\csname PY@tok@m\endcsname{\def\PY@tc##1{\textcolor[rgb]{0.40,0.40,0.40}{##1}}}
\expandafter\def\csname PY@tok@k\endcsname{\let\PY@bf=\textbf\def\PY@tc##1{\textcolor[rgb]{0.00,0.50,0.00}{##1}}}
\expandafter\def\csname PY@tok@mh\endcsname{\def\PY@tc##1{\textcolor[rgb]{0.40,0.40,0.40}{##1}}}
\expandafter\def\csname PY@tok@gr\endcsname{\def\PY@tc##1{\textcolor[rgb]{1.00,0.00,0.00}{##1}}}
\expandafter\def\csname PY@tok@sb\endcsname{\def\PY@tc##1{\textcolor[rgb]{0.73,0.13,0.13}{##1}}}
\expandafter\def\csname PY@tok@sh\endcsname{\def\PY@tc##1{\textcolor[rgb]{0.73,0.13,0.13}{##1}}}
\expandafter\def\csname PY@tok@vg\endcsname{\def\PY@tc##1{\textcolor[rgb]{0.10,0.09,0.49}{##1}}}
\expandafter\def\csname PY@tok@nf\endcsname{\def\PY@tc##1{\textcolor[rgb]{0.00,0.00,1.00}{##1}}}
\expandafter\def\csname PY@tok@err\endcsname{\def\PY@bc##1{\setlength{\fboxsep}{0pt}\fcolorbox[rgb]{1.00,0.00,0.00}{1,1,1}{\strut ##1}}}

\def\PYZbs{\char`\\}
\def\PYZus{\char`\_}
\def\PYZob{\char`\{}
\def\PYZcb{\char`\}}
\def\PYZca{\char`\^}
\def\PYZam{\char`\&}
\def\PYZlt{\char`\<}
\def\PYZgt{\char`\>}
\def\PYZsh{\char`\#}
\def\PYZpc{\char`\%}
\def\PYZdl{\char`\$}
\def\PYZhy{\char`\-}
\def\PYZsq{\char`\'}
\def\PYZdq{\char`\"}
\def\PYZti{\char`\~}
% for compatibility with earlier versions
\def\PYZat{@}
\def\PYZlb{[}
\def\PYZrb{]}
\makeatother


    % Exact colors from NB
    \definecolor{incolor}{rgb}{0.0, 0.0, 0.5}
    \definecolor{outcolor}{rgb}{0.545, 0.0, 0.0}



    
    % Prevent overflowing lines due to hard-to-break entities
    \sloppy 
    % Setup hyperref package
    \hypersetup{
      breaklinks=true,  % so long urls are correctly broken across lines
      colorlinks=true,
      urlcolor=blue,
      linkcolor=darkorange,
      citecolor=darkgreen,
      }
    % Slightly bigger margins than the latex defaults
    
    \geometry{verbose,tmargin=1in,bmargin=1in,lmargin=1in,rmargin=1in}
    
    

    \begin{document}
    
    
    \maketitle
    
    

    
    Now that you have installed ROOT, what's this interactive shell thing
you're running ? It's like this: ROOT leads a double life. It has an
interpreter for macros \href{https://root.cern.ch/cling}{Cling} that you
can run from the command line or like other applications. But it is also
an interactive shell that can evaluate arbitrary statements and
expressions. This is extremely useful for debugging, quick hacking and
testing. Let us first have a look at some very simple examples.

\subsection{2.1 ROOT as calculator}\label{root-as-calculator}

You can even use the ROOT interactive shell instead of a calculator!
Launch the ROOT interactive shell with the command:
root
    on your Linux box. The prompt should appear shortly.

    \begin{Verbatim}[commandchars=\\\{\}]
{\color{incolor}In [{\color{incolor}1}]:} \PY{l+m+mi}{1}\PY{o}{+}\PY{l+m+mi}{1}
\end{Verbatim}

    \begin{Verbatim}[commandchars=\\\{\}]
(int) 2
    \end{Verbatim}

    \begin{Verbatim}[commandchars=\\\{\}]
{\color{incolor}In [{\color{incolor}2}]:} \PY{l+m+mi}{2}\PY{o}{*}\PY{p}{(}\PY{l+m+mi}{4}\PY{o}{+}\PY{l+m+mi}{2}\PY{p}{)}\PY{o}{/}\PY{l+m+mf}{12.}
\end{Verbatim}

    \begin{Verbatim}[commandchars=\\\{\}]
(double) 1.00000
    \end{Verbatim}

    \begin{Verbatim}[commandchars=\\\{\}]
{\color{incolor}In [{\color{incolor}3}]:} \PY{n}{sqrt}\PY{p}{(}\PY{l+m+mf}{3.}\PY{p}{)}
\end{Verbatim}

    \begin{Verbatim}[commandchars=\\\{\}]
(double) 1.73205
    \end{Verbatim}

    \begin{Verbatim}[commandchars=\\\{\}]
{\color{incolor}In [{\color{incolor}4}]:} \PY{l+m+mi}{1}\PY{o}{\PYZgt{}}\PY{l+m+mi}{2}
\end{Verbatim}

    \begin{Verbatim}[commandchars=\\\{\}]
(bool) false
    \end{Verbatim}

    \begin{Verbatim}[commandchars=\\\{\}]
{\color{incolor}In [{\color{incolor}5}]:} \PY{n}{TMath}\PY{o}{:}\PY{o}{:}\PY{n}{Pi}\PY{p}{(}\PY{p}{)}
\end{Verbatim}

    \begin{Verbatim}[commandchars=\\\{\}]
(Double\_t) 3.14159
    \end{Verbatim}

    \begin{Verbatim}[commandchars=\\\{\}]
{\color{incolor}In [{\color{incolor}6}]:} \PY{n}{TMath}\PY{o}{:}\PY{o}{:}\PY{n}{Erf}\PY{p}{(}\PY{l+m+mf}{.2}\PY{p}{)}
\end{Verbatim}

    \begin{Verbatim}[commandchars=\\\{\}]
(Double\_t) 0.222703
    \end{Verbatim}

    Not bad. You can see that ROOT offers you the possibility not only to
type in C++ statements, but also advanced mathematical functions, which
live in the \href{https://root.cern.ch/root/html524/TMath.html}{TMath
namespace}.

Now let's do something more elaborated. A numerical example with the
well known geometrical series:

    \begin{Verbatim}[commandchars=\\\{\}]
{\color{incolor}In [{\color{incolor}7}]:} \PY{k+kt}{double} \PY{n}{x}\PY{o}{=}\PY{l+m+mf}{.5}
\end{Verbatim}

    \begin{Verbatim}[commandchars=\\\{\}]
(double) 0.500000
    \end{Verbatim}

    \begin{Verbatim}[commandchars=\\\{\}]
{\color{incolor}In [{\color{incolor}8}]:} \PY{k+kt}{int} \PY{n}{N}\PY{o}{=}\PY{l+m+mi}{30}
\end{Verbatim}

    \begin{Verbatim}[commandchars=\\\{\}]
(int) 30
    \end{Verbatim}

    \begin{Verbatim}[commandchars=\\\{\}]
{\color{incolor}In [{\color{incolor}9}]:} \PY{k+kt}{double} \PY{n}{geom\PYZus{}series}\PY{o}{=}\PY{l+m+mi}{0}
\end{Verbatim}

    \begin{Verbatim}[commandchars=\\\{\}]
(double) 0.00000
    \end{Verbatim}

    \begin{Verbatim}[commandchars=\\\{\}]
{\color{incolor}In [{\color{incolor}10}]:} \PY{k}{for} \PY{p}{(}\PY{k+kt}{int} \PY{n}{i}\PY{o}{=}\PY{l+m+mi}{0}\PY{p}{;}\PY{n}{i}\PY{o}{\PYZlt{}}\PY{n}{N}\PY{p}{;}\PY{o}{+}\PY{o}{+}\PY{n}{i}\PY{p}{)}\PY{n}{geom\PYZus{}series}\PY{o}{+}\PY{o}{=}\PY{n}{TMath}\PY{o}{:}\PY{o}{:}\PY{n}{Power}\PY{p}{(}\PY{n}{x}\PY{p}{,}\PY{n}{i}\PY{p}{)}
\end{Verbatim}

    \begin{Verbatim}[commandchars=\\\{\}]
{\color{incolor}In [{\color{incolor}11}]:} \PY{n}{TMath}\PY{o}{:}\PY{o}{:}\PY{n}{Abs}\PY{p}{(}\PY{n}{geom\PYZus{}series} \PY{o}{\PYZhy{}} \PY{p}{(}\PY{l+m+mi}{1}\PY{o}{\PYZhy{}}\PY{n}{TMath}\PY{o}{:}\PY{o}{:}\PY{n}{Power}\PY{p}{(}\PY{n}{x}\PY{p}{,}\PY{n}{N}\PY{o}{\PYZhy{}}\PY{l+m+mi}{1}\PY{p}{)}\PY{p}{)}\PY{o}{/}\PY{p}{(}\PY{l+m+mi}{1}\PY{o}{\PYZhy{}}\PY{n}{x}\PY{p}{)}\PY{p}{)}
\end{Verbatim}

    \begin{Verbatim}[commandchars=\\\{\}]
(Double\_t) 1.86265e-09
    \end{Verbatim}

    Here we made a step forward. We even declared variables and used a for
control structure. Note that there are some subtle differences between
Cling and the standard C++ language. You do not need the ``;'' at the
end of line in interactive mode -- try the difference e.g.~declare a
different double like in the command above.

\subsection{2.2 Learn C++ at the ROOT
prompt}\label{learn-c-at-the-root-prompt}

Behind the ROOT prompt there is an interpreter based on a real compiler
toolkit: \href{http://llvm.org}{LLVM}. It is therefore possible to
exercise many features of C++ and the standard library. For example in
the following snippet we define a lambda function, a vector and we sort
it in different ways:

    \begin{Verbatim}[commandchars=\\\{\}]
{\color{incolor}In [{\color{incolor}12}]:} \PY{k}{using} \PY{n}{doubles} \PY{o}{=} \PY{n}{std}\PY{o}{:}\PY{o}{:}\PY{n}{vector}\PY{o}{\PYZlt{}}\PY{k+kt}{double}\PY{o}{\PYZgt{}}\PY{p}{;}
         \PY{k}{auto} \PY{n}{pVec} \PY{o}{=} \PY{p}{[}\PY{p}{]}\PY{p}{(}\PY{k}{const} \PY{n}{doubles}\PY{o}{\PYZam{}} \PY{n}{v}\PY{p}{)}\PY{p}{\PYZob{}}\PY{k}{for} \PY{p}{(}\PY{k}{auto}\PY{o}{\PYZam{}}\PY{o}{\PYZam{}} \PY{n+nl}{x}\PY{p}{:}\PY{n}{v}\PY{p}{)} \PY{n}{cout} \PY{o}{\PYZlt{}}\PY{o}{\PYZlt{}} \PY{n}{x} \PY{o}{\PYZlt{}}\PY{o}{\PYZlt{}} \PY{n}{endl}\PY{p}{;}\PY{p}{\PYZcb{}}\PY{p}{;}
         \PY{n}{doubles} \PY{n}{v}\PY{p}{\PYZob{}}\PY{l+m+mi}{0}\PY{p}{,}\PY{l+m+mi}{3}\PY{p}{,}\PY{l+m+mi}{5}\PY{p}{,}\PY{l+m+mi}{4}\PY{p}{,}\PY{l+m+mi}{1}\PY{p}{,}\PY{l+m+mi}{2}\PY{p}{\PYZcb{}}\PY{p}{;}
         \PY{n}{pVec}\PY{p}{(}\PY{n}{v}\PY{p}{)}\PY{p}{;}
\end{Verbatim}

    \begin{Verbatim}[commandchars=\\\{\}]
0
3
5
4
1
2
    \end{Verbatim}

    \begin{Verbatim}[commandchars=\\\{\}]
{\color{incolor}In [{\color{incolor}13}]:} \PY{n}{std}\PY{o}{:}\PY{o}{:}\PY{n}{sort}\PY{p}{(}\PY{n}{v}\PY{p}{.}\PY{n}{begin}\PY{p}{(}\PY{p}{)}\PY{p}{,}\PY{n}{v}\PY{p}{.}\PY{n}{end}\PY{p}{(}\PY{p}{)}\PY{p}{)}\PY{p}{;}
         \PY{n}{pVec}\PY{p}{(}\PY{n}{v}\PY{p}{)}\PY{p}{;}
\end{Verbatim}

    \begin{Verbatim}[commandchars=\\\{\}]
0
1
2
3
4
5
    \end{Verbatim}

    Or, if you prefer random number generation:

    \begin{Verbatim}[commandchars=\\\{\}]
{\color{incolor}In [{\color{incolor}14}]:} \PY{n}{std}\PY{o}{:}\PY{o}{:}\PY{n}{default\PYZus{}random\PYZus{}engine} \PY{n}{generator}\PY{p}{;}
         \PY{n}{std}\PY{o}{:}\PY{o}{:}\PY{n}{normal\PYZus{}distribution}\PY{o}{\PYZlt{}}\PY{k+kt}{double}\PY{o}{\PYZgt{}} \PY{n}{distribution}\PY{p}{(}\PY{l+m+mf}{0.}\PY{p}{,}\PY{l+m+mf}{1.}\PY{p}{)}\PY{p}{;}
         \PY{n}{distribution}\PY{p}{(}\PY{n}{generator}\PY{p}{)}\PY{p}{;}
         \PY{n}{std}\PY{o}{:}\PY{o}{:}\PY{n}{cout} \PY{o}{\PYZlt{}}\PY{o}{\PYZlt{}} \PY{n}{distribution}\PY{p}{(}\PY{n}{generator}\PY{p}{)}\PY{p}{;}
\end{Verbatim}

    \begin{Verbatim}[commandchars=\\\{\}]
-0.407472
    \end{Verbatim}

    \begin{Verbatim}[commandchars=\\\{\}]
{\color{incolor}In [{\color{incolor}15}]:} \PY{n}{distribution}\PY{p}{(}\PY{n}{generator}\PY{p}{)}\PY{p}{;}
         \PY{n}{std}\PY{o}{:}\PY{o}{:}\PY{n}{cout} \PY{o}{\PYZlt{}}\PY{o}{\PYZlt{}} \PY{n}{distribution}\PY{p}{(}\PY{n}{generator}\PY{p}{)}\PY{p}{;}
\end{Verbatim}

    \begin{Verbatim}[commandchars=\\\{\}]
0.399771
    \end{Verbatim}

    \begin{Verbatim}[commandchars=\\\{\}]
{\color{incolor}In [{\color{incolor}16}]:} \PY{n}{distribution}\PY{p}{(}\PY{n}{generator}\PY{p}{)}\PY{p}{;}
         \PY{n}{std}\PY{o}{:}\PY{o}{:}\PY{n}{cout} \PY{o}{\PYZlt{}}\PY{o}{\PYZlt{}} \PY{n}{distribution}\PY{p}{(}\PY{n}{generator}\PY{p}{)}\PY{p}{;}
\end{Verbatim}

    \begin{Verbatim}[commandchars=\\\{\}]
0.0523187
    \end{Verbatim}

    \subsection{2.3 ROOT as function
plotter}\label{root-as-function-plotter}

Using one of ROOT's powerful classes, here
\href{https://root.cern.ch/doc/master/classTF1.html}{TF1} will allow us
to display a function of one variable, x. Try the following:

    \begin{Verbatim}[commandchars=\\\{\}]
{\color{incolor}In [{\color{incolor}17}]:} \PY{n}{TCanvas} \PY{n}{canvas\PYZus{}2}\PY{p}{;}
         \PY{n}{TF1} \PY{n+nf}{f1}\PY{p}{(}\PY{l+s}{\PYZdq{}}\PY{l+s}{f1}\PY{l+s}{\PYZdq{}}\PY{p}{,}\PY{l+s}{\PYZdq{}}\PY{l+s}{sin(x)}\PY{o}{/}\PY{n}{x}\PY{l+s}{\PYZdq{}}\PY{l+s}{,0.,10.)}\PY{p}{;}
\end{Verbatim}

    \texttt{f1} is an instance of a \texttt{TF1} class, the arguments are
used in the constructor; the first one of type string is a name to be
entered in the internal ROOT memory management system, the second string
type parameter defines the function, here sin(x)/x, and the two
parameters of type double define the range of the variable x. The Draw()
method, here without any parameters, displays the function in a window
which should pop up after you typed the above two lines.

    \begin{Verbatim}[commandchars=\\\{\}]
{\color{incolor}In [{\color{incolor}18}]:} \PY{n}{f1}\PY{p}{.}\PY{n}{Draw}\PY{p}{(}\PY{p}{)}\PY{p}{;}
         \PY{n}{canvas\PYZus{}2}\PY{p}{.}\PY{n}{Draw}\PY{p}{(}\PY{p}{)}\PY{p}{;}
\end{Verbatim}

    \begin{center}
    \adjustimage{max size={0.9\linewidth}{0.9\paperheight}}{2-ROOT-Basics_files/2-ROOT-Basics_25_0.png}
    \end{center}
    { \hspace*{\fill} \\}
    
    A slightly extended version of this example is the definition of a
function with parameters, called {[}0{]}, {[}1{]} and so on in the ROOT
formula syntax. We now need a way to assign values to these parameters;
this is achieved with the method
\href{https://root.cern.ch/doc/master/classTF1.html\#ade6e54171210c6b1b955c9f813040eb8}{SetParameter}(,)
of class TF1. Here is an example:

    \begin{Verbatim}[commandchars=\\\{\}]
{\color{incolor}In [{\color{incolor}19}]:} \PY{n}{TF1} \PY{n+nf}{f2}\PY{p}{(}\PY{l+s}{\PYZdq{}}\PY{l+s}{f2}\PY{l+s}{\PYZdq{}}\PY{p}{,}\PY{l+s}{\PYZdq{}}\PY{l+s}{[0]*sin([1]*x)}\PY{o}{/}\PY{n}{x}\PY{l+s}{\PYZdq{}}\PY{l+s}{,0.,10.)}\PY{p}{;}
\end{Verbatim}

    You can try to change the parameters of the input bellow and try the
results.

    \begin{Verbatim}[commandchars=\\\{\}]
{\color{incolor}In [{\color{incolor}20}]:} \PY{n}{f2}\PY{p}{.}\PY{n}{SetParameter}\PY{p}{(}\PY{l+m+mi}{0}\PY{p}{,}\PY{l+m+mi}{1}\PY{p}{)}\PY{p}{;}
         \PY{n}{f2}\PY{p}{.}\PY{n}{SetParameter}\PY{p}{(}\PY{l+m+mi}{1}\PY{p}{,}\PY{l+m+mi}{1}\PY{p}{)}\PY{p}{;}
         \PY{n}{f2}\PY{p}{.}\PY{n}{Draw}\PY{p}{(}\PY{p}{)}\PY{p}{;}
         \PY{n}{canvas\PYZus{}2}\PY{p}{.}\PY{n}{Draw}\PY{p}{(}\PY{p}{)}\PY{p}{;}
\end{Verbatim}

    \begin{center}
    \adjustimage{max size={0.9\linewidth}{0.9\paperheight}}{2-ROOT-Basics_files/2-ROOT-Basics_29_0.png}
    \end{center}
    { \hspace*{\fill} \\}
    
    Of course, this version shows the same results as the initial one. Try
playing with the parameters and plot the function again. The class TF1
has a large number of very useful methods, including integration and
differentiation. To make full use of this and other ROOT classes, visit
the documentation on the Internet under
http://root.cern.ch/drupal/content/reference-guide. Formulae in ROOT are
evaluated using the class
\href{https://root.cern.ch/doc/master/classTFormula.html}{TFormula}, so
also look up the relevant class documentation for examples, implemented
functions and syntax.

You should definitely download this guide to your own system to have it
at you disposal whenever you need it.

To extend a little bit on the above example, consider a more complex
function you would like to define. You can also do this using standard C
or C++ code.

Consider the example below, which calculates and displays the
interference pattern produced by light falling on a multiple slit.
Please do not type in the example below at the ROOT command line, there
is a much simpler way: Make sure you have the file slits.C on disk, and
type root slits.C in the shell. This will start root and make it read
the ``macro'' slits.C, i.e.~all the lines in the file will be executed
one after the other.

In this example drawing the interference pattern of light falling on a
grid with n slits and ratio r of slit width over distance between slits.

    \begin{Verbatim}[commandchars=\\\{\}]
{\color{incolor}In [{\color{incolor}21}]:} \PY{o}{\PYZpc{}}\PY{o}{\PYZpc{}}\PY{n}{cpp} \PY{o}{\PYZhy{}}\PY{n}{d}
\end{Verbatim}

    As always in the notebook envirement we need to
{[}\ldots{}\ldots{}\ldots{}\ldots{}\ldots{}\ldots{}\ldots{}.{]}.
Something you will not need to do in your machine.

    \begin{Verbatim}[commandchars=\\\{\}]
{\color{incolor}In [{\color{incolor}22}]:} \PY{k}{auto} \PY{n}{pi} \PY{o}{=} \PY{n}{TMath}\PY{o}{:}\PY{o}{:}\PY{n}{Pi}\PY{p}{(}\PY{p}{)}\PY{p}{;}
\end{Verbatim}

    Bellow you can see the function code.

We define the necessary functions in C++ code, split into three separate
functions, as suggested by the problem considered. The full interference
pattern is given by the product of a function depending on the ratio of
the width and distance of the slits, and a second one depending on the
number of slits. More important for us here is the definition of the
interface of these functions to make them usable for the ROOT class TF1:
the first argument is the pointer to x, the second one points to the
array of parameters.

    \begin{Verbatim}[commandchars=\\\{\}]
{\color{incolor}In [{\color{incolor}23}]:} \PY{o}{\PYZpc{}}\PY{o}{\PYZpc{}}\PY{n}{cpp} \PY{o}{\PYZhy{}}\PY{n}{d}
         \PY{k+kt}{double} \PY{n}{single}\PY{p}{(}\PY{k+kt}{double} \PY{o}{*}\PY{n}{x}\PY{p}{,} \PY{k+kt}{double} \PY{o}{*}\PY{n}{par}\PY{p}{)} \PY{p}{\PYZob{}}
           \PY{k}{return} \PY{n}{pow}\PY{p}{(}\PY{n}{sin}\PY{p}{(}\PY{n}{pi}\PY{o}{*}\PY{n}{par}\PY{p}{[}\PY{l+m+mi}{0}\PY{p}{]}\PY{o}{*}\PY{n}{x}\PY{p}{[}\PY{l+m+mi}{0}\PY{p}{]}\PY{p}{)}\PY{o}{/}\PY{p}{(}\PY{n}{pi}\PY{o}{*}\PY{n}{par}\PY{p}{[}\PY{l+m+mi}{0}\PY{p}{]}\PY{o}{*}\PY{n}{x}\PY{p}{[}\PY{l+m+mi}{0}\PY{p}{]}\PY{p}{)}\PY{p}{,}\PY{l+m+mi}{2}\PY{p}{)}\PY{p}{;}
         \PY{p}{\PYZcb{}}
         
         \PY{k+kt}{double} \PY{n}{nslit0}\PY{p}{(}\PY{k+kt}{double} \PY{o}{*}\PY{n}{x}\PY{p}{,}\PY{k+kt}{double} \PY{o}{*}\PY{n}{par}\PY{p}{)}\PY{p}{\PYZob{}}
           \PY{k}{return} \PY{n}{pow}\PY{p}{(}\PY{n}{sin}\PY{p}{(}\PY{n}{pi}\PY{o}{*}\PY{n}{par}\PY{p}{[}\PY{l+m+mi}{1}\PY{p}{]}\PY{o}{*}\PY{n}{x}\PY{p}{[}\PY{l+m+mi}{0}\PY{p}{]}\PY{p}{)}\PY{o}{/}\PY{n}{sin}\PY{p}{(}\PY{n}{pi}\PY{o}{*}\PY{n}{x}\PY{p}{[}\PY{l+m+mi}{0}\PY{p}{]}\PY{p}{)}\PY{p}{,}\PY{l+m+mi}{2}\PY{p}{)}\PY{p}{;}
         \PY{p}{\PYZcb{}}
         
         \PY{k+kt}{double} \PY{n}{nslit}\PY{p}{(}\PY{k+kt}{double} \PY{o}{*}\PY{n}{x}\PY{p}{,} \PY{k+kt}{double} \PY{o}{*}\PY{n}{par}\PY{p}{)}\PY{p}{\PYZob{}}
           \PY{k}{return} \PY{n}{single}\PY{p}{(}\PY{n}{x}\PY{p}{,}\PY{n}{par}\PY{p}{)} \PY{o}{*} \PY{n}{nslit0}\PY{p}{(}\PY{n}{x}\PY{p}{,}\PY{n}{par}\PY{p}{)}\PY{p}{;}
         \PY{p}{\PYZcb{}}
\end{Verbatim}

    Here is how the main program should look like.

It starts with the definition of a function slits() of type void. After
asking for user input, a ROOT function is defined using the C-type
function given in the beginning. We can now use all methods of the TF1
class to control the behaviour of our function -- nice, isn't it ?

    \begin{Verbatim}[commandchars=\\\{\}]
{\color{incolor}In [{\color{incolor}24}]:} \PY{o}{\PYZpc{}}\PY{o}{\PYZpc{}}\PY{n}{cpp} \PY{o}{\PYZhy{}}\PY{n}{d}
           \PY{k+kt}{void} \PY{n}{slits}\PY{p}{(}\PY{p}{)} \PY{p}{\PYZob{}}
           \PY{k+kt}{float} \PY{n}{r}\PY{p}{,}\PY{n}{ns}\PY{p}{;}
         
         
           \PY{n}{r} \PY{o}{=} \PY{l+m+mi}{1}\PY{p}{;}
           \PY{n}{ns}\PY{o}{=}\PY{l+m+mf}{0.45}\PY{p}{;}
           
           \PY{c+cm}{/* // request user input}
         \PY{c+cm}{  cout \PYZlt{}\PYZlt{} \PYZdq{}slit width / g ? \PYZdq{};}
         \PY{c+cm}{  scanf(\PYZdq{}\PYZpc{}f\PYZdq{},\PYZam{}r);}
         \PY{c+cm}{  cout \PYZlt{}\PYZlt{} \PYZdq{}\PYZsh{} of slits? \PYZdq{};}
         \PY{c+cm}{  scanf(\PYZdq{}\PYZpc{}f\PYZdq{},\PYZam{}ns);}
         \PY{c+cm}{  cout \PYZlt{}\PYZlt{}\PYZdq{}interference pattern for \PYZdq{}\PYZlt{}\PYZlt{} ns}
         \PY{c+cm}{       \PYZlt{}\PYZlt{}\PYZdq{} slits, width/distance: \PYZdq{}\PYZlt{}\PYZlt{}r\PYZlt{}\PYZlt{}endl;}
         \PY{c+cm}{  */}
         
           \PY{c+c1}{// define function and set options}
           \PY{n}{TF1} \PY{o}{*}\PY{n}{Fnslit}  \PY{o}{=} \PY{k}{new} \PY{n}{TF1}\PY{p}{(}\PY{l+s}{\PYZdq{}}\PY{l+s}{Fnslit}\PY{l+s}{\PYZdq{}}\PY{p}{,}\PY{n}{nslit}\PY{p}{,}\PY{o}{\PYZhy{}}\PY{l+m+mf}{5.001}\PY{p}{,}\PY{l+m+mf}{5.}\PY{p}{,}\PY{l+m+mi}{2}\PY{p}{)}\PY{p}{;}
           \PY{n}{Fnslit}\PY{o}{\PYZhy{}}\PY{o}{\PYZgt{}}\PY{n}{SetNpx}\PY{p}{(}\PY{l+m+mi}{500}\PY{p}{)}\PY{p}{;}
         
           \PY{c+c1}{// set parameters, as read in above}
           \PY{n}{Fnslit}\PY{o}{\PYZhy{}}\PY{o}{\PYZgt{}}\PY{n}{SetParameter}\PY{p}{(}\PY{l+m+mi}{0}\PY{p}{,}\PY{n}{r}\PY{p}{)}\PY{p}{;}
           \PY{n}{Fnslit}\PY{o}{\PYZhy{}}\PY{o}{\PYZgt{}}\PY{n}{SetParameter}\PY{p}{(}\PY{l+m+mi}{1}\PY{p}{,}\PY{n}{ns}\PY{p}{)}\PY{p}{;}
         
           \PY{c+c1}{// draw the interference pattern for a grid with n slits}
           \PY{n}{Fnslit}\PY{o}{\PYZhy{}}\PY{o}{\PYZgt{}}\PY{n}{Draw}\PY{p}{(}\PY{p}{)}\PY{p}{;}
         \PY{p}{\PYZcb{}}
\end{Verbatim}

    \begin{Verbatim}[commandchars=\\\{\}]
{\color{incolor}In [{\color{incolor}25}]:} \PY{n}{slits}\PY{p}{(}\PY{p}{)}\PY{p}{;}
         \PY{n}{canvas\PYZus{}2}\PY{p}{.}\PY{n}{Draw}\PY{p}{(}\PY{p}{)}\PY{p}{;}
\end{Verbatim}

    \begin{center}
    \adjustimage{max size={0.9\linewidth}{0.9\paperheight}}{2-ROOT-Basics_files/2-ROOT-Basics_38_0.png}
    \end{center}
    { \hspace*{\fill} \\}
    
    Output of slits.C with parameters 0.2 and 2.

In the commented out section the example asks for user input, namely the
ratio of slit width over slit distance, and the number of slits. After
entering this information, you should see the graphical output as above.

This is a more complicated example than the ones we have seen before, so
spend some time analysing it carefully, you should have understood it
before continuing.

If you like, you can easily extend the example to also plot the
interference pattern of a single slit, using function double single, or
of a grid with narrow slits, function double nslit0, in TF1 instances.

Here, we used a macro, some sort of lightweight program, that the
interpreter distributed with ROOT, Cling, is able to execute. This is a
rather extraordinary situation, since C++ is not natively an interpreted
language! There is much more to say: chapter 3 is indeed dedicated to
macros.

\subsection{2.4 Controlling ROOT}\label{controlling-root}

One more remark at this point: as every command you type into ROOT is
usually interpreted by Cling, an ``escape character'' is needed to pass
commands to ROOT directly. This character is the dot at the beginning of
a line:

\begin{verbatim}
root [1] .<command>
\end{verbatim}

This is a selection of the most common commands. * \textbf{quit root},
simply type \texttt{.q}

\begin{itemize}
\item
  obtain a \textbf{list of commands}, use \texttt{.?}
\item
  \textbf{access the shell} of the operating system, type
  \texttt{.!\textless{}OS\_command\textgreater{}}; try, e.g.
  \texttt{.!ls} or \texttt{.!pwd}
\item
  \textbf{execute a macro}, enter
  \texttt{.x\ \textless{}file\_name\textgreater{}}; in the above
  example, you might have used \texttt{.x\ slits.C} at the ROOT prompt
\item
  \textbf{load a macro}, type
  \texttt{.L\ \textless{}file\_name\textgreater{}}; in the above
  example, you might instead have used the command \texttt{.L\ slits.C}
  followed by the function call \texttt{slits();}. Note that after
  loading a macro all functions and procedures defined therein are
  available at the ROOT prompt.

  \begin{itemize}
  \tightlist
  \item
    \textbf{compile a macro}, type
    \texttt{.L\ \textless{}file\_name\textgreater{}+}; ROOT is able to
    manage for you the \texttt{C++} compiler behind the scenes and to
    produce machine code starting from your macro. One could decide to
    compile a macro in order to obtain better performance or to get
    nearer to the production environment.
  \end{itemize}
\end{itemize}

Use \texttt{.help} at the prompt to inspect the full list.

\subsection{2.5 Plotting Measurements}\label{plotting-measurements}

To display measurements in ROOT, including errors, there exists a
powerful class
\href{https://root.cern.ch/doc/master/classTGraphErrors.html}{\texttt{TGraphErrors}}
with different types of constructors. In the example here, we use data
from the file \texttt{ExampleData.txt} in text format:

    \begin{Verbatim}[commandchars=\\\{\}]
{\color{incolor}In [{\color{incolor}26}]:} \PY{n}{TCanvas} \PY{n}{canvas\PYZus{}2\PYZus{}5}\PY{p}{;}
         \PY{n}{TGraphErrors} \PY{n+nf}{gr}\PY{p}{(}\PY{l+s}{\PYZdq{}}\PY{l+s}{../data/ExampleData.txt}\PY{l+s}{\PYZdq{}}\PY{p}{)}\PY{p}{;}
         \PY{n}{gr}\PY{p}{.}\PY{n}{Draw}\PY{p}{(}\PY{l+s}{\PYZdq{}}\PY{l+s}{AP}\PY{l+s}{\PYZdq{}}\PY{p}{)}\PY{p}{;}
         \PY{n}{canvas\PYZus{}2\PYZus{}5}\PY{p}{.}\PY{n}{Draw}\PY{p}{(}\PY{p}{)}\PY{p}{;}
\end{Verbatim}

    \begin{center}
    \adjustimage{max size={0.9\linewidth}{0.9\paperheight}}{2-ROOT-Basics_files/2-ROOT-Basics_40_0.png}
    \end{center}
    { \hspace*{\fill} \\}
    
    Make sure the file \texttt{ExampleData.txt} is available in the
directory from which you started ROOT. Inspect this file now with your
favourite editor, or use the command \texttt{less\ ExampleData.txt} to
inspect the file, you will see something like that:

\begin{verbatim}
# fake data to demonstrate the use of TGraphErrors

# x    y    ex    ey
  1.   0.4  0.1   0.05
  1.3  0.3  0.05  0.1
  1.7  0.5  0.15  0.1
  1.9  0.7  0.05  0.1
  2.3  1.3  0.07  0.1
  2.9  1.5  0.2   0.1
\end{verbatim}

The format is very simple and easy to understand. Lines beginning with
\texttt{\#} are ignored. It is very convenient to add some comments
about the type of data. The data itself consist of lines with four real
numbers each, representing the x- and y- coordinates and their errors of
each data point.

The argument of the method
\href{https://root.cern.ch/doc/v606/classTObject.html\#adaa7be22dce34ebb73fbf22e4bdf33a2}{\texttt{Draw("AP")}}
is important here. Behind the scenes, it tells the
\href{https://root.cern.ch/doc/v606/classTHistPainter.html}{TGraphPainter}
class to show the axes and to plot markers at the x and y positions of
the specified data points. Note that this simple example relies on the
default settings of ROOT, concerning the size of the canvas holding the
plot, the marker type and the line colours and thickness used and so on.
In a well-written, complete example, all this would need to be specified
explicitly in order to obtain nice and well readable results. A full
chapter on graphs will explain many more of the features of the class
\texttt{TGraphErrors} and its relation to other ROOT classes in much
more detail.

\subsection{2.6 Histograms in ROOT}\label{histograms-in-root}

Frequency distributions in ROOT are handled by a set of classes derived
from the histogram class
\href{https://root.cern.ch/doc/master/classTH1.html}{\texttt{TH1}}, in
our case
\href{https://root.cern.ch/doc/master/classTH1F.html}{\texttt{TH1F}}.
The letter F stands for float, meaning that the data type \texttt{float}
is used to store the entries in one histogram bin.

    \begin{Verbatim}[commandchars=\\\{\}]
{\color{incolor}In [{\color{incolor}27}]:} \PY{n}{TCanvas} \PY{n}{canvas\PYZus{}2\PYZus{}6}\PY{p}{;}
         \PY{n}{TF1} \PY{n+nf}{efunc}\PY{p}{(}\PY{l+s}{\PYZdq{}}\PY{l+s}{efunc}\PY{l+s}{\PYZdq{}}\PY{p}{,}\PY{l+s}{\PYZdq{}}\PY{l+s}{exp([0]+[1]*x)}\PY{l+s}{\PYZdq{}}\PY{l+s}{,0.,5.)}\PY{p}{;}
         \PY{n}{efunc}\PY{p}{.}\PY{n}{SetParameter}\PY{p}{(}\PY{l+m+mi}{0}\PY{p}{,}\PY{l+m+mi}{1}\PY{p}{)}\PY{p}{;}
         \PY{n}{efunc}\PY{p}{.}\PY{n}{SetParameter}\PY{p}{(}\PY{l+m+mi}{1}\PY{p}{,}\PY{o}{\PYZhy{}}\PY{l+m+mi}{1}\PY{p}{)}\PY{p}{;}
\end{Verbatim}

    The first lines of this example define a function, an exponential in
this case, and set its parameters.

    \begin{Verbatim}[commandchars=\\\{\}]
{\color{incolor}In [{\color{incolor}28}]:} \PY{n}{TH1F} \PY{n+nf}{hist\PYZus{}2\PYZus{}6\PYZus{}1}\PY{p}{(}\PY{l+s}{\PYZdq{}}\PY{l+s}{histogram 2.6.1}\PY{l+s}{\PYZdq{}}\PY{p}{,}\PY{l+s}{\PYZdq{}}\PY{l+s}{example histogram}\PY{l+s}{\PYZdq{}}\PY{p}{,}\PY{l+m+mi}{100}\PY{p}{,}\PY{l+m+mf}{0.}\PY{p}{,}\PY{l+m+mf}{5.}\PY{p}{)}\PY{p}{;}
\end{Verbatim}

    In this line a histogram is instantiated, with a name, a title, a
certain number of bins (100 of them, equidistant, equally sized) in the
range from 0 to 5.

We use yet another new feature of ROOT to fill this histogram with data,
namely pseudo-random numbers generated with the method
\href{https://root.cern.ch/doc/master/classTF1.html\#ab44c5f63db88a3831d74c7c84dc6316b}{\texttt{TF1::GetRandom}},
which in turn uses an instance of the ROOT class
\href{https://root.cern.ch/doc/master/classTRandom.html}{\texttt{TRandom}}
created when ROOT is started.

    \begin{Verbatim}[commandchars=\\\{\}]
{\color{incolor}In [{\color{incolor}29}]:} \PY{k}{for} \PY{p}{(}\PY{k+kt}{int} \PY{n}{i}\PY{o}{=}\PY{l+m+mi}{0}\PY{p}{;}\PY{n}{i}\PY{o}{\PYZlt{}}\PY{l+m+mi}{1000}\PY{p}{;}\PY{n}{i}\PY{o}{+}\PY{o}{+}\PY{p}{)} \PY{p}{\PYZob{}}\PY{n}{hist\PYZus{}2\PYZus{}6\PYZus{}1}\PY{p}{.}\PY{n}{Fill}\PY{p}{(}\PY{n}{efunc}\PY{p}{.}\PY{n}{GetRandom}\PY{p}{(}\PY{p}{)}\PY{p}{)}\PY{p}{;}\PY{p}{\PYZcb{}}
\end{Verbatim}

    Data is entered in the histogram using the method
\href{https://root.cern.ch/doc/master/classTH1.html\#a77e71290a82517d317ea8d05e96b6c4a}{\texttt{TH1F::Fill}}
in a loop construct. As a result, the histogram is filled with 1000
random numbers distributed according to the defined function.

    \begin{Verbatim}[commandchars=\\\{\}]
{\color{incolor}In [{\color{incolor}30}]:} \PY{n}{hist\PYZus{}2\PYZus{}6\PYZus{}1}\PY{p}{.}\PY{n}{Draw}\PY{p}{(}\PY{p}{)}\PY{p}{;}
         \PY{n}{canvas\PYZus{}2\PYZus{}6}\PY{p}{.}\PY{n}{Draw}\PY{p}{(}\PY{p}{)}\PY{p}{;}
\end{Verbatim}

    \begin{center}
    \adjustimage{max size={0.9\linewidth}{0.9\paperheight}}{2-ROOT-Basics_files/2-ROOT-Basics_48_0.png}
    \end{center}
    { \hspace*{\fill} \\}
    
    The histogram is displayed using the method \texttt{TH1F::Draw()}. You
may think of this example as repeated measurements of the life time of a
quantum mechanical state, which are entered into the histogram, thus
giving a visual impression of the probability density distribution. The
plot is shown above.

Note that you will not obtain an identical plot when executing the lines
above, depending on how the random number generator is initialised.

The class \texttt{TH1F} does not contain a convenient input format from
plain text files. The following lines of \texttt{C++} code do the job.
One number per line stored in the text file ``expo.dat'' is read in via
an input stream and filled in the histogram until end of file is
reached.

    \begin{Verbatim}[commandchars=\\\{\}]
{\color{incolor}In [{\color{incolor}31}]:} \PY{n}{TH1F} \PY{n+nf}{hist\PYZus{}2\PYZus{}6\PYZus{}2}\PY{p}{(}\PY{l+s}{\PYZdq{}}\PY{l+s}{histogram 2.6.2}\PY{l+s}{\PYZdq{}}\PY{p}{,}\PY{l+s}{\PYZdq{}}\PY{l+s}{example histogram}\PY{l+s}{\PYZdq{}}\PY{p}{,}\PY{l+m+mi}{100}\PY{p}{,}\PY{l+m+mf}{0.}\PY{p}{,}\PY{l+m+mf}{5.}\PY{p}{)}\PY{p}{;}
         \PY{n}{ifstream} \PY{n}{inp}\PY{p}{;} 
         \PY{n}{inp}\PY{p}{.}\PY{n}{open}\PY{p}{(}\PY{l+s}{\PYZdq{}}\PY{l+s}{expo.dat}\PY{l+s}{\PYZdq{}}\PY{p}{)}\PY{p}{;}
         \PY{k}{while} \PY{p}{(}\PY{n}{inp} \PY{o}{\PYZgt{}}\PY{o}{\PYZgt{}} \PY{n}{x}\PY{p}{)} \PY{p}{\PYZob{}} \PY{n}{hist\PYZus{}2\PYZus{}6\PYZus{}2}\PY{p}{.}\PY{n}{Fill}\PY{p}{(}\PY{n}{x}\PY{p}{)}\PY{p}{;} \PY{p}{\PYZcb{}}
         \PY{n}{hist\PYZus{}2\PYZus{}6\PYZus{}2}\PY{p}{.}\PY{n}{Draw}\PY{p}{(}\PY{p}{)}\PY{p}{;}
         \PY{n}{inp}\PY{p}{.}\PY{n}{close}\PY{p}{(}\PY{p}{)}\PY{p}{;}
         \PY{n}{canvas\PYZus{}2\PYZus{}6}\PY{p}{.}\PY{n}{Draw}\PY{p}{(}\PY{p}{)}\PY{p}{;}
\end{Verbatim}

    \begin{center}
    \adjustimage{max size={0.9\linewidth}{0.9\paperheight}}{2-ROOT-Basics_files/2-ROOT-Basics_50_0.png}
    \end{center}
    { \hspace*{\fill} \\}
    
    \subsection{2.7 Interactive ROOT}\label{interactive-root}

Look at one of your plots again and move the mouse across. You will
notice that this is much more than a static picture, as the mouse
pointer changes its shape when touching objects on the plot. When the
mouse is over an object, a right-click opens a pull-down menu displaying
in the top line the name of the ROOT class you are dealing with, e.g.
\href{https://root.cern.ch/doc/master/classTCanvas.html}{\texttt{TCanvas}}
for the display window itself,
\href{https://root.cern.ch/doc/v606/classTFrame.html}{\texttt{TFrame}}
for the frame of the plot,
\href{https://root.cern.ch/doc/master/classTAxis.html}{\texttt{TAxis}}
for the axes,
\href{https://root.cern.ch/doc/v606/classTPaveText.html}{\texttt{TPaveText}}
for the plot name. Depending on which plot you are investigating, menus
for the ROOT classes \texttt{TF1}, \texttt{TGraphErrors} or
\texttt{TH1F} will show up when a right-click is performed on the
respective graphical representations. The menu items allow direct access
to the members of the various classes, and you can even modify them,
e.g.~change colour and size of the axis ticks or labels, the function
lines, marker types and so on. Try it!

You will probably like the following: in the output produced by the
example \texttt{slits.C}, right-click on the function line and select
``SetLineAttributes'', then left-click on ``Set Parameters''. This gives
access to a panel allowing you to interactively change the parameters of
the function, as shown in the figure above. Change the slit width, or go
from one to two and then three or more slits, just as you like. When
clicking on ``Apply'', the function plot is updated to reflect the
actual value of the parameters you have set.

Another very useful interactive tool is the
\href{https://root.cern.ch/fit-panel}{\texttt{FitPanel}}, available for
the classes \texttt{TGraphErrors} and \texttt{TH1F}. Predefined fit
functions can be selected from a pull-down menu, including
``\texttt{gaus}'', ``\texttt{expo}'' and ``\texttt{pol0}'' -
``\texttt{pol9}'' for Gaussian and exponential functions or polynomials
of degree 0 to 9, respectively. In addition, user-defined functions
using the same syntax as for functions with parameters are possible.

After setting the initial parameters, a fit of the selected function to
the data of a graph or histogram can be performed and the result
displayed on the plot. The fit panel has a number of control options to
select the fit method, fix or release individual parameters in the fit,
to steer the level of output printed on the console, or to extract and
display additional information like contour lines showing parameter
correlations. As function fitting is of prime importance in any kind of
data analysis, this topic will again show up later.

If you are satisfied with your plot, you probably want to save it. Just
close all selector boxes you opened previously and select the menu item
\texttt{Save\ as...} from the menu line of the window. It will pop up a
file selector box to allow you to choose the format, file name and
target directory to store the image. There is one very noticeable
feature here: you can store a plot as a root macro! In this macro, you
find the C++ representation of all methods and classes involved in
generating the plot. This is a valuable source of information for your
own macros, which you will hopefully write after having worked through
this tutorial.

Using ROOT's interactive capabilities is useful for a first exploration
of possibilities. Other ROOT classes you will encounter in this tutorial
have such graphical interfaces. We will not comment further on this,
just be aware of the existence of ROOT's interactive features and use
them if you find them convenient. Some trial-and-error is certainly
necessary to find your way through the huge number of menus and
parameter settings.

\subsection{2.8 ROOT Beginners' FAQ}\label{root-beginners-faq}

At this point of the guide, some basic questions could have already come
to your mind. We will try to clarify some of them with further
explanations in the following.

\subsubsection{2.8.1 ROOT type declarations for basic data
types}\label{root-type-declarations-for-basic-data-types}

In the official ROOT documentation, you find special data types
replacing the normal ones, e.g. \texttt{Double\_t}, \texttt{Float\_t} or
\texttt{Int\_t} replacing the standard \texttt{double}, \texttt{float}
or \texttt{int} types. Using the ROOT types makes it easier to port code
between platforms (64/32 bit) or operating systems (windows/Linux), as
these types are mapped to suitable ones in the ROOT header files. If you
want adaptive code of this type, use the ROOT type declarations.
However, usually you do not need such adaptive code, and you can safely
use the standard C type declarations for your private code, as we did
and will do throughout this guide. If you intend to become a ROOT
developer, however, you better stick to the official coding rules!

\subsubsection{2.8.2 Configure ROOT at
start-up}\label{configure-root-at-start-up}

The behaviour of a ROOT session can be tailored with the options in the
\texttt{.rootrc} file. Examples of the tunable parameters are the ones
related to the operating and window system, to the fonts to be used, to
the location of start-up files. At start-up, ROOT looks for a
\texttt{.rootrc} file in the following order:

\begin{itemize}
\tightlist
\item
  \texttt{./.rootrc\ //local\ directory}
\item
  \texttt{\$HOME/.rootrc\ //user\ directory}
\item
  \texttt{\$ROOTSYS/etc/system.rootrc\ //global\ ROOT\ directory}
\end{itemize}

If more than one \texttt{.rootrc} files are found in the search paths
above, the options are merged, with precedence local, user, global. The
parsing and interpretation of this file is handled by the ROOT class
TEnv. Have a look to its documentation if you need such rather advanced
features. The file \texttt{.rootrc} defines the location of two rather
important files inspected at start-up: \texttt{rootalias.C} and
\texttt{rootlogon.C}. They can contain code that needs to be loaded and
executed at ROOT startup. \texttt{rootalias.C} is only loaded and best
used to define some often used functions. rootlogon.C contains code that
will be executed at startup: this file is extremely useful for example
to pre-load a custom style for the plots created with ROOT. This is done
most easily by creating a new TStyle object with your preferred
settings, as described in the class reference guide, and then use the
command \texttt{gROOT-\textgreater{}SetStyle("MyStyleName");} to make
this new style definition the default one. As an example, have a look in
the file rootlogon.C coming with this tutorial. Another relevant file is
\texttt{rootlogoff.C} that it called when the session is finished.

\subsubsection{2.8.3 ROOT command history}\label{root-command-history}

Every command typed at the ROOT prompt is stored in a file
\texttt{.root\_hist} in your home directory. ROOT uses this file to
allow for navigation in the command history with the up-arrow and
down-arrow keys. It is also convenient to extract successful ROOT
commands with the help of a text editor for use in your own macros.

\subsubsection{2.8.4 ROOT Global Pointers}\label{root-global-pointers}

All global pointers in ROOT begin with a small ``g''. Some of them were
already implicitly introduced (for example in the section Configure ROOT
at start-up). The most important among them are presented in the
following:

\begin{itemize}
\item
  \textbf{\href{http://root.cern.ch/root/htmldoc/TROOT.html}{gROOT}:}
  the \texttt{gROOT} variable is the entry point to the \texttt{ROOT}
  system. Technically it is an instance of the \texttt{TROOT} class.
  Using the gROOT pointer one has access to basically every object
  created in a ROOT based program. The \texttt{TROOT} object is
  essentially a container of several lists pointing to the main
  \texttt{ROOT} objects.
\item
  \textbf{\href{http://root.cern.ch/root/htmldoc/TStyle.html}{gStyle}:}
  By default ROOT creates a default style that can be accessed via the
  \texttt{gStyle} pointer. This class includes functions to set some of
  the following object attributes.
\item
  Canvas
\item
  Pad
\item
  Histogram axis
\item
  Lines
\item
  Fill areas
\item
  Text
\item
  Markers
\item
  Functions
\item
  Histogram Statistics and Titles
\item
  etc \ldots{}
\item
  \textbf{\href{http://root.cern.ch/root/htmldoc/TSystem.html}{gSystem}:}
  An instance of a base class defining a generic interface to the
  underlying Operating System, in our case \texttt{TUnixSystem}.
\item
  \textbf{\href{http://root.cern.ch/htmldoc/html/TInterpreter.html}{gInterpreter}:}
  The entry point for the ROOT interpreter. Technically an abstraction
  level over a singleton instance of \texttt{TCling}.
\end{itemize}

At this point you have already learnt quite a bit about some basic
features of ROOT.

\textbf{Please move on to become an expert!}


    % Add a bibliography block to the postdoc
    
    
    
    \end{document}
