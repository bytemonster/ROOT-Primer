
% Default to the notebook output style

    


% Inherit from the specified cell style.




    
\documentclass{article}

    
    
    \usepackage{graphicx} % Used to insert images
    \usepackage{adjustbox} % Used to constrain images to a maximum size 
    \usepackage{color} % Allow colors to be defined
    \usepackage{enumerate} % Needed for markdown enumerations to work
    \usepackage{geometry} % Used to adjust the document margins
    \usepackage{amsmath} % Equations
    \usepackage{amssymb} % Equations
    \usepackage{eurosym} % defines \euro
    \usepackage[mathletters]{ucs} % Extended unicode (utf-8) support
    \usepackage[utf8x]{inputenc} % Allow utf-8 characters in the tex document
    \usepackage{fancyvrb} % verbatim replacement that allows latex
    \usepackage{grffile} % extends the file name processing of package graphics 
                         % to support a larger range 
    % The hyperref package gives us a pdf with properly built
    % internal navigation ('pdf bookmarks' for the table of contents,
    % internal cross-reference links, web links for URLs, etc.)
    \usepackage{hyperref}
    \usepackage{longtable} % longtable support required by pandoc >1.10
    \usepackage{booktabs}  % table support for pandoc > 1.12.2
    \usepackage{ulem} % ulem is needed to support strikethroughs (\sout)
    

    
    
    \definecolor{orange}{cmyk}{0,0.4,0.8,0.2}
    \definecolor{darkorange}{rgb}{.71,0.21,0.01}
    \definecolor{darkgreen}{rgb}{.12,.54,.11}
    \definecolor{myteal}{rgb}{.26, .44, .56}
    \definecolor{gray}{gray}{0.45}
    \definecolor{lightgray}{gray}{.95}
    \definecolor{mediumgray}{gray}{.8}
    \definecolor{inputbackground}{rgb}{.95, .95, .85}
    \definecolor{outputbackground}{rgb}{.95, .95, .95}
    \definecolor{traceback}{rgb}{1, .95, .95}
    % ansi colors
    \definecolor{red}{rgb}{.6,0,0}
    \definecolor{green}{rgb}{0,.65,0}
    \definecolor{brown}{rgb}{0.6,0.6,0}
    \definecolor{blue}{rgb}{0,.145,.698}
    \definecolor{purple}{rgb}{.698,.145,.698}
    \definecolor{cyan}{rgb}{0,.698,.698}
    \definecolor{lightgray}{gray}{0.5}
    
    % bright ansi colors
    \definecolor{darkgray}{gray}{0.25}
    \definecolor{lightred}{rgb}{1.0,0.39,0.28}
    \definecolor{lightgreen}{rgb}{0.48,0.99,0.0}
    \definecolor{lightblue}{rgb}{0.53,0.81,0.92}
    \definecolor{lightpurple}{rgb}{0.87,0.63,0.87}
    \definecolor{lightcyan}{rgb}{0.5,1.0,0.83}
    
    % commands and environments needed by pandoc snippets
    % extracted from the output of `pandoc -s`
    \providecommand{\tightlist}{%
      \setlength{\itemsep}{0pt}\setlength{\parskip}{0pt}}
    \DefineVerbatimEnvironment{Highlighting}{Verbatim}{commandchars=\\\{\}}
    % Add ',fontsize=\small' for more characters per line
    \newenvironment{Shaded}{}{}
    \newcommand{\KeywordTok}[1]{\textcolor[rgb]{0.00,0.44,0.13}{\textbf{{#1}}}}
    \newcommand{\DataTypeTok}[1]{\textcolor[rgb]{0.56,0.13,0.00}{{#1}}}
    \newcommand{\DecValTok}[1]{\textcolor[rgb]{0.25,0.63,0.44}{{#1}}}
    \newcommand{\BaseNTok}[1]{\textcolor[rgb]{0.25,0.63,0.44}{{#1}}}
    \newcommand{\FloatTok}[1]{\textcolor[rgb]{0.25,0.63,0.44}{{#1}}}
    \newcommand{\CharTok}[1]{\textcolor[rgb]{0.25,0.44,0.63}{{#1}}}
    \newcommand{\StringTok}[1]{\textcolor[rgb]{0.25,0.44,0.63}{{#1}}}
    \newcommand{\CommentTok}[1]{\textcolor[rgb]{0.38,0.63,0.69}{\textit{{#1}}}}
    \newcommand{\OtherTok}[1]{\textcolor[rgb]{0.00,0.44,0.13}{{#1}}}
    \newcommand{\AlertTok}[1]{\textcolor[rgb]{1.00,0.00,0.00}{\textbf{{#1}}}}
    \newcommand{\FunctionTok}[1]{\textcolor[rgb]{0.02,0.16,0.49}{{#1}}}
    \newcommand{\RegionMarkerTok}[1]{{#1}}
    \newcommand{\ErrorTok}[1]{\textcolor[rgb]{1.00,0.00,0.00}{\textbf{{#1}}}}
    \newcommand{\NormalTok}[1]{{#1}}
    
    % Additional commands for more recent versions of Pandoc
    \newcommand{\ConstantTok}[1]{\textcolor[rgb]{0.53,0.00,0.00}{{#1}}}
    \newcommand{\SpecialCharTok}[1]{\textcolor[rgb]{0.25,0.44,0.63}{{#1}}}
    \newcommand{\VerbatimStringTok}[1]{\textcolor[rgb]{0.25,0.44,0.63}{{#1}}}
    \newcommand{\SpecialStringTok}[1]{\textcolor[rgb]{0.73,0.40,0.53}{{#1}}}
    \newcommand{\ImportTok}[1]{{#1}}
    \newcommand{\DocumentationTok}[1]{\textcolor[rgb]{0.73,0.13,0.13}{\textit{{#1}}}}
    \newcommand{\AnnotationTok}[1]{\textcolor[rgb]{0.38,0.63,0.69}{\textbf{\textit{{#1}}}}}
    \newcommand{\CommentVarTok}[1]{\textcolor[rgb]{0.38,0.63,0.69}{\textbf{\textit{{#1}}}}}
    \newcommand{\VariableTok}[1]{\textcolor[rgb]{0.10,0.09,0.49}{{#1}}}
    \newcommand{\ControlFlowTok}[1]{\textcolor[rgb]{0.00,0.44,0.13}{\textbf{{#1}}}}
    \newcommand{\OperatorTok}[1]{\textcolor[rgb]{0.40,0.40,0.40}{{#1}}}
    \newcommand{\BuiltInTok}[1]{{#1}}
    \newcommand{\ExtensionTok}[1]{{#1}}
    \newcommand{\PreprocessorTok}[1]{\textcolor[rgb]{0.74,0.48,0.00}{{#1}}}
    \newcommand{\AttributeTok}[1]{\textcolor[rgb]{0.49,0.56,0.16}{{#1}}}
    \newcommand{\InformationTok}[1]{\textcolor[rgb]{0.38,0.63,0.69}{\textbf{\textit{{#1}}}}}
    \newcommand{\WarningTok}[1]{\textcolor[rgb]{0.38,0.63,0.69}{\textbf{\textit{{#1}}}}}
    
    
    % Define a nice break command that doesn't care if a line doesn't already
    % exist.
    \def\br{\hspace*{\fill} \\* }
    % Math Jax compatability definitions
    \def\gt{>}
    \def\lt{<}
    % Document parameters
    \title{3-ROOT-Macros}
    
    
    

    % Pygments definitions
    
\makeatletter
\def\PY@reset{\let\PY@it=\relax \let\PY@bf=\relax%
    \let\PY@ul=\relax \let\PY@tc=\relax%
    \let\PY@bc=\relax \let\PY@ff=\relax}
\def\PY@tok#1{\csname PY@tok@#1\endcsname}
\def\PY@toks#1+{\ifx\relax#1\empty\else%
    \PY@tok{#1}\expandafter\PY@toks\fi}
\def\PY@do#1{\PY@bc{\PY@tc{\PY@ul{%
    \PY@it{\PY@bf{\PY@ff{#1}}}}}}}
\def\PY#1#2{\PY@reset\PY@toks#1+\relax+\PY@do{#2}}

\expandafter\def\csname PY@tok@nd\endcsname{\def\PY@tc##1{\textcolor[rgb]{0.67,0.13,1.00}{##1}}}
\expandafter\def\csname PY@tok@mb\endcsname{\def\PY@tc##1{\textcolor[rgb]{0.40,0.40,0.40}{##1}}}
\expandafter\def\csname PY@tok@gs\endcsname{\let\PY@bf=\textbf}
\expandafter\def\csname PY@tok@nb\endcsname{\def\PY@tc##1{\textcolor[rgb]{0.00,0.50,0.00}{##1}}}
\expandafter\def\csname PY@tok@mf\endcsname{\def\PY@tc##1{\textcolor[rgb]{0.40,0.40,0.40}{##1}}}
\expandafter\def\csname PY@tok@bp\endcsname{\def\PY@tc##1{\textcolor[rgb]{0.00,0.50,0.00}{##1}}}
\expandafter\def\csname PY@tok@gh\endcsname{\let\PY@bf=\textbf\def\PY@tc##1{\textcolor[rgb]{0.00,0.00,0.50}{##1}}}
\expandafter\def\csname PY@tok@si\endcsname{\let\PY@bf=\textbf\def\PY@tc##1{\textcolor[rgb]{0.73,0.40,0.53}{##1}}}
\expandafter\def\csname PY@tok@gt\endcsname{\def\PY@tc##1{\textcolor[rgb]{0.00,0.27,0.87}{##1}}}
\expandafter\def\csname PY@tok@s\endcsname{\def\PY@tc##1{\textcolor[rgb]{0.73,0.13,0.13}{##1}}}
\expandafter\def\csname PY@tok@gu\endcsname{\let\PY@bf=\textbf\def\PY@tc##1{\textcolor[rgb]{0.50,0.00,0.50}{##1}}}
\expandafter\def\csname PY@tok@ge\endcsname{\let\PY@it=\textit}
\expandafter\def\csname PY@tok@nt\endcsname{\let\PY@bf=\textbf\def\PY@tc##1{\textcolor[rgb]{0.00,0.50,0.00}{##1}}}
\expandafter\def\csname PY@tok@kr\endcsname{\let\PY@bf=\textbf\def\PY@tc##1{\textcolor[rgb]{0.00,0.50,0.00}{##1}}}
\expandafter\def\csname PY@tok@cpf\endcsname{\let\PY@it=\textit\def\PY@tc##1{\textcolor[rgb]{0.25,0.50,0.50}{##1}}}
\expandafter\def\csname PY@tok@vi\endcsname{\def\PY@tc##1{\textcolor[rgb]{0.10,0.09,0.49}{##1}}}
\expandafter\def\csname PY@tok@sx\endcsname{\def\PY@tc##1{\textcolor[rgb]{0.00,0.50,0.00}{##1}}}
\expandafter\def\csname PY@tok@nc\endcsname{\let\PY@bf=\textbf\def\PY@tc##1{\textcolor[rgb]{0.00,0.00,1.00}{##1}}}
\expandafter\def\csname PY@tok@s1\endcsname{\def\PY@tc##1{\textcolor[rgb]{0.73,0.13,0.13}{##1}}}
\expandafter\def\csname PY@tok@sc\endcsname{\def\PY@tc##1{\textcolor[rgb]{0.73,0.13,0.13}{##1}}}
\expandafter\def\csname PY@tok@sr\endcsname{\def\PY@tc##1{\textcolor[rgb]{0.73,0.40,0.53}{##1}}}
\expandafter\def\csname PY@tok@nn\endcsname{\let\PY@bf=\textbf\def\PY@tc##1{\textcolor[rgb]{0.00,0.00,1.00}{##1}}}
\expandafter\def\csname PY@tok@gp\endcsname{\let\PY@bf=\textbf\def\PY@tc##1{\textcolor[rgb]{0.00,0.00,0.50}{##1}}}
\expandafter\def\csname PY@tok@cm\endcsname{\let\PY@it=\textit\def\PY@tc##1{\textcolor[rgb]{0.25,0.50,0.50}{##1}}}
\expandafter\def\csname PY@tok@kn\endcsname{\let\PY@bf=\textbf\def\PY@tc##1{\textcolor[rgb]{0.00,0.50,0.00}{##1}}}
\expandafter\def\csname PY@tok@kc\endcsname{\let\PY@bf=\textbf\def\PY@tc##1{\textcolor[rgb]{0.00,0.50,0.00}{##1}}}
\expandafter\def\csname PY@tok@mo\endcsname{\def\PY@tc##1{\textcolor[rgb]{0.40,0.40,0.40}{##1}}}
\expandafter\def\csname PY@tok@cs\endcsname{\let\PY@it=\textit\def\PY@tc##1{\textcolor[rgb]{0.25,0.50,0.50}{##1}}}
\expandafter\def\csname PY@tok@na\endcsname{\def\PY@tc##1{\textcolor[rgb]{0.49,0.56,0.16}{##1}}}
\expandafter\def\csname PY@tok@vc\endcsname{\def\PY@tc##1{\textcolor[rgb]{0.10,0.09,0.49}{##1}}}
\expandafter\def\csname PY@tok@nl\endcsname{\def\PY@tc##1{\textcolor[rgb]{0.63,0.63,0.00}{##1}}}
\expandafter\def\csname PY@tok@ow\endcsname{\let\PY@bf=\textbf\def\PY@tc##1{\textcolor[rgb]{0.67,0.13,1.00}{##1}}}
\expandafter\def\csname PY@tok@sd\endcsname{\let\PY@it=\textit\def\PY@tc##1{\textcolor[rgb]{0.73,0.13,0.13}{##1}}}
\expandafter\def\csname PY@tok@gd\endcsname{\def\PY@tc##1{\textcolor[rgb]{0.63,0.00,0.00}{##1}}}
\expandafter\def\csname PY@tok@c1\endcsname{\let\PY@it=\textit\def\PY@tc##1{\textcolor[rgb]{0.25,0.50,0.50}{##1}}}
\expandafter\def\csname PY@tok@kp\endcsname{\def\PY@tc##1{\textcolor[rgb]{0.00,0.50,0.00}{##1}}}
\expandafter\def\csname PY@tok@il\endcsname{\def\PY@tc##1{\textcolor[rgb]{0.40,0.40,0.40}{##1}}}
\expandafter\def\csname PY@tok@ni\endcsname{\let\PY@bf=\textbf\def\PY@tc##1{\textcolor[rgb]{0.60,0.60,0.60}{##1}}}
\expandafter\def\csname PY@tok@ss\endcsname{\def\PY@tc##1{\textcolor[rgb]{0.10,0.09,0.49}{##1}}}
\expandafter\def\csname PY@tok@c\endcsname{\let\PY@it=\textit\def\PY@tc##1{\textcolor[rgb]{0.25,0.50,0.50}{##1}}}
\expandafter\def\csname PY@tok@cp\endcsname{\def\PY@tc##1{\textcolor[rgb]{0.74,0.48,0.00}{##1}}}
\expandafter\def\csname PY@tok@o\endcsname{\def\PY@tc##1{\textcolor[rgb]{0.40,0.40,0.40}{##1}}}
\expandafter\def\csname PY@tok@kd\endcsname{\let\PY@bf=\textbf\def\PY@tc##1{\textcolor[rgb]{0.00,0.50,0.00}{##1}}}
\expandafter\def\csname PY@tok@go\endcsname{\def\PY@tc##1{\textcolor[rgb]{0.53,0.53,0.53}{##1}}}
\expandafter\def\csname PY@tok@kt\endcsname{\def\PY@tc##1{\textcolor[rgb]{0.69,0.00,0.25}{##1}}}
\expandafter\def\csname PY@tok@mi\endcsname{\def\PY@tc##1{\textcolor[rgb]{0.40,0.40,0.40}{##1}}}
\expandafter\def\csname PY@tok@no\endcsname{\def\PY@tc##1{\textcolor[rgb]{0.53,0.00,0.00}{##1}}}
\expandafter\def\csname PY@tok@ch\endcsname{\let\PY@it=\textit\def\PY@tc##1{\textcolor[rgb]{0.25,0.50,0.50}{##1}}}
\expandafter\def\csname PY@tok@ne\endcsname{\let\PY@bf=\textbf\def\PY@tc##1{\textcolor[rgb]{0.82,0.25,0.23}{##1}}}
\expandafter\def\csname PY@tok@gi\endcsname{\def\PY@tc##1{\textcolor[rgb]{0.00,0.63,0.00}{##1}}}
\expandafter\def\csname PY@tok@w\endcsname{\def\PY@tc##1{\textcolor[rgb]{0.73,0.73,0.73}{##1}}}
\expandafter\def\csname PY@tok@se\endcsname{\let\PY@bf=\textbf\def\PY@tc##1{\textcolor[rgb]{0.73,0.40,0.13}{##1}}}
\expandafter\def\csname PY@tok@s2\endcsname{\def\PY@tc##1{\textcolor[rgb]{0.73,0.13,0.13}{##1}}}
\expandafter\def\csname PY@tok@nv\endcsname{\def\PY@tc##1{\textcolor[rgb]{0.10,0.09,0.49}{##1}}}
\expandafter\def\csname PY@tok@m\endcsname{\def\PY@tc##1{\textcolor[rgb]{0.40,0.40,0.40}{##1}}}
\expandafter\def\csname PY@tok@k\endcsname{\let\PY@bf=\textbf\def\PY@tc##1{\textcolor[rgb]{0.00,0.50,0.00}{##1}}}
\expandafter\def\csname PY@tok@mh\endcsname{\def\PY@tc##1{\textcolor[rgb]{0.40,0.40,0.40}{##1}}}
\expandafter\def\csname PY@tok@gr\endcsname{\def\PY@tc##1{\textcolor[rgb]{1.00,0.00,0.00}{##1}}}
\expandafter\def\csname PY@tok@sb\endcsname{\def\PY@tc##1{\textcolor[rgb]{0.73,0.13,0.13}{##1}}}
\expandafter\def\csname PY@tok@sh\endcsname{\def\PY@tc##1{\textcolor[rgb]{0.73,0.13,0.13}{##1}}}
\expandafter\def\csname PY@tok@vg\endcsname{\def\PY@tc##1{\textcolor[rgb]{0.10,0.09,0.49}{##1}}}
\expandafter\def\csname PY@tok@nf\endcsname{\def\PY@tc##1{\textcolor[rgb]{0.00,0.00,1.00}{##1}}}
\expandafter\def\csname PY@tok@err\endcsname{\def\PY@bc##1{\setlength{\fboxsep}{0pt}\fcolorbox[rgb]{1.00,0.00,0.00}{1,1,1}{\strut ##1}}}

\def\PYZbs{\char`\\}
\def\PYZus{\char`\_}
\def\PYZob{\char`\{}
\def\PYZcb{\char`\}}
\def\PYZca{\char`\^}
\def\PYZam{\char`\&}
\def\PYZlt{\char`\<}
\def\PYZgt{\char`\>}
\def\PYZsh{\char`\#}
\def\PYZpc{\char`\%}
\def\PYZdl{\char`\$}
\def\PYZhy{\char`\-}
\def\PYZsq{\char`\'}
\def\PYZdq{\char`\"}
\def\PYZti{\char`\~}
% for compatibility with earlier versions
\def\PYZat{@}
\def\PYZlb{[}
\def\PYZrb{]}
\makeatother


    % Exact colors from NB
    \definecolor{incolor}{rgb}{0.0, 0.0, 0.5}
    \definecolor{outcolor}{rgb}{0.545, 0.0, 0.0}



    
    % Prevent overflowing lines due to hard-to-break entities
    \sloppy 
    % Setup hyperref package
    \hypersetup{
      breaklinks=true,  % so long urls are correctly broken across lines
      colorlinks=true,
      urlcolor=blue,
      linkcolor=darkorange,
      citecolor=darkgreen,
      }
    % Slightly bigger margins than the latex defaults
    
    \geometry{verbose,tmargin=1in,bmargin=1in,lmargin=1in,rmargin=1in}
    
    

    \begin{document}
    
    
    \maketitle
    
    

    
    You know how other books go on and on about programming fundamentals and
finally work up to building a complete, working program? Let's skip all
that. In this guide, we will describe macros executed by the ROOT C++
interpreter Cling.

It is relatively easy to compile a macro, either as a pre-compiled
library to load into ROOT, or as a stand-alone application, by adding
some include statements for header file or some ``dressing code'' to any
macro.

\subsection{3.1 General Remarks on ROOT
macros}\label{general-remarks-on-root-macros}

If you have a number of lines which you were able to execute at the ROOT
prompt, they can be turned into a ROOT macro by giving them a name which
corresponds to the file name without extension. The general structure
for a macro stored in file \texttt{MacroName.C} is:

\begin{Shaded}
\begin{Highlighting}[]
\DataTypeTok{void} \NormalTok{MacroName() \{}
        \NormalTok{<          ...}
          \NormalTok{your lines of C++ code}
                   \NormalTok{...             >}
\NormalTok{\}}
\end{Highlighting}
\end{Shaded}

The macro is executed by typing:

\begin{Shaded}
\begin{Highlighting}[]
 \KeywordTok{>} \KeywordTok{root} \NormalTok{MacroName.C}
\end{Highlighting}
\end{Shaded}

at the system prompt, or executed using \texttt{Bash\ .x} at the ROOT
prompt.

\begin{Shaded}
\begin{Highlighting}[]
 \KeywordTok{>} \NormalTok{root }
 \NormalTok{root [0] .x MacroName.C}
\end{Highlighting}
\end{Shaded}

Or it can be loaded into a ROOT session and then be executed by typing:

\begin{Shaded}
\begin{Highlighting}[]
\KeywordTok{root} \NormalTok{[0].L MacroName.C}
\KeywordTok{root} \NormalTok{[1] MacroName();}
\end{Highlighting}
\end{Shaded}

at the ROOT prompt.Note that more than one macro can be loaded this way,
as each macro has a unique name in the ROOT name space. A small set of
options can help making your plot nicer.

\begin{Shaded}
\begin{Highlighting}[]
\NormalTok{gROOT->SetStyle(}\StringTok{"Plain"}\NormalTok{);   }\CommentTok{// set plain TStyle}
\NormalTok{gStyle->SetOptStat(}\DecValTok{111111}\NormalTok{); }\CommentTok{// draw statistics on plots,}
                            \CommentTok{// (0) for no output}
\NormalTok{gStyle->SetOptFit(}\DecValTok{1111}\NormalTok{);    }\CommentTok{// draw fit results on plot,}
                            \CommentTok{// (0) for no ouput}
\NormalTok{gStyle->SetPalette(}\DecValTok{57}\NormalTok{);     }\CommentTok{// set color map}
\NormalTok{gStyle->SetOptTitle(}\DecValTok{0}\NormalTok{);     }\CommentTok{// suppress title box}
\end{Highlighting}
\end{Shaded}

Next, you should create a canvas for graphical output, with size,
subdivisions and format suitable to your needs, see documentation of
class \texttt{TCanvas}:

    \begin{Verbatim}[commandchars=\\\{\}]
{\color{incolor}In [{\color{incolor}1}]:} \PY{n}{TCanvas} \PY{n+nf}{canvas\PYZus{}3\PYZus{}1}\PY{p}{(}\PY{l+s}{\PYZdq{}}\PY{l+s}{3\PYZhy{}1\PYZhy{}Canvas}\PY{l+s}{\PYZdq{}}\PY{p}{,}\PY{l+s}{\PYZdq{}}\PY{l+s}{\PYZlt{}Title\PYZgt{}}\PY{l+s}{\PYZdq{}}\PY{p}{,}\PY{l+m+mi}{0}\PY{p}{,}\PY{l+m+mi}{0}\PY{p}{,}\PY{l+m+mi}{900}\PY{p}{,}\PY{l+m+mi}{400}\PY{p}{)}\PY{p}{;} 
        \PY{n}{canvas\PYZus{}3\PYZus{}1}\PY{p}{.}\PY{n}{Divide}\PY{p}{(}\PY{l+m+mi}{2}\PY{p}{,}\PY{l+m+mi}{1}\PY{p}{)}\PY{p}{;} 
        \PY{n}{canvas\PYZus{}3\PYZus{}1}\PY{p}{.}\PY{n}{cd}\PY{p}{(}\PY{l+m+mi}{1}\PY{p}{)}\PY{p}{;} 
        \PY{n}{TF1} \PY{n+nf}{f1}\PY{p}{(}\PY{l+s}{\PYZdq{}}\PY{l+s}{f1}\PY{l+s}{\PYZdq{}}\PY{p}{,}\PY{l+s}{\PYZdq{}}\PY{l+s}{sin(x)}\PY{o}{/}\PY{n}{x}\PY{l+s}{\PYZdq{}}\PY{l+s}{,0.,10.)}\PY{p}{;}
        \PY{n}{f1}\PY{p}{.}\PY{n}{Draw}\PY{p}{(}\PY{p}{)}\PY{p}{;}
        \PY{n}{canvas\PYZus{}3\PYZus{}1}\PY{p}{.}\PY{n}{cd}\PY{p}{(}\PY{l+m+mi}{2}\PY{p}{)}\PY{p}{;}
        \PY{n}{TF1} \PY{n+nf}{f2}\PY{p}{(}\PY{l+s}{\PYZdq{}}\PY{l+s}{f2}\PY{l+s}{\PYZdq{}}\PY{p}{,}\PY{l+s}{\PYZdq{}}\PY{l+s}{sin(x)}\PY{o}{/}\PY{n}{x}\PY{l+s}{\PYZdq{}}\PY{l+s}{,0.,10.)}\PY{p}{;}
        \PY{n}{f2}\PY{p}{.}\PY{n}{Draw}\PY{p}{(}\PY{p}{)}\PY{p}{;}
        \PY{n}{canvas\PYZus{}3\PYZus{}1}\PY{p}{.}\PY{n}{Draw}\PY{p}{(}\PY{p}{)}\PY{p}{;}
\end{Verbatim}

    \begin{center}
    \adjustimage{max size={0.9\linewidth}{0.9\paperheight}}{3-ROOT-Macros_files/3-ROOT-Macros_1_0.png}
    \end{center}
    { \hspace*{\fill} \\}
    
    These parts of a well-written macro are pretty standard, and you should
remember to include pieces of code like in the examples above to make
sure your plots always look as you had intended.

Below, in section Interpretation and Compilation, some more code
fragments will be shown, allowing you to use the system compiler to
compile macros for more efficient execution, or turn macros into
stand-alone applications linked against the ROOT libraries.

\subsection{3.2 A more complete example}\label{a-more-complete-example}

Let us now look at a rather complete example of a typical task in data
analysis, a macro that constructs a graph with errors, fits a (linear)
model to it and saves it as an image. To run this macro, simply type in
the shell:

\begin{verbatim}
 > root macro1.C
\end{verbatim}

The code is built around the ROOT class \texttt{TGraphErrors}, which was
already introduced previously. Have a look at it in the class reference
guide, where you will also find further examples. The macro shown below
uses additional classes, \texttt{TF1} to define a function, TCanvas to
define size and properties of the window used for our plot, and
\texttt{TLegend} to add a nice legend. For the moment, ignore the
commented include statements for header files, they will only become
important at the end in section Interpretation and Compilation.

    \begin{Verbatim}[commandchars=\\\{\}]
{\color{incolor}In [{\color{incolor}2}]:} \PY{o}{\PYZpc{}}\PY{o}{\PYZpc{}}\PY{n}{cpp} \PY{o}{\PYZhy{}}\PY{n}{d}
        
        \PY{c+c1}{// Builds a graph with errors, displays it and saves it as}
        \PY{c+c1}{// image. First, include some header files}
        \PY{c+c1}{// (not necessary for Cling)}
        
        \PY{c+cp}{\PYZsh{}}\PY{c+cp}{include} \PY{c+cpf}{\PYZdq{}TCanvas.h\PYZdq{}}
        \PY{c+cp}{\PYZsh{}}\PY{c+cp}{include} \PY{c+cpf}{\PYZdq{}TROOT.h\PYZdq{}}
        \PY{c+cp}{\PYZsh{}}\PY{c+cp}{include} \PY{c+cpf}{\PYZdq{}TGraphErrors.h\PYZdq{}}
        \PY{c+cp}{\PYZsh{}}\PY{c+cp}{include} \PY{c+cpf}{\PYZdq{}TF1.h\PYZdq{}}
        \PY{c+cp}{\PYZsh{}}\PY{c+cp}{include} \PY{c+cpf}{\PYZdq{}TLegend.h\PYZdq{}}
        \PY{c+cp}{\PYZsh{}}\PY{c+cp}{include} \PY{c+cpf}{\PYZdq{}TArrow.h\PYZdq{}}
        \PY{c+cp}{\PYZsh{}}\PY{c+cp}{include} \PY{c+cpf}{\PYZdq{}TLatex.h\PYZdq{}}
            
            
        \PY{k+kt}{void} \PY{n}{macro3\PYZus{}2\PYZus{}1}\PY{p}{(}\PY{p}{)}\PY{p}{\PYZob{}}   \PY{c+c1}{//\PYZsh{}1}
            \PY{c+c1}{// The values and the errors on the Y axis}
            \PY{k}{const} \PY{k+kt}{int} \PY{n}{n\PYZus{}points}\PY{o}{=}\PY{l+m+mi}{10}\PY{p}{;}
            \PY{k+kt}{double} \PY{n}{x\PYZus{}vals}\PY{p}{[}\PY{n}{n\PYZus{}points}\PY{p}{]}\PY{o}{=}
                    \PY{p}{\PYZob{}}\PY{l+m+mi}{1}\PY{p}{,}\PY{l+m+mi}{2}\PY{p}{,}\PY{l+m+mi}{3}\PY{p}{,}\PY{l+m+mi}{4}\PY{p}{,}\PY{l+m+mi}{5}\PY{p}{,}\PY{l+m+mi}{6}\PY{p}{,}\PY{l+m+mi}{7}\PY{p}{,}\PY{l+m+mi}{8}\PY{p}{,}\PY{l+m+mi}{9}\PY{p}{,}\PY{l+m+mi}{10}\PY{p}{\PYZcb{}}\PY{p}{;}
            \PY{k+kt}{double} \PY{n}{y\PYZus{}vals}\PY{p}{[}\PY{n}{n\PYZus{}points}\PY{p}{]}\PY{o}{=}
                    \PY{p}{\PYZob{}}\PY{l+m+mi}{6}\PY{p}{,}\PY{l+m+mi}{12}\PY{p}{,}\PY{l+m+mi}{14}\PY{p}{,}\PY{l+m+mi}{20}\PY{p}{,}\PY{l+m+mi}{22}\PY{p}{,}\PY{l+m+mi}{24}\PY{p}{,}\PY{l+m+mi}{35}\PY{p}{,}\PY{l+m+mi}{45}\PY{p}{,}\PY{l+m+mi}{44}\PY{p}{,}\PY{l+m+mi}{53}\PY{p}{\PYZcb{}}\PY{p}{;}
            \PY{k+kt}{double} \PY{n}{y\PYZus{}errs}\PY{p}{[}\PY{n}{n\PYZus{}points}\PY{p}{]}\PY{o}{=}
                    \PY{p}{\PYZob{}}\PY{l+m+mi}{5}\PY{p}{,}\PY{l+m+mi}{5}\PY{p}{,}\PY{l+m+mf}{4.7}\PY{p}{,}\PY{l+m+mf}{4.5}\PY{p}{,}\PY{l+m+mf}{4.2}\PY{p}{,}\PY{l+m+mf}{5.1}\PY{p}{,}\PY{l+m+mf}{2.9}\PY{p}{,}\PY{l+m+mf}{4.1}\PY{p}{,}\PY{l+m+mf}{4.8}\PY{p}{,}\PY{l+m+mf}{5.43}\PY{p}{\PYZcb{}}\PY{p}{;}
        
            \PY{c+c1}{// Instance of the graph}
            \PY{c+c1}{//\PYZsh{}2}
            \PY{n}{TGraphErrors} \PY{n+nf}{graph}\PY{p}{(}\PY{n}{n\PYZus{}points}\PY{p}{,}\PY{n}{x\PYZus{}vals}\PY{p}{,}\PY{n}{y\PYZus{}vals}\PY{p}{,}\PY{k}{nullptr}\PY{p}{,}\PY{n}{y\PYZus{}errs}\PY{p}{)}\PY{p}{;}
            \PY{n}{graph}\PY{p}{.}\PY{n}{SetTitle}\PY{p}{(}\PY{l+s}{\PYZdq{}}\PY{l+s}{Measurement XYZ;lenght [cm];Arb.Units}\PY{l+s}{\PYZdq{}}\PY{p}{)}\PY{p}{;}
        
            \PY{c+c1}{// Make the plot estetically better}
            \PY{c+c1}{//\PYZsh{}3}
            \PY{n}{graph}\PY{p}{.}\PY{n}{SetMarkerStyle}\PY{p}{(}\PY{n}{kOpenCircle}\PY{p}{)}\PY{p}{;}
            \PY{n}{graph}\PY{p}{.}\PY{n}{SetMarkerColor}\PY{p}{(}\PY{n}{kBlue}\PY{p}{)}\PY{p}{;}
            \PY{n}{graph}\PY{p}{.}\PY{n}{SetLineColor}\PY{p}{(}\PY{n}{kBlue}\PY{p}{)}\PY{p}{;}
        
            \PY{c+c1}{// The canvas on which we\PYZsq{}ll draw the graph}
            \PY{c+c1}{//\PYZsh{}4}
            \PY{k}{auto}  \PY{n}{Canvas\PYZus{}3\PYZus{}2\PYZus{}1} \PY{o}{=} \PY{k}{new} \PY{n}{TCanvas}\PY{p}{(}\PY{p}{)}\PY{p}{;}
        
            \PY{c+c1}{// Draw the graph !}
            \PY{c+c1}{//\PYZsh{}5}
            \PY{n}{graph}\PY{p}{.}\PY{n}{DrawClone}\PY{p}{(}\PY{l+s}{\PYZdq{}}\PY{l+s}{APE}\PY{l+s}{\PYZdq{}}\PY{p}{)}\PY{p}{;}
        
            \PY{c+c1}{// Define a linear function}
            \PY{c+c1}{//\PYZsh{}6}
            \PY{n}{TF1} \PY{n+nf}{function\PYZus{}3\PYZus{}2\PYZus{}1}\PY{p}{(}\PY{l+s}{\PYZdq{}}\PY{l+s}{Linear law}\PY{l+s}{\PYZdq{}}\PY{p}{,}\PY{l+s}{\PYZdq{}}\PY{l+s}{[0]+x*[1]}\PY{l+s}{\PYZdq{}}\PY{p}{,}\PY{l+m+mf}{.5}\PY{p}{,}\PY{l+m+mf}{10.5}\PY{p}{)}\PY{p}{;}
            \PY{c+c1}{// Let\PYZsq{}s make the funcion line nicer}
            \PY{c+c1}{//\PYZsh{}7}
            \PY{n}{function\PYZus{}3\PYZus{}2\PYZus{}1}\PY{p}{.}\PY{n}{SetLineColor}\PY{p}{(}\PY{n}{kRed}\PY{p}{)}\PY{p}{;}  \PY{n}{function\PYZus{}3\PYZus{}2\PYZus{}1}\PY{p}{.}\PY{n}{SetLineStyle}\PY{p}{(}\PY{l+m+mi}{2}\PY{p}{)}\PY{p}{;}
            \PY{c+c1}{// Fit it to the graph and draw it}
            \PY{c+c1}{//\PYZsh{}8}
            \PY{n}{graph}\PY{p}{.}\PY{n}{Fit}\PY{p}{(}\PY{o}{\PYZam{}}\PY{n}{function\PYZus{}3\PYZus{}2\PYZus{}1}\PY{p}{)}\PY{p}{;}
            \PY{n}{function\PYZus{}3\PYZus{}2\PYZus{}1}\PY{p}{.}\PY{n}{DrawClone}\PY{p}{(}\PY{l+s}{\PYZdq{}}\PY{l+s}{Same}\PY{l+s}{\PYZdq{}}\PY{p}{)}\PY{p}{;}
        
            \PY{c+c1}{// Build and Draw a legend}
            \PY{c+c1}{//\PYZsh{}9}
            \PY{n}{TLegend} \PY{n+nf}{leg}\PY{p}{(}\PY{l+m+mf}{.1}\PY{p}{,}\PY{l+m+mf}{.7}\PY{p}{,}\PY{l+m+mf}{.3}\PY{p}{,}\PY{l+m+mf}{.9}\PY{p}{,}\PY{l+s}{\PYZdq{}}\PY{l+s}{Lab. Lesson 1}\PY{l+s}{\PYZdq{}}\PY{p}{)}\PY{p}{;}
            \PY{n}{leg}\PY{p}{.}\PY{n}{SetFillColor}\PY{p}{(}\PY{l+m+mi}{0}\PY{p}{)}\PY{p}{;}
            \PY{n}{graph}\PY{p}{.}\PY{n}{SetFillColor}\PY{p}{(}\PY{l+m+mi}{0}\PY{p}{)}\PY{p}{;}
            \PY{n}{leg}\PY{p}{.}\PY{n}{AddEntry}\PY{p}{(}\PY{o}{\PYZam{}}\PY{n}{graph}\PY{p}{,}\PY{l+s}{\PYZdq{}}\PY{l+s}{Exp. Points}\PY{l+s}{\PYZdq{}}\PY{p}{)}\PY{p}{;}
            \PY{n}{leg}\PY{p}{.}\PY{n}{AddEntry}\PY{p}{(}\PY{o}{\PYZam{}}\PY{n}{function\PYZus{}3\PYZus{}2\PYZus{}1}\PY{p}{,}\PY{l+s}{\PYZdq{}}\PY{l+s}{Th. Law}\PY{l+s}{\PYZdq{}}\PY{p}{)}\PY{p}{;}
            \PY{n}{leg}\PY{p}{.}\PY{n}{DrawClone}\PY{p}{(}\PY{l+s}{\PYZdq{}}\PY{l+s}{Same}\PY{l+s}{\PYZdq{}}\PY{p}{)}\PY{p}{;}
        
            \PY{c+c1}{// Draw an arrow on the canvas}
            \PY{c+c1}{//\PYZsh{}10}
            \PY{n}{TArrow} \PY{n+nf}{arrow}\PY{p}{(}\PY{l+m+mi}{8}\PY{p}{,}\PY{l+m+mi}{8}\PY{p}{,}\PY{l+m+mf}{6.2}\PY{p}{,}\PY{l+m+mi}{23}\PY{p}{,}\PY{l+m+mf}{0.02}\PY{p}{,}\PY{l+s}{\PYZdq{}}\PY{l+s}{|\PYZgt{}}\PY{l+s}{\PYZdq{}}\PY{p}{)}\PY{p}{;}
            \PY{n}{arrow}\PY{p}{.}\PY{n}{SetLineWidth}\PY{p}{(}\PY{l+m+mi}{2}\PY{p}{)}\PY{p}{;}
            \PY{n}{arrow}\PY{p}{.}\PY{n}{DrawClone}\PY{p}{(}\PY{p}{)}\PY{p}{;}
        
            \PY{c+c1}{// Add some text to the plot}
            \PY{c+c1}{//\PYZsh{}11}
            \PY{n}{TLatex} \PY{n+nf}{text}\PY{p}{(}\PY{l+m+mf}{8.2}\PY{p}{,}\PY{l+m+mf}{7.5}\PY{p}{,}\PY{l+s}{\PYZdq{}}\PY{l+s}{\PYZsh{}splitline\PYZob{}Maximum\PYZcb{}\PYZob{}Deviation\PYZcb{}}\PY{l+s}{\PYZdq{}}\PY{p}{)}\PY{p}{;}
            \PY{n}{text}\PY{p}{.}\PY{n}{DrawClone}\PY{p}{(}\PY{p}{)}\PY{p}{;}
        
            \PY{c+cm}{/*this command will create a pdf file with the graph in the same folder. }
        \PY{c+cm}{    If you want to use it you can uncoment it and comment the Draw command bellow.*/}
            \PY{c+c1}{//\PYZsh{}12}
            
            \PY{c+c1}{//mycanvas\PYZhy{}\PYZgt{}Print(\PYZdq{}graph\PYZus{}with\PYZus{}law.pdf\PYZdq{});}
            \PY{n}{Canvas\PYZus{}3\PYZus{}2\PYZus{}1}\PY{o}{\PYZhy{}}\PY{o}{\PYZgt{}}\PY{n}{Draw}\PY{p}{(}\PY{p}{)}\PY{p}{;}
        \PY{p}{\PYZcb{}}
\end{Verbatim}

    Let's give a look to the obtained plot. Beautiful outcome for such a
small bunch of lines, isn't it ?

Your first plot with data points, a fit of an analytical function, a
legend and some additional information in the form of graphics
primitives and text. A well formatted plot, clear for the reader is
crucial to communicate the relevance of your results to the reader.

    \begin{Verbatim}[commandchars=\\\{\}]
{\color{incolor}In [{\color{incolor}3}]:} \PY{n}{macro3\PYZus{}2\PYZus{}1}\PY{p}{(}\PY{p}{)}\PY{p}{;}
\end{Verbatim}

    \begin{center}
    \adjustimage{max size={0.9\linewidth}{0.9\paperheight}}{3-ROOT-Macros_files/3-ROOT-Macros_5_0.png}
    \end{center}
    { \hspace*{\fill} \\}
    
    \begin{Verbatim}[commandchars=\\\{\}]
FCN=3.84883 FROM MIGRAD    STATUS=CONVERGED      31 CALLS          32 TOTAL
                     EDM=5.96982e-22    STRATEGY= 1      ERROR MATRIX ACCURATE 
  EXT PARAMETER                                   STEP         FIRST   
  NO.   NAME      VALUE            ERROR          SIZE      DERIVATIVE 
   1  p0          -1.01604e+00   3.33409e+00   1.48321e-03  -8.98235e-12
   2  p1           5.18756e+00   5.30717e-01   2.36095e-04   9.40487e-12
    \end{Verbatim}

    Let's comment it in detail:

\begin{itemize}
\item
  \textbf{Point \#1:} the name of the principal function (it plays the
  role of the ``main'' function in compiled programs) in the macro file.
  It has to be the same as the file name without extension.
\item
  \textbf{Point \#2:} instance of the TGraphErrors class. The
  constructor takes the number of points and the pointers to the arrays
  of x values, y values, x errors (in this case none, represented by the
  NULL pointer) and y errors. The second line defines in one shot the
  title of the graph and the titles of the two axes, separated by a
  ``;''.
\item
  \textbf{Point \#3:} These three lines are rather intuitive right ? To
  understand better the enumerators for colours and styles see the
  reference for the TColor and TMarker classes.
\item
  \textbf{Point \#4:} the canvas object that will host the drawn
  objects. The ``memory leak'' is intentional, to make the object
  existing also out of the macro1 scope.
\item
  \textbf{Point \#5:} the method
  \href{https://root.cern.ch/doc/master/classTObject.html\#a45d0875bf30660d0903a93d690ff9f7e}{DrawClone}
  draws a clone of the object on the canvas. It has to be a clone, to
  survive after the scope of macro1, and be displayed on screen after
  the end of the macro execution. The string option ``APE'' stands for:
\item
  A imposes the drawing of the Axes.
\item
  P imposes the drawing of the graph's markers.
\item
  E imposes the drawing of the graph's error bars.
\item
  \textbf{Point \#6:} define a mathematical function. There are several
  ways to accomplish this, but in this case the constructor accepts the
  name of the function, the formula, and the function range.
\item
  \textbf{Point \#7:} maquillage. Try to give a look to the line styles
  at your disposal visiting the documentation of the TLine class.
\item
  \textbf{Point \#8:} fits the f function to the graph, observe that the
  pointer is passed. It is more interesting to look at the output on the
  screen to see the parameters values and other crucial information that
  we will learn to read at the end of this guide. The DrawClone comand
  tha follows draws the clone of the object on the canvas again. The
  ``Same'' option avoids the cancellation of the already drawn objects,
  in our case, the graph. The function f will be drawn using the same
  axis system defined by the previously drawn graph.
\item
  \textbf{Point \#9:} completes the plot with a legend, represented by a
  TLegend instance. The constructor takes as parameters the lower left
  and upper right corners coordinates with respect to the total size of
  the canvas, assumed to be 1, and the legend header string. You can add
  to the legend the objects, previously drawn or not drawn, through the
  addEntry method. Observe how the legend is drawn at the end: looks
  familiar now, right ?
\item
  \textbf{Point \#10:} defines an arrow with a triangle on the right
  hand side, a thickness of 2 and draws it.
\item
  \textbf{Point \#11:} interpret a Latex string which hast its lower
  left corner located in the specified coordinate. The
  \#splitline\{\}\{\} construct allows to store multiple lines in the
  same TLatex object.
\item
  \textbf{Point \#12:} save the canvas as image. The format is
  automatically inferred from the file extension (it could have been
  eps, gif, \ldots{}).
\end{itemize}

\subsection{3.3 Summary of Visual
effects}\label{summary-of-visual-effects}

\subsubsection{3.3.1 Colours and Graph
Markers}\label{colours-and-graph-markers}

We have seen that to specify a colour, some identifiers like kWhite,
kRed or kBlue can be specified for markers, lines, arrows etc. The
complete summary of colours is represented by the ROOT
\href{http://root.cern.ch/root/htmldoc/TColor.html\#C02}{``colour
wheel''}. To know more about the full story, refer to the online
documentation of \texttt{TColor}.

ROOT provides several
\href{http://root.cern.ch/root/htmldoc/TAttMarker.html\#M2}{graphics
markers} types. Select the most suited symbols for your plot among dots,
triangles, crosses or stars. An alternative set of names for the markers
is available.

\subsubsection{3.3.2 Arrows and Lines}\label{arrows-and-lines}

The macro line 55 shows how to define an arrow and draw it. The class
representing arrows is \texttt{TArrow}, which inherits from
\texttt{TLine}. The constructors of lines and arrows always contain the
coordinates of the endpoints. Arrows also foresee parameters to
\href{http://root.cern.ch/root/htmldoc/TArrow.html}{specify their}
shapes. Do not underestimate the role of lines and arrows in your plots.
Since each plot should contain a message, it is convenient to stress it
with additional graphics primitives.

\subsubsection{3.3.3 Text}\label{text}

Also text plays a fundamental role in making the plots self-explanatory.
A possibility to add text in your plot is provided by the TLatex class.
The objects of this class are constructed with the coordinates of the
bottom-left corner of the text and a string which contains the text
itself. The real twist is that ordinary
\href{http://root.cern.ch/root/htmldoc/TLatex.html\#L5}{Latex
mathematical symbols} are automatically interpreted, you just need to
replace the ``'' by a ``\#''.

If \href{http://root.cern.ch/root/htmldoc/TLatex.html\#L14}{``'' is used
as control character} , then the
\href{http://root.cern.ch/root/htmldoc/TMathText.html}{TMathText
interface} is invoked. It provides the plain TeX syntax and allow to
access character's set like Russian and Japenese.

\subsection{3.4 Interpretation and
Compilation}\label{interpretation-and-compilation}

As you observed, up to now we heavily exploited the capabilities of ROOT
for interpreting our code, more than compiling and then executing. This
is sufficient for a wide range of applications, but you might have
already asked yourself ``how can this code be compiled ?''. There are
two answers.

\subsubsection{3.4.1 Compile a Macro with
ACLiC}\label{compile-a-macro-with-aclic}

\href{https://root.cern.ch/compiling-your-code-also-known-aclic}{ACLiC}
will create for you a compiled dynamic library for your macro, without
any effort from your side, except the insertion of the appropriate
header files at the top of the code. In this example, they are already
included. To generate an object library from the macro code, from inside
the interpreter type (please note the ``+''):

\begin{Shaded}
\begin{Highlighting}[]
 \KeywordTok{root} \NormalTok{[1] .L macro1.C+}
\end{Highlighting}
\end{Shaded}

Once this operation is accomplished, the macro symbols will be available
in memory and you will be able to execute it simply by calling from
inside the interpreter:

\begin{Shaded}
\begin{Highlighting}[]
\KeywordTok{root} \NormalTok{[2] macro1()}
\end{Highlighting}
\end{Shaded}

\subsubsection{3.4.2 Compile a Macro with the
Compiler}\label{compile-a-macro-with-the-compiler}

A plethora of excellent compilers are available, both free and
commercial. We will refer to the
\href{https://gcc.gnu.org/onlinedocs/}{\texttt{GCC}} compiler in the
following. In this case, you have to include the appropriate headers in
the code and then exploit the root-config tool for the automatic
settings of all the compiler flags. root-config is a script that comes
with ROOT; it prints all flags and libraries needed to compile code and
link it with the ROOT libraries. In order to make the code executable
stand-alone, an entry point for the operating system is needed, in C++
this is the procedure \texttt{int\ main();}. The easiest way to turn a
ROOT macro code into a stand-alone application is to add the following
``dressing code'' at the end of the macro file. This defines the
procedure main, the only purpose of which is to call your macro:

\begin{Shaded}
\begin{Highlighting}[]
\DataTypeTok{int} \NormalTok{main() \{}
  \NormalTok{ExampleMacro();}
  \KeywordTok{return} \DecValTok{0}\NormalTok{;}
\NormalTok{\}}
\end{Highlighting}
\end{Shaded}

To create a stand-alone program from a macro called
\texttt{ExampleMacro.C}, simply type

\begin{Shaded}
\begin{Highlighting}[]
 \KeywordTok{>} \KeywordTok{g++} \NormalTok{-o ExampleMacro ExampleMacro.C }\StringTok{'root-config --cflags --libs'}
\end{Highlighting}
\end{Shaded}

and execute it by typing:

\begin{Shaded}
\begin{Highlighting}[]
 \KeywordTok{>} \KeywordTok{./ExampleMacro}
\end{Highlighting}
\end{Shaded}

This procedure will, however, not give access to the ROOT graphics, as
neither control of mouse or keyboard events nor access to the graphics
windows of ROOT is available. If you want your stand-alone application
have display graphics output and respond to mouse and keyboard, a
slightly more complex piece of code can be used. In the example below, a
macro \texttt{ExampleMacro\_GUI} is executed by the ROOT class
\href{https://root.cern.ch/doc/master/classTApplication.html}{\texttt{TApplication}}.
As a additional feature, this code example offers access to parameters
eventually passed to the program when started from the command line.
Here is the code fragment:

    \begin{Verbatim}[commandchars=\\\{\}]
{\color{incolor}In [{\color{incolor}4}]:} \PY{o}{\PYZpc{}}\PY{o}{\PYZpc{}}\PY{n}{cpp} \PY{o}{\PYZhy{}}\PY{n}{d}
        \PY{c+cm}{/*}
        \PY{c+cm}{ This piece of code demonstrates how a root macro is used as a standalone}
        \PY{c+cm}{    application with full acces the grapical user interface (GUI) of ROOT   */}
        
        \PY{c+c1}{//     ==\PYZgt{}\PYZgt{}  put the code of your macro here}
        \PY{k+kt}{void} \PY{n}{ExampleMacro\PYZus{}GUI}\PY{p}{(}\PY{p}{)} \PY{p}{\PYZob{}}
          \PY{c+c1}{// Create a histogram, fill it with random gaussian numbers}
          \PY{n}{TH1F} \PY{o}{*}\PY{n}{histogram\PYZus{}3\PYZus{}1} \PY{o}{=} \PY{k}{new} \PY{n}{TH1F} \PY{p}{(}\PY{l+s}{\PYZdq{}}\PY{l+s}{histogram\PYZus{}3\PYZus{}1}\PY{l+s}{\PYZdq{}}\PY{p}{,} \PY{l+s}{\PYZdq{}}\PY{l+s}{example histogram}\PY{l+s}{\PYZdq{}}\PY{p}{,} \PY{l+m+mi}{100}\PY{p}{,} \PY{o}{\PYZhy{}}\PY{l+m+mf}{5.}\PY{p}{,}\PY{l+m+mf}{5.}\PY{p}{)}\PY{p}{;}
          \PY{n}{histogram\PYZus{}3\PYZus{}1}\PY{o}{\PYZhy{}}\PY{o}{\PYZgt{}}\PY{n}{FillRandom}\PY{p}{(}\PY{l+s}{\PYZdq{}}\PY{l+s}{gaus}\PY{l+s}{\PYZdq{}}\PY{p}{,}\PY{l+m+mi}{1000}\PY{p}{)}\PY{p}{;}
          
          \PY{k}{auto}  \PY{n}{mycanvas} \PY{o}{=} \PY{k}{new} \PY{n}{TCanvas}\PY{p}{(}\PY{p}{)}\PY{p}{;}
          \PY{c+c1}{// draw the histogram}
          \PY{n}{histogram\PYZus{}3\PYZus{}1}\PY{o}{\PYZhy{}}\PY{o}{\PYZgt{}}\PY{n}{DrawClone}\PY{p}{(}\PY{p}{)}\PY{p}{;}
          
        \PY{c+cm}{/* \PYZhy{} Create a new ROOT file for output}
        \PY{c+cm}{   \PYZhy{} Note that this file may contain any kind of ROOT objects, histograms,}
        \PY{c+cm}{     pictures, graphics objects etc.}
        \PY{c+cm}{   \PYZhy{} the new file is now becoming the current directory */}
          \PY{n}{TFile} \PY{o}{*}\PY{n}{file\PYZus{}3\PYZus{}1} \PY{o}{=} \PY{k}{new} \PY{n}{TFile}\PY{p}{(}\PY{l+s}{\PYZdq{}}\PY{l+s}{ExampleMacro.root}\PY{l+s}{\PYZdq{}}\PY{p}{,}\PY{l+s}{\PYZdq{}}\PY{l+s}{RECREATE}\PY{l+s}{\PYZdq{}}\PY{p}{,}\PY{l+s}{\PYZdq{}}\PY{l+s}{ExampleMacro}\PY{l+s}{\PYZdq{}}\PY{p}{)}\PY{p}{;}
        
          \PY{c+c1}{// write Histogram to current directory (i.e. the file just opened)}
          \PY{n}{histogram\PYZus{}3\PYZus{}1}\PY{o}{\PYZhy{}}\PY{o}{\PYZgt{}}\PY{n}{Write}\PY{p}{(}\PY{p}{)}\PY{p}{;}
        
          \PY{c+c1}{// Close the file.}
          \PY{c+c1}{//   (You may inspect your histogram in the file using the TBrowser class)}
          \PY{n}{file\PYZus{}3\PYZus{}1}\PY{o}{\PYZhy{}}\PY{o}{\PYZgt{}}\PY{n}{Close}\PY{p}{(}\PY{p}{)}\PY{p}{;}
          
          \PY{n}{mycanvas}\PY{o}{\PYZhy{}}\PY{o}{\PYZgt{}}\PY{n}{Draw}\PY{p}{(}\PY{p}{)}\PY{p}{;}
        \PY{p}{\PYZcb{}}
        
        
        \PY{k+kt}{void} \PY{n}{StandaloneApplication}\PY{p}{(}\PY{p}{)} \PY{p}{\PYZob{}}
          \PY{c+c1}{// ==\PYZgt{}\PYZgt{} this application calls the ROOT macro}
          \PY{n}{ExampleMacro\PYZus{}GUI}\PY{p}{(}\PY{p}{)}\PY{p}{;}
        \PY{p}{\PYZcb{}}
\end{Verbatim}

    \begin{Verbatim}[commandchars=\\\{\}]
{\color{incolor}In [{\color{incolor}5}]:} \PY{n}{StandaloneApplication}\PY{p}{(}\PY{p}{)}\PY{p}{;}
\end{Verbatim}

    \begin{center}
    \adjustimage{max size={0.9\linewidth}{0.9\paperheight}}{3-ROOT-Macros_files/3-ROOT-Macros_8_0.png}
    \end{center}
    { \hspace*{\fill} \\}
    
    Compile the code with:

\begin{verbatim}
g++ -o ExampleMacro_GUI ExampleMacro_GUI 'root-config --cflags --libs'
\end{verbatim}

and execute the program with

\begin{verbatim}
> ./ExampleMacro_GUI
\end{verbatim}


    % Add a bibliography block to the postdoc
    
    
    
    \end{document}
