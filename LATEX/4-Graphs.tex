
% Default to the notebook output style

    


% Inherit from the specified cell style.




    
\documentclass{article}

    
    
    \usepackage{graphicx} % Used to insert images
    \usepackage{adjustbox} % Used to constrain images to a maximum size 
    \usepackage{color} % Allow colors to be defined
    \usepackage{enumerate} % Needed for markdown enumerations to work
    \usepackage{geometry} % Used to adjust the document margins
    \usepackage{amsmath} % Equations
    \usepackage{amssymb} % Equations
    \usepackage{eurosym} % defines \euro
    \usepackage[mathletters]{ucs} % Extended unicode (utf-8) support
    \usepackage[utf8x]{inputenc} % Allow utf-8 characters in the tex document
    \usepackage{fancyvrb} % verbatim replacement that allows latex
    \usepackage{grffile} % extends the file name processing of package graphics 
                         % to support a larger range 
    % The hyperref package gives us a pdf with properly built
    % internal navigation ('pdf bookmarks' for the table of contents,
    % internal cross-reference links, web links for URLs, etc.)
    \usepackage{hyperref}
    \usepackage{longtable} % longtable support required by pandoc >1.10
    \usepackage{booktabs}  % table support for pandoc > 1.12.2
    \usepackage{ulem} % ulem is needed to support strikethroughs (\sout)
    

    
    
    \definecolor{orange}{cmyk}{0,0.4,0.8,0.2}
    \definecolor{darkorange}{rgb}{.71,0.21,0.01}
    \definecolor{darkgreen}{rgb}{.12,.54,.11}
    \definecolor{myteal}{rgb}{.26, .44, .56}
    \definecolor{gray}{gray}{0.45}
    \definecolor{lightgray}{gray}{.95}
    \definecolor{mediumgray}{gray}{.8}
    \definecolor{inputbackground}{rgb}{.95, .95, .85}
    \definecolor{outputbackground}{rgb}{.95, .95, .95}
    \definecolor{traceback}{rgb}{1, .95, .95}
    % ansi colors
    \definecolor{red}{rgb}{.6,0,0}
    \definecolor{green}{rgb}{0,.65,0}
    \definecolor{brown}{rgb}{0.6,0.6,0}
    \definecolor{blue}{rgb}{0,.145,.698}
    \definecolor{purple}{rgb}{.698,.145,.698}
    \definecolor{cyan}{rgb}{0,.698,.698}
    \definecolor{lightgray}{gray}{0.5}
    
    % bright ansi colors
    \definecolor{darkgray}{gray}{0.25}
    \definecolor{lightred}{rgb}{1.0,0.39,0.28}
    \definecolor{lightgreen}{rgb}{0.48,0.99,0.0}
    \definecolor{lightblue}{rgb}{0.53,0.81,0.92}
    \definecolor{lightpurple}{rgb}{0.87,0.63,0.87}
    \definecolor{lightcyan}{rgb}{0.5,1.0,0.83}
    
    % commands and environments needed by pandoc snippets
    % extracted from the output of `pandoc -s`
    \providecommand{\tightlist}{%
      \setlength{\itemsep}{0pt}\setlength{\parskip}{0pt}}
    \DefineVerbatimEnvironment{Highlighting}{Verbatim}{commandchars=\\\{\}}
    % Add ',fontsize=\small' for more characters per line
    \newenvironment{Shaded}{}{}
    \newcommand{\KeywordTok}[1]{\textcolor[rgb]{0.00,0.44,0.13}{\textbf{{#1}}}}
    \newcommand{\DataTypeTok}[1]{\textcolor[rgb]{0.56,0.13,0.00}{{#1}}}
    \newcommand{\DecValTok}[1]{\textcolor[rgb]{0.25,0.63,0.44}{{#1}}}
    \newcommand{\BaseNTok}[1]{\textcolor[rgb]{0.25,0.63,0.44}{{#1}}}
    \newcommand{\FloatTok}[1]{\textcolor[rgb]{0.25,0.63,0.44}{{#1}}}
    \newcommand{\CharTok}[1]{\textcolor[rgb]{0.25,0.44,0.63}{{#1}}}
    \newcommand{\StringTok}[1]{\textcolor[rgb]{0.25,0.44,0.63}{{#1}}}
    \newcommand{\CommentTok}[1]{\textcolor[rgb]{0.38,0.63,0.69}{\textit{{#1}}}}
    \newcommand{\OtherTok}[1]{\textcolor[rgb]{0.00,0.44,0.13}{{#1}}}
    \newcommand{\AlertTok}[1]{\textcolor[rgb]{1.00,0.00,0.00}{\textbf{{#1}}}}
    \newcommand{\FunctionTok}[1]{\textcolor[rgb]{0.02,0.16,0.49}{{#1}}}
    \newcommand{\RegionMarkerTok}[1]{{#1}}
    \newcommand{\ErrorTok}[1]{\textcolor[rgb]{1.00,0.00,0.00}{\textbf{{#1}}}}
    \newcommand{\NormalTok}[1]{{#1}}
    
    % Additional commands for more recent versions of Pandoc
    \newcommand{\ConstantTok}[1]{\textcolor[rgb]{0.53,0.00,0.00}{{#1}}}
    \newcommand{\SpecialCharTok}[1]{\textcolor[rgb]{0.25,0.44,0.63}{{#1}}}
    \newcommand{\VerbatimStringTok}[1]{\textcolor[rgb]{0.25,0.44,0.63}{{#1}}}
    \newcommand{\SpecialStringTok}[1]{\textcolor[rgb]{0.73,0.40,0.53}{{#1}}}
    \newcommand{\ImportTok}[1]{{#1}}
    \newcommand{\DocumentationTok}[1]{\textcolor[rgb]{0.73,0.13,0.13}{\textit{{#1}}}}
    \newcommand{\AnnotationTok}[1]{\textcolor[rgb]{0.38,0.63,0.69}{\textbf{\textit{{#1}}}}}
    \newcommand{\CommentVarTok}[1]{\textcolor[rgb]{0.38,0.63,0.69}{\textbf{\textit{{#1}}}}}
    \newcommand{\VariableTok}[1]{\textcolor[rgb]{0.10,0.09,0.49}{{#1}}}
    \newcommand{\ControlFlowTok}[1]{\textcolor[rgb]{0.00,0.44,0.13}{\textbf{{#1}}}}
    \newcommand{\OperatorTok}[1]{\textcolor[rgb]{0.40,0.40,0.40}{{#1}}}
    \newcommand{\BuiltInTok}[1]{{#1}}
    \newcommand{\ExtensionTok}[1]{{#1}}
    \newcommand{\PreprocessorTok}[1]{\textcolor[rgb]{0.74,0.48,0.00}{{#1}}}
    \newcommand{\AttributeTok}[1]{\textcolor[rgb]{0.49,0.56,0.16}{{#1}}}
    \newcommand{\InformationTok}[1]{\textcolor[rgb]{0.38,0.63,0.69}{\textbf{\textit{{#1}}}}}
    \newcommand{\WarningTok}[1]{\textcolor[rgb]{0.38,0.63,0.69}{\textbf{\textit{{#1}}}}}
    
    
    % Define a nice break command that doesn't care if a line doesn't already
    % exist.
    \def\br{\hspace*{\fill} \\* }
    % Math Jax compatability definitions
    \def\gt{>}
    \def\lt{<}
    % Document parameters
    \title{4-Graphs}
    
    
    

    % Pygments definitions
    
\makeatletter
\def\PY@reset{\let\PY@it=\relax \let\PY@bf=\relax%
    \let\PY@ul=\relax \let\PY@tc=\relax%
    \let\PY@bc=\relax \let\PY@ff=\relax}
\def\PY@tok#1{\csname PY@tok@#1\endcsname}
\def\PY@toks#1+{\ifx\relax#1\empty\else%
    \PY@tok{#1}\expandafter\PY@toks\fi}
\def\PY@do#1{\PY@bc{\PY@tc{\PY@ul{%
    \PY@it{\PY@bf{\PY@ff{#1}}}}}}}
\def\PY#1#2{\PY@reset\PY@toks#1+\relax+\PY@do{#2}}

\expandafter\def\csname PY@tok@nd\endcsname{\def\PY@tc##1{\textcolor[rgb]{0.67,0.13,1.00}{##1}}}
\expandafter\def\csname PY@tok@mb\endcsname{\def\PY@tc##1{\textcolor[rgb]{0.40,0.40,0.40}{##1}}}
\expandafter\def\csname PY@tok@gs\endcsname{\let\PY@bf=\textbf}
\expandafter\def\csname PY@tok@nb\endcsname{\def\PY@tc##1{\textcolor[rgb]{0.00,0.50,0.00}{##1}}}
\expandafter\def\csname PY@tok@mf\endcsname{\def\PY@tc##1{\textcolor[rgb]{0.40,0.40,0.40}{##1}}}
\expandafter\def\csname PY@tok@bp\endcsname{\def\PY@tc##1{\textcolor[rgb]{0.00,0.50,0.00}{##1}}}
\expandafter\def\csname PY@tok@gh\endcsname{\let\PY@bf=\textbf\def\PY@tc##1{\textcolor[rgb]{0.00,0.00,0.50}{##1}}}
\expandafter\def\csname PY@tok@si\endcsname{\let\PY@bf=\textbf\def\PY@tc##1{\textcolor[rgb]{0.73,0.40,0.53}{##1}}}
\expandafter\def\csname PY@tok@gt\endcsname{\def\PY@tc##1{\textcolor[rgb]{0.00,0.27,0.87}{##1}}}
\expandafter\def\csname PY@tok@s\endcsname{\def\PY@tc##1{\textcolor[rgb]{0.73,0.13,0.13}{##1}}}
\expandafter\def\csname PY@tok@gu\endcsname{\let\PY@bf=\textbf\def\PY@tc##1{\textcolor[rgb]{0.50,0.00,0.50}{##1}}}
\expandafter\def\csname PY@tok@ge\endcsname{\let\PY@it=\textit}
\expandafter\def\csname PY@tok@nt\endcsname{\let\PY@bf=\textbf\def\PY@tc##1{\textcolor[rgb]{0.00,0.50,0.00}{##1}}}
\expandafter\def\csname PY@tok@kr\endcsname{\let\PY@bf=\textbf\def\PY@tc##1{\textcolor[rgb]{0.00,0.50,0.00}{##1}}}
\expandafter\def\csname PY@tok@cpf\endcsname{\let\PY@it=\textit\def\PY@tc##1{\textcolor[rgb]{0.25,0.50,0.50}{##1}}}
\expandafter\def\csname PY@tok@vi\endcsname{\def\PY@tc##1{\textcolor[rgb]{0.10,0.09,0.49}{##1}}}
\expandafter\def\csname PY@tok@sx\endcsname{\def\PY@tc##1{\textcolor[rgb]{0.00,0.50,0.00}{##1}}}
\expandafter\def\csname PY@tok@nc\endcsname{\let\PY@bf=\textbf\def\PY@tc##1{\textcolor[rgb]{0.00,0.00,1.00}{##1}}}
\expandafter\def\csname PY@tok@s1\endcsname{\def\PY@tc##1{\textcolor[rgb]{0.73,0.13,0.13}{##1}}}
\expandafter\def\csname PY@tok@sc\endcsname{\def\PY@tc##1{\textcolor[rgb]{0.73,0.13,0.13}{##1}}}
\expandafter\def\csname PY@tok@sr\endcsname{\def\PY@tc##1{\textcolor[rgb]{0.73,0.40,0.53}{##1}}}
\expandafter\def\csname PY@tok@nn\endcsname{\let\PY@bf=\textbf\def\PY@tc##1{\textcolor[rgb]{0.00,0.00,1.00}{##1}}}
\expandafter\def\csname PY@tok@gp\endcsname{\let\PY@bf=\textbf\def\PY@tc##1{\textcolor[rgb]{0.00,0.00,0.50}{##1}}}
\expandafter\def\csname PY@tok@cm\endcsname{\let\PY@it=\textit\def\PY@tc##1{\textcolor[rgb]{0.25,0.50,0.50}{##1}}}
\expandafter\def\csname PY@tok@kn\endcsname{\let\PY@bf=\textbf\def\PY@tc##1{\textcolor[rgb]{0.00,0.50,0.00}{##1}}}
\expandafter\def\csname PY@tok@kc\endcsname{\let\PY@bf=\textbf\def\PY@tc##1{\textcolor[rgb]{0.00,0.50,0.00}{##1}}}
\expandafter\def\csname PY@tok@mo\endcsname{\def\PY@tc##1{\textcolor[rgb]{0.40,0.40,0.40}{##1}}}
\expandafter\def\csname PY@tok@cs\endcsname{\let\PY@it=\textit\def\PY@tc##1{\textcolor[rgb]{0.25,0.50,0.50}{##1}}}
\expandafter\def\csname PY@tok@na\endcsname{\def\PY@tc##1{\textcolor[rgb]{0.49,0.56,0.16}{##1}}}
\expandafter\def\csname PY@tok@vc\endcsname{\def\PY@tc##1{\textcolor[rgb]{0.10,0.09,0.49}{##1}}}
\expandafter\def\csname PY@tok@nl\endcsname{\def\PY@tc##1{\textcolor[rgb]{0.63,0.63,0.00}{##1}}}
\expandafter\def\csname PY@tok@ow\endcsname{\let\PY@bf=\textbf\def\PY@tc##1{\textcolor[rgb]{0.67,0.13,1.00}{##1}}}
\expandafter\def\csname PY@tok@sd\endcsname{\let\PY@it=\textit\def\PY@tc##1{\textcolor[rgb]{0.73,0.13,0.13}{##1}}}
\expandafter\def\csname PY@tok@gd\endcsname{\def\PY@tc##1{\textcolor[rgb]{0.63,0.00,0.00}{##1}}}
\expandafter\def\csname PY@tok@c1\endcsname{\let\PY@it=\textit\def\PY@tc##1{\textcolor[rgb]{0.25,0.50,0.50}{##1}}}
\expandafter\def\csname PY@tok@kp\endcsname{\def\PY@tc##1{\textcolor[rgb]{0.00,0.50,0.00}{##1}}}
\expandafter\def\csname PY@tok@il\endcsname{\def\PY@tc##1{\textcolor[rgb]{0.40,0.40,0.40}{##1}}}
\expandafter\def\csname PY@tok@ni\endcsname{\let\PY@bf=\textbf\def\PY@tc##1{\textcolor[rgb]{0.60,0.60,0.60}{##1}}}
\expandafter\def\csname PY@tok@ss\endcsname{\def\PY@tc##1{\textcolor[rgb]{0.10,0.09,0.49}{##1}}}
\expandafter\def\csname PY@tok@c\endcsname{\let\PY@it=\textit\def\PY@tc##1{\textcolor[rgb]{0.25,0.50,0.50}{##1}}}
\expandafter\def\csname PY@tok@cp\endcsname{\def\PY@tc##1{\textcolor[rgb]{0.74,0.48,0.00}{##1}}}
\expandafter\def\csname PY@tok@o\endcsname{\def\PY@tc##1{\textcolor[rgb]{0.40,0.40,0.40}{##1}}}
\expandafter\def\csname PY@tok@kd\endcsname{\let\PY@bf=\textbf\def\PY@tc##1{\textcolor[rgb]{0.00,0.50,0.00}{##1}}}
\expandafter\def\csname PY@tok@go\endcsname{\def\PY@tc##1{\textcolor[rgb]{0.53,0.53,0.53}{##1}}}
\expandafter\def\csname PY@tok@kt\endcsname{\def\PY@tc##1{\textcolor[rgb]{0.69,0.00,0.25}{##1}}}
\expandafter\def\csname PY@tok@mi\endcsname{\def\PY@tc##1{\textcolor[rgb]{0.40,0.40,0.40}{##1}}}
\expandafter\def\csname PY@tok@no\endcsname{\def\PY@tc##1{\textcolor[rgb]{0.53,0.00,0.00}{##1}}}
\expandafter\def\csname PY@tok@ch\endcsname{\let\PY@it=\textit\def\PY@tc##1{\textcolor[rgb]{0.25,0.50,0.50}{##1}}}
\expandafter\def\csname PY@tok@ne\endcsname{\let\PY@bf=\textbf\def\PY@tc##1{\textcolor[rgb]{0.82,0.25,0.23}{##1}}}
\expandafter\def\csname PY@tok@gi\endcsname{\def\PY@tc##1{\textcolor[rgb]{0.00,0.63,0.00}{##1}}}
\expandafter\def\csname PY@tok@w\endcsname{\def\PY@tc##1{\textcolor[rgb]{0.73,0.73,0.73}{##1}}}
\expandafter\def\csname PY@tok@se\endcsname{\let\PY@bf=\textbf\def\PY@tc##1{\textcolor[rgb]{0.73,0.40,0.13}{##1}}}
\expandafter\def\csname PY@tok@s2\endcsname{\def\PY@tc##1{\textcolor[rgb]{0.73,0.13,0.13}{##1}}}
\expandafter\def\csname PY@tok@nv\endcsname{\def\PY@tc##1{\textcolor[rgb]{0.10,0.09,0.49}{##1}}}
\expandafter\def\csname PY@tok@m\endcsname{\def\PY@tc##1{\textcolor[rgb]{0.40,0.40,0.40}{##1}}}
\expandafter\def\csname PY@tok@k\endcsname{\let\PY@bf=\textbf\def\PY@tc##1{\textcolor[rgb]{0.00,0.50,0.00}{##1}}}
\expandafter\def\csname PY@tok@mh\endcsname{\def\PY@tc##1{\textcolor[rgb]{0.40,0.40,0.40}{##1}}}
\expandafter\def\csname PY@tok@gr\endcsname{\def\PY@tc##1{\textcolor[rgb]{1.00,0.00,0.00}{##1}}}
\expandafter\def\csname PY@tok@sb\endcsname{\def\PY@tc##1{\textcolor[rgb]{0.73,0.13,0.13}{##1}}}
\expandafter\def\csname PY@tok@sh\endcsname{\def\PY@tc##1{\textcolor[rgb]{0.73,0.13,0.13}{##1}}}
\expandafter\def\csname PY@tok@vg\endcsname{\def\PY@tc##1{\textcolor[rgb]{0.10,0.09,0.49}{##1}}}
\expandafter\def\csname PY@tok@nf\endcsname{\def\PY@tc##1{\textcolor[rgb]{0.00,0.00,1.00}{##1}}}
\expandafter\def\csname PY@tok@err\endcsname{\def\PY@bc##1{\setlength{\fboxsep}{0pt}\fcolorbox[rgb]{1.00,0.00,0.00}{1,1,1}{\strut ##1}}}

\def\PYZbs{\char`\\}
\def\PYZus{\char`\_}
\def\PYZob{\char`\{}
\def\PYZcb{\char`\}}
\def\PYZca{\char`\^}
\def\PYZam{\char`\&}
\def\PYZlt{\char`\<}
\def\PYZgt{\char`\>}
\def\PYZsh{\char`\#}
\def\PYZpc{\char`\%}
\def\PYZdl{\char`\$}
\def\PYZhy{\char`\-}
\def\PYZsq{\char`\'}
\def\PYZdq{\char`\"}
\def\PYZti{\char`\~}
% for compatibility with earlier versions
\def\PYZat{@}
\def\PYZlb{[}
\def\PYZrb{]}
\makeatother


    % Exact colors from NB
    \definecolor{incolor}{rgb}{0.0, 0.0, 0.5}
    \definecolor{outcolor}{rgb}{0.545, 0.0, 0.0}



    
    % Prevent overflowing lines due to hard-to-break entities
    \sloppy 
    % Setup hyperref package
    \hypersetup{
      breaklinks=true,  % so long urls are correctly broken across lines
      colorlinks=true,
      urlcolor=blue,
      linkcolor=darkorange,
      citecolor=darkgreen,
      }
    % Slightly bigger margins than the latex defaults
    
    \geometry{verbose,tmargin=1in,bmargin=1in,lmargin=1in,rmargin=1in}
    
    

    \begin{document}
    
    
    \maketitle
    
    

    
    In this Chapter we will learn how to exploit some of the functionalities
ROOT provides to display data exploiting the class
\href{https://root.cern.ch/doc/master/classTGraphErrors.html}{\texttt{TGraphErrors}},
which you already got to know previously.

\subsection{4.1 Read Graph Points from
File}\label{read-graph-points-from-file}

The fastest way in which you can fill a graph with experimental data is
to use the constructor which reads data points and their errors from an
ASCII file (i.e.~standard text) format:

\begin{verbatim}
TGraphErrors(const char *filename, const char *format="%lg %lg %lg %lg", Option_t *option="");
\end{verbatim}

The format string can be:

\begin{itemize}
\item
  ``\%lg \%lg'' read only 2 first columns into X,Y
\item
  ``\%lg \%lg \%lg'' read only 3 first columns into X,Y and EY
\item
  ``\%lg \%lg \%lg \%lg'' read only 4 first columns into X,Y,EX,EY
\end{itemize}

This approach has the nice feature of allowing the user to reuse the
macro for many different data sets. Here is an example of an input file.
The nice graphic result shown is produced by the macro below, which
reads two such input files and uses different options to display the
data points.

``` \# Measurement of Friday 26 March \# Experiment 2 Physics Lab

1 6 5 2 12 5 3 14 4.7 4 20 4.5 5 22 4.2 6 24 5.1 7 35 2.9 8 45 4.1 9 44
4.8 10 53 5.43 ```

    \begin{Verbatim}[commandchars=\\\{\}]
{\color{incolor}In [{\color{incolor}1}]:} \PY{o}{\PYZpc{}}\PY{o}{\PYZpc{}}\PY{n}{jsroot} \PY{n}{on}
\end{Verbatim}

    \begin{Verbatim}[commandchars=\\\{\}]
{\color{incolor}In [{\color{incolor}2}]:} \PY{o}{\PYZpc{}}\PY{o}{\PYZpc{}}\PY{n}{cpp} \PY{o}{\PYZhy{}}\PY{n}{d}
        \PY{c+c1}{// Reads the points from a file and produces a simple graph.}
        \PY{k+kt}{int} \PY{n}{macro\PYZus{}4\PYZus{}1}\PY{p}{(}\PY{p}{)}\PY{p}{\PYZob{}}
            \PY{k}{auto} \PY{n}{canvas\PYZus{}4\PYZus{}1}\PY{o}{=}\PY{k}{new} \PY{n}{TCanvas}\PY{p}{(}\PY{p}{)}\PY{p}{;}
            \PY{n}{canvas\PYZus{}4\PYZus{}1}\PY{o}{\PYZhy{}}\PY{o}{\PYZgt{}}\PY{n}{SetGrid}\PY{p}{(}\PY{p}{)}\PY{p}{;}
            
            \PY{n}{TGraphErrors} \PY{n+nf}{graph\PYZus{}expected}\PY{p}{(}\PY{l+s}{\PYZdq{}}\PY{l+s}{../data/macro4\PYZus{}1\PYZus{}input\PYZus{}expected.txt}\PY{l+s}{\PYZdq{}}\PY{p}{,} \PY{l+s}{\PYZdq{}}\PY{l+s}{\PYZpc{}lg \PYZpc{}lg \PYZpc{}lg}\PY{l+s}{\PYZdq{}}\PY{p}{)}\PY{p}{;}
            \PY{n}{graph\PYZus{}expected}\PY{p}{.}\PY{n}{SetTitle}\PY{p}{(}
               \PY{l+s}{\PYZdq{}}\PY{l+s}{Measurement XYZ and Expectation;}\PY{l+s}{\PYZdq{}}
               \PY{l+s}{\PYZdq{}}\PY{l+s}{lenght [cm];}\PY{l+s}{\PYZdq{}}
               \PY{l+s}{\PYZdq{}}\PY{l+s}{Arb.Units}\PY{l+s}{\PYZdq{}}\PY{p}{)}\PY{p}{;}
            \PY{n}{graph\PYZus{}expected}\PY{p}{.}\PY{n}{SetFillColor}\PY{p}{(}\PY{n}{kYellow}\PY{p}{)}\PY{p}{;}
            \PY{n}{graph\PYZus{}expected}\PY{p}{.}\PY{n}{DrawClone}\PY{p}{(}\PY{l+s}{\PYZdq{}}\PY{l+s}{E3AL}\PY{l+s}{\PYZdq{}}\PY{p}{)}\PY{p}{;} \PY{c+c1}{// E3 draws the band}
        
            \PY{n}{TGraphErrors} \PY{n+nf}{graph}\PY{p}{(}\PY{l+s}{\PYZdq{}}\PY{l+s}{../data/macro4\PYZus{}1\PYZus{}input.txt}\PY{l+s}{\PYZdq{}}\PY{p}{,}\PY{l+s}{\PYZdq{}}\PY{l+s}{\PYZpc{}lg \PYZpc{}lg \PYZpc{}lg}\PY{l+s}{\PYZdq{}}\PY{p}{)}\PY{p}{;}
            \PY{n}{graph}\PY{p}{.}\PY{n}{SetMarkerStyle}\PY{p}{(}\PY{n}{kCircle}\PY{p}{)}\PY{p}{;}
            \PY{n}{graph}\PY{p}{.}\PY{n}{SetFillColor}\PY{p}{(}\PY{l+m+mi}{0}\PY{p}{)}\PY{p}{;}
            \PY{n}{graph}\PY{p}{.}\PY{n}{DrawClone}\PY{p}{(}\PY{l+s}{\PYZdq{}}\PY{l+s}{PESame}\PY{l+s}{\PYZdq{}}\PY{p}{)}\PY{p}{;}
        
            \PY{c+c1}{// Draw the Legend}
            \PY{n}{TLegend} \PY{n+nf}{leg}\PY{p}{(}\PY{l+m+mf}{.1}\PY{p}{,}\PY{l+m+mf}{.7}\PY{p}{,}\PY{l+m+mf}{.3}\PY{p}{,}\PY{l+m+mf}{.9}\PY{p}{,}\PY{l+s}{\PYZdq{}}\PY{l+s}{Lab. Lesson 2}\PY{l+s}{\PYZdq{}}\PY{p}{)}\PY{p}{;}
            \PY{n}{leg}\PY{p}{.}\PY{n}{SetFillColor}\PY{p}{(}\PY{l+m+mi}{0}\PY{p}{)}\PY{p}{;}
            \PY{n}{leg}\PY{p}{.}\PY{n}{AddEntry}\PY{p}{(}\PY{o}{\PYZam{}}\PY{n}{graph\PYZus{}expected}\PY{p}{,}\PY{l+s}{\PYZdq{}}\PY{l+s}{Expected Points}\PY{l+s}{\PYZdq{}}\PY{p}{)}\PY{p}{;}
            \PY{n}{leg}\PY{p}{.}\PY{n}{AddEntry}\PY{p}{(}\PY{o}{\PYZam{}}\PY{n}{graph}\PY{p}{,}\PY{l+s}{\PYZdq{}}\PY{l+s}{Measured Points}\PY{l+s}{\PYZdq{}}\PY{p}{)}\PY{p}{;}
            \PY{n}{leg}\PY{p}{.}\PY{n}{DrawClone}\PY{p}{(}\PY{l+s}{\PYZdq{}}\PY{l+s}{Same}\PY{l+s}{\PYZdq{}}\PY{p}{)}\PY{p}{;}
        
            \PY{n}{graph}\PY{p}{.}\PY{n}{Print}\PY{p}{(}\PY{p}{)}\PY{p}{;}
            \PY{n}{canvas\PYZus{}4\PYZus{}1}\PY{o}{\PYZhy{}}\PY{o}{\PYZgt{}}\PY{n}{Draw}\PY{p}{(}\PY{p}{)}\PY{p}{;}
            \PY{k}{return} \PY{l+m+mi}{0}\PY{p}{;}
        \PY{p}{\PYZcb{}}
\end{Verbatim}

    \begin{Verbatim}[commandchars=\\\{\}]
{\color{incolor}In [{\color{incolor}3}]:} \PY{n}{macro\PYZus{}4\PYZus{}1}\PY{p}{(}\PY{p}{)}\PY{p}{;}
\end{Verbatim}

    
    \begin{verbatim}
<IPython.core.display.HTML object>
    \end{verbatim}

    
    \begin{Verbatim}[commandchars=\\\{\}]
x[0]=1, y[0]=6, ex[0]=0, ey[0]=5
x[1]=2, y[1]=12, ex[1]=0, ey[1]=5
x[2]=3, y[2]=14, ex[2]=0, ey[2]=4.7
x[3]=4, y[3]=20, ex[3]=0, ey[3]=4.5
x[4]=5, y[4]=22, ex[4]=0, ey[4]=4.2
x[5]=6, y[5]=24, ex[5]=0, ey[5]=5.1
x[6]=7, y[6]=35, ex[6]=0, ey[6]=2.9
x[7]=8, y[7]=45, ex[7]=0, ey[7]=4.1
x[8]=9, y[8]=44, ex[8]=0, ey[8]=4.8
x[9]=10, y[9]=53, ex[9]=0, ey[9]=5.43
    \end{Verbatim}

    In addition to the inspection of the plot, you can check the actual
contents of the graph with the
\href{https://root.cern.ch/doc/master/classTGraph.html\#aaa3ad04cb8017339ac8e96543ba3b5fb}{\texttt{TGraph::Print()}}
method at any time, obtaining a printout of the coordinates of data
points on screen. The macro also shows us how to print a coloured band
around a graph instead of error bars, quite useful for example to
represent the errors of a theoretical prediction.

\subsection{4.2 Polar Graphs}\label{polar-graphs}

With ROOT you can profit from rather advanced plotting routines, like
the ones implemented in the
\href{https://root.cern.ch/doc/v606/classTGraphPolar.html}{\texttt{TPolarGraph}},
a class to draw graphs in polar coordinates. You can see the example
macro in the following:

    \begin{Verbatim}[commandchars=\\\{\}]
{\color{incolor}In [{\color{incolor}4}]:} \PY{k}{auto} \PY{n}{canvas\PYZus{}4\PYZus{}2} \PY{o}{=} \PY{k}{new} \PY{n}{TCanvas}\PY{p}{(}\PY{l+s}{\PYZdq{}}\PY{l+s}{myCanvas}\PY{l+s}{\PYZdq{}}\PY{p}{,}\PY{l+s}{\PYZdq{}}\PY{l+s}{myCanvas}\PY{l+s}{\PYZdq{}}\PY{p}{,}\PY{l+m+mi}{600}\PY{p}{,}\PY{l+m+mi}{600}\PY{p}{)}\PY{p}{;}
        \PY{n}{Double\PYZus{}t} \PY{n}{rmin}\PY{o}{=}\PY{l+m+mf}{0.}\PY{p}{;}
        \PY{n}{Double\PYZus{}t} \PY{n}{rmax}\PY{o}{=}\PY{n}{TMath}\PY{o}{:}\PY{o}{:}\PY{n}{Pi}\PY{p}{(}\PY{p}{)}\PY{o}{*}\PY{l+m+mf}{6.}\PY{p}{;}
        \PY{k}{const} \PY{n}{Int\PYZus{}t} \PY{n}{npoints}\PY{o}{=}\PY{l+m+mi}{1000}\PY{p}{;}
        \PY{n}{Double\PYZus{}t} \PY{n}{r}\PY{p}{[}\PY{n}{npoints}\PY{p}{]}\PY{p}{;}
        \PY{n}{Double\PYZus{}t} \PY{n}{theta}\PY{p}{[}\PY{n}{npoints}\PY{p}{]}\PY{p}{;}
        \PY{k}{for} \PY{p}{(}\PY{n}{Int\PYZus{}t} \PY{n}{ipt} \PY{o}{=} \PY{l+m+mi}{0}\PY{p}{;} \PY{n}{ipt} \PY{o}{\PYZlt{}} \PY{n}{npoints}\PY{p}{;} \PY{n}{ipt}\PY{o}{+}\PY{o}{+}\PY{p}{)} \PY{p}{\PYZob{}}
            \PY{n}{r}\PY{p}{[}\PY{n}{ipt}\PY{p}{]} \PY{o}{=} \PY{n}{ipt}\PY{o}{*}\PY{p}{(}\PY{n}{rmax}\PY{o}{\PYZhy{}}\PY{n}{rmin}\PY{p}{)}\PY{o}{/}\PY{n}{npoints}\PY{o}{+}\PY{n}{rmin}\PY{p}{;}
            \PY{n}{theta}\PY{p}{[}\PY{n}{ipt}\PY{p}{]} \PY{o}{=} \PY{n}{TMath}\PY{o}{:}\PY{o}{:}\PY{n}{Sin}\PY{p}{(}\PY{n}{r}\PY{p}{[}\PY{n}{ipt}\PY{p}{]}\PY{p}{)}\PY{p}{;}
        \PY{p}{\PYZcb{}}
        \PY{n}{TGraphPolar} \PY{n}{grP1} \PY{p}{(}\PY{n}{npoints}\PY{p}{,}\PY{n}{r}\PY{p}{,}\PY{n}{theta}\PY{p}{)}\PY{p}{;}
        \PY{n}{grP1}\PY{p}{.}\PY{n}{SetTitle}\PY{p}{(}\PY{l+s}{\PYZdq{}}\PY{l+s}{A Fan}\PY{l+s}{\PYZdq{}}\PY{p}{)}\PY{p}{;}
        \PY{n}{grP1}\PY{p}{.}\PY{n}{SetLineWidth}\PY{p}{(}\PY{l+m+mi}{3}\PY{p}{)}\PY{p}{;}
        \PY{n}{grP1}\PY{p}{.}\PY{n}{SetLineColor}\PY{p}{(}\PY{l+m+mi}{2}\PY{p}{)}\PY{p}{;}
        \PY{n}{grP1}\PY{p}{.}\PY{n}{DrawClone}\PY{p}{(}\PY{l+s}{\PYZdq{}}\PY{l+s}{L}\PY{l+s}{\PYZdq{}}\PY{p}{)}\PY{p}{;}
        \PY{n}{canvas\PYZus{}4\PYZus{}2}\PY{o}{\PYZhy{}}\PY{o}{\PYZgt{}}\PY{n}{Draw}\PY{p}{(}\PY{p}{)}\PY{p}{;}
\end{Verbatim}

    \begin{center}
    \adjustimage{max size={0.9\linewidth}{0.9\paperheight}}{4-Graphs_files/4-Graphs_5_0.png}
    \end{center}
    { \hspace*{\fill} \\}
    
    A new element was added on the canvas declaration, the size of the
canvas: it is sometimes optically better to show plots in specific
canvas sizes.

    \subsection{4.3 2D Graphs}\label{d-graphs}

Under specific circumstances, it might be useful to plot some quantities
versus two variables, therefore creating a bi-dimensional graph. Of
course ROOT can help you in this task, with the
\href{https://root.cern.ch/doc/master/classTGraph2DErrors.html}{\texttt{TGraph2DErrors}}
class. The following macro produces a bi-dimensional graph representing
a hypothetical measurement, fits a bi-dimensional function to it and
draws it together with its x and y projections. Some points of the code
will be explained in detail. This time, the graph is populated with data
points using random numbers, introducing a new and very important
ingredient, the ROOT \texttt{TRandom3} random number generator using the
\href{http://www.math.sci.hiroshima-u.ac.jp/~m-mat/MT/emt.html}{Mersenne
Twister algorithm (Matsumoto 1997)}.

Let's go through the code, step by step to understand what is going on:

    \begin{itemize}
\tightlist
\item
  The instance of the random generator. You can then draw out of this
  instance random numbers distributed according to different probability
  density functions, like the Uniform one at point 2. See the on-line
  documentation to appreciate the full power of this ROOT feature.
\end{itemize}

    \begin{Verbatim}[commandchars=\\\{\}]
{\color{incolor}In [{\color{incolor}5}]:} \PY{c+c1}{//\PYZsh{}1}
        \PY{n}{TRandom3} \PY{n}{my\PYZus{}random\PYZus{}generator}\PY{p}{;}
\end{Verbatim}

    \begin{itemize}
\tightlist
\item
  You are already familiar with the TF1 class. This is its
  two-dimensional version. At line 16 two random numbers distributed
  according to the TF2 formula are drawn with the method
  TF2::GetRandom2(double\& a, double\&b).
\end{itemize}

    \begin{Verbatim}[commandchars=\\\{\}]
{\color{incolor}In [{\color{incolor}6}]:} \PY{c+c1}{//\PYZsh{}2}
        \PY{n}{TF2} \PY{n+nf}{function\PYZus{}4\PYZus{}3}\PY{p}{(}\PY{l+s}{\PYZdq{}}\PY{l+s}{f2}\PY{l+s}{\PYZdq{}}\PY{p}{,}\PY{l+s}{\PYZdq{}}\PY{l+s}{1000*(([0]*sin(x)}\PY{o}{/}\PY{n}{x}\PY{p}{)}\PY{o}{*}\PY{p}{(}\PY{p}{[}\PY{l+m+mi}{1}\PY{p}{]}\PY{o}{*}\PY{n}{sin}\PY{p}{(}\PY{n}{y}\PY{p}{)}\PY{o}{/}\PY{n}{y}\PY{p}{)}\PY{p}{)}\PY{o}{+}\PY{l+m+mi}{2000}\PY{l+s}{\PYZdq{}}\PY{l+s}{,\PYZhy{}6,6,\PYZhy{}6,6)}\PY{p}{;}
        \PY{n}{function\PYZus{}4\PYZus{}3}\PY{p}{.}\PY{n}{SetParameters}\PY{p}{(}\PY{l+m+mi}{1}\PY{p}{,}\PY{l+m+mi}{1}\PY{p}{)}\PY{p}{;}
        \PY{n}{TGraph2DErrors} \PY{n+nf}{dte}\PY{p}{(}\PY{l+m+mi}{500}\PY{p}{)}\PY{p}{;}
        \PY{c+c1}{// Fill the 2D graph}
        \PY{k+kt}{double} \PY{n}{rnd}\PY{p}{,} \PY{n}{x}\PY{p}{,} \PY{n}{y}\PY{p}{,} \PY{n}{z}\PY{p}{,} \PY{n}{ex}\PY{p}{,} \PY{n}{ey}\PY{p}{,} \PY{n}{ez}\PY{p}{;}
        \PY{k}{for} \PY{p}{(}\PY{n}{Int\PYZus{}t} \PY{n}{i}\PY{o}{=}\PY{l+m+mi}{0}\PY{p}{;} \PY{n}{i}\PY{o}{\PYZlt{}}\PY{l+m+mi}{500}\PY{p}{;} \PY{n}{i}\PY{o}{+}\PY{o}{+}\PY{p}{)} \PY{p}{\PYZob{}}
            \PY{n}{function\PYZus{}4\PYZus{}3}\PY{p}{.}\PY{n}{GetRandom2}\PY{p}{(}\PY{n}{x}\PY{p}{,}\PY{n}{y}\PY{p}{)}\PY{p}{;}
            \PY{c+c1}{// A random number in [\PYZhy{}e,e]}
            \PY{n}{rnd} \PY{o}{=} \PY{n}{my\PYZus{}random\PYZus{}generator}\PY{p}{.}\PY{n}{Uniform}\PY{p}{(}\PY{o}{\PYZhy{}}\PY{l+m+mf}{0.3}\PY{p}{,}\PY{l+m+mf}{0.3}\PY{p}{)}\PY{p}{;}
            \PY{n}{z} \PY{o}{=} \PY{n}{function\PYZus{}4\PYZus{}3}\PY{p}{.}\PY{n}{Eval}\PY{p}{(}\PY{n}{x}\PY{p}{,}\PY{n}{y}\PY{p}{)}\PY{o}{*}\PY{p}{(}\PY{l+m+mi}{1}\PY{o}{+}\PY{n}{rnd}\PY{p}{)}\PY{p}{;}
            \PY{n}{dte}\PY{p}{.}\PY{n}{SetPoint}\PY{p}{(}\PY{n}{i}\PY{p}{,}\PY{n}{x}\PY{p}{,}\PY{n}{y}\PY{p}{,}\PY{n}{z}\PY{p}{)}\PY{p}{;}
            \PY{n}{ex} \PY{o}{=} \PY{l+m+mf}{0.05}\PY{o}{*}\PY{n}{my\PYZus{}random\PYZus{}generator}\PY{p}{.}\PY{n}{Uniform}\PY{p}{(}\PY{p}{)}\PY{p}{;}
            \PY{n}{ey} \PY{o}{=} \PY{l+m+mf}{0.05}\PY{o}{*}\PY{n}{my\PYZus{}random\PYZus{}generator}\PY{p}{.}\PY{n}{Uniform}\PY{p}{(}\PY{p}{)}\PY{p}{;}
            \PY{n}{ez} \PY{o}{=} \PY{n}{fabs}\PY{p}{(}\PY{n}{z}\PY{o}{*}\PY{n}{rnd}\PY{p}{)}\PY{p}{;}
            \PY{n}{dte}\PY{p}{.}\PY{n}{SetPointError}\PY{p}{(}\PY{n}{i}\PY{p}{,}\PY{n}{ex}\PY{p}{,}\PY{n}{ey}\PY{p}{,}\PY{n}{ez}\PY{p}{)}\PY{p}{;}
        \PY{p}{\PYZcb{}}
\end{Verbatim}

    \begin{itemize}
\tightlist
\item
  Fitting a 2-dimensional function just works like in the
  one-dimensional case, i.e.~initialisation of parameters and calling of
  the Fit() method.
\end{itemize}

    \begin{Verbatim}[commandchars=\\\{\}]
{\color{incolor}In [{\color{incolor}7}]:} \PY{c+c1}{//\PYZsh{}4}
        \PY{c+c1}{// Fit function to generated data}
        \PY{n}{function\PYZus{}4\PYZus{}3}\PY{p}{.}\PY{n}{SetParameters}\PY{p}{(}\PY{l+m+mf}{0.7}\PY{p}{,}\PY{l+m+mf}{1.5}\PY{p}{)}\PY{p}{;}  \PY{c+c1}{// set initial values for fit}
        \PY{n}{function\PYZus{}4\PYZus{}3}\PY{p}{.}\PY{n}{SetTitle}\PY{p}{(}\PY{l+s}{\PYZdq{}}\PY{l+s}{Fitted 2D function}\PY{l+s}{\PYZdq{}}\PY{p}{)}\PY{p}{;}
        \PY{n}{dte}\PY{p}{.}\PY{n}{Fit}\PY{p}{(}\PY{o}{\PYZam{}}\PY{n}{function\PYZus{}4\PYZus{}3}\PY{p}{)}\PY{p}{;}
        \PY{c+c1}{// Plot the result}
        \PY{k}{auto} \PY{n}{canvas\PYZus{}4\PYZus{}3\PYZus{}1} \PY{o}{=} \PY{k}{new} \PY{n}{TCanvas}\PY{p}{(}\PY{p}{)}\PY{p}{;}
        \PY{n}{function\PYZus{}4\PYZus{}3}\PY{p}{.}\PY{n}{SetLineWidth}\PY{p}{(}\PY{l+m+mi}{1}\PY{p}{)}\PY{p}{;}
        \PY{n}{function\PYZus{}4\PYZus{}3}\PY{p}{.}\PY{n}{SetLineColor}\PY{p}{(}\PY{n}{kBlue}\PY{o}{\PYZhy{}}\PY{l+m+mi}{5}\PY{p}{)}\PY{p}{;}
\end{Verbatim}

    \begin{Verbatim}[commandchars=\\\{\}]
FCN=550.868 FROM HESSE     STATUS=NOT POSDEF     14 CALLS          48 TOTAL
                     EDM=9.72579e-13    STRATEGY= 1      ERR MATRIX NOT POS-DEF
  EXT PARAMETER                APPROXIMATE        STEP         FIRST   
  NO.   NAME      VALUE            ERROR          SIZE      DERIVATIVE 
   1  p0           6.83686e-01   1.49858e-01   4.34655e-06  -3.65525e-05
   2  p1           1.46504e+00   3.21126e-01   9.31404e-06  -1.28163e-05
    \end{Verbatim}

    \begin{itemize}
\tightlist
\item
  The Surf1 option draws the TF2 objects (but also bi-dimensional
  histograms) as coloured surfaces with a wire-frame on
  three-dimensional canvases.
\end{itemize}

    \begin{Verbatim}[commandchars=\\\{\}]
{\color{incolor}In [{\color{incolor}8}]:} \PY{c+c1}{//\PYZsh{}5}
        \PY{n}{TF2}   \PY{o}{*}\PY{n}{function\PYZus{}4\PYZus{}3\PYZus{}c} \PY{o}{=} \PY{p}{(}\PY{n}{TF2}\PY{o}{*}\PY{p}{)}\PY{n}{function\PYZus{}4\PYZus{}3}\PY{p}{.}\PY{n}{DrawClone}\PY{p}{(}\PY{l+s}{\PYZdq{}}\PY{l+s}{Surf1}\PY{l+s}{\PYZdq{}}\PY{p}{)}\PY{p}{;}
\end{Verbatim}

    \begin{itemize}
\tightlist
\item
  Retrieve the axis pointer and define the axis titles.
\end{itemize}

    \begin{Verbatim}[commandchars=\\\{\}]
{\color{incolor}In [{\color{incolor}9}]:} \PY{c+c1}{//\PYZsh{}6}
        \PY{n}{TAxis} \PY{o}{*}\PY{n}{Xaxis} \PY{o}{=} \PY{n}{function\PYZus{}4\PYZus{}3\PYZus{}c}\PY{o}{\PYZhy{}}\PY{o}{\PYZgt{}}\PY{n}{GetXaxis}\PY{p}{(}\PY{p}{)}\PY{p}{;}
        \PY{n}{TAxis} \PY{o}{*}\PY{n}{Yaxis} \PY{o}{=} \PY{n}{function\PYZus{}4\PYZus{}3\PYZus{}c}\PY{o}{\PYZhy{}}\PY{o}{\PYZgt{}}\PY{n}{GetYaxis}\PY{p}{(}\PY{p}{)}\PY{p}{;}
        \PY{n}{TAxis} \PY{o}{*}\PY{n}{Zaxis} \PY{o}{=} \PY{n}{function\PYZus{}4\PYZus{}3\PYZus{}c}\PY{o}{\PYZhy{}}\PY{o}{\PYZgt{}}\PY{n}{GetZaxis}\PY{p}{(}\PY{p}{)}\PY{p}{;}
        \PY{n}{Xaxis}\PY{o}{\PYZhy{}}\PY{o}{\PYZgt{}}\PY{n}{SetTitle}\PY{p}{(}\PY{l+s}{\PYZdq{}}\PY{l+s}{X Title}\PY{l+s}{\PYZdq{}}\PY{p}{)}\PY{p}{;} \PY{n}{Xaxis}\PY{o}{\PYZhy{}}\PY{o}{\PYZgt{}}\PY{n}{SetTitleOffset}\PY{p}{(}\PY{l+m+mf}{1.5}\PY{p}{)}\PY{p}{;}
        \PY{n}{Yaxis}\PY{o}{\PYZhy{}}\PY{o}{\PYZgt{}}\PY{n}{SetTitle}\PY{p}{(}\PY{l+s}{\PYZdq{}}\PY{l+s}{Y Title}\PY{l+s}{\PYZdq{}}\PY{p}{)}\PY{p}{;} \PY{n}{Yaxis}\PY{o}{\PYZhy{}}\PY{o}{\PYZgt{}}\PY{n}{SetTitleOffset}\PY{p}{(}\PY{l+m+mf}{1.5}\PY{p}{)}\PY{p}{;}
        \PY{n}{Zaxis}\PY{o}{\PYZhy{}}\PY{o}{\PYZgt{}}\PY{n}{SetTitle}\PY{p}{(}\PY{l+s}{\PYZdq{}}\PY{l+s}{Z Title}\PY{l+s}{\PYZdq{}}\PY{p}{)}\PY{p}{;} \PY{n}{Zaxis}\PY{o}{\PYZhy{}}\PY{o}{\PYZgt{}}\PY{n}{SetTitleOffset}\PY{p}{(}\PY{l+m+mf}{1.5}\PY{p}{)}\PY{p}{;}
\end{Verbatim}

    \begin{itemize}
\tightlist
\item
  Draw the cloud of points on top of the coloured surface.
\end{itemize}

    \begin{Verbatim}[commandchars=\\\{\}]
{\color{incolor}In [{\color{incolor}10}]:} \PY{c+c1}{//\PYZsh{}7}
         \PY{n}{dte}\PY{p}{.}\PY{n}{DrawClone}\PY{p}{(}\PY{l+s}{\PYZdq{}}\PY{l+s}{P0 Same}\PY{l+s}{\PYZdq{}}\PY{p}{)}\PY{p}{;}
         \PY{c+c1}{// Make the x and y projections}
\end{Verbatim}

    \begin{itemize}
\tightlist
\item
  Here you learn how to create a canvas, partition it in two sub-pads
  and access them. It is very handy to show multiple plots in the same
  window or image.
\end{itemize}

    \begin{Verbatim}[commandchars=\\\{\}]
{\color{incolor}In [{\color{incolor}11}]:} \PY{c+c1}{//\PYZsh{}8}
         \PY{k}{auto} \PY{n}{canvas\PYZus{}4\PYZus{}3\PYZus{}2}\PY{o}{=} \PY{k}{new} \PY{n}{TCanvas}\PY{p}{(}\PY{l+s}{\PYZdq{}}\PY{l+s}{ProjCan}\PY{l+s}{\PYZdq{}}\PY{p}{,}\PY{l+s}{\PYZdq{}}\PY{l+s}{The Projections}\PY{l+s}{\PYZdq{}}\PY{p}{,}\PY{l+m+mi}{1000}\PY{p}{,}\PY{l+m+mi}{400}\PY{p}{)}\PY{p}{;}
         \PY{n}{canvas\PYZus{}4\PYZus{}3\PYZus{}2}\PY{o}{\PYZhy{}}\PY{o}{\PYZgt{}}\PY{n}{Divide}\PY{p}{(}\PY{l+m+mi}{2}\PY{p}{,}\PY{l+m+mi}{1}\PY{p}{)}\PY{p}{;}
         \PY{n}{canvas\PYZus{}4\PYZus{}3\PYZus{}2}\PY{o}{\PYZhy{}}\PY{o}{\PYZgt{}}\PY{n}{cd}\PY{p}{(}\PY{l+m+mi}{1}\PY{p}{)}\PY{p}{;}
         \PY{n}{dte}\PY{p}{.}\PY{n}{Project}\PY{p}{(}\PY{l+s}{\PYZdq{}}\PY{l+s}{x}\PY{l+s}{\PYZdq{}}\PY{p}{)}\PY{o}{\PYZhy{}}\PY{o}{\PYZgt{}}\PY{n}{Draw}\PY{p}{(}\PY{p}{)}\PY{p}{;}
         \PY{n}{canvas\PYZus{}4\PYZus{}3\PYZus{}2}\PY{o}{\PYZhy{}}\PY{o}{\PYZgt{}}\PY{n}{cd}\PY{p}{(}\PY{l+m+mi}{2}\PY{p}{)}\PY{p}{;}
         \PY{n}{dte}\PY{p}{.}\PY{n}{Project}\PY{p}{(}\PY{l+s}{\PYZdq{}}\PY{l+s}{y}\PY{l+s}{\PYZdq{}}\PY{p}{)}\PY{o}{\PYZhy{}}\PY{o}{\PYZgt{}}\PY{n}{Draw}\PY{p}{(}\PY{p}{)}\PY{p}{;}
            
         \PY{n}{canvas\PYZus{}4\PYZus{}3\PYZus{}2} \PY{o}{\PYZhy{}}\PY{o}{\PYZgt{}}\PY{n}{Draw}\PY{p}{(}\PY{p}{)}\PY{p}{;}
\end{Verbatim}

    
    \begin{verbatim}
<IPython.core.display.HTML object>
    \end{verbatim}

    
    \subsection{4.4 Multiple graphs}\label{multiple-graphs}

The class
\href{https://root.cern.ch/doc/master/classTMultiGraph.html}{\texttt{TMultigraph}}
allows to manipulate a set of graphs as a single entity. It is a
collection of TGraph (or derived) objects. When drawn, the X and Y axis
ranges are automatically computed such as all the graphs will be
visible.

    \begin{Verbatim}[commandchars=\\\{\}]
{\color{incolor}In [{\color{incolor}12}]:} \PY{n}{TCanvas} \PY{o}{*}\PY{n}{canvas\PYZus{}4\PYZus{}4} \PY{o}{=} \PY{k}{new} \PY{n}{TCanvas}\PY{p}{(}\PY{l+s}{\PYZdq{}}\PY{l+s}{canvas\PYZus{}4\PYZus{}4}\PY{l+s}{\PYZdq{}}\PY{p}{,}\PY{l+s}{\PYZdq{}}\PY{l+s}{multigraph}\PY{l+s}{\PYZdq{}}\PY{p}{,}\PY{l+m+mi}{700}\PY{p}{,}\PY{l+m+mi}{500}\PY{p}{)}\PY{p}{;}
         \PY{n}{canvas\PYZus{}4\PYZus{}4}\PY{o}{\PYZhy{}}\PY{o}{\PYZgt{}}\PY{n}{SetGrid}\PY{p}{(}\PY{p}{)}\PY{p}{;}
\end{Verbatim}

    Here we create the multigraph.

    \begin{Verbatim}[commandchars=\\\{\}]
{\color{incolor}In [{\color{incolor}13}]:} \PY{n}{TMultiGraph} \PY{o}{*}\PY{n}{multigraph\PYZus{}4\PYZus{}4} \PY{o}{=} \PY{k}{new} \PY{n}{TMultiGraph}\PY{p}{(}\PY{p}{)}\PY{p}{;}
         
         \PY{c+c1}{// create first graph}
         \PY{k}{const} \PY{n}{Int\PYZus{}t} \PY{n}{n1} \PY{o}{=} \PY{l+m+mi}{10}\PY{p}{;}
         \PY{n}{Double\PYZus{}t} \PY{n}{px1}\PY{p}{[}\PY{p}{]} \PY{o}{=} \PY{p}{\PYZob{}}\PY{o}{\PYZhy{}}\PY{l+m+mf}{0.1}\PY{p}{,} \PY{l+m+mf}{0.05}\PY{p}{,} \PY{l+m+mf}{0.25}\PY{p}{,} \PY{l+m+mf}{0.35}\PY{p}{,} \PY{l+m+mf}{0.5}\PY{p}{,} \PY{l+m+mf}{0.61}\PY{p}{,}\PY{l+m+mf}{0.7}\PY{p}{,}\PY{l+m+mf}{0.85}\PY{p}{,}\PY{l+m+mf}{0.89}\PY{p}{,}\PY{l+m+mf}{0.95}\PY{p}{\PYZcb{}}\PY{p}{;}
         \PY{n}{Double\PYZus{}t} \PY{n}{py1}\PY{p}{[}\PY{p}{]} \PY{o}{=} \PY{p}{\PYZob{}}\PY{o}{\PYZhy{}}\PY{l+m+mi}{1}\PY{p}{,}\PY{l+m+mf}{2.9}\PY{p}{,}\PY{l+m+mf}{5.6}\PY{p}{,}\PY{l+m+mf}{7.4}\PY{p}{,}\PY{l+m+mi}{9}\PY{p}{,}\PY{l+m+mf}{9.6}\PY{p}{,}\PY{l+m+mf}{8.7}\PY{p}{,}\PY{l+m+mf}{6.3}\PY{p}{,}\PY{l+m+mf}{4.5}\PY{p}{,}\PY{l+m+mi}{1}\PY{p}{\PYZcb{}}\PY{p}{;}
         \PY{n}{Double\PYZus{}t} \PY{n}{ex1}\PY{p}{[}\PY{p}{]} \PY{o}{=} \PY{p}{\PYZob{}}\PY{l+m+mf}{.05}\PY{p}{,}\PY{l+m+mf}{.1}\PY{p}{,}\PY{l+m+mf}{.07}\PY{p}{,}\PY{l+m+mf}{.07}\PY{p}{,}\PY{l+m+mf}{.04}\PY{p}{,}\PY{l+m+mf}{.05}\PY{p}{,}\PY{l+m+mf}{.06}\PY{p}{,}\PY{l+m+mf}{.07}\PY{p}{,}\PY{l+m+mf}{.08}\PY{p}{,}\PY{l+m+mf}{.05}\PY{p}{\PYZcb{}}\PY{p}{;}
         \PY{n}{Double\PYZus{}t} \PY{n}{ey1}\PY{p}{[}\PY{p}{]} \PY{o}{=} \PY{p}{\PYZob{}}\PY{l+m+mf}{.8}\PY{p}{,}\PY{l+m+mf}{.7}\PY{p}{,}\PY{l+m+mf}{.6}\PY{p}{,}\PY{l+m+mf}{.5}\PY{p}{,}\PY{l+m+mf}{.4}\PY{p}{,}\PY{l+m+mf}{.4}\PY{p}{,}\PY{l+m+mf}{.5}\PY{p}{,}\PY{l+m+mf}{.6}\PY{p}{,}\PY{l+m+mf}{.7}\PY{p}{,}\PY{l+m+mf}{.8}\PY{p}{\PYZcb{}}\PY{p}{;}
         \PY{n}{TGraphErrors} \PY{o}{*}\PY{n}{error\PYZus{}graph\PYZus{}1} \PY{o}{=} \PY{k}{new} \PY{n}{TGraphErrors}\PY{p}{(}\PY{n}{n1}\PY{p}{,}\PY{n}{px1}\PY{p}{,}\PY{n}{py1}\PY{p}{,}\PY{n}{ex1}\PY{p}{,}\PY{n}{ey1}\PY{p}{)}\PY{p}{;}
         \PY{n}{error\PYZus{}graph\PYZus{}1}\PY{o}{\PYZhy{}}\PY{o}{\PYZgt{}}\PY{n}{SetMarkerColor}\PY{p}{(}\PY{n}{kBlue}\PY{p}{)}\PY{p}{;}
         \PY{n}{error\PYZus{}graph\PYZus{}1}\PY{o}{\PYZhy{}}\PY{o}{\PYZgt{}}\PY{n}{SetMarkerStyle}\PY{p}{(}\PY{l+m+mi}{21}\PY{p}{)}\PY{p}{;}
         \PY{n}{multigraph\PYZus{}4\PYZus{}4}\PY{o}{\PYZhy{}}\PY{o}{\PYZgt{}}\PY{n}{Add}\PY{p}{(}\PY{n}{error\PYZus{}graph\PYZus{}1}\PY{p}{)}\PY{p}{;}
\end{Verbatim}

    Here we create two graphs with errors and add them in the multigraph.

    \begin{Verbatim}[commandchars=\\\{\}]
{\color{incolor}In [{\color{incolor}14}]:} \PY{c+c1}{// create second graph}
         \PY{k}{const} \PY{n}{Int\PYZus{}t} \PY{n}{n2} \PY{o}{=} \PY{l+m+mi}{10}\PY{p}{;}
         \PY{n}{Float\PYZus{}t} \PY{n}{x2}\PY{p}{[}\PY{p}{]}  \PY{o}{=} \PY{p}{\PYZob{}}\PY{o}{\PYZhy{}}\PY{l+m+mf}{0.28}\PY{p}{,} \PY{l+m+mf}{0.005}\PY{p}{,} \PY{l+m+mf}{0.19}\PY{p}{,} \PY{l+m+mf}{0.29}\PY{p}{,} \PY{l+m+mf}{0.45}\PY{p}{,} \PY{l+m+mf}{0.56}\PY{p}{,}\PY{l+m+mf}{0.65}\PY{p}{,}\PY{l+m+mf}{0.80}\PY{p}{,}\PY{l+m+mf}{0.90}\PY{p}{,}\PY{l+m+mf}{1.01}\PY{p}{\PYZcb{}}\PY{p}{;}
         \PY{n}{Float\PYZus{}t} \PY{n}{y2}\PY{p}{[}\PY{p}{]}  \PY{o}{=} \PY{p}{\PYZob{}}\PY{l+m+mf}{2.1}\PY{p}{,}\PY{l+m+mf}{3.86}\PY{p}{,}\PY{l+m+mi}{7}\PY{p}{,}\PY{l+m+mi}{9}\PY{p}{,}\PY{l+m+mi}{10}\PY{p}{,}\PY{l+m+mf}{10.55}\PY{p}{,}\PY{l+m+mf}{9.64}\PY{p}{,}\PY{l+m+mf}{7.26}\PY{p}{,}\PY{l+m+mf}{5.42}\PY{p}{,}\PY{l+m+mi}{2}\PY{p}{\PYZcb{}}\PY{p}{;}
         \PY{n}{Float\PYZus{}t} \PY{n}{ex2}\PY{p}{[}\PY{p}{]} \PY{o}{=} \PY{p}{\PYZob{}}\PY{l+m+mf}{.04}\PY{p}{,}\PY{l+m+mf}{.12}\PY{p}{,}\PY{l+m+mf}{.08}\PY{p}{,}\PY{l+m+mf}{.06}\PY{p}{,}\PY{l+m+mf}{.05}\PY{p}{,}\PY{l+m+mf}{.04}\PY{p}{,}\PY{l+m+mf}{.07}\PY{p}{,}\PY{l+m+mf}{.06}\PY{p}{,}\PY{l+m+mf}{.08}\PY{p}{,}\PY{l+m+mf}{.04}\PY{p}{\PYZcb{}}\PY{p}{;}
         \PY{n}{Float\PYZus{}t} \PY{n}{ey2}\PY{p}{[}\PY{p}{]} \PY{o}{=} \PY{p}{\PYZob{}}\PY{l+m+mf}{.6}\PY{p}{,}\PY{l+m+mf}{.8}\PY{p}{,}\PY{l+m+mf}{.7}\PY{p}{,}\PY{l+m+mf}{.4}\PY{p}{,}\PY{l+m+mf}{.3}\PY{p}{,}\PY{l+m+mf}{.3}\PY{p}{,}\PY{l+m+mf}{.4}\PY{p}{,}\PY{l+m+mf}{.5}\PY{p}{,}\PY{l+m+mf}{.6}\PY{p}{,}\PY{l+m+mf}{.7}\PY{p}{\PYZcb{}}\PY{p}{;}
         \PY{n}{TGraphErrors} \PY{o}{*}\PY{n}{error\PYZus{}graph\PYZus{}2} \PY{o}{=} \PY{k}{new} \PY{n}{TGraphErrors}\PY{p}{(}\PY{n}{n2}\PY{p}{,}\PY{n}{x2}\PY{p}{,}\PY{n}{y2}\PY{p}{,}\PY{n}{ex2}\PY{p}{,}\PY{n}{ey2}\PY{p}{)}\PY{p}{;}
         \PY{n}{error\PYZus{}graph\PYZus{}1}\PY{o}{\PYZhy{}}\PY{o}{\PYZgt{}}\PY{n}{SetMarkerColor}\PY{p}{(}\PY{n}{kRed}\PY{p}{)}\PY{p}{;}
         \PY{n}{error\PYZus{}graph\PYZus{}1}\PY{o}{\PYZhy{}}\PY{o}{\PYZgt{}}\PY{n}{SetMarkerStyle}\PY{p}{(}\PY{l+m+mi}{20}\PY{p}{)}\PY{p}{;}
         \PY{n}{multigraph\PYZus{}4\PYZus{}4}\PY{o}{\PYZhy{}}\PY{o}{\PYZgt{}}\PY{n}{Add}\PY{p}{(}\PY{n}{error\PYZus{}graph\PYZus{}1}\PY{p}{)}\PY{p}{;}
         
         \PY{n}{multigraph\PYZus{}4\PYZus{}4}\PY{o}{\PYZhy{}}\PY{o}{\PYZgt{}}\PY{n}{Draw}\PY{p}{(}\PY{l+s}{\PYZdq{}}\PY{l+s}{apl}\PY{l+s}{\PYZdq{}}\PY{p}{)}\PY{p}{;}
\end{Verbatim}

    Here we draw the multigraph. The axis limits are computed automatically
to make sure all the graphs' points will be in range.

    \begin{Verbatim}[commandchars=\\\{\}]
{\color{incolor}In [{\color{incolor}15}]:} \PY{n}{multigraph\PYZus{}4\PYZus{}4}\PY{o}{\PYZhy{}}\PY{o}{\PYZgt{}}\PY{n}{GetXaxis}\PY{p}{(}\PY{p}{)}\PY{o}{\PYZhy{}}\PY{o}{\PYZgt{}}\PY{n}{SetTitle}\PY{p}{(}\PY{l+s}{\PYZdq{}}\PY{l+s}{X values}\PY{l+s}{\PYZdq{}}\PY{p}{)}\PY{p}{;}
         \PY{n}{multigraph\PYZus{}4\PYZus{}4}\PY{o}{\PYZhy{}}\PY{o}{\PYZgt{}}\PY{n}{GetYaxis}\PY{p}{(}\PY{p}{)}\PY{o}{\PYZhy{}}\PY{o}{\PYZgt{}}\PY{n}{SetTitle}\PY{p}{(}\PY{l+s}{\PYZdq{}}\PY{l+s}{Y values}\PY{l+s}{\PYZdq{}}\PY{p}{)}\PY{p}{;}
         
         \PY{n}{gPad}\PY{o}{\PYZhy{}}\PY{o}{\PYZgt{}}\PY{n}{Update}\PY{p}{(}\PY{p}{)}\PY{p}{;}
         \PY{n}{gPad}\PY{o}{\PYZhy{}}\PY{o}{\PYZgt{}}\PY{n}{Modified}\PY{p}{(}\PY{p}{)}\PY{p}{;}
         \PY{n}{canvas\PYZus{}4\PYZus{}4}\PY{o}{\PYZhy{}}\PY{o}{\PYZgt{}}\PY{n}{Draw}\PY{p}{(}\PY{p}{)}\PY{p}{;}
\end{Verbatim}

    
    \begin{verbatim}
<IPython.core.display.HTML object>
    \end{verbatim}

    

    % Add a bibliography block to the postdoc
    
    
    
    \end{document}
