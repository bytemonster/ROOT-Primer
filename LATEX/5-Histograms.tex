
% Default to the notebook output style

    


% Inherit from the specified cell style.




    
\documentclass{article}

    
    
    \usepackage{graphicx} % Used to insert images
    \usepackage{adjustbox} % Used to constrain images to a maximum size 
    \usepackage{color} % Allow colors to be defined
    \usepackage{enumerate} % Needed for markdown enumerations to work
    \usepackage{geometry} % Used to adjust the document margins
    \usepackage{amsmath} % Equations
    \usepackage{amssymb} % Equations
    \usepackage{eurosym} % defines \euro
    \usepackage[mathletters]{ucs} % Extended unicode (utf-8) support
    \usepackage[utf8x]{inputenc} % Allow utf-8 characters in the tex document
    \usepackage{fancyvrb} % verbatim replacement that allows latex
    \usepackage{grffile} % extends the file name processing of package graphics 
                         % to support a larger range 
    % The hyperref package gives us a pdf with properly built
    % internal navigation ('pdf bookmarks' for the table of contents,
    % internal cross-reference links, web links for URLs, etc.)
    \usepackage{hyperref}
    \usepackage{longtable} % longtable support required by pandoc >1.10
    \usepackage{booktabs}  % table support for pandoc > 1.12.2
    \usepackage{ulem} % ulem is needed to support strikethroughs (\sout)
    

    
    
    \definecolor{orange}{cmyk}{0,0.4,0.8,0.2}
    \definecolor{darkorange}{rgb}{.71,0.21,0.01}
    \definecolor{darkgreen}{rgb}{.12,.54,.11}
    \definecolor{myteal}{rgb}{.26, .44, .56}
    \definecolor{gray}{gray}{0.45}
    \definecolor{lightgray}{gray}{.95}
    \definecolor{mediumgray}{gray}{.8}
    \definecolor{inputbackground}{rgb}{.95, .95, .85}
    \definecolor{outputbackground}{rgb}{.95, .95, .95}
    \definecolor{traceback}{rgb}{1, .95, .95}
    % ansi colors
    \definecolor{red}{rgb}{.6,0,0}
    \definecolor{green}{rgb}{0,.65,0}
    \definecolor{brown}{rgb}{0.6,0.6,0}
    \definecolor{blue}{rgb}{0,.145,.698}
    \definecolor{purple}{rgb}{.698,.145,.698}
    \definecolor{cyan}{rgb}{0,.698,.698}
    \definecolor{lightgray}{gray}{0.5}
    
    % bright ansi colors
    \definecolor{darkgray}{gray}{0.25}
    \definecolor{lightred}{rgb}{1.0,0.39,0.28}
    \definecolor{lightgreen}{rgb}{0.48,0.99,0.0}
    \definecolor{lightblue}{rgb}{0.53,0.81,0.92}
    \definecolor{lightpurple}{rgb}{0.87,0.63,0.87}
    \definecolor{lightcyan}{rgb}{0.5,1.0,0.83}
    
    % commands and environments needed by pandoc snippets
    % extracted from the output of `pandoc -s`
    \providecommand{\tightlist}{%
      \setlength{\itemsep}{0pt}\setlength{\parskip}{0pt}}
    \DefineVerbatimEnvironment{Highlighting}{Verbatim}{commandchars=\\\{\}}
    % Add ',fontsize=\small' for more characters per line
    \newenvironment{Shaded}{}{}
    \newcommand{\KeywordTok}[1]{\textcolor[rgb]{0.00,0.44,0.13}{\textbf{{#1}}}}
    \newcommand{\DataTypeTok}[1]{\textcolor[rgb]{0.56,0.13,0.00}{{#1}}}
    \newcommand{\DecValTok}[1]{\textcolor[rgb]{0.25,0.63,0.44}{{#1}}}
    \newcommand{\BaseNTok}[1]{\textcolor[rgb]{0.25,0.63,0.44}{{#1}}}
    \newcommand{\FloatTok}[1]{\textcolor[rgb]{0.25,0.63,0.44}{{#1}}}
    \newcommand{\CharTok}[1]{\textcolor[rgb]{0.25,0.44,0.63}{{#1}}}
    \newcommand{\StringTok}[1]{\textcolor[rgb]{0.25,0.44,0.63}{{#1}}}
    \newcommand{\CommentTok}[1]{\textcolor[rgb]{0.38,0.63,0.69}{\textit{{#1}}}}
    \newcommand{\OtherTok}[1]{\textcolor[rgb]{0.00,0.44,0.13}{{#1}}}
    \newcommand{\AlertTok}[1]{\textcolor[rgb]{1.00,0.00,0.00}{\textbf{{#1}}}}
    \newcommand{\FunctionTok}[1]{\textcolor[rgb]{0.02,0.16,0.49}{{#1}}}
    \newcommand{\RegionMarkerTok}[1]{{#1}}
    \newcommand{\ErrorTok}[1]{\textcolor[rgb]{1.00,0.00,0.00}{\textbf{{#1}}}}
    \newcommand{\NormalTok}[1]{{#1}}
    
    % Additional commands for more recent versions of Pandoc
    \newcommand{\ConstantTok}[1]{\textcolor[rgb]{0.53,0.00,0.00}{{#1}}}
    \newcommand{\SpecialCharTok}[1]{\textcolor[rgb]{0.25,0.44,0.63}{{#1}}}
    \newcommand{\VerbatimStringTok}[1]{\textcolor[rgb]{0.25,0.44,0.63}{{#1}}}
    \newcommand{\SpecialStringTok}[1]{\textcolor[rgb]{0.73,0.40,0.53}{{#1}}}
    \newcommand{\ImportTok}[1]{{#1}}
    \newcommand{\DocumentationTok}[1]{\textcolor[rgb]{0.73,0.13,0.13}{\textit{{#1}}}}
    \newcommand{\AnnotationTok}[1]{\textcolor[rgb]{0.38,0.63,0.69}{\textbf{\textit{{#1}}}}}
    \newcommand{\CommentVarTok}[1]{\textcolor[rgb]{0.38,0.63,0.69}{\textbf{\textit{{#1}}}}}
    \newcommand{\VariableTok}[1]{\textcolor[rgb]{0.10,0.09,0.49}{{#1}}}
    \newcommand{\ControlFlowTok}[1]{\textcolor[rgb]{0.00,0.44,0.13}{\textbf{{#1}}}}
    \newcommand{\OperatorTok}[1]{\textcolor[rgb]{0.40,0.40,0.40}{{#1}}}
    \newcommand{\BuiltInTok}[1]{{#1}}
    \newcommand{\ExtensionTok}[1]{{#1}}
    \newcommand{\PreprocessorTok}[1]{\textcolor[rgb]{0.74,0.48,0.00}{{#1}}}
    \newcommand{\AttributeTok}[1]{\textcolor[rgb]{0.49,0.56,0.16}{{#1}}}
    \newcommand{\InformationTok}[1]{\textcolor[rgb]{0.38,0.63,0.69}{\textbf{\textit{{#1}}}}}
    \newcommand{\WarningTok}[1]{\textcolor[rgb]{0.38,0.63,0.69}{\textbf{\textit{{#1}}}}}
    
    
    % Define a nice break command that doesn't care if a line doesn't already
    % exist.
    \def\br{\hspace*{\fill} \\* }
    % Math Jax compatability definitions
    \def\gt{>}
    \def\lt{<}
    % Document parameters
    \title{5-Histograms}
    
    
    

    % Pygments definitions
    
\makeatletter
\def\PY@reset{\let\PY@it=\relax \let\PY@bf=\relax%
    \let\PY@ul=\relax \let\PY@tc=\relax%
    \let\PY@bc=\relax \let\PY@ff=\relax}
\def\PY@tok#1{\csname PY@tok@#1\endcsname}
\def\PY@toks#1+{\ifx\relax#1\empty\else%
    \PY@tok{#1}\expandafter\PY@toks\fi}
\def\PY@do#1{\PY@bc{\PY@tc{\PY@ul{%
    \PY@it{\PY@bf{\PY@ff{#1}}}}}}}
\def\PY#1#2{\PY@reset\PY@toks#1+\relax+\PY@do{#2}}

\expandafter\def\csname PY@tok@nd\endcsname{\def\PY@tc##1{\textcolor[rgb]{0.67,0.13,1.00}{##1}}}
\expandafter\def\csname PY@tok@mb\endcsname{\def\PY@tc##1{\textcolor[rgb]{0.40,0.40,0.40}{##1}}}
\expandafter\def\csname PY@tok@gs\endcsname{\let\PY@bf=\textbf}
\expandafter\def\csname PY@tok@nb\endcsname{\def\PY@tc##1{\textcolor[rgb]{0.00,0.50,0.00}{##1}}}
\expandafter\def\csname PY@tok@mf\endcsname{\def\PY@tc##1{\textcolor[rgb]{0.40,0.40,0.40}{##1}}}
\expandafter\def\csname PY@tok@bp\endcsname{\def\PY@tc##1{\textcolor[rgb]{0.00,0.50,0.00}{##1}}}
\expandafter\def\csname PY@tok@gh\endcsname{\let\PY@bf=\textbf\def\PY@tc##1{\textcolor[rgb]{0.00,0.00,0.50}{##1}}}
\expandafter\def\csname PY@tok@si\endcsname{\let\PY@bf=\textbf\def\PY@tc##1{\textcolor[rgb]{0.73,0.40,0.53}{##1}}}
\expandafter\def\csname PY@tok@gt\endcsname{\def\PY@tc##1{\textcolor[rgb]{0.00,0.27,0.87}{##1}}}
\expandafter\def\csname PY@tok@s\endcsname{\def\PY@tc##1{\textcolor[rgb]{0.73,0.13,0.13}{##1}}}
\expandafter\def\csname PY@tok@gu\endcsname{\let\PY@bf=\textbf\def\PY@tc##1{\textcolor[rgb]{0.50,0.00,0.50}{##1}}}
\expandafter\def\csname PY@tok@ge\endcsname{\let\PY@it=\textit}
\expandafter\def\csname PY@tok@nt\endcsname{\let\PY@bf=\textbf\def\PY@tc##1{\textcolor[rgb]{0.00,0.50,0.00}{##1}}}
\expandafter\def\csname PY@tok@kr\endcsname{\let\PY@bf=\textbf\def\PY@tc##1{\textcolor[rgb]{0.00,0.50,0.00}{##1}}}
\expandafter\def\csname PY@tok@cpf\endcsname{\let\PY@it=\textit\def\PY@tc##1{\textcolor[rgb]{0.25,0.50,0.50}{##1}}}
\expandafter\def\csname PY@tok@vi\endcsname{\def\PY@tc##1{\textcolor[rgb]{0.10,0.09,0.49}{##1}}}
\expandafter\def\csname PY@tok@sx\endcsname{\def\PY@tc##1{\textcolor[rgb]{0.00,0.50,0.00}{##1}}}
\expandafter\def\csname PY@tok@nc\endcsname{\let\PY@bf=\textbf\def\PY@tc##1{\textcolor[rgb]{0.00,0.00,1.00}{##1}}}
\expandafter\def\csname PY@tok@s1\endcsname{\def\PY@tc##1{\textcolor[rgb]{0.73,0.13,0.13}{##1}}}
\expandafter\def\csname PY@tok@sc\endcsname{\def\PY@tc##1{\textcolor[rgb]{0.73,0.13,0.13}{##1}}}
\expandafter\def\csname PY@tok@sr\endcsname{\def\PY@tc##1{\textcolor[rgb]{0.73,0.40,0.53}{##1}}}
\expandafter\def\csname PY@tok@nn\endcsname{\let\PY@bf=\textbf\def\PY@tc##1{\textcolor[rgb]{0.00,0.00,1.00}{##1}}}
\expandafter\def\csname PY@tok@gp\endcsname{\let\PY@bf=\textbf\def\PY@tc##1{\textcolor[rgb]{0.00,0.00,0.50}{##1}}}
\expandafter\def\csname PY@tok@cm\endcsname{\let\PY@it=\textit\def\PY@tc##1{\textcolor[rgb]{0.25,0.50,0.50}{##1}}}
\expandafter\def\csname PY@tok@kn\endcsname{\let\PY@bf=\textbf\def\PY@tc##1{\textcolor[rgb]{0.00,0.50,0.00}{##1}}}
\expandafter\def\csname PY@tok@kc\endcsname{\let\PY@bf=\textbf\def\PY@tc##1{\textcolor[rgb]{0.00,0.50,0.00}{##1}}}
\expandafter\def\csname PY@tok@mo\endcsname{\def\PY@tc##1{\textcolor[rgb]{0.40,0.40,0.40}{##1}}}
\expandafter\def\csname PY@tok@cs\endcsname{\let\PY@it=\textit\def\PY@tc##1{\textcolor[rgb]{0.25,0.50,0.50}{##1}}}
\expandafter\def\csname PY@tok@na\endcsname{\def\PY@tc##1{\textcolor[rgb]{0.49,0.56,0.16}{##1}}}
\expandafter\def\csname PY@tok@vc\endcsname{\def\PY@tc##1{\textcolor[rgb]{0.10,0.09,0.49}{##1}}}
\expandafter\def\csname PY@tok@nl\endcsname{\def\PY@tc##1{\textcolor[rgb]{0.63,0.63,0.00}{##1}}}
\expandafter\def\csname PY@tok@ow\endcsname{\let\PY@bf=\textbf\def\PY@tc##1{\textcolor[rgb]{0.67,0.13,1.00}{##1}}}
\expandafter\def\csname PY@tok@sd\endcsname{\let\PY@it=\textit\def\PY@tc##1{\textcolor[rgb]{0.73,0.13,0.13}{##1}}}
\expandafter\def\csname PY@tok@gd\endcsname{\def\PY@tc##1{\textcolor[rgb]{0.63,0.00,0.00}{##1}}}
\expandafter\def\csname PY@tok@c1\endcsname{\let\PY@it=\textit\def\PY@tc##1{\textcolor[rgb]{0.25,0.50,0.50}{##1}}}
\expandafter\def\csname PY@tok@kp\endcsname{\def\PY@tc##1{\textcolor[rgb]{0.00,0.50,0.00}{##1}}}
\expandafter\def\csname PY@tok@il\endcsname{\def\PY@tc##1{\textcolor[rgb]{0.40,0.40,0.40}{##1}}}
\expandafter\def\csname PY@tok@ni\endcsname{\let\PY@bf=\textbf\def\PY@tc##1{\textcolor[rgb]{0.60,0.60,0.60}{##1}}}
\expandafter\def\csname PY@tok@ss\endcsname{\def\PY@tc##1{\textcolor[rgb]{0.10,0.09,0.49}{##1}}}
\expandafter\def\csname PY@tok@c\endcsname{\let\PY@it=\textit\def\PY@tc##1{\textcolor[rgb]{0.25,0.50,0.50}{##1}}}
\expandafter\def\csname PY@tok@cp\endcsname{\def\PY@tc##1{\textcolor[rgb]{0.74,0.48,0.00}{##1}}}
\expandafter\def\csname PY@tok@o\endcsname{\def\PY@tc##1{\textcolor[rgb]{0.40,0.40,0.40}{##1}}}
\expandafter\def\csname PY@tok@kd\endcsname{\let\PY@bf=\textbf\def\PY@tc##1{\textcolor[rgb]{0.00,0.50,0.00}{##1}}}
\expandafter\def\csname PY@tok@go\endcsname{\def\PY@tc##1{\textcolor[rgb]{0.53,0.53,0.53}{##1}}}
\expandafter\def\csname PY@tok@kt\endcsname{\def\PY@tc##1{\textcolor[rgb]{0.69,0.00,0.25}{##1}}}
\expandafter\def\csname PY@tok@mi\endcsname{\def\PY@tc##1{\textcolor[rgb]{0.40,0.40,0.40}{##1}}}
\expandafter\def\csname PY@tok@no\endcsname{\def\PY@tc##1{\textcolor[rgb]{0.53,0.00,0.00}{##1}}}
\expandafter\def\csname PY@tok@ch\endcsname{\let\PY@it=\textit\def\PY@tc##1{\textcolor[rgb]{0.25,0.50,0.50}{##1}}}
\expandafter\def\csname PY@tok@ne\endcsname{\let\PY@bf=\textbf\def\PY@tc##1{\textcolor[rgb]{0.82,0.25,0.23}{##1}}}
\expandafter\def\csname PY@tok@gi\endcsname{\def\PY@tc##1{\textcolor[rgb]{0.00,0.63,0.00}{##1}}}
\expandafter\def\csname PY@tok@w\endcsname{\def\PY@tc##1{\textcolor[rgb]{0.73,0.73,0.73}{##1}}}
\expandafter\def\csname PY@tok@se\endcsname{\let\PY@bf=\textbf\def\PY@tc##1{\textcolor[rgb]{0.73,0.40,0.13}{##1}}}
\expandafter\def\csname PY@tok@s2\endcsname{\def\PY@tc##1{\textcolor[rgb]{0.73,0.13,0.13}{##1}}}
\expandafter\def\csname PY@tok@nv\endcsname{\def\PY@tc##1{\textcolor[rgb]{0.10,0.09,0.49}{##1}}}
\expandafter\def\csname PY@tok@m\endcsname{\def\PY@tc##1{\textcolor[rgb]{0.40,0.40,0.40}{##1}}}
\expandafter\def\csname PY@tok@k\endcsname{\let\PY@bf=\textbf\def\PY@tc##1{\textcolor[rgb]{0.00,0.50,0.00}{##1}}}
\expandafter\def\csname PY@tok@mh\endcsname{\def\PY@tc##1{\textcolor[rgb]{0.40,0.40,0.40}{##1}}}
\expandafter\def\csname PY@tok@gr\endcsname{\def\PY@tc##1{\textcolor[rgb]{1.00,0.00,0.00}{##1}}}
\expandafter\def\csname PY@tok@sb\endcsname{\def\PY@tc##1{\textcolor[rgb]{0.73,0.13,0.13}{##1}}}
\expandafter\def\csname PY@tok@sh\endcsname{\def\PY@tc##1{\textcolor[rgb]{0.73,0.13,0.13}{##1}}}
\expandafter\def\csname PY@tok@vg\endcsname{\def\PY@tc##1{\textcolor[rgb]{0.10,0.09,0.49}{##1}}}
\expandafter\def\csname PY@tok@nf\endcsname{\def\PY@tc##1{\textcolor[rgb]{0.00,0.00,1.00}{##1}}}
\expandafter\def\csname PY@tok@err\endcsname{\def\PY@bc##1{\setlength{\fboxsep}{0pt}\fcolorbox[rgb]{1.00,0.00,0.00}{1,1,1}{\strut ##1}}}

\def\PYZbs{\char`\\}
\def\PYZus{\char`\_}
\def\PYZob{\char`\{}
\def\PYZcb{\char`\}}
\def\PYZca{\char`\^}
\def\PYZam{\char`\&}
\def\PYZlt{\char`\<}
\def\PYZgt{\char`\>}
\def\PYZsh{\char`\#}
\def\PYZpc{\char`\%}
\def\PYZdl{\char`\$}
\def\PYZhy{\char`\-}
\def\PYZsq{\char`\'}
\def\PYZdq{\char`\"}
\def\PYZti{\char`\~}
% for compatibility with earlier versions
\def\PYZat{@}
\def\PYZlb{[}
\def\PYZrb{]}
\makeatother


    % Exact colors from NB
    \definecolor{incolor}{rgb}{0.0, 0.0, 0.5}
    \definecolor{outcolor}{rgb}{0.545, 0.0, 0.0}



    
    % Prevent overflowing lines due to hard-to-break entities
    \sloppy 
    % Setup hyperref package
    \hypersetup{
      breaklinks=true,  % so long urls are correctly broken across lines
      colorlinks=true,
      urlcolor=blue,
      linkcolor=darkorange,
      citecolor=darkgreen,
      }
    % Slightly bigger margins than the latex defaults
    
    \geometry{verbose,tmargin=1in,bmargin=1in,lmargin=1in,rmargin=1in}
    
    

    \begin{document}
    
    
    \maketitle
    
    

    
    Histograms play a fundamental role in any type of physics analysis, not
only to visualise measurements but being a powerful form of data
reduction. ROOT offers many classes that represent histograms, all
inheriting from the
\href{https://root.cern.ch/doc/master/classTH1.html}{\texttt{TH1}}
class. We will focus in this chapter on uni- and bi- dimensional
histograms the bin contents of which are represented by floating point
numbers the
\href{https://root.cern.ch/doc/master/classTH1F.html}{\texttt{TH1F}} and
\href{https://root.cern.ch/doc/master/classTH2F.html}{\texttt{TH2F}}
classes respectively. To optimise the memory usage you might go for one
byte
(\href{https://root.cern.ch/doc/master/classTH1C.html}{\texttt{TH1C}}),
short
(\href{https://root.cern.ch/doc/master/classTH1S.html}{\texttt{TH1S}}),
integer
(\href{https://root.cern.ch/doc/master/classTH1I.html}{\texttt{TH1I}})
or double-precision
(\href{https://root.cern.ch/doc/master/classTH1D.html}{\texttt{TH1D}})
bin-content.

\subsection{5.1 Your First Histogram}\label{your-first-histogram}

Let's suppose you want to measure the counts of a Geiger detector
located in proximity of a radioactive source in a given time interval.
This would give you an idea of the activity of your source. The count
distribution in this case is a Poisson distribution. Let's see how
operatively you can fill and draw a histogram with the following example
macro.

The result of a counting (pseudo) experiment. Only bins corresponding to
integer values are filled given the discrete nature of the poissonian
distribution.

Using histograms is rather simple. The main differences with respect to
graphs that emerge from the example are:

\begin{itemize}
\tightlist
\item
  In the begining the histograms have a name and a title right from the
  start, no predefined number of entries but a number of bins and a
  lower-upper range.
\end{itemize}

    \begin{Verbatim}[commandchars=\\\{\}]
{\color{incolor}In [{\color{incolor}1}]:} \PY{o}{\PYZpc{}}\PY{o}{\PYZpc{}}\PY{n}{jsroot} \PY{n}{on}
\end{Verbatim}

    \begin{Verbatim}[commandchars=\\\{\}]
{\color{incolor}In [{\color{incolor}2}]:} \PY{k}{auto} \PY{n}{cnt\PYZus{}r\PYZus{}h}\PY{o}{=}\PY{k}{new} \PY{n}{TH1F}\PY{p}{(}\PY{l+s}{\PYZdq{}}\PY{l+s}{count\PYZus{}rate}\PY{l+s}{\PYZdq{}}\PY{p}{,}
                        \PY{l+s}{\PYZdq{}}\PY{l+s}{Count Rate;N\PYZus{}\PYZob{}Counts\PYZcb{};\PYZsh{} occurencies}\PY{l+s}{\PYZdq{}}\PY{p}{,}
                        \PY{l+m+mi}{100}\PY{p}{,} \PY{c+c1}{// Number of Bins}
                        \PY{o}{\PYZhy{}}\PY{l+m+mf}{0.5}\PY{p}{,} \PY{c+c1}{// Lower X Boundary}
                        \PY{l+m+mf}{15.5}\PY{p}{)}\PY{p}{;} \PY{c+c1}{// Upper X Boundary}
        
            \PY{k}{auto} \PY{n}{mean\PYZus{}count}\PY{o}{=}\PY{l+m+mf}{3.6f}\PY{p}{;}
            \PY{n}{TRandom3} \PY{n}{rndgen\PYZus{}5\PYZus{}1}\PY{p}{;}
            \PY{c+c1}{// simulate the measurements}
\end{Verbatim}

    \begin{itemize}
\tightlist
\item
  During each loop of the following for an entry is stored in the
  histogram through the TH1F::Fill method.
\end{itemize}

    \begin{Verbatim}[commandchars=\\\{\}]
{\color{incolor}In [{\color{incolor}3}]:} \PY{k}{for} \PY{p}{(}\PY{k+kt}{int} \PY{n}{imeas}\PY{o}{=}\PY{l+m+mi}{0}\PY{p}{;}\PY{n}{imeas}\PY{o}{\PYZlt{}}\PY{l+m+mi}{400}\PY{p}{;}\PY{n}{imeas}\PY{o}{+}\PY{o}{+}\PY{p}{)}
                \PY{n}{cnt\PYZus{}r\PYZus{}h}\PY{o}{\PYZhy{}}\PY{o}{\PYZgt{}}\PY{n}{Fill}\PY{p}{(}\PY{n}{rndgen\PYZus{}5\PYZus{}1}\PY{p}{.}\PY{n}{Poisson}\PY{p}{(}\PY{n}{mean\PYZus{}count}\PY{p}{)}\PY{p}{)}\PY{p}{;}
        
            \PY{k}{auto} \PY{n}{canvas\PYZus{}5\PYZus{}1}\PY{o}{=} \PY{k}{new} \PY{n}{TCanvas}\PY{p}{(}\PY{p}{)}\PY{p}{;}
            \PY{n}{cnt\PYZus{}r\PYZus{}h}\PY{o}{\PYZhy{}}\PY{o}{\PYZgt{}}\PY{n}{Draw}\PY{p}{(}\PY{p}{)}\PY{p}{;}
        
            \PY{k}{auto} \PY{n}{canvas\PYZus{}5\PYZus{}2}\PY{o}{=} \PY{k}{new} \PY{n}{TCanvas}\PY{p}{(}\PY{p}{)}\PY{p}{;}
\end{Verbatim}

    \begin{itemize}
\tightlist
\item
  The histogram can be drawn also normalised, ROOT automatically takes
  cares of the necessary rescaling.
\end{itemize}

    \begin{Verbatim}[commandchars=\\\{\}]
{\color{incolor}In [{\color{incolor}4}]:} \PY{n}{cnt\PYZus{}r\PYZus{}h}\PY{o}{\PYZhy{}}\PY{o}{\PYZgt{}}\PY{n}{DrawNormalized}\PY{p}{(}\PY{p}{)}\PY{p}{;}
\end{Verbatim}

    \begin{itemize}
\tightlist
\item
  This small snippet shows how easy it is to access the moments and
  associated errors of a histogram.
\end{itemize}

    \begin{Verbatim}[commandchars=\\\{\}]
{\color{incolor}In [{\color{incolor}5}]:} \PY{c+c1}{// Print summary}
            \PY{n}{cout} \PY{o}{\PYZlt{}}\PY{o}{\PYZlt{}} \PY{l+s}{\PYZdq{}}\PY{l+s}{Moments of Distribution:}\PY{l+s+se}{\PYZbs{}n}\PY{l+s}{\PYZdq{}}
                 \PY{o}{\PYZlt{}}\PY{o}{\PYZlt{}} \PY{l+s}{\PYZdq{}}\PY{l+s}{ \PYZhy{} Mean     = }\PY{l+s}{\PYZdq{}} \PY{o}{\PYZlt{}}\PY{o}{\PYZlt{}} \PY{n}{cnt\PYZus{}r\PYZus{}h}\PY{o}{\PYZhy{}}\PY{o}{\PYZgt{}}\PY{n}{GetMean}\PY{p}{(}\PY{p}{)} \PY{o}{\PYZlt{}}\PY{o}{\PYZlt{}} \PY{l+s}{\PYZdq{}}\PY{l+s}{ +\PYZhy{} }\PY{l+s}{\PYZdq{}}
                                     \PY{o}{\PYZlt{}}\PY{o}{\PYZlt{}} \PY{n}{cnt\PYZus{}r\PYZus{}h}\PY{o}{\PYZhy{}}\PY{o}{\PYZgt{}}\PY{n}{GetMeanError}\PY{p}{(}\PY{p}{)} \PY{o}{\PYZlt{}}\PY{o}{\PYZlt{}} \PY{l+s}{\PYZdq{}}\PY{l+s+se}{\PYZbs{}n}\PY{l+s}{\PYZdq{}}
                 \PY{o}{\PYZlt{}}\PY{o}{\PYZlt{}} \PY{l+s}{\PYZdq{}}\PY{l+s}{ \PYZhy{} Std Dev  = }\PY{l+s}{\PYZdq{}} \PY{o}{\PYZlt{}}\PY{o}{\PYZlt{}} \PY{n}{cnt\PYZus{}r\PYZus{}h}\PY{o}{\PYZhy{}}\PY{o}{\PYZgt{}}\PY{n}{GetStdDev}\PY{p}{(}\PY{p}{)} \PY{o}{\PYZlt{}}\PY{o}{\PYZlt{}} \PY{l+s}{\PYZdq{}}\PY{l+s}{ +\PYZhy{} }\PY{l+s}{\PYZdq{}}
                                     \PY{o}{\PYZlt{}}\PY{o}{\PYZlt{}} \PY{n}{cnt\PYZus{}r\PYZus{}h}\PY{o}{\PYZhy{}}\PY{o}{\PYZgt{}}\PY{n}{GetStdDevError}\PY{p}{(}\PY{p}{)} \PY{o}{\PYZlt{}}\PY{o}{\PYZlt{}} \PY{l+s}{\PYZdq{}}\PY{l+s+se}{\PYZbs{}n}\PY{l+s}{\PYZdq{}}
                 \PY{o}{\PYZlt{}}\PY{o}{\PYZlt{}} \PY{l+s}{\PYZdq{}}\PY{l+s}{ \PYZhy{} Skewness = }\PY{l+s}{\PYZdq{}} \PY{o}{\PYZlt{}}\PY{o}{\PYZlt{}} \PY{n}{cnt\PYZus{}r\PYZus{}h}\PY{o}{\PYZhy{}}\PY{o}{\PYZgt{}}\PY{n}{GetSkewness}\PY{p}{(}\PY{p}{)} \PY{o}{\PYZlt{}}\PY{o}{\PYZlt{}} \PY{l+s}{\PYZdq{}}\PY{l+s+se}{\PYZbs{}n}\PY{l+s}{\PYZdq{}}
                 \PY{o}{\PYZlt{}}\PY{o}{\PYZlt{}} \PY{l+s}{\PYZdq{}}\PY{l+s}{ \PYZhy{} Kurtosis = }\PY{l+s}{\PYZdq{}} \PY{o}{\PYZlt{}}\PY{o}{\PYZlt{}} \PY{n}{cnt\PYZus{}r\PYZus{}h}\PY{o}{\PYZhy{}}\PY{o}{\PYZgt{}}\PY{n}{GetKurtosis}\PY{p}{(}\PY{p}{)} \PY{o}{\PYZlt{}}\PY{o}{\PYZlt{}} \PY{l+s}{\PYZdq{}}\PY{l+s+se}{\PYZbs{}n}\PY{l+s}{\PYZdq{}}\PY{p}{;}
            
            \PY{n}{canvas\PYZus{}5\PYZus{}1}\PY{o}{\PYZhy{}}\PY{o}{\PYZgt{}}\PY{n}{Draw}\PY{p}{(}\PY{p}{)}\PY{p}{;}
            \PY{n}{canvas\PYZus{}5\PYZus{}2}\PY{o}{\PYZhy{}}\PY{o}{\PYZgt{}}\PY{n}{Draw}\PY{p}{(}\PY{p}{)}\PY{p}{;}
\end{Verbatim}

    \begin{Verbatim}[commandchars=\\\{\}]
Moments of Distribution:
 - Mean     = 3.5625 +- 0.0895976
 - Std Dev  = 1.79195 +- 0.0633551
 - Skewness = 0.326374
 - Kurtosis = -0.242483
    \end{Verbatim}

    
    \begin{verbatim}
<IPython.core.display.HTML object>
    \end{verbatim}

    
    
    \begin{verbatim}
<IPython.core.display.HTML object>
    \end{verbatim}

    
    \subsection{5.2 Add and Divide
Histograms}\label{add-and-divide-histograms}

Quite a large number of operations can be carried out with histograms.
The most useful are addition and division. In the following macro we
will learn how to manage these procedures within ROOT.

Some lines now need a bit of clarification:

\begin{itemize}
\tightlist
\item
  Cling, as we know, is also able to interpret more than one function
  per file. In this case the format\_h function simply sets up some
  parameters to conveniently set the line of histograms.
\end{itemize}

    \begin{Verbatim}[commandchars=\\\{\}]
{\color{incolor}In [{\color{incolor}6}]:} \PY{o}{\PYZpc{}}\PY{o}{\PYZpc{}}\PY{n}{cpp} \PY{o}{\PYZhy{}}\PY{n}{d}
        \PY{c+c1}{// Divide and add 1D Histograms}
        
        \PY{k+kt}{void} \PY{n}{format\PYZus{}h}\PY{p}{(}\PY{n}{TH1F}\PY{o}{*} \PY{n}{h}\PY{p}{,} \PY{k+kt}{int} \PY{n}{linecolor}\PY{p}{)}\PY{p}{\PYZob{}}
            \PY{n}{h}\PY{o}{\PYZhy{}}\PY{o}{\PYZgt{}}\PY{n}{SetLineWidth}\PY{p}{(}\PY{l+m+mi}{3}\PY{p}{)}\PY{p}{;}
            \PY{n}{h}\PY{o}{\PYZhy{}}\PY{o}{\PYZgt{}}\PY{n}{SetLineColor}\PY{p}{(}\PY{n}{linecolor}\PY{p}{)}\PY{p}{;}
            \PY{p}{\PYZcb{}}
\end{Verbatim}

    \begin{Verbatim}[commandchars=\\\{\}]
{\color{incolor}In [{\color{incolor}7}]:} \PY{k}{auto} \PY{n}{sig\PYZus{}h}\PY{o}{=}\PY{k}{new} \PY{n}{TH1F}\PY{p}{(}\PY{l+s}{\PYZdq{}}\PY{l+s}{sig\PYZus{}h}\PY{l+s}{\PYZdq{}}\PY{p}{,}\PY{l+s}{\PYZdq{}}\PY{l+s}{Signal Histo}\PY{l+s}{\PYZdq{}}\PY{p}{,}\PY{l+m+mi}{50}\PY{p}{,}\PY{l+m+mi}{0}\PY{p}{,}\PY{l+m+mi}{10}\PY{p}{)}\PY{p}{;}
            \PY{k}{auto} \PY{n}{gaus\PYZus{}h1}\PY{o}{=}\PY{k}{new} \PY{n}{TH1F}\PY{p}{(}\PY{l+s}{\PYZdq{}}\PY{l+s}{gaus\PYZus{}h1}\PY{l+s}{\PYZdq{}}\PY{p}{,}\PY{l+s}{\PYZdq{}}\PY{l+s}{Gauss Histo 1}\PY{l+s}{\PYZdq{}}\PY{p}{,}\PY{l+m+mi}{30}\PY{p}{,}\PY{l+m+mi}{0}\PY{p}{,}\PY{l+m+mi}{10}\PY{p}{)}\PY{p}{;}
            \PY{k}{auto} \PY{n}{gaus\PYZus{}h2}\PY{o}{=}\PY{k}{new} \PY{n}{TH1F}\PY{p}{(}\PY{l+s}{\PYZdq{}}\PY{l+s}{gaus\PYZus{}h2}\PY{l+s}{\PYZdq{}}\PY{p}{,}\PY{l+s}{\PYZdq{}}\PY{l+s}{Gauss Histo 2}\PY{l+s}{\PYZdq{}}\PY{p}{,}\PY{l+m+mi}{30}\PY{p}{,}\PY{l+m+mi}{0}\PY{p}{,}\PY{l+m+mi}{10}\PY{p}{)}\PY{p}{;}
            \PY{k}{auto} \PY{n}{bkg\PYZus{}h}\PY{o}{=}\PY{k}{new} \PY{n}{TH1F}\PY{p}{(}\PY{l+s}{\PYZdq{}}\PY{l+s}{exp\PYZus{}h}\PY{l+s}{\PYZdq{}}\PY{p}{,}\PY{l+s}{\PYZdq{}}\PY{l+s}{Exponential Histo}\PY{l+s}{\PYZdq{}}\PY{p}{,}\PY{l+m+mi}{50}\PY{p}{,}\PY{l+m+mi}{0}\PY{p}{,}\PY{l+m+mi}{10}\PY{p}{)}\PY{p}{;}
        
            \PY{c+c1}{// simulate the measurements}
            \PY{n}{TRandom3} \PY{n}{rndgen\PYZus{}5\PYZus{}2}\PY{p}{;}
\end{Verbatim}

    \begin{itemize}
\tightlist
\item
  Here some C++ syntax for conditional statements is used to fill the
  histograms with different numbers of entries inside the loop.
\end{itemize}

    \begin{Verbatim}[commandchars=\\\{\}]
{\color{incolor}In [{\color{incolor}8}]:} \PY{k}{for} \PY{p}{(}\PY{k+kt}{int} \PY{n}{imeas}\PY{o}{=}\PY{l+m+mi}{0}\PY{p}{;} \PY{n}{imeas}\PY{o}{\PYZlt{}}\PY{l+m+mi}{4000}\PY{p}{;} \PY{n}{imeas}\PY{o}{+}\PY{o}{+}\PY{p}{)}\PY{p}{\PYZob{}}
                \PY{n}{bkg\PYZus{}h}\PY{o}{\PYZhy{}}\PY{o}{\PYZgt{}}\PY{n}{Fill}\PY{p}{(}\PY{n}{rndgen\PYZus{}5\PYZus{}2}\PY{p}{.}\PY{n}{Exp}\PY{p}{(}\PY{l+m+mi}{4}\PY{p}{)}\PY{p}{)}\PY{p}{;}
                \PY{k}{if} \PY{p}{(}\PY{n}{imeas}\PY{o}{\PYZpc{}}\PY{l+m+mi}{4}\PY{o}{=}\PY{o}{=}\PY{l+m+mi}{0}\PY{p}{)} \PY{n}{gaus\PYZus{}h1}\PY{o}{\PYZhy{}}\PY{o}{\PYZgt{}}\PY{n}{Fill}\PY{p}{(}\PY{n}{rndgen\PYZus{}5\PYZus{}2}\PY{p}{.}\PY{n}{Gaus}\PY{p}{(}\PY{l+m+mi}{5}\PY{p}{,}\PY{l+m+mi}{2}\PY{p}{)}\PY{p}{)}\PY{p}{;}
                \PY{k}{if} \PY{p}{(}\PY{n}{imeas}\PY{o}{\PYZpc{}}\PY{l+m+mi}{4}\PY{o}{=}\PY{o}{=}\PY{l+m+mi}{0}\PY{p}{)} \PY{n}{gaus\PYZus{}h2}\PY{o}{\PYZhy{}}\PY{o}{\PYZgt{}}\PY{n}{Fill}\PY{p}{(}\PY{n}{rndgen\PYZus{}5\PYZus{}2}\PY{p}{.}\PY{n}{Gaus}\PY{p}{(}\PY{l+m+mi}{5}\PY{p}{,}\PY{l+m+mi}{2}\PY{p}{)}\PY{p}{)}\PY{p}{;}
                \PY{k}{if} \PY{p}{(}\PY{n}{imeas}\PY{o}{\PYZpc{}}\PY{l+m+mi}{10}\PY{o}{=}\PY{o}{=}\PY{l+m+mi}{0}\PY{p}{)}\PY{n}{sig\PYZus{}h}\PY{o}{\PYZhy{}}\PY{o}{\PYZgt{}}\PY{n}{Fill}\PY{p}{(}\PY{n}{rndgen\PYZus{}5\PYZus{}2}\PY{p}{.}\PY{n}{Gaus}\PY{p}{(}\PY{l+m+mi}{5}\PY{p}{,}\PY{l+m+mf}{.5}\PY{p}{)}\PY{p}{)}\PY{p}{;}\PY{p}{\PYZcb{}}
            
            \PY{c+c1}{// Format Histograms}
            \PY{k+kt}{int} \PY{n}{i}\PY{o}{=}\PY{l+m+mi}{0}\PY{p}{;}
            \PY{k}{for} \PY{p}{(}\PY{k}{auto} \PY{n+nl}{hist} \PY{p}{:} \PY{p}{\PYZob{}}\PY{n}{sig\PYZus{}h}\PY{p}{,}\PY{n}{bkg\PYZus{}h}\PY{p}{,}\PY{n}{gaus\PYZus{}h1}\PY{p}{,}\PY{n}{gaus\PYZus{}h2}\PY{p}{\PYZcb{}}\PY{p}{)}
                \PY{n}{format\PYZus{}h}\PY{p}{(}\PY{n}{hist}\PY{p}{,}\PY{l+m+mi}{1}\PY{o}{+}\PY{n}{i}\PY{o}{+}\PY{o}{+}\PY{p}{)}\PY{p}{;}
        
            \PY{c+c1}{// Sum}
            \PY{k}{auto} \PY{n}{sum\PYZus{}h}\PY{o}{=} \PY{k}{new} \PY{n}{TH1F}\PY{p}{(}\PY{o}{*}\PY{n}{bkg\PYZus{}h}\PY{p}{)}\PY{p}{;}
\end{Verbatim}

    \begin{itemize}
\tightlist
\item
  Here the sum of two histograms. A weight, which can be negative, can
  be assigned to the added histogram.
\end{itemize}

    \begin{Verbatim}[commandchars=\\\{\}]
{\color{incolor}In [{\color{incolor}9}]:} \PY{n}{sum\PYZus{}h}\PY{o}{\PYZhy{}}\PY{o}{\PYZgt{}}\PY{n}{Add}\PY{p}{(}\PY{n}{sig\PYZus{}h}\PY{p}{,}\PY{l+m+mf}{1.}\PY{p}{)}\PY{p}{;}
            \PY{n}{sum\PYZus{}h}\PY{o}{\PYZhy{}}\PY{o}{\PYZgt{}}\PY{n}{SetTitle}\PY{p}{(}\PY{l+s}{\PYZdq{}}\PY{l+s}{Exponential + Gaussian;X variable;Y variable}\PY{l+s}{\PYZdq{}}\PY{p}{)}\PY{p}{;}
            \PY{n}{format\PYZus{}h}\PY{p}{(}\PY{n}{sum\PYZus{}h}\PY{p}{,}\PY{n}{kBlue}\PY{p}{)}\PY{p}{;}
        
            \PY{k}{auto} \PY{n}{canvas\PYZus{}5\PYZus{}2\PYZus{}sum}\PY{o}{=} \PY{k}{new} \PY{n}{TCanvas}\PY{p}{(}\PY{p}{)}\PY{p}{;}
            \PY{n}{sum\PYZus{}h}\PY{o}{\PYZhy{}}\PY{o}{\PYZgt{}}\PY{n}{Draw}\PY{p}{(}\PY{l+s}{\PYZdq{}}\PY{l+s}{hist}\PY{l+s}{\PYZdq{}}\PY{p}{)}\PY{p}{;}
            \PY{n}{bkg\PYZus{}h}\PY{o}{\PYZhy{}}\PY{o}{\PYZgt{}}\PY{n}{Draw}\PY{p}{(}\PY{l+s}{\PYZdq{}}\PY{l+s}{SameHist}\PY{l+s}{\PYZdq{}}\PY{p}{)}\PY{p}{;}
            \PY{n}{sig\PYZus{}h}\PY{o}{\PYZhy{}}\PY{o}{\PYZgt{}}\PY{n}{Draw}\PY{p}{(}\PY{l+s}{\PYZdq{}}\PY{l+s}{SameHist}\PY{l+s}{\PYZdq{}}\PY{p}{)}\PY{p}{;}
\end{Verbatim}

    \begin{itemize}
\tightlist
\item
  The division of two histograms is rather straightforward.
\end{itemize}

    \begin{Verbatim}[commandchars=\\\{\}]
{\color{incolor}In [{\color{incolor}10}]:} \PY{c+c1}{// Divide}
             \PY{k}{auto} \PY{n}{dividend}\PY{o}{=}\PY{k}{new} \PY{n}{TH1F}\PY{p}{(}\PY{o}{*}\PY{n}{gaus\PYZus{}h1}\PY{p}{)}\PY{p}{;}
             \PY{n}{dividend}\PY{o}{\PYZhy{}}\PY{o}{\PYZgt{}}\PY{n}{Divide}\PY{p}{(}\PY{n}{gaus\PYZus{}h2}\PY{p}{)}\PY{p}{;}
\end{Verbatim}

    \begin{itemize}
\tightlist
\item
  When you draw two quantities and their ratios, it is much better if
  all the information is condensed in one single plot. These lines
  provide a skeleton to perform this operation.
\end{itemize}

    \begin{Verbatim}[commandchars=\\\{\}]
{\color{incolor}In [{\color{incolor}11}]:} \PY{c+c1}{// Graphical Maquillage}
             \PY{n}{dividend}\PY{o}{\PYZhy{}}\PY{o}{\PYZgt{}}\PY{n}{SetTitle}\PY{p}{(}\PY{l+s}{\PYZdq{}}\PY{l+s}{;X axis;Gaus Histo 1 / Gaus Histo 2}\PY{l+s}{\PYZdq{}}\PY{p}{)}\PY{p}{;}
             \PY{n}{format\PYZus{}h}\PY{p}{(}\PY{n}{dividend}\PY{p}{,}\PY{n}{kOrange}\PY{p}{)}\PY{p}{;}
             \PY{n}{gaus\PYZus{}h1}\PY{o}{\PYZhy{}}\PY{o}{\PYZgt{}}\PY{n}{SetTitle}\PY{p}{(}\PY{l+s}{\PYZdq{}}\PY{l+s}{;;Gaus Histo 1 and Gaus Histo 2}\PY{l+s}{\PYZdq{}}\PY{p}{)}\PY{p}{;}
             \PY{n}{gStyle}\PY{o}{\PYZhy{}}\PY{o}{\PYZgt{}}\PY{n}{SetOptStat}\PY{p}{(}\PY{l+m+mi}{0}\PY{p}{)}\PY{p}{;}
         
             \PY{n}{TCanvas}\PY{o}{*} \PY{n}{canvas\PYZus{}5\PYZus{}2\PYZus{}divide}\PY{o}{=} \PY{k}{new} \PY{n}{TCanvas}\PY{p}{(}\PY{p}{)}\PY{p}{;}
             \PY{n}{canvas\PYZus{}5\PYZus{}2\PYZus{}divide}\PY{o}{\PYZhy{}}\PY{o}{\PYZgt{}}\PY{n}{Divide}\PY{p}{(}\PY{l+m+mi}{1}\PY{p}{,}\PY{l+m+mi}{2}\PY{p}{,}\PY{l+m+mi}{0}\PY{p}{,}\PY{l+m+mi}{0}\PY{p}{)}\PY{p}{;}
             \PY{n}{canvas\PYZus{}5\PYZus{}2\PYZus{}divide}\PY{o}{\PYZhy{}}\PY{o}{\PYZgt{}}\PY{n}{cd}\PY{p}{(}\PY{l+m+mi}{1}\PY{p}{)}\PY{p}{;}
             \PY{n}{canvas\PYZus{}5\PYZus{}2\PYZus{}divide}\PY{o}{\PYZhy{}}\PY{o}{\PYZgt{}}\PY{n}{GetPad}\PY{p}{(}\PY{l+m+mi}{1}\PY{p}{)}\PY{o}{\PYZhy{}}\PY{o}{\PYZgt{}}\PY{n}{SetRightMargin}\PY{p}{(}\PY{l+m+mf}{.01}\PY{p}{)}\PY{p}{;}
             \PY{n}{gaus\PYZus{}h1}\PY{o}{\PYZhy{}}\PY{o}{\PYZgt{}}\PY{n}{DrawNormalized}\PY{p}{(}\PY{l+s}{\PYZdq{}}\PY{l+s}{Hist}\PY{l+s}{\PYZdq{}}\PY{p}{)}\PY{p}{;}
             \PY{n}{gaus\PYZus{}h2}\PY{o}{\PYZhy{}}\PY{o}{\PYZgt{}}\PY{n}{DrawNormalized}\PY{p}{(}\PY{l+s}{\PYZdq{}}\PY{l+s}{HistSame}\PY{l+s}{\PYZdq{}}\PY{p}{)}\PY{p}{;}
         
             \PY{n}{canvas\PYZus{}5\PYZus{}2\PYZus{}divide}\PY{o}{\PYZhy{}}\PY{o}{\PYZgt{}}\PY{n}{cd}\PY{p}{(}\PY{l+m+mi}{2}\PY{p}{)}\PY{p}{;}
             \PY{n}{dividend}\PY{o}{\PYZhy{}}\PY{o}{\PYZgt{}}\PY{n}{GetYaxis}\PY{p}{(}\PY{p}{)}\PY{o}{\PYZhy{}}\PY{o}{\PYZgt{}}\PY{n}{SetRangeUser}\PY{p}{(}\PY{l+m+mi}{0}\PY{p}{,}\PY{l+m+mf}{2.49}\PY{p}{)}\PY{p}{;}
             \PY{n}{canvas\PYZus{}5\PYZus{}2\PYZus{}divide}\PY{o}{\PYZhy{}}\PY{o}{\PYZgt{}}\PY{n}{GetPad}\PY{p}{(}\PY{l+m+mi}{2}\PY{p}{)}\PY{o}{\PYZhy{}}\PY{o}{\PYZgt{}}\PY{n}{SetGridy}\PY{p}{(}\PY{p}{)}\PY{p}{;}
             \PY{n}{canvas\PYZus{}5\PYZus{}2\PYZus{}divide}\PY{o}{\PYZhy{}}\PY{o}{\PYZgt{}}\PY{n}{GetPad}\PY{p}{(}\PY{l+m+mi}{2}\PY{p}{)}\PY{o}{\PYZhy{}}\PY{o}{\PYZgt{}}\PY{n}{SetRightMargin}\PY{p}{(}\PY{l+m+mf}{.01}\PY{p}{)}\PY{p}{;}
             \PY{n}{dividend}\PY{o}{\PYZhy{}}\PY{o}{\PYZgt{}}\PY{n}{Draw}\PY{p}{(}\PY{p}{)}\PY{p}{;}
             
             \PY{n}{canvas\PYZus{}5\PYZus{}2\PYZus{}sum}\PY{o}{\PYZhy{}}\PY{o}{\PYZgt{}}\PY{n}{Draw}\PY{p}{(}\PY{p}{)}\PY{p}{;}
             \PY{n}{canvas\PYZus{}5\PYZus{}2\PYZus{}divide}\PY{o}{\PYZhy{}}\PY{o}{\PYZgt{}}\PY{n}{Draw}\PY{p}{(}\PY{p}{)}\PY{p}{;}
\end{Verbatim}

    
    \begin{verbatim}
<IPython.core.display.HTML object>
    \end{verbatim}

    
    
    \begin{verbatim}
<IPython.core.display.HTML object>
    \end{verbatim}

    
    \subsection{5.3 Two-dimensional
Histograms}\label{two-dimensional-histograms}

Two-dimensional histograms are a very useful tool, for example to
inspect correlations between variables. You can exploit the
bi-dimensional histogram classes provided by ROOT in a simple way. Let's
see how in this code:

    \begin{Verbatim}[commandchars=\\\{\}]
{\color{incolor}In [{\color{incolor}12}]:} \PY{c+c1}{// Draw a Bidimensional Histogram in many ways}
         \PY{c+c1}{// together with its profiles and projections}
         
             \PY{n}{gStyle}\PY{o}{\PYZhy{}}\PY{o}{\PYZgt{}}\PY{n}{SetPalette}\PY{p}{(}\PY{n}{kBird}\PY{p}{)}\PY{p}{;}
             \PY{n}{gStyle}\PY{o}{\PYZhy{}}\PY{o}{\PYZgt{}}\PY{n}{SetOptStat}\PY{p}{(}\PY{l+m+mi}{0}\PY{p}{)}\PY{p}{;}
             \PY{n}{gStyle}\PY{o}{\PYZhy{}}\PY{o}{\PYZgt{}}\PY{n}{SetOptTitle}\PY{p}{(}\PY{l+m+mi}{0}\PY{p}{)}\PY{p}{;}
         
             \PY{n}{TH2F} \PY{n}{bidi\PYZus{}h}\PY{p}{(}\PY{l+s}{\PYZdq{}}\PY{l+s}{bidi\PYZus{}h}\PY{l+s}{\PYZdq{}}\PY{p}{,}\PY{l+s}{\PYZdq{}}\PY{l+s}{2D Histo;Gaussian Vals;Exp. Vals}\PY{l+s}{\PYZdq{}}\PY{p}{,}
                         \PY{l+m+mi}{30}\PY{p}{,}\PY{o}{\PYZhy{}}\PY{l+m+mi}{5}\PY{p}{,}\PY{l+m+mi}{5}\PY{p}{,}  \PY{c+c1}{// X axis}
                         \PY{l+m+mi}{30}\PY{p}{,}\PY{l+m+mi}{0}\PY{p}{,}\PY{l+m+mi}{10}\PY{p}{)}\PY{p}{;} \PY{c+c1}{// Y axis}
         
             \PY{n}{TRandom3} \PY{n}{rgen\PYZus{}5\PYZus{}3}\PY{p}{;}
             \PY{k}{for} \PY{p}{(}\PY{k+kt}{int} \PY{n}{i}\PY{o}{=}\PY{l+m+mi}{0}\PY{p}{;}\PY{n}{i}\PY{o}{\PYZlt{}}\PY{l+m+mi}{500000}\PY{p}{;}\PY{n}{i}\PY{o}{+}\PY{o}{+}\PY{p}{)}
                 \PY{n}{bidi\PYZus{}h}\PY{p}{.}\PY{n}{Fill}\PY{p}{(}\PY{n}{rgen\PYZus{}5\PYZus{}3}\PY{p}{.}\PY{n}{Gaus}\PY{p}{(}\PY{l+m+mi}{0}\PY{p}{,}\PY{l+m+mi}{2}\PY{p}{)}\PY{p}{,}\PY{l+m+mi}{10}\PY{o}{\PYZhy{}}\PY{n}{rgen\PYZus{}5\PYZus{}3}\PY{p}{.}\PY{n}{Exp}\PY{p}{(}\PY{l+m+mi}{4}\PY{p}{)}\PY{p}{,}\PY{l+m+mf}{.1}\PY{p}{)}\PY{p}{;}
         
             \PY{k}{auto} \PY{n}{canvas\PYZus{}5\PYZus{}3\PYZus{}1}\PY{o}{=}\PY{k}{new} \PY{n}{TCanvas}\PY{p}{(}\PY{l+s}{\PYZdq{}}\PY{l+s}{canvas\PYZus{}5\PYZus{}3}\PY{l+s}{\PYZdq{}}\PY{p}{,}\PY{l+s}{\PYZdq{}}\PY{l+s}{canvas\PYZus{}5\PYZus{}3}\PY{l+s}{\PYZdq{}}\PY{p}{,}\PY{l+m+mi}{800}\PY{p}{,}\PY{l+m+mi}{800}\PY{p}{)}\PY{p}{;}
             \PY{n}{canvas\PYZus{}5\PYZus{}3\PYZus{}1}\PY{o}{\PYZhy{}}\PY{o}{\PYZgt{}}\PY{n}{Divide}\PY{p}{(}\PY{l+m+mi}{2}\PY{p}{,}\PY{l+m+mi}{2}\PY{p}{)}\PY{p}{;}
             \PY{n}{canvas\PYZus{}5\PYZus{}3\PYZus{}1}\PY{o}{\PYZhy{}}\PY{o}{\PYZgt{}}\PY{n}{cd}\PY{p}{(}\PY{l+m+mi}{1}\PY{p}{)}\PY{p}{;}\PY{n}{bidi\PYZus{}h}\PY{p}{.}\PY{n}{DrawClone}\PY{p}{(}\PY{l+s}{\PYZdq{}}\PY{l+s}{Cont1}\PY{l+s}{\PYZdq{}}\PY{p}{)}\PY{p}{;}
             \PY{n}{canvas\PYZus{}5\PYZus{}3\PYZus{}1}\PY{o}{\PYZhy{}}\PY{o}{\PYZgt{}}\PY{n}{cd}\PY{p}{(}\PY{l+m+mi}{2}\PY{p}{)}\PY{p}{;}\PY{n}{bidi\PYZus{}h}\PY{p}{.}\PY{n}{DrawClone}\PY{p}{(}\PY{l+s}{\PYZdq{}}\PY{l+s}{Colz}\PY{l+s}{\PYZdq{}}\PY{p}{)}\PY{p}{;}
             \PY{n}{canvas\PYZus{}5\PYZus{}3\PYZus{}1}\PY{o}{\PYZhy{}}\PY{o}{\PYZgt{}}\PY{n}{cd}\PY{p}{(}\PY{l+m+mi}{3}\PY{p}{)}\PY{p}{;}\PY{n}{bidi\PYZus{}h}\PY{p}{.}\PY{n}{DrawClone}\PY{p}{(}\PY{l+s}{\PYZdq{}}\PY{l+s}{lego2}\PY{l+s}{\PYZdq{}}\PY{p}{)}\PY{p}{;}
             \PY{n}{canvas\PYZus{}5\PYZus{}3\PYZus{}1}\PY{o}{\PYZhy{}}\PY{o}{\PYZgt{}}\PY{n}{cd}\PY{p}{(}\PY{l+m+mi}{4}\PY{p}{)}\PY{p}{;}\PY{n}{bidi\PYZus{}h}\PY{p}{.}\PY{n}{DrawClone}\PY{p}{(}\PY{l+s}{\PYZdq{}}\PY{l+s}{surf3}\PY{l+s}{\PYZdq{}}\PY{p}{)}\PY{p}{;}
         
             \PY{c+c1}{// Profiles and Projections}
             \PY{k}{auto} \PY{n}{canvas\PYZus{}5\PYZus{}3\PYZus{}2}\PY{o}{=}\PY{k}{new} \PY{n}{TCanvas}\PY{p}{(}\PY{l+s}{\PYZdq{}}\PY{l+s}{canvas\PYZus{}5\PYZus{}3\PYZus{}2}\PY{l+s}{\PYZdq{}}\PY{p}{,}\PY{l+s}{\PYZdq{}}\PY{l+s}{canvas\PYZus{}5\PYZus{}3\PYZus{}2}\PY{l+s}{\PYZdq{}}\PY{p}{,}\PY{l+m+mi}{800}\PY{p}{,}\PY{l+m+mi}{800}\PY{p}{)}\PY{p}{;}
             \PY{n}{canvas\PYZus{}5\PYZus{}3\PYZus{}2}\PY{o}{\PYZhy{}}\PY{o}{\PYZgt{}}\PY{n}{Divide}\PY{p}{(}\PY{l+m+mi}{2}\PY{p}{,}\PY{l+m+mi}{2}\PY{p}{)}\PY{p}{;}
             \PY{n}{canvas\PYZus{}5\PYZus{}3\PYZus{}2}\PY{o}{\PYZhy{}}\PY{o}{\PYZgt{}}\PY{n}{cd}\PY{p}{(}\PY{l+m+mi}{1}\PY{p}{)}\PY{p}{;}\PY{n}{bidi\PYZus{}h}\PY{p}{.}\PY{n}{ProjectionX}\PY{p}{(}\PY{p}{)}\PY{o}{\PYZhy{}}\PY{o}{\PYZgt{}}\PY{n}{DrawClone}\PY{p}{(}\PY{p}{)}\PY{p}{;}
             \PY{n}{canvas\PYZus{}5\PYZus{}3\PYZus{}2}\PY{o}{\PYZhy{}}\PY{o}{\PYZgt{}}\PY{n}{cd}\PY{p}{(}\PY{l+m+mi}{2}\PY{p}{)}\PY{p}{;}\PY{n}{bidi\PYZus{}h}\PY{p}{.}\PY{n}{ProjectionY}\PY{p}{(}\PY{p}{)}\PY{o}{\PYZhy{}}\PY{o}{\PYZgt{}}\PY{n}{DrawClone}\PY{p}{(}\PY{p}{)}\PY{p}{;}
             \PY{n}{canvas\PYZus{}5\PYZus{}3\PYZus{}2}\PY{o}{\PYZhy{}}\PY{o}{\PYZgt{}}\PY{n}{cd}\PY{p}{(}\PY{l+m+mi}{3}\PY{p}{)}\PY{p}{;}\PY{n}{bidi\PYZus{}h}\PY{p}{.}\PY{n}{ProfileX}\PY{p}{(}\PY{p}{)}\PY{o}{\PYZhy{}}\PY{o}{\PYZgt{}}\PY{n}{DrawClone}\PY{p}{(}\PY{p}{)}\PY{p}{;}
             \PY{n}{canvas\PYZus{}5\PYZus{}3\PYZus{}2}\PY{o}{\PYZhy{}}\PY{o}{\PYZgt{}}\PY{n}{cd}\PY{p}{(}\PY{l+m+mi}{4}\PY{p}{)}\PY{p}{;}\PY{n}{bidi\PYZus{}h}\PY{p}{.}\PY{n}{ProfileY}\PY{p}{(}\PY{p}{)}\PY{o}{\PYZhy{}}\PY{o}{\PYZgt{}}\PY{n}{DrawClone}\PY{p}{(}\PY{p}{)}\PY{p}{;}
             
             \PY{n}{canvas\PYZus{}5\PYZus{}3\PYZus{}1}\PY{o}{\PYZhy{}}\PY{o}{\PYZgt{}}\PY{n}{Draw}\PY{p}{(}\PY{p}{)}\PY{p}{;}
             \PY{n}{canvas\PYZus{}5\PYZus{}3\PYZus{}2}\PY{o}{\PYZhy{}}\PY{o}{\PYZgt{}}\PY{n}{Draw}\PY{p}{(}\PY{p}{)}\PY{p}{;}
\end{Verbatim}

    
    \begin{verbatim}
<IPython.core.display.HTML object>
    \end{verbatim}

    
    
    \begin{verbatim}
<IPython.core.display.HTML object>
    \end{verbatim}

    
    Two kinds of plots are provided within the code, the first one
containing three-dimensional representations and the second one
projections and profiles of the bi-dimensional histogram.

The projections and profiles of bi-dimensional histograms.

When a projection is performed along the x (y) direction, for every bin
along the x (y) axis, all bin contents along the y (x) axis are summed
up. When a profile is performed along the x (y) direction, for every bin
along the x (y) axis, the average of all the bin contents along the y
(x) is calculated together with their RMS and displayed as a symbol with
error bar.

Correlations between the variables are quantified by the methods
\texttt{Double\_t\ GetCovariance()} and
\texttt{Double\_t\ GetCorrelationFactor()}.

\subsection{5.4 Multiple histograms}\label{multiple-histograms}

The class \texttt{THStack} allows to manipulate a set of histograms as a
single entity. It is a collection of \texttt{TH1} (or derived) objects.
When drawn, the X and Y axis ranges are automatically computed such as
all the histograms will be visible. Several drawing option are available
for both 1D and 2D histograms. The next macros shows how it looks for 2D
histograms:

    \begin{Verbatim}[commandchars=\\\{\}]
{\color{incolor}In [{\color{incolor}13}]:} \PY{c+c1}{// Example of stacked histograms using the class THStack}
            \PY{k}{auto} \PY{n}{canvas\PYZus{}5\PYZus{}4}\PY{o}{=}\PY{k}{new} \PY{n}{TCanvas}\PY{p}{(}\PY{l+s}{\PYZdq{}}\PY{l+s}{canvas\PYZus{}5\PYZus{}4}\PY{l+s}{\PYZdq{}}\PY{p}{,}\PY{l+s}{\PYZdq{}}\PY{l+s}{canvas\PYZus{}5\PYZus{}4}\PY{l+s}{\PYZdq{}}\PY{p}{,} \PY{l+m+mi}{900}\PY{p}{,} \PY{l+m+mi}{700}\PY{p}{)}\PY{p}{;}
\end{Verbatim}

    \begin{itemize}
\tightlist
\item
  Here we create the stack.
\end{itemize}

    \begin{Verbatim}[commandchars=\\\{\}]
{\color{incolor}In [{\color{incolor}14}]:} \PY{n}{THStack} \PY{o}{*}\PY{n}{stHistogram\PYZus{}5\PYZus{}4} \PY{o}{=} \PY{k}{new} \PY{n}{THStack}\PY{p}{(}\PY{l+s}{\PYZdq{}}\PY{l+s}{stHistogram\PYZus{}5\PYZus{}4}\PY{l+s}{\PYZdq{}}\PY{p}{,}\PY{l+s}{\PYZdq{}}\PY{l+s}{Stacked 2D histograms}\PY{l+s}{\PYZdq{}}\PY{p}{)}\PY{p}{;}
\end{Verbatim}

    \begin{itemize}
\tightlist
\item
  Here we create two histograms to be added in the stack.
\end{itemize}

    \begin{Verbatim}[commandchars=\\\{\}]
{\color{incolor}In [{\color{incolor}15}]:} \PY{n}{TF2} \PY{o}{*}\PY{n}{f1} \PY{o}{=} \PY{k}{new} \PY{n}{TF2}\PY{p}{(}\PY{l+s}{\PYZdq{}}\PY{l+s}{f1}\PY{l+s}{\PYZdq{}}\PY{p}{,}\PY{l+s}{\PYZdq{}}\PY{l+s}{xygaus + xygaus(5) + xylandau(10)}\PY{l+s}{\PYZdq{}}\PY{p}{,}\PY{o}{\PYZhy{}}\PY{l+m+mi}{4}\PY{p}{,}\PY{l+m+mi}{4}\PY{p}{,}\PY{o}{\PYZhy{}}\PY{l+m+mi}{4}\PY{p}{,}\PY{l+m+mi}{4}\PY{p}{)}\PY{p}{;}
            \PY{n}{Double\PYZus{}t} \PY{n}{params1}\PY{p}{[}\PY{p}{]} \PY{o}{=} \PY{p}{\PYZob{}}\PY{l+m+mi}{130}\PY{p}{,}\PY{o}{\PYZhy{}}\PY{l+m+mf}{1.4}\PY{p}{,}\PY{l+m+mf}{1.8}\PY{p}{,}\PY{l+m+mf}{1.5}\PY{p}{,}\PY{l+m+mi}{1}\PY{p}{,} \PY{l+m+mi}{150}\PY{p}{,}\PY{l+m+mi}{2}\PY{p}{,}\PY{l+m+mf}{0.5}\PY{p}{,}\PY{o}{\PYZhy{}}\PY{l+m+mi}{2}\PY{p}{,}\PY{l+m+mf}{0.5}\PY{p}{,} \PY{l+m+mi}{3600}\PY{p}{,}\PY{o}{\PYZhy{}}\PY{l+m+mi}{2}\PY{p}{,}\PY{l+m+mf}{0.7}\PY{p}{,}\PY{o}{\PYZhy{}}\PY{l+m+mi}{3}\PY{p}{,}\PY{l+m+mf}{0.3}\PY{p}{\PYZcb{}}\PY{p}{;}
            \PY{n}{f1}\PY{o}{\PYZhy{}}\PY{o}{\PYZgt{}}\PY{n}{SetParameters}\PY{p}{(}\PY{n}{params1}\PY{p}{)}\PY{p}{;}
            \PY{n}{TH2F} \PY{o}{*}\PY{n}{histogram\PYZus{}5\PYZus{}4\PYZus{}1} \PY{o}{=} \PY{k}{new} \PY{n}{TH2F}\PY{p}{(}\PY{l+s}{\PYZdq{}}\PY{l+s}{histogram\PYZus{}5\PYZus{}4\PYZus{}1}\PY{l+s}{\PYZdq{}}\PY{p}{,}\PY{l+s}{\PYZdq{}}\PY{l+s}{histogram\PYZus{}5\PYZus{}4\PYZus{}1}\PY{l+s}{\PYZdq{}}\PY{p}{,}\PY{l+m+mi}{20}\PY{p}{,}\PY{o}{\PYZhy{}}\PY{l+m+mi}{4}\PY{p}{,}\PY{l+m+mi}{4}\PY{p}{,}\PY{l+m+mi}{20}\PY{p}{,}\PY{o}{\PYZhy{}}\PY{l+m+mi}{4}\PY{p}{,}\PY{l+m+mi}{4}\PY{p}{)}\PY{p}{;}
            \PY{n}{histogram\PYZus{}5\PYZus{}4\PYZus{}1}\PY{o}{\PYZhy{}}\PY{o}{\PYZgt{}}\PY{n}{SetFillColor}\PY{p}{(}\PY{l+m+mi}{38}\PY{p}{)}\PY{p}{;}
            \PY{n}{histogram\PYZus{}5\PYZus{}4\PYZus{}1}\PY{o}{\PYZhy{}}\PY{o}{\PYZgt{}}\PY{n}{FillRandom}\PY{p}{(}\PY{l+s}{\PYZdq{}}\PY{l+s}{f1}\PY{l+s}{\PYZdq{}}\PY{p}{,}\PY{l+m+mi}{4000}\PY{p}{)}\PY{p}{;}
         
            \PY{n}{TF2} \PY{o}{*}\PY{n}{f2} \PY{o}{=} \PY{k}{new} \PY{n}{TF2}\PY{p}{(}\PY{l+s}{\PYZdq{}}\PY{l+s}{f2}\PY{l+s}{\PYZdq{}}\PY{p}{,}\PY{l+s}{\PYZdq{}}\PY{l+s}{xygaus + xygaus(5)}\PY{l+s}{\PYZdq{}}\PY{p}{,}\PY{o}{\PYZhy{}}\PY{l+m+mi}{4}\PY{p}{,}\PY{l+m+mi}{4}\PY{p}{,}\PY{o}{\PYZhy{}}\PY{l+m+mi}{4}\PY{p}{,}\PY{l+m+mi}{4}\PY{p}{)}\PY{p}{;}
            \PY{n}{Double\PYZus{}t} \PY{n}{params2}\PY{p}{[}\PY{p}{]} \PY{o}{=} \PY{p}{\PYZob{}}\PY{l+m+mi}{100}\PY{p}{,}\PY{o}{\PYZhy{}}\PY{l+m+mf}{1.4}\PY{p}{,}\PY{l+m+mf}{1.9}\PY{p}{,}\PY{l+m+mf}{1.1}\PY{p}{,}\PY{l+m+mi}{2}\PY{p}{,} \PY{l+m+mi}{80}\PY{p}{,}\PY{l+m+mi}{2}\PY{p}{,}\PY{l+m+mf}{0.7}\PY{p}{,}\PY{o}{\PYZhy{}}\PY{l+m+mi}{2}\PY{p}{,}\PY{l+m+mf}{0.5}\PY{p}{\PYZcb{}}\PY{p}{;}
            \PY{n}{f2}\PY{o}{\PYZhy{}}\PY{o}{\PYZgt{}}\PY{n}{SetParameters}\PY{p}{(}\PY{n}{params2}\PY{p}{)}\PY{p}{;}
            \PY{n}{TH2F} \PY{o}{*}\PY{n}{histogram\PYZus{}5\PYZus{}4\PYZus{}2} \PY{o}{=} \PY{k}{new} \PY{n}{TH2F}\PY{p}{(}\PY{l+s}{\PYZdq{}}\PY{l+s}{histogram\PYZus{}5\PYZus{}4\PYZus{}2}\PY{l+s}{\PYZdq{}}\PY{p}{,}\PY{l+s}{\PYZdq{}}\PY{l+s}{histogram\PYZus{}5\PYZus{}4\PYZus{}2}\PY{l+s}{\PYZdq{}}\PY{p}{,}\PY{l+m+mi}{20}\PY{p}{,}\PY{o}{\PYZhy{}}\PY{l+m+mi}{4}\PY{p}{,}\PY{l+m+mi}{4}\PY{p}{,}\PY{l+m+mi}{20}\PY{p}{,}\PY{o}{\PYZhy{}}\PY{l+m+mi}{4}\PY{p}{,}\PY{l+m+mi}{4}\PY{p}{)}\PY{p}{;}
            \PY{n}{histogram\PYZus{}5\PYZus{}4\PYZus{}2}\PY{o}{\PYZhy{}}\PY{o}{\PYZgt{}}\PY{n}{SetFillColor}\PY{p}{(}\PY{l+m+mi}{46}\PY{p}{)}\PY{p}{;}
            \PY{n}{histogram\PYZus{}5\PYZus{}4\PYZus{}2}\PY{o}{\PYZhy{}}\PY{o}{\PYZgt{}}\PY{n}{FillRandom}\PY{p}{(}\PY{l+s}{\PYZdq{}}\PY{l+s}{f2}\PY{l+s}{\PYZdq{}}\PY{p}{,}\PY{l+m+mi}{3000}\PY{p}{)}\PY{p}{;}
\end{Verbatim}

    \begin{itemize}
\tightlist
\item
  Here we add the histograms in the stack.
\end{itemize}

    \begin{Verbatim}[commandchars=\\\{\}]
{\color{incolor}In [{\color{incolor}16}]:} \PY{n}{stHistogram\PYZus{}5\PYZus{}4}\PY{o}{\PYZhy{}}\PY{o}{\PYZgt{}}\PY{n}{Add}\PY{p}{(}\PY{n}{histogram\PYZus{}5\PYZus{}4\PYZus{}1}\PY{p}{)}\PY{p}{;}
            \PY{n}{stHistogram\PYZus{}5\PYZus{}4}\PY{o}{\PYZhy{}}\PY{o}{\PYZgt{}}\PY{n}{Add}\PY{p}{(}\PY{n}{histogram\PYZus{}5\PYZus{}4\PYZus{}2}\PY{p}{)}\PY{p}{;}
\end{Verbatim}

    \begin{itemize}
\tightlist
\item
  Finally we draw the stack as a lego plot. In which the colour
  distinguish the two histograms.
\end{itemize}

    \begin{Verbatim}[commandchars=\\\{\}]
{\color{incolor}In [{\color{incolor}17}]:} \PY{n}{stHistogram\PYZus{}5\PYZus{}4}\PY{o}{\PYZhy{}}\PY{o}{\PYZgt{}}\PY{n}{Draw}\PY{p}{(}\PY{p}{)}\PY{p}{;}
            \PY{n}{canvas\PYZus{}5\PYZus{}4}\PY{o}{\PYZhy{}}\PY{o}{\PYZgt{}}\PY{n}{Draw}\PY{p}{(}\PY{p}{)}\PY{p}{;}
\end{Verbatim}

    
    \begin{verbatim}
<IPython.core.display.HTML object>
    \end{verbatim}

    

    % Add a bibliography block to the postdoc
    
    
    
    \end{document}
