
% Default to the notebook output style

    


% Inherit from the specified cell style.




    
\documentclass{article}

    
    
    \usepackage{graphicx} % Used to insert images
    \usepackage{adjustbox} % Used to constrain images to a maximum size 
    \usepackage{color} % Allow colors to be defined
    \usepackage{enumerate} % Needed for markdown enumerations to work
    \usepackage{geometry} % Used to adjust the document margins
    \usepackage{amsmath} % Equations
    \usepackage{amssymb} % Equations
    \usepackage{eurosym} % defines \euro
    \usepackage[mathletters]{ucs} % Extended unicode (utf-8) support
    \usepackage[utf8x]{inputenc} % Allow utf-8 characters in the tex document
    \usepackage{fancyvrb} % verbatim replacement that allows latex
    \usepackage{grffile} % extends the file name processing of package graphics 
                         % to support a larger range 
    % The hyperref package gives us a pdf with properly built
    % internal navigation ('pdf bookmarks' for the table of contents,
    % internal cross-reference links, web links for URLs, etc.)
    \usepackage{hyperref}
    \usepackage{longtable} % longtable support required by pandoc >1.10
    \usepackage{booktabs}  % table support for pandoc > 1.12.2
    \usepackage{ulem} % ulem is needed to support strikethroughs (\sout)
    

    
    
    \definecolor{orange}{cmyk}{0,0.4,0.8,0.2}
    \definecolor{darkorange}{rgb}{.71,0.21,0.01}
    \definecolor{darkgreen}{rgb}{.12,.54,.11}
    \definecolor{myteal}{rgb}{.26, .44, .56}
    \definecolor{gray}{gray}{0.45}
    \definecolor{lightgray}{gray}{.95}
    \definecolor{mediumgray}{gray}{.8}
    \definecolor{inputbackground}{rgb}{.95, .95, .85}
    \definecolor{outputbackground}{rgb}{.95, .95, .95}
    \definecolor{traceback}{rgb}{1, .95, .95}
    % ansi colors
    \definecolor{red}{rgb}{.6,0,0}
    \definecolor{green}{rgb}{0,.65,0}
    \definecolor{brown}{rgb}{0.6,0.6,0}
    \definecolor{blue}{rgb}{0,.145,.698}
    \definecolor{purple}{rgb}{.698,.145,.698}
    \definecolor{cyan}{rgb}{0,.698,.698}
    \definecolor{lightgray}{gray}{0.5}
    
    % bright ansi colors
    \definecolor{darkgray}{gray}{0.25}
    \definecolor{lightred}{rgb}{1.0,0.39,0.28}
    \definecolor{lightgreen}{rgb}{0.48,0.99,0.0}
    \definecolor{lightblue}{rgb}{0.53,0.81,0.92}
    \definecolor{lightpurple}{rgb}{0.87,0.63,0.87}
    \definecolor{lightcyan}{rgb}{0.5,1.0,0.83}
    
    % commands and environments needed by pandoc snippets
    % extracted from the output of `pandoc -s`
    \providecommand{\tightlist}{%
      \setlength{\itemsep}{0pt}\setlength{\parskip}{0pt}}
    \DefineVerbatimEnvironment{Highlighting}{Verbatim}{commandchars=\\\{\}}
    % Add ',fontsize=\small' for more characters per line
    \newenvironment{Shaded}{}{}
    \newcommand{\KeywordTok}[1]{\textcolor[rgb]{0.00,0.44,0.13}{\textbf{{#1}}}}
    \newcommand{\DataTypeTok}[1]{\textcolor[rgb]{0.56,0.13,0.00}{{#1}}}
    \newcommand{\DecValTok}[1]{\textcolor[rgb]{0.25,0.63,0.44}{{#1}}}
    \newcommand{\BaseNTok}[1]{\textcolor[rgb]{0.25,0.63,0.44}{{#1}}}
    \newcommand{\FloatTok}[1]{\textcolor[rgb]{0.25,0.63,0.44}{{#1}}}
    \newcommand{\CharTok}[1]{\textcolor[rgb]{0.25,0.44,0.63}{{#1}}}
    \newcommand{\StringTok}[1]{\textcolor[rgb]{0.25,0.44,0.63}{{#1}}}
    \newcommand{\CommentTok}[1]{\textcolor[rgb]{0.38,0.63,0.69}{\textit{{#1}}}}
    \newcommand{\OtherTok}[1]{\textcolor[rgb]{0.00,0.44,0.13}{{#1}}}
    \newcommand{\AlertTok}[1]{\textcolor[rgb]{1.00,0.00,0.00}{\textbf{{#1}}}}
    \newcommand{\FunctionTok}[1]{\textcolor[rgb]{0.02,0.16,0.49}{{#1}}}
    \newcommand{\RegionMarkerTok}[1]{{#1}}
    \newcommand{\ErrorTok}[1]{\textcolor[rgb]{1.00,0.00,0.00}{\textbf{{#1}}}}
    \newcommand{\NormalTok}[1]{{#1}}
    
    % Additional commands for more recent versions of Pandoc
    \newcommand{\ConstantTok}[1]{\textcolor[rgb]{0.53,0.00,0.00}{{#1}}}
    \newcommand{\SpecialCharTok}[1]{\textcolor[rgb]{0.25,0.44,0.63}{{#1}}}
    \newcommand{\VerbatimStringTok}[1]{\textcolor[rgb]{0.25,0.44,0.63}{{#1}}}
    \newcommand{\SpecialStringTok}[1]{\textcolor[rgb]{0.73,0.40,0.53}{{#1}}}
    \newcommand{\ImportTok}[1]{{#1}}
    \newcommand{\DocumentationTok}[1]{\textcolor[rgb]{0.73,0.13,0.13}{\textit{{#1}}}}
    \newcommand{\AnnotationTok}[1]{\textcolor[rgb]{0.38,0.63,0.69}{\textbf{\textit{{#1}}}}}
    \newcommand{\CommentVarTok}[1]{\textcolor[rgb]{0.38,0.63,0.69}{\textbf{\textit{{#1}}}}}
    \newcommand{\VariableTok}[1]{\textcolor[rgb]{0.10,0.09,0.49}{{#1}}}
    \newcommand{\ControlFlowTok}[1]{\textcolor[rgb]{0.00,0.44,0.13}{\textbf{{#1}}}}
    \newcommand{\OperatorTok}[1]{\textcolor[rgb]{0.40,0.40,0.40}{{#1}}}
    \newcommand{\BuiltInTok}[1]{{#1}}
    \newcommand{\ExtensionTok}[1]{{#1}}
    \newcommand{\PreprocessorTok}[1]{\textcolor[rgb]{0.74,0.48,0.00}{{#1}}}
    \newcommand{\AttributeTok}[1]{\textcolor[rgb]{0.49,0.56,0.16}{{#1}}}
    \newcommand{\InformationTok}[1]{\textcolor[rgb]{0.38,0.63,0.69}{\textbf{\textit{{#1}}}}}
    \newcommand{\WarningTok}[1]{\textcolor[rgb]{0.38,0.63,0.69}{\textbf{\textit{{#1}}}}}
    
    
    % Define a nice break command that doesn't care if a line doesn't already
    % exist.
    \def\br{\hspace*{\fill} \\* }
    % Math Jax compatability definitions
    \def\gt{>}
    \def\lt{<}
    % Document parameters
    \title{6-Functions-and-Parameter-Estimation}
    
    
    

    % Pygments definitions
    
\makeatletter
\def\PY@reset{\let\PY@it=\relax \let\PY@bf=\relax%
    \let\PY@ul=\relax \let\PY@tc=\relax%
    \let\PY@bc=\relax \let\PY@ff=\relax}
\def\PY@tok#1{\csname PY@tok@#1\endcsname}
\def\PY@toks#1+{\ifx\relax#1\empty\else%
    \PY@tok{#1}\expandafter\PY@toks\fi}
\def\PY@do#1{\PY@bc{\PY@tc{\PY@ul{%
    \PY@it{\PY@bf{\PY@ff{#1}}}}}}}
\def\PY#1#2{\PY@reset\PY@toks#1+\relax+\PY@do{#2}}

\expandafter\def\csname PY@tok@nd\endcsname{\def\PY@tc##1{\textcolor[rgb]{0.67,0.13,1.00}{##1}}}
\expandafter\def\csname PY@tok@mb\endcsname{\def\PY@tc##1{\textcolor[rgb]{0.40,0.40,0.40}{##1}}}
\expandafter\def\csname PY@tok@gs\endcsname{\let\PY@bf=\textbf}
\expandafter\def\csname PY@tok@nb\endcsname{\def\PY@tc##1{\textcolor[rgb]{0.00,0.50,0.00}{##1}}}
\expandafter\def\csname PY@tok@mf\endcsname{\def\PY@tc##1{\textcolor[rgb]{0.40,0.40,0.40}{##1}}}
\expandafter\def\csname PY@tok@bp\endcsname{\def\PY@tc##1{\textcolor[rgb]{0.00,0.50,0.00}{##1}}}
\expandafter\def\csname PY@tok@gh\endcsname{\let\PY@bf=\textbf\def\PY@tc##1{\textcolor[rgb]{0.00,0.00,0.50}{##1}}}
\expandafter\def\csname PY@tok@si\endcsname{\let\PY@bf=\textbf\def\PY@tc##1{\textcolor[rgb]{0.73,0.40,0.53}{##1}}}
\expandafter\def\csname PY@tok@gt\endcsname{\def\PY@tc##1{\textcolor[rgb]{0.00,0.27,0.87}{##1}}}
\expandafter\def\csname PY@tok@s\endcsname{\def\PY@tc##1{\textcolor[rgb]{0.73,0.13,0.13}{##1}}}
\expandafter\def\csname PY@tok@gu\endcsname{\let\PY@bf=\textbf\def\PY@tc##1{\textcolor[rgb]{0.50,0.00,0.50}{##1}}}
\expandafter\def\csname PY@tok@ge\endcsname{\let\PY@it=\textit}
\expandafter\def\csname PY@tok@nt\endcsname{\let\PY@bf=\textbf\def\PY@tc##1{\textcolor[rgb]{0.00,0.50,0.00}{##1}}}
\expandafter\def\csname PY@tok@kr\endcsname{\let\PY@bf=\textbf\def\PY@tc##1{\textcolor[rgb]{0.00,0.50,0.00}{##1}}}
\expandafter\def\csname PY@tok@cpf\endcsname{\let\PY@it=\textit\def\PY@tc##1{\textcolor[rgb]{0.25,0.50,0.50}{##1}}}
\expandafter\def\csname PY@tok@vi\endcsname{\def\PY@tc##1{\textcolor[rgb]{0.10,0.09,0.49}{##1}}}
\expandafter\def\csname PY@tok@sx\endcsname{\def\PY@tc##1{\textcolor[rgb]{0.00,0.50,0.00}{##1}}}
\expandafter\def\csname PY@tok@nc\endcsname{\let\PY@bf=\textbf\def\PY@tc##1{\textcolor[rgb]{0.00,0.00,1.00}{##1}}}
\expandafter\def\csname PY@tok@s1\endcsname{\def\PY@tc##1{\textcolor[rgb]{0.73,0.13,0.13}{##1}}}
\expandafter\def\csname PY@tok@sc\endcsname{\def\PY@tc##1{\textcolor[rgb]{0.73,0.13,0.13}{##1}}}
\expandafter\def\csname PY@tok@sr\endcsname{\def\PY@tc##1{\textcolor[rgb]{0.73,0.40,0.53}{##1}}}
\expandafter\def\csname PY@tok@nn\endcsname{\let\PY@bf=\textbf\def\PY@tc##1{\textcolor[rgb]{0.00,0.00,1.00}{##1}}}
\expandafter\def\csname PY@tok@gp\endcsname{\let\PY@bf=\textbf\def\PY@tc##1{\textcolor[rgb]{0.00,0.00,0.50}{##1}}}
\expandafter\def\csname PY@tok@cm\endcsname{\let\PY@it=\textit\def\PY@tc##1{\textcolor[rgb]{0.25,0.50,0.50}{##1}}}
\expandafter\def\csname PY@tok@kn\endcsname{\let\PY@bf=\textbf\def\PY@tc##1{\textcolor[rgb]{0.00,0.50,0.00}{##1}}}
\expandafter\def\csname PY@tok@kc\endcsname{\let\PY@bf=\textbf\def\PY@tc##1{\textcolor[rgb]{0.00,0.50,0.00}{##1}}}
\expandafter\def\csname PY@tok@mo\endcsname{\def\PY@tc##1{\textcolor[rgb]{0.40,0.40,0.40}{##1}}}
\expandafter\def\csname PY@tok@cs\endcsname{\let\PY@it=\textit\def\PY@tc##1{\textcolor[rgb]{0.25,0.50,0.50}{##1}}}
\expandafter\def\csname PY@tok@na\endcsname{\def\PY@tc##1{\textcolor[rgb]{0.49,0.56,0.16}{##1}}}
\expandafter\def\csname PY@tok@vc\endcsname{\def\PY@tc##1{\textcolor[rgb]{0.10,0.09,0.49}{##1}}}
\expandafter\def\csname PY@tok@nl\endcsname{\def\PY@tc##1{\textcolor[rgb]{0.63,0.63,0.00}{##1}}}
\expandafter\def\csname PY@tok@ow\endcsname{\let\PY@bf=\textbf\def\PY@tc##1{\textcolor[rgb]{0.67,0.13,1.00}{##1}}}
\expandafter\def\csname PY@tok@sd\endcsname{\let\PY@it=\textit\def\PY@tc##1{\textcolor[rgb]{0.73,0.13,0.13}{##1}}}
\expandafter\def\csname PY@tok@gd\endcsname{\def\PY@tc##1{\textcolor[rgb]{0.63,0.00,0.00}{##1}}}
\expandafter\def\csname PY@tok@c1\endcsname{\let\PY@it=\textit\def\PY@tc##1{\textcolor[rgb]{0.25,0.50,0.50}{##1}}}
\expandafter\def\csname PY@tok@kp\endcsname{\def\PY@tc##1{\textcolor[rgb]{0.00,0.50,0.00}{##1}}}
\expandafter\def\csname PY@tok@il\endcsname{\def\PY@tc##1{\textcolor[rgb]{0.40,0.40,0.40}{##1}}}
\expandafter\def\csname PY@tok@ni\endcsname{\let\PY@bf=\textbf\def\PY@tc##1{\textcolor[rgb]{0.60,0.60,0.60}{##1}}}
\expandafter\def\csname PY@tok@ss\endcsname{\def\PY@tc##1{\textcolor[rgb]{0.10,0.09,0.49}{##1}}}
\expandafter\def\csname PY@tok@c\endcsname{\let\PY@it=\textit\def\PY@tc##1{\textcolor[rgb]{0.25,0.50,0.50}{##1}}}
\expandafter\def\csname PY@tok@cp\endcsname{\def\PY@tc##1{\textcolor[rgb]{0.74,0.48,0.00}{##1}}}
\expandafter\def\csname PY@tok@o\endcsname{\def\PY@tc##1{\textcolor[rgb]{0.40,0.40,0.40}{##1}}}
\expandafter\def\csname PY@tok@kd\endcsname{\let\PY@bf=\textbf\def\PY@tc##1{\textcolor[rgb]{0.00,0.50,0.00}{##1}}}
\expandafter\def\csname PY@tok@go\endcsname{\def\PY@tc##1{\textcolor[rgb]{0.53,0.53,0.53}{##1}}}
\expandafter\def\csname PY@tok@kt\endcsname{\def\PY@tc##1{\textcolor[rgb]{0.69,0.00,0.25}{##1}}}
\expandafter\def\csname PY@tok@mi\endcsname{\def\PY@tc##1{\textcolor[rgb]{0.40,0.40,0.40}{##1}}}
\expandafter\def\csname PY@tok@no\endcsname{\def\PY@tc##1{\textcolor[rgb]{0.53,0.00,0.00}{##1}}}
\expandafter\def\csname PY@tok@ch\endcsname{\let\PY@it=\textit\def\PY@tc##1{\textcolor[rgb]{0.25,0.50,0.50}{##1}}}
\expandafter\def\csname PY@tok@ne\endcsname{\let\PY@bf=\textbf\def\PY@tc##1{\textcolor[rgb]{0.82,0.25,0.23}{##1}}}
\expandafter\def\csname PY@tok@gi\endcsname{\def\PY@tc##1{\textcolor[rgb]{0.00,0.63,0.00}{##1}}}
\expandafter\def\csname PY@tok@w\endcsname{\def\PY@tc##1{\textcolor[rgb]{0.73,0.73,0.73}{##1}}}
\expandafter\def\csname PY@tok@se\endcsname{\let\PY@bf=\textbf\def\PY@tc##1{\textcolor[rgb]{0.73,0.40,0.13}{##1}}}
\expandafter\def\csname PY@tok@s2\endcsname{\def\PY@tc##1{\textcolor[rgb]{0.73,0.13,0.13}{##1}}}
\expandafter\def\csname PY@tok@nv\endcsname{\def\PY@tc##1{\textcolor[rgb]{0.10,0.09,0.49}{##1}}}
\expandafter\def\csname PY@tok@m\endcsname{\def\PY@tc##1{\textcolor[rgb]{0.40,0.40,0.40}{##1}}}
\expandafter\def\csname PY@tok@k\endcsname{\let\PY@bf=\textbf\def\PY@tc##1{\textcolor[rgb]{0.00,0.50,0.00}{##1}}}
\expandafter\def\csname PY@tok@mh\endcsname{\def\PY@tc##1{\textcolor[rgb]{0.40,0.40,0.40}{##1}}}
\expandafter\def\csname PY@tok@gr\endcsname{\def\PY@tc##1{\textcolor[rgb]{1.00,0.00,0.00}{##1}}}
\expandafter\def\csname PY@tok@sb\endcsname{\def\PY@tc##1{\textcolor[rgb]{0.73,0.13,0.13}{##1}}}
\expandafter\def\csname PY@tok@sh\endcsname{\def\PY@tc##1{\textcolor[rgb]{0.73,0.13,0.13}{##1}}}
\expandafter\def\csname PY@tok@vg\endcsname{\def\PY@tc##1{\textcolor[rgb]{0.10,0.09,0.49}{##1}}}
\expandafter\def\csname PY@tok@nf\endcsname{\def\PY@tc##1{\textcolor[rgb]{0.00,0.00,1.00}{##1}}}
\expandafter\def\csname PY@tok@err\endcsname{\def\PY@bc##1{\setlength{\fboxsep}{0pt}\fcolorbox[rgb]{1.00,0.00,0.00}{1,1,1}{\strut ##1}}}

\def\PYZbs{\char`\\}
\def\PYZus{\char`\_}
\def\PYZob{\char`\{}
\def\PYZcb{\char`\}}
\def\PYZca{\char`\^}
\def\PYZam{\char`\&}
\def\PYZlt{\char`\<}
\def\PYZgt{\char`\>}
\def\PYZsh{\char`\#}
\def\PYZpc{\char`\%}
\def\PYZdl{\char`\$}
\def\PYZhy{\char`\-}
\def\PYZsq{\char`\'}
\def\PYZdq{\char`\"}
\def\PYZti{\char`\~}
% for compatibility with earlier versions
\def\PYZat{@}
\def\PYZlb{[}
\def\PYZrb{]}
\makeatother


    % Exact colors from NB
    \definecolor{incolor}{rgb}{0.0, 0.0, 0.5}
    \definecolor{outcolor}{rgb}{0.545, 0.0, 0.0}



    
    % Prevent overflowing lines due to hard-to-break entities
    \sloppy 
    % Setup hyperref package
    \hypersetup{
      breaklinks=true,  % so long urls are correctly broken across lines
      colorlinks=true,
      urlcolor=blue,
      linkcolor=darkorange,
      citecolor=darkgreen,
      }
    % Slightly bigger margins than the latex defaults
    
    \geometry{verbose,tmargin=1in,bmargin=1in,lmargin=1in,rmargin=1in}
    
    

    \begin{document}
    
    
    \maketitle
    
    

    
    \subsection{6.1 Fitting Functions to Pseudo
Data}\label{fitting-functions-to-pseudo-data}

In the example below, a pseudo-data set is produced and a model fitted
to it.

ROOT offers various minimisation algorithms to minimise a
\textbackslash{}(\textbackslash{}chi\^{}\{2\}\textbackslash{}) or a
negative log-likelihood function. The default minimiser is MINUIT, a
package originally implemented in the FORTRAN programming language. A
C++ version is also available, MINUIT2, as well as Fumili (Silin 1983)
an algorithm optimised for fitting. The minimisation algorithms can be
selected using the static functions of the
\href{https://root.cern.ch/doc/v606/classROOT_1_1Math_1_1MinimizerOptions.html}{\texttt{ROOT::Math::MinimizerOptions}}
class. Steering options for the minimiser, such as the convergence
tolerance or the maximum number of function calls, can also be set using
the methods of this class. All currently implemented minimisers are
documented in the reference documentation of ROOT: have a look for
example to the
\href{https://root.cern.ch/doc/v606/classROOT_1_1Math_1_1Minimizer.html}{\texttt{ROOT::Math::Minimizer}}
class documentation. The complication level of the code below is
intentionally a little higher than in the previous examples.

Let's go through the code, step by step to understand what is going on:

\begin{itemize}
\tightlist
\item
  First we create a simple function to ease the make-up of lines.
  Remember that the class TF1 inherits from
  \href{https://root.cern.ch/doc/master/classTAttLine.html}{TAttLine}.
\end{itemize}

    \begin{Verbatim}[commandchars=\\\{\}]
{\color{incolor}In [{\color{incolor}1}]:} \PY{o}{\PYZpc{}}\PY{o}{\PYZpc{}}\PY{n}{cpp} \PY{o}{\PYZhy{}}\PY{n}{d}
        \PY{k+kt}{void} \PY{n}{format\PYZus{}line}\PY{p}{(}\PY{n}{TAttLine}\PY{o}{*} \PY{n}{line}\PY{p}{,}\PY{k+kt}{int} \PY{n}{col}\PY{p}{,}\PY{k+kt}{int} \PY{n}{sty}\PY{p}{)}\PY{p}{\PYZob{}}
        \PY{n}{line}\PY{o}{\PYZhy{}}\PY{o}{\PYZgt{}}\PY{n}{SetLineWidth}\PY{p}{(}\PY{l+m+mi}{5}\PY{p}{)}\PY{p}{;} \PY{n}{line}\PY{o}{\PYZhy{}}\PY{o}{\PYZgt{}}\PY{n}{SetLineColor}\PY{p}{(}\PY{n}{col}\PY{p}{)}\PY{p}{;}
        \PY{n}{line}\PY{o}{\PYZhy{}}\PY{o}{\PYZgt{}}\PY{n}{SetLineStyle}\PY{p}{(}\PY{n}{sty}\PY{p}{)}\PY{p}{;}\PY{p}{\PYZcb{}}
\end{Verbatim}

    \begin{itemize}
\tightlist
\item
  Here we creat a definition of a customised function, namely a Gaussian
  (the ``signal'') plus a parabolic function, the ``background''.
\end{itemize}

    \begin{Verbatim}[commandchars=\\\{\}]
{\color{incolor}In [{\color{incolor}2}]:} \PY{o}{\PYZpc{}}\PY{o}{\PYZpc{}}\PY{n}{cpp} \PY{o}{\PYZhy{}}\PY{n}{d}
        \PY{k+kt}{double} \PY{n}{the\PYZus{}gausppar}\PY{p}{(}\PY{k+kt}{double}\PY{o}{*} \PY{n}{vars}\PY{p}{,} \PY{k+kt}{double}\PY{o}{*} \PY{n}{pars}\PY{p}{)}\PY{p}{\PYZob{}}
        \PY{k}{return} \PY{n}{pars}\PY{p}{[}\PY{l+m+mi}{0}\PY{p}{]}\PY{o}{*}\PY{n}{TMath}\PY{o}{:}\PY{o}{:}\PY{n}{Gaus}\PY{p}{(}\PY{n}{vars}\PY{p}{[}\PY{l+m+mi}{0}\PY{p}{]}\PY{p}{,}\PY{n}{pars}\PY{p}{[}\PY{l+m+mi}{1}\PY{p}{]}\PY{p}{,}\PY{n}{pars}\PY{p}{[}\PY{l+m+mi}{2}\PY{p}{]}\PY{p}{)}\PY{o}{+}
        \PY{n}{pars}\PY{p}{[}\PY{l+m+mi}{3}\PY{p}{]}\PY{o}{+}\PY{n}{pars}\PY{p}{[}\PY{l+m+mi}{4}\PY{p}{]}\PY{o}{*}\PY{n}{vars}\PY{p}{[}\PY{l+m+mi}{0}\PY{p}{]}\PY{o}{+}\PY{n}{pars}\PY{p}{[}\PY{l+m+mi}{5}\PY{p}{]}\PY{o}{*}\PY{n}{vars}\PY{p}{[}\PY{l+m+mi}{0}\PY{p}{]}\PY{o}{*}\PY{n}{vars}\PY{p}{[}\PY{l+m+mi}{0}\PY{p}{]}\PY{p}{;}\PY{p}{\PYZcb{}}
\end{Verbatim}

    \begin{itemize}
\tightlist
\item
  Some make-up for the Canvas. In particular we want that the parameters
  of the fit appear very clearly and nicely on the plot.
\end{itemize}

    \begin{Verbatim}[commandchars=\\\{\}]
{\color{incolor}In [{\color{incolor}3}]:} \PY{k}{auto} \PY{n}{canvas\PYZus{}6\PYZus{}1}\PY{o}{=}\PY{k}{new} \PY{n}{TCanvas}\PY{p}{(}\PY{l+s}{\PYZdq{}}\PY{l+s}{canvas\PYZus{}6\PYZus{}1}\PY{l+s}{\PYZdq{}}\PY{p}{,}\PY{l+s}{\PYZdq{}}\PY{l+s}{canvas\PYZus{}6\PYZus{}1}\PY{l+s}{\PYZdq{}}\PY{p}{)}\PY{p}{;}
        \PY{n}{gStyle}\PY{o}{\PYZhy{}}\PY{o}{\PYZgt{}}\PY{n}{SetOptTitle}\PY{p}{(}\PY{l+m+mi}{0}\PY{p}{)}\PY{p}{;} \PY{n}{gStyle}\PY{o}{\PYZhy{}}\PY{o}{\PYZgt{}}\PY{n}{SetOptStat}\PY{p}{(}\PY{l+m+mi}{0}\PY{p}{)}\PY{p}{;}
        \PY{n}{gStyle}\PY{o}{\PYZhy{}}\PY{o}{\PYZgt{}}\PY{n}{SetOptFit}\PY{p}{(}\PY{l+m+mi}{1111}\PY{p}{)}\PY{p}{;} \PY{n}{gStyle}\PY{o}{\PYZhy{}}\PY{o}{\PYZgt{}}\PY{n}{SetStatBorderSize}\PY{p}{(}\PY{l+m+mi}{0}\PY{p}{)}\PY{p}{;}
        \PY{n}{gStyle}\PY{o}{\PYZhy{}}\PY{o}{\PYZgt{}}\PY{n}{SetStatX}\PY{p}{(}\PY{l+m+mf}{.89}\PY{p}{)}\PY{p}{;} \PY{n}{gStyle}\PY{o}{\PYZhy{}}\PY{o}{\PYZgt{}}\PY{n}{SetStatY}\PY{p}{(}\PY{l+m+mf}{.89}\PY{p}{)}\PY{p}{;}
        
        \PY{n}{TF1} \PY{n+nf}{parabola}\PY{p}{(}\PY{l+s}{\PYZdq{}}\PY{l+s}{parabola}\PY{l+s}{\PYZdq{}}\PY{p}{,}\PY{l+s}{\PYZdq{}}\PY{l+s}{[0]+[1]*x+[2]*x**2}\PY{l+s}{\PYZdq{}}\PY{p}{,}\PY{l+m+mi}{0}\PY{p}{,}\PY{l+m+mi}{20}\PY{p}{)}\PY{p}{;}
        \PY{n}{format\PYZus{}line}\PY{p}{(}\PY{o}{\PYZam{}}\PY{n}{parabola}\PY{p}{,}\PY{n}{kBlue}\PY{p}{,}\PY{l+m+mi}{2}\PY{p}{)}\PY{p}{;}
        
        \PY{n}{TF1} \PY{n+nf}{gaussian}\PY{p}{(}\PY{l+s}{\PYZdq{}}\PY{l+s}{gaussian}\PY{l+s}{\PYZdq{}}\PY{p}{,}\PY{l+s}{\PYZdq{}}\PY{l+s}{[0]*TMath::Gaus(x,[1],[2])}\PY{l+s}{\PYZdq{}}\PY{l+s}{,0,20)}\PY{p}{;}
        \PY{n}{format\PYZus{}line}\PY{p}{(}\PY{o}{\PYZam{}}\PY{n}{gaussian}\PY{p}{,}\PY{n}{kRed}\PY{p}{,}\PY{l+m+mi}{2}\PY{p}{)}\PY{p}{;}
\end{Verbatim}

    \begin{itemize}
\tightlist
\item
  Next we define and initialise an instance of TF1.
\end{itemize}

    \begin{Verbatim}[commandchars=\\\{\}]
{\color{incolor}In [{\color{incolor}4}]:} \PY{n}{TF1} \PY{n+nf}{gausppar}\PY{p}{(}\PY{l+s}{\PYZdq{}}\PY{l+s}{gausppar}\PY{l+s}{\PYZdq{}}\PY{p}{,}\PY{n}{the\PYZus{}gausppar}\PY{p}{,}\PY{o}{\PYZhy{}}\PY{l+m+mi}{0}\PY{p}{,}\PY{l+m+mi}{20}\PY{p}{,}\PY{l+m+mi}{6}\PY{p}{)}\PY{p}{;}
        \PY{k+kt}{double} \PY{n}{a}\PY{o}{=}\PY{l+m+mi}{15}\PY{p}{;} \PY{k+kt}{double} \PY{n}{b}\PY{o}{=}\PY{o}{\PYZhy{}}\PY{l+m+mf}{1.2}\PY{p}{;} \PY{k+kt}{double} \PY{n}{c}\PY{o}{=}\PY{l+m+mf}{.03}\PY{p}{;}
        \PY{k+kt}{double} \PY{n}{normal}\PY{o}{=}\PY{l+m+mi}{4}\PY{p}{;} \PY{k+kt}{double} \PY{n}{mean}\PY{o}{=}\PY{l+m+mi}{7}\PY{p}{;} \PY{k+kt}{double} \PY{n}{sigma}\PY{o}{=}\PY{l+m+mi}{1}\PY{p}{;}
        \PY{n}{gausppar}\PY{p}{.}\PY{n}{SetParameters}\PY{p}{(}\PY{n}{normal}\PY{p}{,}\PY{n}{mean}\PY{p}{,}\PY{n}{sigma}\PY{p}{,}\PY{n}{a}\PY{p}{,}\PY{n}{b}\PY{p}{,}\PY{n}{c}\PY{p}{)}\PY{p}{;}
        \PY{n}{gausppar}\PY{p}{.}\PY{n}{SetParNames}\PY{p}{(}\PY{l+s}{\PYZdq{}}\PY{l+s}{Normal}\PY{l+s}{\PYZdq{}}\PY{p}{,}\PY{l+s}{\PYZdq{}}\PY{l+s}{Mean}\PY{l+s}{\PYZdq{}}\PY{p}{,}\PY{l+s}{\PYZdq{}}\PY{l+s}{Sigma}\PY{l+s}{\PYZdq{}}\PY{p}{,}\PY{l+s}{\PYZdq{}}\PY{l+s}{a}\PY{l+s}{\PYZdq{}}\PY{p}{,}\PY{l+s}{\PYZdq{}}\PY{l+s}{b}\PY{l+s}{\PYZdq{}}\PY{p}{,}\PY{l+s}{\PYZdq{}}\PY{l+s}{c}\PY{l+s}{\PYZdq{}}\PY{p}{)}\PY{p}{;}
        \PY{n}{format\PYZus{}line}\PY{p}{(}\PY{o}{\PYZam{}}\PY{n}{gausppar}\PY{p}{,}\PY{n}{kBlue}\PY{p}{,}\PY{l+m+mi}{1}\PY{p}{)}\PY{p}{;}
\end{Verbatim}

    \begin{itemize}
\tightlist
\item
  Followed by the definition and the filling of a histogram.
\end{itemize}

    \begin{Verbatim}[commandchars=\\\{\}]
{\color{incolor}In [{\color{incolor}5}]:} \PY{n}{TH1F} \PY{n}{histo}\PY{p}{(}\PY{l+s}{\PYZdq{}}\PY{l+s}{histo}\PY{l+s}{\PYZdq{}}\PY{p}{,}\PY{l+s}{\PYZdq{}}\PY{l+s}{Signal plus background;X vals;Y Vals}\PY{l+s}{\PYZdq{}}\PY{p}{,}\PY{l+m+mi}{50}\PY{p}{,}\PY{l+m+mi}{0}\PY{p}{,}\PY{l+m+mi}{20}\PY{p}{)}\PY{p}{;}
        \PY{n}{histo}\PY{p}{.}\PY{n}{SetMarkerStyle}\PY{p}{(}\PY{l+m+mi}{8}\PY{p}{)}\PY{p}{;}
        
        \PY{c+c1}{// Fake the data}
        \PY{k}{for} \PY{p}{(}\PY{k+kt}{int} \PY{n}{i}\PY{o}{=}\PY{l+m+mi}{1}\PY{p}{;}\PY{n}{i}\PY{o}{\PYZlt{}}\PY{o}{=}\PY{l+m+mi}{5000}\PY{p}{;}\PY{o}{+}\PY{o}{+}\PY{n}{i}\PY{p}{)} \PY{n}{histo}\PY{p}{.}\PY{n}{Fill}\PY{p}{(}\PY{n}{gausppar}\PY{p}{.}\PY{n}{GetRandom}\PY{p}{(}\PY{p}{)}\PY{p}{)}\PY{p}{;}
\end{Verbatim}

    \begin{itemize}
\tightlist
\item
  For convenience, the same function as for the generation of the
  pseudo-data is used in the fit; hence, we need to reset the function
  parameters. This part of the code is very important for each fit
  procedure, as it sets the initial values of the fit.
\end{itemize}

    \begin{Verbatim}[commandchars=\\\{\}]
{\color{incolor}In [{\color{incolor}6}]:} \PY{c+c1}{// Reset the parameters before the fit and set}
        \PY{c+c1}{// by eye a peak at 6 with an area of more or less 50}
        \PY{n}{gausppar}\PY{p}{.}\PY{n}{SetParameter}\PY{p}{(}\PY{l+m+mi}{0}\PY{p}{,}\PY{l+m+mi}{50}\PY{p}{)}\PY{p}{;}
        \PY{n}{gausppar}\PY{p}{.}\PY{n}{SetParameter}\PY{p}{(}\PY{l+m+mi}{1}\PY{p}{,}\PY{l+m+mi}{6}\PY{p}{)}\PY{p}{;}
        \PY{k+kt}{int} \PY{n}{npar}\PY{o}{=}\PY{n}{gausppar}\PY{p}{.}\PY{n}{GetNpar}\PY{p}{(}\PY{p}{)}\PY{p}{;}
        \PY{k}{for} \PY{p}{(}\PY{k+kt}{int} \PY{n}{ipar}\PY{o}{=}\PY{l+m+mi}{2}\PY{p}{;}\PY{n}{ipar}\PY{o}{\PYZlt{}}\PY{n}{npar}\PY{p}{;}\PY{o}{+}\PY{o}{+}\PY{n}{ipar}\PY{p}{)} \PY{n}{gausppar}\PY{p}{.}\PY{n}{SetParameter}\PY{p}{(}\PY{n}{ipar}\PY{p}{,}\PY{l+m+mi}{1}\PY{p}{)}\PY{p}{;}
\end{Verbatim}

    \begin{itemize}
\tightlist
\item
  Next a very simple command, well known by now: fit the function to the
  histogram.
\end{itemize}

    \begin{Verbatim}[commandchars=\\\{\}]
{\color{incolor}In [{\color{incolor}7}]:} \PY{c+c1}{// perform fit ...}
        \PY{k}{auto} \PY{n}{fitResPtr} \PY{o}{=} \PY{n}{histo}\PY{p}{.}\PY{n}{Fit}\PY{p}{(}\PY{o}{\PYZam{}}\PY{n}{gausppar}\PY{p}{,} \PY{l+s}{\PYZdq{}}\PY{l+s}{S}\PY{l+s}{\PYZdq{}}\PY{p}{)}\PY{p}{;}
\end{Verbatim}

    \begin{Verbatim}[commandchars=\\\{\}]
FCN=42.0305 FROM MIGRAD    STATUS=CONVERGED    1207 CALLS        1208 TOTAL
                     EDM=5.17306e-07    STRATEGY= 1  ERROR MATRIX UNCERTAINTY   2.2 per cent
  EXT PARAMETER                                   STEP         FIRST   
  NO.   NAME      VALUE            ERROR          SIZE      DERIVATIVE 
   1  Normal       5.78269e+01   7.94068e+00   5.77561e-02   4.28977e-05
   2  Mean         7.01009e+00   1.35383e-01  -3.22514e-04   7.29243e-05
   3  Sigma        9.23834e-01   1.57793e-01  -2.59746e-04   2.77085e-03
   4  a            2.00626e+02   5.50591e+00   2.33533e-03  -2.57315e-04
   5  b           -1.67332e+01   1.03021e+00  -1.03565e-04   1.76478e-04
   6  c            4.43814e-01   4.63690e-02  -4.30880e-05   2.61849e-02
    \end{Verbatim}

    \begin{itemize}
\tightlist
\item
  We then retrieve the output from the fit. Here, we simply print the
  fit result and access and print the covariance matrix of the
  parameters.
\end{itemize}

    \begin{Verbatim}[commandchars=\\\{\}]
{\color{incolor}In [{\color{incolor}8}]:} \PY{c+c1}{// ... and retrieve fit results}
        \PY{n}{fitResPtr}\PY{o}{\PYZhy{}}\PY{o}{\PYZgt{}}\PY{n}{Print}\PY{p}{(}\PY{p}{)}\PY{p}{;} \PY{c+c1}{// print fit results}
        \PY{c+c1}{// get covariance Matrix an print it}
        \PY{n}{TMatrixDSym} \PY{n+nf}{covMatrix} \PY{p}{(}\PY{n}{fitResPtr}\PY{o}{\PYZhy{}}\PY{o}{\PYZgt{}}\PY{n}{GetCovarianceMatrix}\PY{p}{(}\PY{p}{)}\PY{p}{)}\PY{p}{;}
        \PY{n}{covMatrix}\PY{p}{.}\PY{n}{Print}\PY{p}{(}\PY{p}{)}\PY{p}{;}
        
        \PY{c+c1}{// Set the values of the gaussian and parabola}
        \PY{k}{for} \PY{p}{(}\PY{k+kt}{int} \PY{n}{ipar}\PY{o}{=}\PY{l+m+mi}{0}\PY{p}{;}\PY{n}{ipar}\PY{o}{\PYZlt{}}\PY{l+m+mi}{3}\PY{p}{;}\PY{n}{ipar}\PY{o}{+}\PY{o}{+}\PY{p}{)}\PY{p}{\PYZob{}}
            \PY{n}{gaussian}\PY{p}{.}\PY{n}{SetParameter}\PY{p}{(}\PY{n}{ipar}\PY{p}{,}\PY{n}{gausppar}\PY{p}{.}\PY{n}{GetParameter}\PY{p}{(}\PY{n}{ipar}\PY{p}{)}\PY{p}{)}\PY{p}{;}
            \PY{n}{parabola}\PY{p}{.}\PY{n}{SetParameter}\PY{p}{(}\PY{n}{ipar}\PY{p}{,}\PY{n}{gausppar}\PY{p}{.}\PY{n}{GetParameter}\PY{p}{(}\PY{n}{ipar}\PY{o}{+}\PY{l+m+mi}{3}\PY{p}{)}\PY{p}{)}\PY{p}{;}
        \PY{p}{\PYZcb{}}
\end{Verbatim}

    \begin{Verbatim}[commandchars=\\\{\}]
****************************************
Minimizer is Minuit / Migrad
Chi2                      =      42.0305
NDf                       =           44
Edm                       =  5.17306e-07
NCalls                    =         1208
Normal                    =      57.8269   +/-   7.94068     
Mean                      =      7.01009   +/-   0.135383    
Sigma                     =     0.923834   +/-   0.157793    
a                         =      200.626   +/-   5.50591     
b                         =     -16.7332   +/-   1.03021     
c                         =     0.443814   +/-   0.046369    

6x6 matrix is as follows

     |      0    |      1    |      2    |      3    |      4    |
----------------------------------------------------------------------
   0 |      63.05     0.08809     -0.5976      -2.663     -0.3044 
   1 |    0.08809     0.01833   -0.005144      0.1111    -0.01496 
   2 |    -0.5976   -0.005144      0.0249     -0.1629   -0.009859 
   3 |     -2.663      0.1111     -0.1629       30.32      -4.738 
   4 |    -0.3044    -0.01496   -0.009859      -4.738       1.061 
   5 |    0.02711   0.0004654    0.001147      0.1721    -0.04605 


     |      5    |
----------------------------------------------------------------------
   0 |    0.02711 
   1 |  0.0004654 
   2 |   0.001147 
   3 |     0.1721 
   4 |   -0.04605 
   5 |    0.00215
    \end{Verbatim}

    \begin{itemize}
\tightlist
\item
  Finally we plot the pseudo-data, the fitted function and the signal
  and background components at the best-fit values.
\end{itemize}

    \begin{Verbatim}[commandchars=\\\{\}]
{\color{incolor}In [{\color{incolor}9}]:} \PY{n}{histo}\PY{p}{.}\PY{n}{GetYaxis}\PY{p}{(}\PY{p}{)}\PY{o}{\PYZhy{}}\PY{o}{\PYZgt{}}\PY{n}{SetRangeUser}\PY{p}{(}\PY{l+m+mi}{0}\PY{p}{,}\PY{l+m+mi}{250}\PY{p}{)}\PY{p}{;}
        \PY{n}{histo}\PY{p}{.}\PY{n}{DrawClone}\PY{p}{(}\PY{l+s}{\PYZdq{}}\PY{l+s}{PE}\PY{l+s}{\PYZdq{}}\PY{p}{)}\PY{p}{;}
        \PY{n}{parabola}\PY{p}{.}\PY{n}{DrawClone}\PY{p}{(}\PY{l+s}{\PYZdq{}}\PY{l+s}{Same}\PY{l+s}{\PYZdq{}}\PY{p}{)}\PY{p}{;} \PY{n}{gaussian}\PY{p}{.}\PY{n}{DrawClone}\PY{p}{(}\PY{l+s}{\PYZdq{}}\PY{l+s}{Same}\PY{l+s}{\PYZdq{}}\PY{p}{)}\PY{p}{;}
        \PY{n}{TLatex} \PY{n+nf}{latex}\PY{p}{(}\PY{l+m+mi}{2}\PY{p}{,}\PY{l+m+mi}{220}\PY{p}{,}\PY{l+s}{\PYZdq{}}\PY{l+s}{\PYZsh{}splitline\PYZob{}Signal Peak over\PYZcb{}\PYZob{}background\PYZcb{}}\PY{l+s}{\PYZdq{}}\PY{p}{)}\PY{p}{;}
        \PY{n}{latex}\PY{p}{.}\PY{n}{DrawClone}\PY{p}{(}\PY{l+s}{\PYZdq{}}\PY{l+s}{Same}\PY{l+s}{\PYZdq{}}\PY{p}{)}\PY{p}{;}
        \PY{n}{canvas\PYZus{}6\PYZus{}1}\PY{o}{\PYZhy{}}\PY{o}{\PYZgt{}}\PY{n}{Draw}\PY{p}{(}\PY{p}{)}\PY{p}{;}
        \PY{k}{return} \PY{l+m+mi}{0}\PY{p}{;}
\end{Verbatim}

    \begin{center}
    \adjustimage{max size={0.9\linewidth}{0.9\paperheight}}{6-Functions-and-Parameter-Estimation_files/6-Functions-and-Parameter-Estimation_17_0.png}
    \end{center}
    { \hspace*{\fill} \\}
    
    Fit of pseudo data: a signal shape over a background trend. This plot is
another example of how making a plot ``self-explanatory'' can help you
better displaying your results.

\subsection{6.2 Toy Monte Carlo
Experiments}\label{toy-monte-carlo-experiments}

Let us look at a simple example of a toy experiment comparing two
methods to fit a function to a histogram, the
\textbackslash{}(\textbackslash{}chi\^{}\{2\}\textbackslash{})

method and a method called ``binned log-likelihood fit'', both available
in ROOT.

As a very simple yet powerful quantity to check the quality of the fit
results, we construct for each pseudo-data set the so-called ``pull'',
the difference of the estimated and the true value of a parameter,
normalised to the estimated error on the parameter,
\textbackslash{}(\textbackslash{}frac\{(p\_\{estim\} -
p\_\{true\})\}\{\textbackslash{}sigma\_\{p\}\}\textbackslash{}). If
everything is OK, the distribution of the pull values is a standard
normal distribution, i.e.~a Gaussian distribution centred around zero
with a standard deviation of one.

The macro performs a rather big number of toy experiments, where a
histogram is repeatedly filled with Gaussian distributed numbers,
representing the pseudo-data in this example. Each time, a fit is
performed according to the selected method, and the pull is calculated
and filled into a histogram. Here is the code:

    \begin{Verbatim}[commandchars=\\\{\}]
{\color{incolor}In [{\color{incolor}10}]:} \PY{o}{\PYZpc{}}\PY{o}{\PYZpc{}}\PY{n}{cpp} \PY{o}{\PYZhy{}}\PY{n}{d}
         \PY{c+c1}{// Toy Monte Carlo example.}
         \PY{c+c1}{// Check pull distribution to compare chi2 and binned}
         \PY{c+c1}{// log\PYZhy{}likelihood methods.}
         
         \PY{k+kt}{void} \PY{n}{pull}\PY{p}{(} \PY{k+kt}{int} \PY{n}{n\PYZus{}toys} \PY{o}{=} \PY{l+m+mi}{10000}\PY{p}{,}
             \PY{k+kt}{int} \PY{n}{n\PYZus{}tot\PYZus{}entries} \PY{o}{=} \PY{l+m+mi}{100}\PY{p}{,}
             \PY{k+kt}{int} \PY{n}{nbins} \PY{o}{=} \PY{l+m+mi}{40}\PY{p}{,}
             \PY{k+kt}{bool} \PY{n}{do\PYZus{}chi2}\PY{o}{=}\PY{n+nb}{true} \PY{p}{)}\PY{p}{\PYZob{}}
         
             \PY{n}{TString} \PY{n}{method\PYZus{}prefix}\PY{p}{(}\PY{l+s}{\PYZdq{}}\PY{l+s}{Log\PYZhy{}Likelihood }\PY{l+s}{\PYZdq{}}\PY{p}{)}\PY{p}{;}
             \PY{k}{if} \PY{p}{(}\PY{n}{do\PYZus{}chi2}\PY{p}{)}
                 \PY{n}{method\PYZus{}prefix}\PY{o}{=}\PY{l+s}{\PYZdq{}}\PY{l+s}{\PYZsh{}chi\PYZca{}\PYZob{}2\PYZcb{} }\PY{l+s}{\PYZdq{}}\PY{p}{;}
         
             \PY{c+c1}{// Create histo}
             \PY{n}{TH1F} \PY{n+nf}{h4}\PY{p}{(}\PY{n}{method\PYZus{}prefix}\PY{o}{+}\PY{l+s}{\PYZdq{}}\PY{l+s}{h4}\PY{l+s}{\PYZdq{}}\PY{p}{,}
                     \PY{n}{method\PYZus{}prefix}\PY{o}{+}\PY{l+s}{\PYZdq{}}\PY{l+s}{ Random Gauss}\PY{l+s}{\PYZdq{}}\PY{p}{,}
                     \PY{n}{nbins}\PY{p}{,}\PY{o}{\PYZhy{}}\PY{l+m+mi}{4}\PY{p}{,}\PY{l+m+mi}{4}\PY{p}{)}\PY{p}{;}
             \PY{n}{h4}\PY{p}{.}\PY{n}{SetMarkerStyle}\PY{p}{(}\PY{l+m+mi}{21}\PY{p}{)}\PY{p}{;}
             \PY{n}{h4}\PY{p}{.}\PY{n}{SetMarkerSize}\PY{p}{(}\PY{l+m+mf}{0.8}\PY{p}{)}\PY{p}{;}
             \PY{n}{h4}\PY{p}{.}\PY{n}{SetMarkerColor}\PY{p}{(}\PY{n}{kRed}\PY{p}{)}\PY{p}{;}
         
             \PY{c+c1}{// Histogram for sigma and pull}
             \PY{n}{TH1F} \PY{n+nf}{sigma}\PY{p}{(}\PY{n}{method\PYZus{}prefix}\PY{o}{+}\PY{l+s}{\PYZdq{}}\PY{l+s}{sigma}\PY{l+s}{\PYZdq{}}\PY{p}{,}
                        \PY{n}{method\PYZus{}prefix}\PY{o}{+}\PY{l+s}{\PYZdq{}}\PY{l+s}{sigma from gaus fit}\PY{l+s}{\PYZdq{}}\PY{p}{,}
                        \PY{l+m+mi}{50}\PY{p}{,}\PY{l+m+mf}{0.5}\PY{p}{,}\PY{l+m+mf}{1.5}\PY{p}{)}\PY{p}{;}
             \PY{n}{TH1F} \PY{n+nf}{pull}\PY{p}{(}\PY{n}{method\PYZus{}prefix}\PY{o}{+}\PY{l+s}{\PYZdq{}}\PY{l+s}{pull}\PY{l+s}{\PYZdq{}}\PY{p}{,}
                       \PY{n}{method\PYZus{}prefix}\PY{o}{+}\PY{l+s}{\PYZdq{}}\PY{l+s}{pull from gaus fit}\PY{l+s}{\PYZdq{}}\PY{p}{,}
                       \PY{l+m+mi}{50}\PY{p}{,}\PY{o}{\PYZhy{}}\PY{l+m+mf}{4.}\PY{p}{,}\PY{l+m+mf}{4.}\PY{p}{)}\PY{p}{;}
             \PY{c+c1}{// Make a nice devided canvas}
             \PY{k}{auto} \PY{o}{*}\PY{n}{canvas\PYZus{}6\PYZus{}2} \PY{o}{=} \PY{k}{new} \PY{n}{TCanvas}\PY{p}{(}\PY{n}{method\PYZus{}prefix}\PY{o}{+}\PY{l+s}{\PYZdq{}}\PY{l+s}{canvas\PYZus{}6\PYZus{}2}\PY{l+s}{\PYZdq{}}\PY{p}{,}\PY{n}{method\PYZus{}prefix}\PY{o}{+}\PY{l+s}{\PYZdq{}}\PY{l+s}{canvas\PYZus{}6\PYZus{}2}\PY{l+s}{\PYZdq{}}\PY{p}{,}\PY{l+m+mi}{800}\PY{p}{,}\PY{l+m+mi}{400}\PY{p}{)}\PY{p}{;}
             \PY{n}{canvas\PYZus{}6\PYZus{}2}\PY{o}{\PYZhy{}}\PY{o}{\PYZgt{}}\PY{n}{Divide}\PY{p}{(}\PY{l+m+mi}{2}\PY{p}{,}\PY{l+m+mi}{1}\PY{p}{)}\PY{p}{;}
             \PY{n}{canvas\PYZus{}6\PYZus{}2}\PY{o}{\PYZhy{}}\PY{o}{\PYZgt{}}\PY{n}{cd}\PY{p}{(}\PY{l+m+mi}{1}\PY{p}{)}\PY{p}{;}\PY{n}{canvas\PYZus{}6\PYZus{}2}\PY{o}{\PYZhy{}}\PY{o}{\PYZgt{}}\PY{n}{SetGrid}\PY{p}{(}\PY{p}{)}\PY{p}{;}
             
             
             \PY{k+kt}{float} \PY{n}{sig}\PY{p}{,} \PY{n}{mean}\PY{p}{;}
             \PY{k}{for} \PY{p}{(}\PY{k+kt}{int} \PY{n}{i}\PY{o}{=}\PY{l+m+mi}{0}\PY{p}{;} \PY{n}{i}\PY{o}{\PYZlt{}}\PY{n}{n\PYZus{}toys}\PY{p}{;} \PY{n}{i}\PY{o}{+}\PY{o}{+}\PY{p}{)}\PY{p}{\PYZob{}}
              \PY{c+c1}{// Reset histo contents}
                 \PY{n}{h4}\PY{p}{.}\PY{n}{Reset}\PY{p}{(}\PY{p}{)}\PY{p}{;}
              \PY{c+c1}{// Fill histo}
                 \PY{k}{for} \PY{p}{(} \PY{k+kt}{int} \PY{n}{j} \PY{o}{=} \PY{l+m+mi}{0}\PY{p}{;} \PY{n}{j}\PY{o}{\PYZlt{}}\PY{n}{n\PYZus{}tot\PYZus{}entries}\PY{p}{;} \PY{n}{j}\PY{o}{+}\PY{o}{+} \PY{p}{)}
                 \PY{n}{h4}\PY{p}{.}\PY{n}{Fill}\PY{p}{(}\PY{n}{gRandom}\PY{o}{\PYZhy{}}\PY{o}{\PYZgt{}}\PY{n}{Gaus}\PY{p}{(}\PY{p}{)}\PY{p}{)}\PY{p}{;}
              \PY{c+c1}{// perform fit}
                 \PY{k}{if} \PY{p}{(}\PY{n}{do\PYZus{}chi2}\PY{p}{)} \PY{n}{h4}\PY{p}{.}\PY{n}{Fit}\PY{p}{(}\PY{l+s}{\PYZdq{}}\PY{l+s}{gaus}\PY{l+s}{\PYZdq{}}\PY{p}{,}\PY{l+s}{\PYZdq{}}\PY{l+s}{q}\PY{l+s}{\PYZdq{}}\PY{p}{)}\PY{p}{;} \PY{c+c1}{// Chi2 fit}
                 \PY{k}{else} \PY{n}{h4}\PY{p}{.}\PY{n}{Fit}\PY{p}{(}\PY{l+s}{\PYZdq{}}\PY{l+s}{gaus}\PY{l+s}{\PYZdq{}}\PY{p}{,}\PY{l+s}{\PYZdq{}}\PY{l+s}{lq}\PY{l+s}{\PYZdq{}}\PY{p}{)}\PY{p}{;} \PY{c+c1}{// Likelihood fit}
              \PY{c+c1}{// some control output on the way}
                 \PY{k}{if} \PY{p}{(}\PY{o}{!}\PY{p}{(}\PY{n}{i}\PY{o}{\PYZpc{}}\PY{l+m+mi}{100}\PY{p}{)}\PY{p}{)}\PY{p}{\PYZob{}}
                     \PY{n}{h4}\PY{p}{.}\PY{n}{Draw}\PY{p}{(}\PY{l+s}{\PYZdq{}}\PY{l+s}{ep}\PY{l+s}{\PYZdq{}}\PY{p}{)}\PY{p}{;}
                     \PY{n}{canvas\PYZus{}6\PYZus{}2}\PY{o}{\PYZhy{}}\PY{o}{\PYZgt{}}\PY{n}{Update}\PY{p}{(}\PY{p}{)}\PY{p}{;}
                 \PY{p}{\PYZcb{}}
         
              \PY{c+c1}{// Get sigma from fit}
                 \PY{n}{TF1} \PY{o}{*}\PY{n}{fit} \PY{o}{=} \PY{n}{h4}\PY{p}{.}\PY{n}{GetFunction}\PY{p}{(}\PY{l+s}{\PYZdq{}}\PY{l+s}{gaus}\PY{l+s}{\PYZdq{}}\PY{p}{)}\PY{p}{;}
                 \PY{n}{sig} \PY{o}{=} \PY{n}{fit}\PY{o}{\PYZhy{}}\PY{o}{\PYZgt{}}\PY{n}{GetParameter}\PY{p}{(}\PY{l+m+mi}{2}\PY{p}{)}\PY{p}{;}
                 \PY{n}{mean}\PY{o}{=} \PY{n}{fit}\PY{o}{\PYZhy{}}\PY{o}{\PYZgt{}}\PY{n}{GetParameter}\PY{p}{(}\PY{l+m+mi}{1}\PY{p}{)}\PY{p}{;}
                 \PY{n}{sigma}\PY{p}{.}\PY{n}{Fill}\PY{p}{(}\PY{n}{sig}\PY{p}{)}\PY{p}{;}
                 \PY{n}{pull}\PY{p}{.}\PY{n}{Fill}\PY{p}{(}\PY{n}{mean}\PY{o}{/}\PY{n}{sig} \PY{o}{*} \PY{n}{sqrt}\PY{p}{(}\PY{n}{n\PYZus{}tot\PYZus{}entries}\PY{p}{)}\PY{p}{)}\PY{p}{;}
                \PY{p}{\PYZcb{}} \PY{c+c1}{// end of toy MC loop }
                \PY{n}{h4}\PY{p}{.}\PY{n}{DrawClone}\PY{p}{(}\PY{l+s}{\PYZdq{}}\PY{l+s}{ep}\PY{l+s}{\PYZdq{}}\PY{p}{)}\PY{p}{;}
                 \PY{n}{canvas\PYZus{}6\PYZus{}2}\PY{o}{\PYZhy{}}\PY{o}{\PYZgt{}}\PY{n}{Draw}\PY{p}{(}\PY{p}{)}\PY{p}{;}
                 \PY{n}{canvas\PYZus{}6\PYZus{}2}\PY{o}{\PYZhy{}}\PY{o}{\PYZgt{}}\PY{n}{cd}\PY{p}{(}\PY{l+m+mi}{2}\PY{p}{)}\PY{p}{;}
              \PY{c+c1}{// print result}
                 \PY{n}{pull}\PY{p}{.}\PY{n}{DrawClone}\PY{p}{(}\PY{p}{)}\PY{p}{;}
                 \PY{n}{canvas\PYZus{}6\PYZus{}2}\PY{o}{\PYZhy{}}\PY{o}{\PYZgt{}}\PY{n}{Draw}\PY{p}{(}\PY{p}{)}\PY{p}{;}
             
         \PY{p}{\PYZcb{}}
\end{Verbatim}

    \begin{Verbatim}[commandchars=\\\{\}]
{\color{incolor}In [{\color{incolor}11}]:} \PY{k+kt}{int} \PY{n}{n\PYZus{}toys}\PY{o}{=}\PY{l+m+mi}{10000}\PY{p}{;}
         \PY{k+kt}{int} \PY{n}{n\PYZus{}tot\PYZus{}entries}\PY{o}{=}\PY{l+m+mi}{100}\PY{p}{;}
         \PY{k+kt}{int} \PY{n}{n\PYZus{}bins}\PY{o}{=}\PY{l+m+mi}{40}\PY{p}{;}
         \PY{n}{cout} \PY{o}{\PYZlt{}}\PY{o}{\PYZlt{}} \PY{l+s}{\PYZdq{}}\PY{l+s}{Performing Pull Experiment with chi2 }\PY{l+s+se}{\PYZbs{}n}\PY{l+s}{\PYZdq{}}\PY{p}{;}
         \PY{n}{pull}\PY{p}{(}\PY{n}{n\PYZus{}toys}\PY{p}{,}\PY{n}{n\PYZus{}tot\PYZus{}entries}\PY{p}{,}\PY{n}{n\PYZus{}bins}\PY{p}{,}\PY{n+nb}{true}\PY{p}{)}\PY{p}{;}
         \PY{n}{cout} \PY{o}{\PYZlt{}}\PY{o}{\PYZlt{}} \PY{l+s}{\PYZdq{}}\PY{l+s}{Performing Pull Experiment with Log Likelihood}\PY{l+s+se}{\PYZbs{}n}\PY{l+s}{\PYZdq{}}\PY{p}{;}
         \PY{n}{pull}\PY{p}{(}\PY{n}{n\PYZus{}toys}\PY{p}{,}\PY{n}{n\PYZus{}tot\PYZus{}entries}\PY{p}{,}\PY{n}{n\PYZus{}bins}\PY{p}{,}\PY{n+nb}{false}\PY{p}{)}\PY{p}{;}
\end{Verbatim}

    \begin{Verbatim}[commandchars=\\\{\}]
Performing Pull Experiment with chi2 
Performing Pull Experiment with Log Likelihood
    \end{Verbatim}

    \begin{center}
    \adjustimage{max size={0.9\linewidth}{0.9\paperheight}}{6-Functions-and-Parameter-Estimation_files/6-Functions-and-Parameter-Estimation_20_1.png}
    \end{center}
    { \hspace*{\fill} \\}
    
    \begin{center}
    \adjustimage{max size={0.9\linewidth}{0.9\paperheight}}{6-Functions-and-Parameter-Estimation_files/6-Functions-and-Parameter-Estimation_20_2.png}
    \end{center}
    { \hspace*{\fill} \\}
    
    Your present knowledge of ROOT should be enough to understand all the
technicalities behind the macro. Note that the variable pull in line 61
is different from the definition above: instead of the parameter error
on mean, the fitted standard deviation of the distribution divided by
the square root of the number of entries,
\texttt{sig/sqrt(n\_tot\_entries)}, is used.

\begin{itemize}
\item
  What method exhibits the better performance with the default
  parameters?
\item
  What happens if you increase the number of entries per histogram by a
  factor of ten? Why?
\end{itemize}

The answers to these questions are well beyond the scope of this guide.
Basically all books about statistical methods provide a complete
treatment of the aforementioned topics.


    % Add a bibliography block to the postdoc
    
    
    
    \end{document}
