
% Default to the notebook output style

    


% Inherit from the specified cell style.




    
\documentclass{article}

    
    
    \usepackage{graphicx} % Used to insert images
    \usepackage{adjustbox} % Used to constrain images to a maximum size 
    \usepackage{color} % Allow colors to be defined
    \usepackage{enumerate} % Needed for markdown enumerations to work
    \usepackage{geometry} % Used to adjust the document margins
    \usepackage{amsmath} % Equations
    \usepackage{amssymb} % Equations
    \usepackage{eurosym} % defines \euro
    \usepackage[mathletters]{ucs} % Extended unicode (utf-8) support
    \usepackage[utf8x]{inputenc} % Allow utf-8 characters in the tex document
    \usepackage{fancyvrb} % verbatim replacement that allows latex
    \usepackage{grffile} % extends the file name processing of package graphics 
                         % to support a larger range 
    % The hyperref package gives us a pdf with properly built
    % internal navigation ('pdf bookmarks' for the table of contents,
    % internal cross-reference links, web links for URLs, etc.)
    \usepackage{hyperref}
    \usepackage{longtable} % longtable support required by pandoc >1.10
    \usepackage{booktabs}  % table support for pandoc > 1.12.2
    \usepackage{ulem} % ulem is needed to support strikethroughs (\sout)
    

    
    
    \definecolor{orange}{cmyk}{0,0.4,0.8,0.2}
    \definecolor{darkorange}{rgb}{.71,0.21,0.01}
    \definecolor{darkgreen}{rgb}{.12,.54,.11}
    \definecolor{myteal}{rgb}{.26, .44, .56}
    \definecolor{gray}{gray}{0.45}
    \definecolor{lightgray}{gray}{.95}
    \definecolor{mediumgray}{gray}{.8}
    \definecolor{inputbackground}{rgb}{.95, .95, .85}
    \definecolor{outputbackground}{rgb}{.95, .95, .95}
    \definecolor{traceback}{rgb}{1, .95, .95}
    % ansi colors
    \definecolor{red}{rgb}{.6,0,0}
    \definecolor{green}{rgb}{0,.65,0}
    \definecolor{brown}{rgb}{0.6,0.6,0}
    \definecolor{blue}{rgb}{0,.145,.698}
    \definecolor{purple}{rgb}{.698,.145,.698}
    \definecolor{cyan}{rgb}{0,.698,.698}
    \definecolor{lightgray}{gray}{0.5}
    
    % bright ansi colors
    \definecolor{darkgray}{gray}{0.25}
    \definecolor{lightred}{rgb}{1.0,0.39,0.28}
    \definecolor{lightgreen}{rgb}{0.48,0.99,0.0}
    \definecolor{lightblue}{rgb}{0.53,0.81,0.92}
    \definecolor{lightpurple}{rgb}{0.87,0.63,0.87}
    \definecolor{lightcyan}{rgb}{0.5,1.0,0.83}
    
    % commands and environments needed by pandoc snippets
    % extracted from the output of `pandoc -s`
    \providecommand{\tightlist}{%
      \setlength{\itemsep}{0pt}\setlength{\parskip}{0pt}}
    \DefineVerbatimEnvironment{Highlighting}{Verbatim}{commandchars=\\\{\}}
    % Add ',fontsize=\small' for more characters per line
    \newenvironment{Shaded}{}{}
    \newcommand{\KeywordTok}[1]{\textcolor[rgb]{0.00,0.44,0.13}{\textbf{{#1}}}}
    \newcommand{\DataTypeTok}[1]{\textcolor[rgb]{0.56,0.13,0.00}{{#1}}}
    \newcommand{\DecValTok}[1]{\textcolor[rgb]{0.25,0.63,0.44}{{#1}}}
    \newcommand{\BaseNTok}[1]{\textcolor[rgb]{0.25,0.63,0.44}{{#1}}}
    \newcommand{\FloatTok}[1]{\textcolor[rgb]{0.25,0.63,0.44}{{#1}}}
    \newcommand{\CharTok}[1]{\textcolor[rgb]{0.25,0.44,0.63}{{#1}}}
    \newcommand{\StringTok}[1]{\textcolor[rgb]{0.25,0.44,0.63}{{#1}}}
    \newcommand{\CommentTok}[1]{\textcolor[rgb]{0.38,0.63,0.69}{\textit{{#1}}}}
    \newcommand{\OtherTok}[1]{\textcolor[rgb]{0.00,0.44,0.13}{{#1}}}
    \newcommand{\AlertTok}[1]{\textcolor[rgb]{1.00,0.00,0.00}{\textbf{{#1}}}}
    \newcommand{\FunctionTok}[1]{\textcolor[rgb]{0.02,0.16,0.49}{{#1}}}
    \newcommand{\RegionMarkerTok}[1]{{#1}}
    \newcommand{\ErrorTok}[1]{\textcolor[rgb]{1.00,0.00,0.00}{\textbf{{#1}}}}
    \newcommand{\NormalTok}[1]{{#1}}
    
    % Additional commands for more recent versions of Pandoc
    \newcommand{\ConstantTok}[1]{\textcolor[rgb]{0.53,0.00,0.00}{{#1}}}
    \newcommand{\SpecialCharTok}[1]{\textcolor[rgb]{0.25,0.44,0.63}{{#1}}}
    \newcommand{\VerbatimStringTok}[1]{\textcolor[rgb]{0.25,0.44,0.63}{{#1}}}
    \newcommand{\SpecialStringTok}[1]{\textcolor[rgb]{0.73,0.40,0.53}{{#1}}}
    \newcommand{\ImportTok}[1]{{#1}}
    \newcommand{\DocumentationTok}[1]{\textcolor[rgb]{0.73,0.13,0.13}{\textit{{#1}}}}
    \newcommand{\AnnotationTok}[1]{\textcolor[rgb]{0.38,0.63,0.69}{\textbf{\textit{{#1}}}}}
    \newcommand{\CommentVarTok}[1]{\textcolor[rgb]{0.38,0.63,0.69}{\textbf{\textit{{#1}}}}}
    \newcommand{\VariableTok}[1]{\textcolor[rgb]{0.10,0.09,0.49}{{#1}}}
    \newcommand{\ControlFlowTok}[1]{\textcolor[rgb]{0.00,0.44,0.13}{\textbf{{#1}}}}
    \newcommand{\OperatorTok}[1]{\textcolor[rgb]{0.40,0.40,0.40}{{#1}}}
    \newcommand{\BuiltInTok}[1]{{#1}}
    \newcommand{\ExtensionTok}[1]{{#1}}
    \newcommand{\PreprocessorTok}[1]{\textcolor[rgb]{0.74,0.48,0.00}{{#1}}}
    \newcommand{\AttributeTok}[1]{\textcolor[rgb]{0.49,0.56,0.16}{{#1}}}
    \newcommand{\InformationTok}[1]{\textcolor[rgb]{0.38,0.63,0.69}{\textbf{\textit{{#1}}}}}
    \newcommand{\WarningTok}[1]{\textcolor[rgb]{0.38,0.63,0.69}{\textbf{\textit{{#1}}}}}
    
    
    % Define a nice break command that doesn't care if a line doesn't already
    % exist.
    \def\br{\hspace*{\fill} \\* }
    % Math Jax compatability definitions
    \def\gt{>}
    \def\lt{<}
    % Document parameters
    \title{7-File-IO-and-Parallel-Analysis}
    
    
    

    % Pygments definitions
    
\makeatletter
\def\PY@reset{\let\PY@it=\relax \let\PY@bf=\relax%
    \let\PY@ul=\relax \let\PY@tc=\relax%
    \let\PY@bc=\relax \let\PY@ff=\relax}
\def\PY@tok#1{\csname PY@tok@#1\endcsname}
\def\PY@toks#1+{\ifx\relax#1\empty\else%
    \PY@tok{#1}\expandafter\PY@toks\fi}
\def\PY@do#1{\PY@bc{\PY@tc{\PY@ul{%
    \PY@it{\PY@bf{\PY@ff{#1}}}}}}}
\def\PY#1#2{\PY@reset\PY@toks#1+\relax+\PY@do{#2}}

\expandafter\def\csname PY@tok@nd\endcsname{\def\PY@tc##1{\textcolor[rgb]{0.67,0.13,1.00}{##1}}}
\expandafter\def\csname PY@tok@mb\endcsname{\def\PY@tc##1{\textcolor[rgb]{0.40,0.40,0.40}{##1}}}
\expandafter\def\csname PY@tok@gs\endcsname{\let\PY@bf=\textbf}
\expandafter\def\csname PY@tok@nb\endcsname{\def\PY@tc##1{\textcolor[rgb]{0.00,0.50,0.00}{##1}}}
\expandafter\def\csname PY@tok@mf\endcsname{\def\PY@tc##1{\textcolor[rgb]{0.40,0.40,0.40}{##1}}}
\expandafter\def\csname PY@tok@bp\endcsname{\def\PY@tc##1{\textcolor[rgb]{0.00,0.50,0.00}{##1}}}
\expandafter\def\csname PY@tok@gh\endcsname{\let\PY@bf=\textbf\def\PY@tc##1{\textcolor[rgb]{0.00,0.00,0.50}{##1}}}
\expandafter\def\csname PY@tok@si\endcsname{\let\PY@bf=\textbf\def\PY@tc##1{\textcolor[rgb]{0.73,0.40,0.53}{##1}}}
\expandafter\def\csname PY@tok@gt\endcsname{\def\PY@tc##1{\textcolor[rgb]{0.00,0.27,0.87}{##1}}}
\expandafter\def\csname PY@tok@s\endcsname{\def\PY@tc##1{\textcolor[rgb]{0.73,0.13,0.13}{##1}}}
\expandafter\def\csname PY@tok@gu\endcsname{\let\PY@bf=\textbf\def\PY@tc##1{\textcolor[rgb]{0.50,0.00,0.50}{##1}}}
\expandafter\def\csname PY@tok@ge\endcsname{\let\PY@it=\textit}
\expandafter\def\csname PY@tok@nt\endcsname{\let\PY@bf=\textbf\def\PY@tc##1{\textcolor[rgb]{0.00,0.50,0.00}{##1}}}
\expandafter\def\csname PY@tok@kr\endcsname{\let\PY@bf=\textbf\def\PY@tc##1{\textcolor[rgb]{0.00,0.50,0.00}{##1}}}
\expandafter\def\csname PY@tok@cpf\endcsname{\let\PY@it=\textit\def\PY@tc##1{\textcolor[rgb]{0.25,0.50,0.50}{##1}}}
\expandafter\def\csname PY@tok@vi\endcsname{\def\PY@tc##1{\textcolor[rgb]{0.10,0.09,0.49}{##1}}}
\expandafter\def\csname PY@tok@sx\endcsname{\def\PY@tc##1{\textcolor[rgb]{0.00,0.50,0.00}{##1}}}
\expandafter\def\csname PY@tok@nc\endcsname{\let\PY@bf=\textbf\def\PY@tc##1{\textcolor[rgb]{0.00,0.00,1.00}{##1}}}
\expandafter\def\csname PY@tok@s1\endcsname{\def\PY@tc##1{\textcolor[rgb]{0.73,0.13,0.13}{##1}}}
\expandafter\def\csname PY@tok@sc\endcsname{\def\PY@tc##1{\textcolor[rgb]{0.73,0.13,0.13}{##1}}}
\expandafter\def\csname PY@tok@sr\endcsname{\def\PY@tc##1{\textcolor[rgb]{0.73,0.40,0.53}{##1}}}
\expandafter\def\csname PY@tok@nn\endcsname{\let\PY@bf=\textbf\def\PY@tc##1{\textcolor[rgb]{0.00,0.00,1.00}{##1}}}
\expandafter\def\csname PY@tok@gp\endcsname{\let\PY@bf=\textbf\def\PY@tc##1{\textcolor[rgb]{0.00,0.00,0.50}{##1}}}
\expandafter\def\csname PY@tok@cm\endcsname{\let\PY@it=\textit\def\PY@tc##1{\textcolor[rgb]{0.25,0.50,0.50}{##1}}}
\expandafter\def\csname PY@tok@kn\endcsname{\let\PY@bf=\textbf\def\PY@tc##1{\textcolor[rgb]{0.00,0.50,0.00}{##1}}}
\expandafter\def\csname PY@tok@kc\endcsname{\let\PY@bf=\textbf\def\PY@tc##1{\textcolor[rgb]{0.00,0.50,0.00}{##1}}}
\expandafter\def\csname PY@tok@mo\endcsname{\def\PY@tc##1{\textcolor[rgb]{0.40,0.40,0.40}{##1}}}
\expandafter\def\csname PY@tok@cs\endcsname{\let\PY@it=\textit\def\PY@tc##1{\textcolor[rgb]{0.25,0.50,0.50}{##1}}}
\expandafter\def\csname PY@tok@na\endcsname{\def\PY@tc##1{\textcolor[rgb]{0.49,0.56,0.16}{##1}}}
\expandafter\def\csname PY@tok@vc\endcsname{\def\PY@tc##1{\textcolor[rgb]{0.10,0.09,0.49}{##1}}}
\expandafter\def\csname PY@tok@nl\endcsname{\def\PY@tc##1{\textcolor[rgb]{0.63,0.63,0.00}{##1}}}
\expandafter\def\csname PY@tok@ow\endcsname{\let\PY@bf=\textbf\def\PY@tc##1{\textcolor[rgb]{0.67,0.13,1.00}{##1}}}
\expandafter\def\csname PY@tok@sd\endcsname{\let\PY@it=\textit\def\PY@tc##1{\textcolor[rgb]{0.73,0.13,0.13}{##1}}}
\expandafter\def\csname PY@tok@gd\endcsname{\def\PY@tc##1{\textcolor[rgb]{0.63,0.00,0.00}{##1}}}
\expandafter\def\csname PY@tok@c1\endcsname{\let\PY@it=\textit\def\PY@tc##1{\textcolor[rgb]{0.25,0.50,0.50}{##1}}}
\expandafter\def\csname PY@tok@kp\endcsname{\def\PY@tc##1{\textcolor[rgb]{0.00,0.50,0.00}{##1}}}
\expandafter\def\csname PY@tok@il\endcsname{\def\PY@tc##1{\textcolor[rgb]{0.40,0.40,0.40}{##1}}}
\expandafter\def\csname PY@tok@ni\endcsname{\let\PY@bf=\textbf\def\PY@tc##1{\textcolor[rgb]{0.60,0.60,0.60}{##1}}}
\expandafter\def\csname PY@tok@ss\endcsname{\def\PY@tc##1{\textcolor[rgb]{0.10,0.09,0.49}{##1}}}
\expandafter\def\csname PY@tok@c\endcsname{\let\PY@it=\textit\def\PY@tc##1{\textcolor[rgb]{0.25,0.50,0.50}{##1}}}
\expandafter\def\csname PY@tok@cp\endcsname{\def\PY@tc##1{\textcolor[rgb]{0.74,0.48,0.00}{##1}}}
\expandafter\def\csname PY@tok@o\endcsname{\def\PY@tc##1{\textcolor[rgb]{0.40,0.40,0.40}{##1}}}
\expandafter\def\csname PY@tok@kd\endcsname{\let\PY@bf=\textbf\def\PY@tc##1{\textcolor[rgb]{0.00,0.50,0.00}{##1}}}
\expandafter\def\csname PY@tok@go\endcsname{\def\PY@tc##1{\textcolor[rgb]{0.53,0.53,0.53}{##1}}}
\expandafter\def\csname PY@tok@kt\endcsname{\def\PY@tc##1{\textcolor[rgb]{0.69,0.00,0.25}{##1}}}
\expandafter\def\csname PY@tok@mi\endcsname{\def\PY@tc##1{\textcolor[rgb]{0.40,0.40,0.40}{##1}}}
\expandafter\def\csname PY@tok@no\endcsname{\def\PY@tc##1{\textcolor[rgb]{0.53,0.00,0.00}{##1}}}
\expandafter\def\csname PY@tok@ch\endcsname{\let\PY@it=\textit\def\PY@tc##1{\textcolor[rgb]{0.25,0.50,0.50}{##1}}}
\expandafter\def\csname PY@tok@ne\endcsname{\let\PY@bf=\textbf\def\PY@tc##1{\textcolor[rgb]{0.82,0.25,0.23}{##1}}}
\expandafter\def\csname PY@tok@gi\endcsname{\def\PY@tc##1{\textcolor[rgb]{0.00,0.63,0.00}{##1}}}
\expandafter\def\csname PY@tok@w\endcsname{\def\PY@tc##1{\textcolor[rgb]{0.73,0.73,0.73}{##1}}}
\expandafter\def\csname PY@tok@se\endcsname{\let\PY@bf=\textbf\def\PY@tc##1{\textcolor[rgb]{0.73,0.40,0.13}{##1}}}
\expandafter\def\csname PY@tok@s2\endcsname{\def\PY@tc##1{\textcolor[rgb]{0.73,0.13,0.13}{##1}}}
\expandafter\def\csname PY@tok@nv\endcsname{\def\PY@tc##1{\textcolor[rgb]{0.10,0.09,0.49}{##1}}}
\expandafter\def\csname PY@tok@m\endcsname{\def\PY@tc##1{\textcolor[rgb]{0.40,0.40,0.40}{##1}}}
\expandafter\def\csname PY@tok@k\endcsname{\let\PY@bf=\textbf\def\PY@tc##1{\textcolor[rgb]{0.00,0.50,0.00}{##1}}}
\expandafter\def\csname PY@tok@mh\endcsname{\def\PY@tc##1{\textcolor[rgb]{0.40,0.40,0.40}{##1}}}
\expandafter\def\csname PY@tok@gr\endcsname{\def\PY@tc##1{\textcolor[rgb]{1.00,0.00,0.00}{##1}}}
\expandafter\def\csname PY@tok@sb\endcsname{\def\PY@tc##1{\textcolor[rgb]{0.73,0.13,0.13}{##1}}}
\expandafter\def\csname PY@tok@sh\endcsname{\def\PY@tc##1{\textcolor[rgb]{0.73,0.13,0.13}{##1}}}
\expandafter\def\csname PY@tok@vg\endcsname{\def\PY@tc##1{\textcolor[rgb]{0.10,0.09,0.49}{##1}}}
\expandafter\def\csname PY@tok@nf\endcsname{\def\PY@tc##1{\textcolor[rgb]{0.00,0.00,1.00}{##1}}}
\expandafter\def\csname PY@tok@err\endcsname{\def\PY@bc##1{\setlength{\fboxsep}{0pt}\fcolorbox[rgb]{1.00,0.00,0.00}{1,1,1}{\strut ##1}}}

\def\PYZbs{\char`\\}
\def\PYZus{\char`\_}
\def\PYZob{\char`\{}
\def\PYZcb{\char`\}}
\def\PYZca{\char`\^}
\def\PYZam{\char`\&}
\def\PYZlt{\char`\<}
\def\PYZgt{\char`\>}
\def\PYZsh{\char`\#}
\def\PYZpc{\char`\%}
\def\PYZdl{\char`\$}
\def\PYZhy{\char`\-}
\def\PYZsq{\char`\'}
\def\PYZdq{\char`\"}
\def\PYZti{\char`\~}
% for compatibility with earlier versions
\def\PYZat{@}
\def\PYZlb{[}
\def\PYZrb{]}
\makeatother


    % Exact colors from NB
    \definecolor{incolor}{rgb}{0.0, 0.0, 0.5}
    \definecolor{outcolor}{rgb}{0.545, 0.0, 0.0}



    
    % Prevent overflowing lines due to hard-to-break entities
    \sloppy 
    % Setup hyperref package
    \hypersetup{
      breaklinks=true,  % so long urls are correctly broken across lines
      colorlinks=true,
      urlcolor=blue,
      linkcolor=darkorange,
      citecolor=darkgreen,
      }
    % Slightly bigger margins than the latex defaults
    
    \geometry{verbose,tmargin=1in,bmargin=1in,lmargin=1in,rmargin=1in}
    
    

    \begin{document}
    
    
    \maketitle
    
    

    
    \subsection{7.1 Storing ROOT Objects}\label{storing-root-objects}

ROOT offers the possibility to write instances of classes on disk, into
a ROOT-file (see the
\href{https://root.cern.ch/doc/master/classTFile.html}{\texttt{TFile}}
class for more details). One says that the object is made ``persistent''
by storing it on disk. When reading the file back, the object is
reconstructed in memory. The requirement to be satisfied to perform I/O
of instances of a certain class is that the ROOT type system is aware of
the layout in memory of that class. This topic is beyond the scope of
this document: it is worth to mention that I/O can be performed out of
the box for the almost complete set of ROOT classes.

We can explore this functionality with histograms and two simple macros.

    \begin{Verbatim}[commandchars=\\\{\}]
{\color{incolor}In [{\color{incolor}1}]:} \PY{o}{\PYZpc{}}\PY{o}{\PYZpc{}}\PY{n}{jsroot} \PY{n}{on}
\end{Verbatim}

    \begin{Verbatim}[commandchars=\\\{\}]
{\color{incolor}In [{\color{incolor}2}]:} \PY{c+c1}{// Instance of our histogram}
        \PY{n}{TH1F} \PY{n}{histogram\PYZus{}7\PYZus{}1}\PY{p}{(}\PY{l+s}{\PYZdq{}}\PY{l+s}{stored\PYZus{}histogram}\PY{l+s}{\PYZdq{}}\PY{p}{,}\PY{l+s}{\PYZdq{}}\PY{l+s}{My Title;X;\PYZsh{} of entries}\PY{l+s}{\PYZdq{}}\PY{p}{,}\PY{l+m+mi}{100}\PY{p}{,}\PY{o}{\PYZhy{}}\PY{l+m+mi}{5}\PY{p}{,}\PY{l+m+mi}{5}\PY{p}{)}\PY{p}{;}
        
        \PY{c+c1}{// Let\PYZsq{}s fill it randomly}
        \PY{n}{histogram\PYZus{}7\PYZus{}1}\PY{p}{.}\PY{n}{FillRandom}\PY{p}{(}\PY{l+s}{\PYZdq{}}\PY{l+s}{gaus}\PY{l+s}{\PYZdq{}}\PY{p}{)}\PY{p}{;}
        
        \PY{c+c1}{// Let\PYZsq{}s open a TFile}
        \PY{n}{TFile} \PY{n+nf}{out\PYZus{}file}\PY{p}{(}\PY{l+s}{\PYZdq{}}\PY{l+s}{../data/my\PYZus{}rootfile.root}\PY{l+s}{\PYZdq{}}\PY{p}{,}\PY{l+s}{\PYZdq{}}\PY{l+s}{RECREATE}\PY{l+s}{\PYZdq{}}\PY{p}{)}\PY{p}{;}
        
        \PY{c+c1}{// Write the histogram in the file}
        \PY{n}{histogram\PYZus{}7\PYZus{}1}\PY{p}{.}\PY{n}{Write}\PY{p}{(}\PY{p}{)}\PY{p}{;}
        \PY{c+c1}{// Close the file}
        \PY{n}{out\PYZus{}file}\PY{p}{.}\PY{n}{Close}\PY{p}{(}\PY{p}{)}\PY{p}{;}
\end{Verbatim}

    Not bad, eh ? Especially for a language that does not foresees
persistency natively like C++. The RECREATE option forces ROOT to create
a new file even if a file with the same name exists on disk.

Now, you may use the Cling command line to access information in the
file and draw the previously written histogram:

\begin{verbatim}
>  root my_rootfile.root
root [0]
Attaching file my_rootfile.root as _file0...
root [1] _file0->ls()
TFile**     my_rootfile.root
 TFile*     my_rootfile.root
  KEY: TH1F my_histogram;1 My Title
root [2] my_histogram->Draw()
\end{verbatim}

Alternatively, you can use a simple macro to carry out the job:

    \begin{Verbatim}[commandchars=\\\{\}]
{\color{incolor}In [{\color{incolor}3}]:} \PY{c+c1}{// \PYZpc{}jsroot debug}
        \PY{k}{auto} \PY{n}{canvas\PYZus{}7\PYZus{}1}\PY{o}{=}\PY{k}{new} \PY{n}{TCanvas}\PY{p}{(}\PY{l+s}{\PYZdq{}}\PY{l+s}{canvas\PYZus{}7\PYZus{}1}\PY{l+s}{\PYZdq{}}\PY{p}{,}\PY{l+s}{\PYZdq{}}\PY{l+s}{canvas\PYZus{}7\PYZus{}1}\PY{l+s}{\PYZdq{}}\PY{p}{)}\PY{p}{;}
        \PY{c+c1}{// Let\PYZsq{}s open the TFile}
        \PY{n}{TFile} \PY{n+nf}{in\PYZus{}file}\PY{p}{(}\PY{l+s}{\PYZdq{}}\PY{l+s}{../data/my\PYZus{}rootfile.root}\PY{l+s}{\PYZdq{}}\PY{p}{)}\PY{p}{;}
        
        \PY{c+c1}{// Get the Histogram out}
        \PY{n}{TH1F}\PY{o}{*} \PY{n}{histogram\PYZus{}7\PYZus{}1\PYZus{}2}\PY{p}{;}
        \PY{n}{in\PYZus{}file}\PY{p}{.}\PY{n}{GetObject}\PY{p}{(}\PY{l+s}{\PYZdq{}}\PY{l+s}{stored\PYZus{}histogram}\PY{l+s}{\PYZdq{}}\PY{p}{,}\PY{n}{histogram\PYZus{}7\PYZus{}1\PYZus{}2}\PY{p}{)}\PY{p}{;}
        
        \PY{c+c1}{// Draw it}
        \PY{n}{histogram\PYZus{}7\PYZus{}1\PYZus{}2}\PY{o}{\PYZhy{}}\PY{o}{\PYZgt{}}\PY{n}{Draw}\PY{p}{(}\PY{p}{)}\PY{p}{;}
        \PY{n}{canvas\PYZus{}7\PYZus{}1}\PY{o}{\PYZhy{}}\PY{o}{\PYZgt{}}\PY{n}{Draw}\PY{p}{(}\PY{p}{)}\PY{p}{;}
\end{Verbatim}

    
    \begin{verbatim}
<IPython.core.display.HTML object>
    \end{verbatim}

    
    \subsection{7.2 N-tuples in ROOT}\label{n-tuples-in-root}

\subsubsection{7.2.1 Storing simple
N-tuples}\label{storing-simple-n-tuples}

Up to now we have seen how to manipulate input read from ASCII files.
ROOT offers the possibility to do much better than that, with its own
n-tuple classes. Among the many advantages provided by these classes one
could cite

\begin{itemize}
\item
  Optimised disk I/O.
\item
  Possibility to store many n-tuple rows.
\item
  Write the n-tuples in ROOT files.
\item
  Interactive inspection with
  \href{https://root.cern.ch/doc/master/classTBrowser.html}{\texttt{TBrowser}}.
\item
  Store not only numbers, but also objects in the columns.
\end{itemize}

In this section we will discuss briefly the
\href{https://root.cern.ch/doc/v606/classTNtuple.html}{\texttt{TNtuple}}
class, which is a simplified version of the
\href{https://root.cern.ch/doc/v606/classTTree.html}{\texttt{TTree}}
class. A ROOT
\href{https://root.cern.ch/doc/v606/classTNtuple.html}{\texttt{TNtuple}}
object can store rows of float entries. Let's tackle the problem
according to the usual strategy commenting a minimal example

    \begin{Verbatim}[commandchars=\\\{\}]
{\color{incolor}In [{\color{incolor}4}]:} \PY{c+c1}{// Fill an n\PYZhy{}tuple and write it to a file simulating measurement of}
        \PY{c+c1}{// conductivity of a material in different conditions of pressure}
        \PY{c+c1}{// and temperature.}
        
            \PY{n}{TFile} \PY{n+nf}{ofile}\PY{p}{(}\PY{l+s}{\PYZdq{}}\PY{l+s}{../data/conductivity\PYZus{}experiment.root}\PY{l+s}{\PYZdq{}}\PY{p}{,}\PY{l+s}{\PYZdq{}}\PY{l+s}{RECREATE}\PY{l+s}{\PYZdq{}}\PY{p}{)}\PY{p}{;}
        
            \PY{c+c1}{// Initialise the TNtuple}
            \PY{n}{TNtuple} \PY{n+nf}{cond\PYZus{}data}\PY{p}{(}\PY{l+s}{\PYZdq{}}\PY{l+s}{cond\PYZus{}data}\PY{l+s}{\PYZdq{}}\PY{p}{,}
                              \PY{l+s}{\PYZdq{}}\PY{l+s}{Example N\PYZhy{}Tuple}\PY{l+s}{\PYZdq{}}\PY{p}{,}
                              \PY{l+s}{\PYZdq{}}\PY{l+s}{Potential:Current:Temperature:Pressure}\PY{l+s}{\PYZdq{}}\PY{p}{)}\PY{p}{;}
        
            \PY{c+c1}{// Fill it randomly to fake the acquired data}
            \PY{n}{TRandom3} \PY{n}{rndm}\PY{p}{;}
            \PY{k+kt}{float} \PY{n}{pot}\PY{p}{,}\PY{n}{cur}\PY{p}{,}\PY{n}{temp}\PY{p}{,}\PY{n}{pres}\PY{p}{;}
            \PY{k}{for} \PY{p}{(}\PY{k+kt}{int} \PY{n}{i}\PY{o}{=}\PY{l+m+mi}{0}\PY{p}{;}\PY{n}{i}\PY{o}{\PYZlt{}}\PY{l+m+mi}{10000}\PY{p}{;}\PY{o}{+}\PY{o}{+}\PY{n}{i}\PY{p}{)}\PY{p}{\PYZob{}}
                \PY{n}{pot}\PY{o}{=}\PY{n}{rndm}\PY{p}{.}\PY{n}{Uniform}\PY{p}{(}\PY{l+m+mf}{0.}\PY{p}{,}\PY{l+m+mf}{10.}\PY{p}{)}\PY{p}{;}      \PY{c+c1}{// get voltage}
                \PY{n}{temp}\PY{o}{=}\PY{n}{rndm}\PY{p}{.}\PY{n}{Uniform}\PY{p}{(}\PY{l+m+mf}{250.}\PY{p}{,}\PY{l+m+mf}{350.}\PY{p}{)}\PY{p}{;}  \PY{c+c1}{// get temperature}
                \PY{n}{pres}\PY{o}{=}\PY{n}{rndm}\PY{p}{.}\PY{n}{Uniform}\PY{p}{(}\PY{l+m+mf}{0.5}\PY{p}{,}\PY{l+m+mf}{1.5}\PY{p}{)}\PY{p}{;}    \PY{c+c1}{// get pressure}
                \PY{n}{cur}\PY{o}{=}\PY{n}{pot}\PY{o}{/}\PY{p}{(}\PY{l+m+mf}{10.}\PY{o}{+}\PY{l+m+mf}{0.05}\PY{o}{*}\PY{p}{(}\PY{n}{temp}\PY{o}{\PYZhy{}}\PY{l+m+mf}{300.}\PY{p}{)}\PY{o}{\PYZhy{}}\PY{l+m+mf}{0.2}\PY{o}{*}\PY{p}{(}\PY{n}{pres}\PY{o}{\PYZhy{}}\PY{l+m+mf}{1.}\PY{p}{)}\PY{p}{)}\PY{p}{;} \PY{c+c1}{// current}
        \PY{c+c1}{// add some random smearing (measurement errors)}
                \PY{n}{pot}\PY{o}{*}\PY{o}{=}\PY{n}{rndm}\PY{p}{.}\PY{n}{Gaus}\PY{p}{(}\PY{l+m+mf}{1.}\PY{p}{,}\PY{l+m+mf}{0.01}\PY{p}{)}\PY{p}{;} \PY{c+c1}{// 1\PYZpc{} error on voltage}
                \PY{n}{temp}\PY{o}{+}\PY{o}{=}\PY{n}{rndm}\PY{p}{.}\PY{n}{Gaus}\PY{p}{(}\PY{l+m+mf}{0.}\PY{p}{,}\PY{l+m+mf}{0.3}\PY{p}{)}\PY{p}{;} \PY{c+c1}{// 0.3 abs. error on temp.}
                \PY{n}{pres}\PY{o}{*}\PY{o}{=}\PY{n}{rndm}\PY{p}{.}\PY{n}{Gaus}\PY{p}{(}\PY{l+m+mf}{1.}\PY{p}{,}\PY{l+m+mf}{0.02}\PY{p}{)}\PY{p}{;}\PY{c+c1}{// 1\PYZpc{} error on pressure}
                \PY{n}{cur}\PY{o}{*}\PY{o}{=}\PY{n}{rndm}\PY{p}{.}\PY{n}{Gaus}\PY{p}{(}\PY{l+m+mf}{1.}\PY{p}{,}\PY{l+m+mf}{0.01}\PY{p}{)}\PY{p}{;} \PY{c+c1}{// 1\PYZpc{} error on current}
        \PY{c+c1}{// write to ntuple}
                \PY{n}{cond\PYZus{}data}\PY{p}{.}\PY{n}{Fill}\PY{p}{(}\PY{n}{pot}\PY{p}{,}\PY{n}{cur}\PY{p}{,}\PY{n}{temp}\PY{p}{,}\PY{n}{pres}\PY{p}{)}\PY{p}{;}
                \PY{p}{\PYZcb{}}
        
            \PY{c+c1}{// Save the ntuple and close the file}
            \PY{n}{cond\PYZus{}data}\PY{p}{.}\PY{n}{Write}\PY{p}{(}\PY{p}{)}\PY{p}{;}
        \PY{c+c1}{//     ofile.Close();}
\end{Verbatim}

    This data written to this example n-tuple represents, in the statistical
sense, three independent variables (Potential or Voltage, Pressure and
Temperature), and one variable (Current) which depends on the others
according to very simple laws, and an additional Gaussian smearing. This
set of variables mimics a measurement of an electrical resistance while
varying pressure and temperature.

Imagine your task now consists in finding the relations among the
variables -- of course without knowing the code used to generate them.
You will see that the possibilities of the
\href{https://root.cern.ch/doc/v606/classTNtuple.html}{\texttt{NTuple}}
class enable you to perform this analysis task. Open the ROOT file
(\texttt{cond\_data.root}) written by the macro above in an interactive
session and use a
\href{https://root.cern.ch/doc/master/classTBrowser.html}{\texttt{TBrowser}}
to interactively inspect it:

\begin{verbatim}
root[0] TBrowser b
\end{verbatim}

You find the columns of your n-tuple written as leafs. Simply clicking
on them you can obtain histograms of the variables!

Next, try the following commands at the shell prompt and in the
interactive ROOT shell, respectively:

\begin{verbatim}
> root conductivity_experiment.root
Attaching file conductivity_experiment.root as _file0...
root [0] cond_data->Draw("Current:Potential")
\end{verbatim}

    \begin{Verbatim}[commandchars=\\\{\}]
{\color{incolor}In [{\color{incolor}5}]:} \PY{n}{cond\PYZus{}data}\PY{p}{.}\PY{n}{Draw}\PY{p}{(}\PY{l+s}{\PYZdq{}}\PY{l+s}{Current:Potential}\PY{l+s}{\PYZdq{}}\PY{p}{)}\PY{p}{;}
        \PY{n}{canvas\PYZus{}7\PYZus{}1}\PY{o}{\PYZhy{}}\PY{o}{\PYZgt{}}\PY{n}{Draw}\PY{p}{(}\PY{p}{)}\PY{p}{;}
\end{Verbatim}

    
    \begin{verbatim}
<IPython.core.display.HTML object>
    \end{verbatim}

    
    You just produced a correlation plot with one single line of code!

Try to extend the syntax typing for example

\begin{verbatim}
root [1] cond_data->Draw("Current:Potential","Temperature<270")
\end{verbatim}

    \begin{Verbatim}[commandchars=\\\{\}]
{\color{incolor}In [{\color{incolor}6}]:} \PY{n}{cond\PYZus{}data}\PY{p}{.}\PY{n}{Draw}\PY{p}{(}\PY{l+s}{\PYZdq{}}\PY{l+s}{Current:Potential}\PY{l+s}{\PYZdq{}}\PY{p}{,}\PY{l+s}{\PYZdq{}}\PY{l+s}{Temperature\PYZlt{}270}\PY{l+s}{\PYZdq{}}\PY{p}{)}\PY{p}{;}
        \PY{n}{canvas\PYZus{}7\PYZus{}1}\PY{o}{\PYZhy{}}\PY{o}{\PYZgt{}}\PY{n}{Draw}\PY{p}{(}\PY{p}{)}\PY{p}{;}
\end{Verbatim}

    
    \begin{verbatim}
<IPython.core.display.HTML object>
    \end{verbatim}

    
    What do you obtain ?

Now try

\begin{verbatim}
root [2] cond_data->Draw("Current/Potential:Temperature")
\end{verbatim}

You will see this result:

    \begin{Verbatim}[commandchars=\\\{\}]
{\color{incolor}In [{\color{incolor}7}]:} \PY{n}{cond\PYZus{}data}\PY{p}{.}\PY{n}{Draw}\PY{p}{(}\PY{l+s}{\PYZdq{}}\PY{l+s}{Current/Potential:Temperature}\PY{l+s}{\PYZdq{}}\PY{p}{)}\PY{p}{;}
        \PY{n}{canvas\PYZus{}7\PYZus{}1}\PY{o}{\PYZhy{}}\PY{o}{\PYZgt{}}\PY{n}{Draw}\PY{p}{(}\PY{p}{)}\PY{p}{;}
\end{Verbatim}

    
    \begin{verbatim}
<IPython.core.display.HTML object>
    \end{verbatim}

    
    It should have become clear from these examples how to navigate in such
a multi-dimensional space of variables and unveil relations between
variables using n-tuples.

\subsubsection{7.2.2 Reading N-tuples}\label{reading-n-tuples}

For completeness, you find here a small macro to read the data back from
a ROOT n-tuple

    \begin{Verbatim}[commandchars=\\\{\}]
{\color{incolor}In [{\color{incolor}8}]:} \PY{o}{\PYZpc{}}\PY{o}{\PYZpc{}}\PY{n}{cpp} \PY{o}{\PYZhy{}}\PY{n}{d}
        \PY{c+c1}{// Read the previously produced N\PYZhy{}Tuple and print on screen}
        \PY{c+c1}{// its content}
        
        \PY{k+kt}{void} \PY{n}{read\PYZus{}ntuple\PYZus{}from\PYZus{}file}\PY{p}{(}\PY{p}{)}\PY{p}{\PYZob{}}
        
            \PY{c+c1}{// Open a file, save the ntuple and close the file}
            \PY{n}{TFile} \PY{n}{in\PYZus{}file}\PY{p}{(}\PY{l+s}{\PYZdq{}}\PY{l+s}{../data/conductivity\PYZus{}experiment.root}\PY{l+s}{\PYZdq{}}\PY{p}{)}\PY{p}{;}
            \PY{n}{TNtuple}\PY{o}{*} \PY{n}{my\PYZus{}tuple}\PY{p}{;}\PY{n}{in\PYZus{}file}\PY{p}{.}\PY{n}{GetObject}\PY{p}{(}\PY{l+s}{\PYZdq{}}\PY{l+s}{cond\PYZus{}data}\PY{l+s}{\PYZdq{}}\PY{p}{,}\PY{n}{my\PYZus{}tuple}\PY{p}{)}\PY{p}{;}
            \PY{k+kt}{float} \PY{n}{pot}\PY{p}{,}\PY{n}{cur}\PY{p}{,}\PY{n}{temp}\PY{p}{,}\PY{n}{pres}\PY{p}{;} \PY{k+kt}{float}\PY{o}{*} \PY{n}{row\PYZus{}content}\PY{p}{;}
        
            \PY{n}{cout} \PY{o}{\PYZlt{}}\PY{o}{\PYZlt{}} \PY{l+s}{\PYZdq{}}\PY{l+s}{Potential}\PY{l+s+se}{\PYZbs{}t}\PY{l+s}{Current}\PY{l+s+se}{\PYZbs{}t}\PY{l+s}{Temperature}\PY{l+s+se}{\PYZbs{}t}\PY{l+s}{Pressure}\PY{l+s+se}{\PYZbs{}n}\PY{l+s}{\PYZdq{}}\PY{p}{;}
            \PY{k}{for} \PY{p}{(}\PY{k+kt}{int} \PY{n}{irow}\PY{o}{=}\PY{l+m+mi}{0}\PY{p}{;}\PY{n}{irow}\PY{o}{\PYZlt{}}\PY{n}{my\PYZus{}tuple}\PY{o}{\PYZhy{}}\PY{o}{\PYZgt{}}\PY{n}{GetEntries}\PY{p}{(}\PY{p}{)}\PY{p}{;}\PY{o}{+}\PY{o}{+}\PY{n}{irow}\PY{p}{)}\PY{p}{\PYZob{}}
                \PY{n}{my\PYZus{}tuple}\PY{o}{\PYZhy{}}\PY{o}{\PYZgt{}}\PY{n}{GetEntry}\PY{p}{(}\PY{n}{irow}\PY{p}{)}\PY{p}{;}
                \PY{n}{row\PYZus{}content} \PY{o}{=} \PY{n}{my\PYZus{}tuple}\PY{o}{\PYZhy{}}\PY{o}{\PYZgt{}}\PY{n}{GetArgs}\PY{p}{(}\PY{p}{)}\PY{p}{;}
                \PY{n}{pot} \PY{o}{=} \PY{n}{row\PYZus{}content}\PY{p}{[}\PY{l+m+mi}{0}\PY{p}{]}\PY{p}{;}
                \PY{n}{cur} \PY{o}{=} \PY{n}{row\PYZus{}content}\PY{p}{[}\PY{l+m+mi}{1}\PY{p}{]}\PY{p}{;}
                \PY{n}{temp} \PY{o}{=} \PY{n}{row\PYZus{}content}\PY{p}{[}\PY{l+m+mi}{2}\PY{p}{]}\PY{p}{;}
                \PY{n}{pres} \PY{o}{=} \PY{n}{row\PYZus{}content}\PY{p}{[}\PY{l+m+mi}{3}\PY{p}{]}\PY{p}{;}
                \PY{n}{cout} \PY{o}{\PYZlt{}}\PY{o}{\PYZlt{}} \PY{n}{pot} \PY{o}{\PYZlt{}}\PY{o}{\PYZlt{}} \PY{l+s}{\PYZdq{}}\PY{l+s+se}{\PYZbs{}t}\PY{l+s}{\PYZdq{}} \PY{o}{\PYZlt{}}\PY{o}{\PYZlt{}} \PY{n}{cur} \PY{o}{\PYZlt{}}\PY{o}{\PYZlt{}} \PY{l+s}{\PYZdq{}}\PY{l+s+se}{\PYZbs{}t}\PY{l+s}{\PYZdq{}} \PY{o}{\PYZlt{}}\PY{o}{\PYZlt{}} \PY{n}{temp}
                     \PY{o}{\PYZlt{}}\PY{o}{\PYZlt{}} \PY{l+s}{\PYZdq{}}\PY{l+s+se}{\PYZbs{}t}\PY{l+s}{\PYZdq{}} \PY{o}{\PYZlt{}}\PY{o}{\PYZlt{}} \PY{n}{pres} \PY{o}{\PYZlt{}}\PY{o}{\PYZlt{}} \PY{n}{endl}\PY{p}{;}
                \PY{p}{\PYZcb{}}
        
            \PY{p}{\PYZcb{}}
\end{Verbatim}

    The macro shows the easiest way of accessing the content of a n-tuple:
after loading the n-tuple, its branches are assigned to variables and
\href{https://root.cern.ch/doc/master/classTTree.html\#a9fc48df5560fce1a2d63ecd1ac5b40cb}{\texttt{GetEntry(long)}}
automatically fills them with the content for a specific row. By doing
so, the logic for reading the n-tuple and the code to process it can be
split and the source code remains clear.

\subsubsection{7.2.3 Storing Arbitrary
N-tuples}\label{storing-arbitrary-n-tuples}

It is also possible to write n-tuples of arbitrary type by using ROOT's
\texttt{TBranch} class. This is especially important as
\href{https://root.cern.ch/doc/v606/classTNtuple.html}{\texttt{TNtuple::Fill()}}
accepts only floats. The following macro creates the same n-tuple as
before but the branches are booked directly. The
\href{https://root.cern.ch/doc/master/classTH1.html}{\texttt{Fill()}}
function then fills the current values of the connected variables to the
tree.

    \begin{Verbatim}[commandchars=\\\{\}]
{\color{incolor}In [{\color{incolor}9}]:} \PY{o}{\PYZpc{}}\PY{o}{\PYZpc{}}\PY{n}{cpp} \PY{o}{\PYZhy{}}\PY{n}{d}
            
        \PY{c+c1}{// Fill an n\PYZhy{}tuple and write it to a file simulating measurement of}
        \PY{c+c1}{// conductivity of a material in different conditions of pressure}
        \PY{c+c1}{// and temperature using branches.}
        
        \PY{k+kt}{void} \PY{n}{write\PYZus{}ntuple\PYZus{}to\PYZus{}file\PYZus{}advanced}\PY{p}{(}
           \PY{k}{const} \PY{n}{std}\PY{o}{:}\PY{o}{:}\PY{n}{string}\PY{o}{\PYZam{}} \PY{n}{outputFileName}\PY{o}{=}\PY{l+s}{\PYZdq{}}\PY{l+s}{../data/conductivity\PYZus{}experiment.root}\PY{l+s}{\PYZdq{}}
           \PY{p}{,}\PY{k+kt}{unsigned} \PY{k+kt}{int} \PY{n}{numDataPoints}\PY{o}{=}\PY{l+m+mi}{1000000}\PY{p}{)}\PY{p}{\PYZob{}}
        
           \PY{n}{TFile} \PY{n}{ofile}\PY{p}{(}\PY{n}{outputFileName}\PY{p}{.}\PY{n}{c\PYZus{}str}\PY{p}{(}\PY{p}{)}\PY{p}{,}\PY{l+s}{\PYZdq{}}\PY{l+s}{RECREATE}\PY{l+s}{\PYZdq{}}\PY{p}{)}\PY{p}{;}
        
           \PY{c+c1}{// Initialise the TNtuple}
           \PY{n}{TTree} \PY{n+nf}{cond\PYZus{}data}\PY{p}{(}\PY{l+s}{\PYZdq{}}\PY{l+s}{cond\PYZus{}data}\PY{l+s}{\PYZdq{}}\PY{p}{,} \PY{l+s}{\PYZdq{}}\PY{l+s}{Example N\PYZhy{}Tuple}\PY{l+s}{\PYZdq{}}\PY{p}{)}\PY{p}{;}
        
           \PY{c+c1}{// define the variables and book them for the ntuple}
           \PY{k+kt}{float} \PY{n}{pot}\PY{p}{,}\PY{n}{cur}\PY{p}{,}\PY{n}{temp}\PY{p}{,}\PY{n}{pres}\PY{p}{;}
           \PY{n}{cond\PYZus{}data}\PY{p}{.}\PY{n}{Branch}\PY{p}{(}\PY{l+s}{\PYZdq{}}\PY{l+s}{Potential}\PY{l+s}{\PYZdq{}}\PY{p}{,} \PY{o}{\PYZam{}}\PY{n}{pot}\PY{p}{,} \PY{l+s}{\PYZdq{}}\PY{l+s}{Potential/F}\PY{l+s}{\PYZdq{}}\PY{p}{)}\PY{p}{;}
           \PY{n}{cond\PYZus{}data}\PY{p}{.}\PY{n}{Branch}\PY{p}{(}\PY{l+s}{\PYZdq{}}\PY{l+s}{Current}\PY{l+s}{\PYZdq{}}\PY{p}{,} \PY{o}{\PYZam{}}\PY{n}{cur}\PY{p}{,} \PY{l+s}{\PYZdq{}}\PY{l+s}{Current/F}\PY{l+s}{\PYZdq{}}\PY{p}{)}\PY{p}{;}
           \PY{n}{cond\PYZus{}data}\PY{p}{.}\PY{n}{Branch}\PY{p}{(}\PY{l+s}{\PYZdq{}}\PY{l+s}{Temperature}\PY{l+s}{\PYZdq{}}\PY{p}{,} \PY{o}{\PYZam{}}\PY{n}{temp}\PY{p}{,} \PY{l+s}{\PYZdq{}}\PY{l+s}{Temperature/F}\PY{l+s}{\PYZdq{}}\PY{p}{)}\PY{p}{;}
           \PY{n}{cond\PYZus{}data}\PY{p}{.}\PY{n}{Branch}\PY{p}{(}\PY{l+s}{\PYZdq{}}\PY{l+s}{Pressure}\PY{l+s}{\PYZdq{}}\PY{p}{,} \PY{o}{\PYZam{}}\PY{n}{pres}\PY{p}{,} \PY{l+s}{\PYZdq{}}\PY{l+s}{Pressure/F}\PY{l+s}{\PYZdq{}}\PY{p}{)}\PY{p}{;}
        
           \PY{k}{for} \PY{p}{(}\PY{k+kt}{int} \PY{n}{i}\PY{o}{=}\PY{l+m+mi}{0}\PY{p}{;}\PY{n}{i}\PY{o}{\PYZlt{}}\PY{n}{numDataPoints}\PY{p}{;}\PY{o}{+}\PY{o}{+}\PY{n}{i}\PY{p}{)}\PY{p}{\PYZob{}}
              \PY{c+c1}{// Fill it randomly to fake the acquired data}
              \PY{n}{pot}\PY{o}{=}\PY{n}{gRandom}\PY{o}{\PYZhy{}}\PY{o}{\PYZgt{}}\PY{n}{Uniform}\PY{p}{(}\PY{l+m+mf}{0.}\PY{p}{,}\PY{l+m+mf}{10.}\PY{p}{)}\PY{o}{*}\PY{n}{gRandom}\PY{o}{\PYZhy{}}\PY{o}{\PYZgt{}}\PY{n}{Gaus}\PY{p}{(}\PY{l+m+mf}{1.}\PY{p}{,}\PY{l+m+mf}{0.01}\PY{p}{)}\PY{p}{;}
              \PY{n}{temp}\PY{o}{=}\PY{n}{gRandom}\PY{o}{\PYZhy{}}\PY{o}{\PYZgt{}}\PY{n}{Uniform}\PY{p}{(}\PY{l+m+mf}{250.}\PY{p}{,}\PY{l+m+mf}{350.}\PY{p}{)}\PY{o}{+}\PY{n}{gRandom}\PY{o}{\PYZhy{}}\PY{o}{\PYZgt{}}\PY{n}{Gaus}\PY{p}{(}\PY{l+m+mf}{0.}\PY{p}{,}\PY{l+m+mf}{0.3}\PY{p}{)}\PY{p}{;}
              \PY{n}{pres}\PY{o}{=}\PY{n}{gRandom}\PY{o}{\PYZhy{}}\PY{o}{\PYZgt{}}\PY{n}{Uniform}\PY{p}{(}\PY{l+m+mf}{0.5}\PY{p}{,}\PY{l+m+mf}{1.5}\PY{p}{)}\PY{o}{*}\PY{n}{gRandom}\PY{o}{\PYZhy{}}\PY{o}{\PYZgt{}}\PY{n}{Gaus}\PY{p}{(}\PY{l+m+mf}{1.}\PY{p}{,}\PY{l+m+mf}{0.02}\PY{p}{)}\PY{p}{;}
              \PY{n}{cur}\PY{o}{=}\PY{n}{pot}\PY{o}{/}\PY{p}{(}\PY{l+m+mf}{10.}\PY{o}{+}\PY{l+m+mf}{0.05}\PY{o}{*}\PY{p}{(}\PY{n}{temp}\PY{o}{\PYZhy{}}\PY{l+m+mf}{300.}\PY{p}{)}\PY{o}{\PYZhy{}}\PY{l+m+mf}{0.2}\PY{o}{*}\PY{p}{(}\PY{n}{pres}\PY{o}{\PYZhy{}}\PY{l+m+mf}{1.}\PY{p}{)}\PY{p}{)}\PY{o}{*}
                            \PY{n}{gRandom}\PY{o}{\PYZhy{}}\PY{o}{\PYZgt{}}\PY{n}{Gaus}\PY{p}{(}\PY{l+m+mf}{1.}\PY{p}{,}\PY{l+m+mf}{0.01}\PY{p}{)}\PY{p}{;}
              \PY{c+c1}{// write to ntuple}
              \PY{n}{cond\PYZus{}data}\PY{p}{.}\PY{n}{Fill}\PY{p}{(}\PY{p}{)}\PY{p}{;}\PY{p}{\PYZcb{}}
        
           \PY{c+c1}{// Save the ntuple and close the file}
           \PY{n}{cond\PYZus{}data}\PY{p}{.}\PY{n}{Write}\PY{p}{(}\PY{p}{)}\PY{p}{;}
           \PY{n}{ofile}\PY{p}{.}\PY{n}{Close}\PY{p}{(}\PY{p}{)}\PY{p}{;}
        \PY{p}{\PYZcb{}}
\end{Verbatim}

    The
\href{https://root.cern.ch/doc/master/classTBranch.html}{\texttt{Branch()}}
function requires a pointer to a variable and a definition of the
variable type. The following table lists some of the possible values.
Please note that ROOT is not checking the input and mistakes are likely
to result in serious problems. This holds especially if values are read
as another type than they have been written, e.g.~when storing a
variable as float and reading it as double.

List of variable types that can be used to define the type of a branch
in ROOT:

\begin{longtable}[]{@{}llll@{}}
\toprule
type & size & C++ & identifier\tabularnewline
\midrule
\endhead
signed integer & 32 bit & int & I\tabularnewline
& 64 bit & long & L\tabularnewline
unsigned integer & 32 bit & unsigned int & i\tabularnewline
& 64 bit & unsigned long & l\tabularnewline
floating point & 32 bit & float & F\tabularnewline
& 64 bit & double & D\tabularnewline
boolean & - & bool & O\tabularnewline
\bottomrule
\end{longtable}

\subsubsection{7.2.4 Processing N-tuples Spanning over Several
Files}\label{processing-n-tuples-spanning-over-several-files}

Usually n-tuples or trees span over many files and it would be difficult
to add them manually. ROOT thus kindly provides a helper class in the
form of
\href{https://root.cern.ch/doc/master/classTChain.html}{\texttt{TChain}}.
Its usage is shown in the following macro which is very similar to the
previous example. The constructor of a
\href{https://root.cern.ch/doc/master/classTChain.html}{\texttt{TChain}}
takes the name of the
\href{https://root.cern.ch/doc/v606/classTTree.html}{\texttt{TTree}} (or
\href{(https://root.cern.ch/doc/v606/classTNtuple.html)}{\texttt{TNuple}})
as an argument. The files are added with the function
\href{https://root.cern.ch/doc/master/classTH1.html}{\texttt{Add(fileName)}},
where one can also use wild-cards as shown in the example.

    \begin{Verbatim}[commandchars=\\\{\}]
{\color{incolor}In [{\color{incolor}10}]:} \PY{o}{\PYZpc{}}\PY{o}{\PYZpc{}}\PY{n}{cpp} \PY{o}{\PYZhy{}}\PY{n}{d}
         \PY{c+c1}{// Read several previously produced N\PYZhy{}Tuples and print on screen its}
         \PY{c+c1}{// content.}
         \PY{c+c1}{//}
         \PY{c+c1}{// you can easily create some files with the following statement:}
         \PY{c+c1}{//}
         \PY{c+c1}{// for i in 0 1 2 3 4 5; \PYZbs{}\PYZbs{}}
         \PY{c+c1}{// do root \PYZhy{}l \PYZhy{}x \PYZhy{}b \PYZhy{}q \PYZbs{}\PYZbs{}}
         \PY{c+c1}{// \PYZdq{}write\PYZus{}ntuple\PYZus{}to\PYZus{}file.cxx \PYZbs{}\PYZbs{}}
         \PY{c+c1}{// (\PYZbs{}\PYZdq{}conductivity\PYZus{}experiment\PYZus{}\PYZdl{}\PYZob{}i\PYZcb{}.root\PYZbs{}\PYZdq{}, 100)\PYZdq{}; \PYZbs{}\PYZbs{}}
         \PY{c+c1}{//  done}
         
         \PY{k+kt}{void} \PY{n}{read\PYZus{}ntuple\PYZus{}with\PYZus{}chain}\PY{p}{(}\PY{p}{)}\PY{p}{\PYZob{}}
            \PY{c+c1}{// initiate a TChain with the name of the TTree to be processed}
            \PY{n}{TChain} \PY{n}{in\PYZus{}chain}\PY{p}{(}\PY{l+s}{\PYZdq{}}\PY{l+s}{cond\PYZus{}data}\PY{l+s}{\PYZdq{}}\PY{p}{)}\PY{p}{;}
            \PY{n}{in\PYZus{}chain}\PY{p}{.}\PY{n}{Add}\PY{p}{(}\PY{l+s}{\PYZdq{}}\PY{l+s}{../data/conductivity\PYZus{}experiment*.root}\PY{l+s}{\PYZdq{}}\PY{p}{)}\PY{p}{;} \PY{c+c1}{// add files,}
                                                           \PY{c+c1}{// wildcards work}
         
            \PY{c+c1}{// define variables and assign them to the corresponding branches}
            \PY{k+kt}{float} \PY{n}{pot}\PY{p}{,} \PY{n}{cur}\PY{p}{,} \PY{n}{temp}\PY{p}{,} \PY{n}{pres}\PY{p}{;}
            \PY{n}{in\PYZus{}chain}\PY{p}{.}\PY{n}{SetBranchAddress}\PY{p}{(}\PY{l+s}{\PYZdq{}}\PY{l+s}{Potential}\PY{l+s}{\PYZdq{}}\PY{p}{,} \PY{o}{\PYZam{}}\PY{n}{pot}\PY{p}{)}\PY{p}{;}
            \PY{n}{in\PYZus{}chain}\PY{p}{.}\PY{n}{SetBranchAddress}\PY{p}{(}\PY{l+s}{\PYZdq{}}\PY{l+s}{Current}\PY{l+s}{\PYZdq{}}\PY{p}{,} \PY{o}{\PYZam{}}\PY{n}{cur}\PY{p}{)}\PY{p}{;}
            \PY{n}{in\PYZus{}chain}\PY{p}{.}\PY{n}{SetBranchAddress}\PY{p}{(}\PY{l+s}{\PYZdq{}}\PY{l+s}{Temperature}\PY{l+s}{\PYZdq{}}\PY{p}{,} \PY{o}{\PYZam{}}\PY{n}{temp}\PY{p}{)}\PY{p}{;}
            \PY{n}{in\PYZus{}chain}\PY{p}{.}\PY{n}{SetBranchAddress}\PY{p}{(}\PY{l+s}{\PYZdq{}}\PY{l+s}{Pressure}\PY{l+s}{\PYZdq{}}\PY{p}{,} \PY{o}{\PYZam{}}\PY{n}{pres}\PY{p}{)}\PY{p}{;}
         
            \PY{n}{cout} \PY{o}{\PYZlt{}}\PY{o}{\PYZlt{}} \PY{l+s}{\PYZdq{}}\PY{l+s}{Potential}\PY{l+s+se}{\PYZbs{}t}\PY{l+s}{Current}\PY{l+s+se}{\PYZbs{}t}\PY{l+s}{Temperature}\PY{l+s+se}{\PYZbs{}t}\PY{l+s}{Pressure}\PY{l+s+se}{\PYZbs{}n}\PY{l+s}{\PYZdq{}}\PY{p}{;}
            \PY{k}{for} \PY{p}{(}\PY{k+kt}{size\PYZus{}t} \PY{n}{irow}\PY{o}{=}\PY{l+m+mi}{0}\PY{p}{;} \PY{n}{irow}\PY{o}{\PYZlt{}}\PY{n}{in\PYZus{}chain}\PY{p}{.}\PY{n}{GetEntries}\PY{p}{(}\PY{p}{)}\PY{p}{;} \PY{o}{+}\PY{o}{+}\PY{n}{irow}\PY{p}{)}\PY{p}{\PYZob{}}
               \PY{n}{in\PYZus{}chain}\PY{p}{.}\PY{n}{GetEntry}\PY{p}{(}\PY{n}{irow}\PY{p}{)}\PY{p}{;} \PY{c+c1}{// loads all variables that have}
                                             \PY{c+c1}{// been connected to branches}
               \PY{n}{cout} \PY{o}{\PYZlt{}}\PY{o}{\PYZlt{}} \PY{n}{pot} \PY{o}{\PYZlt{}}\PY{o}{\PYZlt{}} \PY{l+s}{\PYZdq{}}\PY{l+s+se}{\PYZbs{}t}\PY{l+s}{\PYZdq{}} \PY{o}{\PYZlt{}}\PY{o}{\PYZlt{}} \PY{n}{cur} \PY{o}{\PYZlt{}}\PY{o}{\PYZlt{}} \PY{l+s}{\PYZdq{}}\PY{l+s+se}{\PYZbs{}t}\PY{l+s}{\PYZdq{}} \PY{o}{\PYZlt{}}\PY{o}{\PYZlt{}} \PY{n}{temp} \PY{o}{\PYZlt{}}\PY{o}{\PYZlt{}}
                                   \PY{l+s}{\PYZdq{}}\PY{l+s+se}{\PYZbs{}t}\PY{l+s}{\PYZdq{}} \PY{o}{\PYZlt{}}\PY{o}{\PYZlt{}} \PY{n}{pres} \PY{o}{\PYZlt{}}\PY{o}{\PYZlt{}} \PY{n}{endl}\PY{p}{;}
            \PY{p}{\PYZcb{}}
         \PY{p}{\PYZcb{}}
\end{Verbatim}

    \subsubsection{7.2.5 For the advanced user: Processing trees with a
selector
script}\label{for-the-advanced-user-processing-trees-with-a-selector-script}

Another very general and powerful way of processing a \texttt{TChain} is
provided via the method \texttt{TChain::Process()}. This method takes as
arguments an instance of a -- user-implemented-- class of type
\texttt{TSelector}, and -- optionally -- the number of entries and the
first entry to be processed. A template for the class \texttt{TSelector}
is provided by the method \texttt{TTree::MakeSelector}, as is shown in
the little macro makeSelector.C below.

It opens the n-tuple \texttt{conductivity\_experiment.root} from the
example above and creates from it the header file \texttt{MySelector.h}
and a template to insert your own analysis code, \texttt{MySelector.C}.

\begin{verbatim}
{
// create template class for Selector to run on a tree
//////////////////////////////////////////////////////
//
// open root file containing the Tree
    TFile f("conductivity_experiment.root");
// create TTree object from it
    TTree *t; f.GetObject("cond_data",t);
// this generates the files MySelector.h and MySelector.C
    t->MakeSelector("MySelector");
}
\end{verbatim}

The template contains the entry points \texttt{Begin()} and
\texttt{SlaveBegin()} called before processing of the TChain starts,
Process() called for every entry of the chain, and
\texttt{SlaveTerminate()} and \texttt{Terminate()} called after the last
entry has been processed. Typically, initialization like booking of
histograms is performed in \texttt{SlaveBegin()}, the analysis, i.e.~the
selection of entries, calculations and filling of histograms, is done in
\texttt{Process()}, and final operations like plotting and storing of
results happen in \texttt{SlaveTerminate()} or \texttt{Terminate()}.

The entry points \texttt{SlaveBegin()} and \texttt{SlaveTerminate()} are
called on so-called slave nodes only if parallel processing via PROOF or
PROOF lite is enabled, as will be explained below.

A simple example of a selector class is shown in the macro MySelector.C.
The example is executed with the following sequence of commands:

\begin{verbatim}
> TChain *ch=new TChain("cond_data", "Chain for Example N-Tuple");
> ch->Add("conductivity_experiment*.root");
> ch->Process("MySelector.C+");
\end{verbatim}

As usual, the ``+'' appended to the name of the macro to be executed
initiates the compilation of the \texttt{MySelector.C} with the system
compiler in order to improve performance.

The code in \texttt{MySelector.C}, shown in the listing below, books
some histograms in \texttt{SlaveBegin()} and adds them to the instance
fOutput, which is of the class TList.4 The final processing in
\texttt{Terminate()} allows to access histograms and store, display or
save them as pictures. This is shown in the example via the TList
fOutput. See the commented listing below for more details; most of the
text is actually comments generated automatically by
\texttt{TTree::MakeSelector}.

\begin{verbatim}
#define MySelector_cxx
// The class definition in MySelector.h has been generated automatically
// by the ROOT utility TTree::MakeSelector(). This class is derived
// from the ROOT class TSelector. For more information on the TSelector
// framework see $ROOTSYS/README/README.SELECTOR or the ROOT User Manual.

// The following methods are defined in this file:
//    Begin():        called every time a loop on the tree starts,
//                    a convenient place to create your histograms.
//    SlaveBegin():   called after Begin(), when on PROOF called only on the
//                    slave servers.
//    Process():      called for each event, in this function you decide what
//                    to read and fill your histograms.
//    SlaveTerminate: called at the end of the loop on the tree, when on PROOF
//                    called only on the slave servers.
//    Terminate():    called at the end of the loop on the tree,
//                    a convenient place to draw/fit your histograms.
//
// To use this file, try the following session on your Tree T:
//
// root> T->Process("MySelector.C")
// root> T->Process("MySelector.C","some options")
// root> T->Process("MySelector.C+")
//

#include "MySelector.h"
#include <TH2.h>
#include <TStyle.h>


void MySelector::Begin(TTree * /*tree*/)
{
   // The Begin() function is called at the start of the query.
   // When running with PROOF Begin() is only called on the client.
   // The tree argument is deprecated (on PROOF 0 is passed).

   TString option = GetOption();

}

void MySelector::SlaveBegin(TTree * /*tree*/)
{
   // The SlaveBegin() function is called after the Begin() function.
   // When running with PROOF SlaveBegin() is called on each slave server.
   // The tree argument is deprecated (on PROOF 0 is passed).

   TString option = GetOption();

}

Bool_t MySelector::Process(Long64_t entry)
{
   // The Process() function is called for each entry in the tree (or possibly
   // keyed object in the case of PROOF) to be processed. The entry argument
   // specifies which entry in the currently loaded tree is to be processed.
   // It can be passed to either MySelector::GetEntry() or TBranch::GetEntry()
   // to read either all or the required parts of the data. When processing
   // keyed objects with PROOF, the object is already loaded and is available
   // via the fObject pointer.
   //
   // This function should contain the "body" of the analysis. It can contain
   // simple or elaborate selection criteria, run algorithms on the data
   // of the event and typically fill histograms.
   //
   // The processing can be stopped by calling Abort().
   //
   // Use fStatus to set the return value of TTree::Process().
   //
   // The return value is currently not used.

   return kTRUE;
}

void MySelector::SlaveTerminate()
{
   // The SlaveTerminate() function is called after all entries or objects
   // have been processed. When running with PROOF SlaveTerminate() is called
   // on each slave server.

}

void MySelector::Terminate()
{
   // The Terminate() function is the last function to be called during
   // a query. It always runs on the client, it can be used to present
   // the results graphically or save the results to file.

}
\end{verbatim}

\subsubsection{7.2.6 For power-users: Multi-core processing with PROOF
lite}\label{for-power-users-multi-core-processing-with-proof-lite}

The processing of n-tuples via a selector function of type
\texttt{TSelector} through \texttt{TChain::Process()}, as described at
the end of the previous section, offers an additional advantage in
particular for very large data sets: on distributed systems or
multi-core architectures, portions of data can be processed in parallel,
thus significantly reducing the execution time. On modern computers with
multi-core CPUs or hardware-threading enabled, this allows a much faster
turnaround of analyses, since all the available CPU power is used.

On distributed systems, a PROOF server and worker nodes have to be set
up, as described in detail in the ROOT documentation. On a single
computer with multiple cores, \texttt{PROOF} lite can be used instead.
Try the following little macro, RunMySelector.C, which contains two
extra lines compared to the example above (adjust the number of workers
according to the number of CPU cores):

\begin{verbatim}
{// set up a TChain
TChain *ch=new TChain("cond_data", "My Chain for Example N-Tuple");
 ch->Add("conductivity_experiment*.root");
// eventually, start Proof Lite on cores
TProof::Open("workers=4");
ch->SetProof();
ch->Process("MySelector.C+");}
\end{verbatim}

The first command, \texttt{TProof::Open(const\ char*)} starts a local
PROOF server (if no arguments are specified, all cores will be used),
and the command \texttt{ch-\textgreater{}SetProof();} enables processing
of the chain using PROOF. Now, when issuing the command
\texttt{ch-\textgreater{}Process("MySelector.C+);}, the code in
\texttt{MySelector.C} is compiled and executed on each slave node. The
methods \texttt{Begin()} and \texttt{Terminate()} are executed on the
master only. The list of n-tuple files is analysed, and portions of the
data are assigned to the available slave processes. Histograms booked in
SlaveBegin() exist in the processes on the slave nodes, and are filled
accordingly. Upon termination, the PROOF master collects the histograms
from the slaves and merges them. In \texttt{Terminate()} all merged
histograms are available and can be inspected, analysed or stored. The
histograms are handled via the instances fOutput of class TList in each
slave process, and can be retrieved from this list after merging in
Terminate.

To explore the power of this mechanism, generate some very large
n-tuples using the script from the section Storing Arbitrary N-tuples -
you could try 10 000 000 events (this results in a large n-tuple of
about 160 MByte in size). You could also generate a large number of
files and use wildcards to add the to the TChain. Now execute:
\texttt{\textgreater{}\ root\ -l\ RunMySelector.C} and watch what
happens:

\begin{verbatim}
Processing RunMySelector.C...
 +++ Starting PROOF-Lite with 4 workers +++
Opening connections to workers: OK (4 workers)
Setting up worker servers: OK (4 workers)
PROOF set to parallel mode (4 workers)

Info in <TProofLite::SetQueryRunning>: starting query: 1
Info in <TProofQueryResult::SetRunning>: nwrks: 4
Info in <TUnixSystem::ACLiC>: creating shared library
                             ~/DivingROOT/macros/MySelector_C.so
*==* ----- Begin of Job ----- Date/Time = Wed Feb 15 23:00:04 2012
Looking up for exact location of files: OK (4 files)
Looking up for exact location of files: OK (4 files)
Info in <TPacketizerAdaptive::TPacketizerAdaptive>:
                      Setting max number of workers per node to 4
Validating files: OK (4 files)
Info in <TPacketizerAdaptive::InitStats>:
                      fraction of remote files 1.000000
Info in <TCanvas::Print>:
       file ResistanceDistribution.png has been created
*==* ----- End of Job ----- Date/Time = Wed Feb 15 23:00:08 2012
Lite-0: all output objects have been merged
\end{verbatim}

Log files of the whole processing chain are kept in the directory
\texttt{\textasciitilde{}.proof} for each worker node. This is very
helpful for debugging or if something goes wrong. As the method
described here also works without using PROOF, the development work on
an analysis script can be done in the standard way on a small subset of
the data, and only for the full processing one would use parallelism via
PROOF.

It is worth to remind the reader that the speed of typical data analysis
programs limited by the I/O speed (for example the latencies implied by
reading data from a hard drive). It is therefore expected that this
limitation cannot be eliminated with the usage of any parallel analysis
toolkit.

\subsection{7.2.7 Optimisation Regarding
N-tuples}\label{optimisation-regarding-n-tuples}

ROOT automatically applies compression algorithms on n-tuples to reduce
the memory consumption. A value that is in most cases the same will
consume only small space on your disk (but it has to be decompressed on
reading). Nevertheless, you should think about the design of your
n-tuples and your analyses as soon as the processing time exceeds some
minutes.

\begin{itemize}
\item
  Try to keep your n-tuples simple and use appropriate variable types.
  If your measurement has only a limited precision, it is needless to
  store it with double precision.
\item
  Experimental conditions that do not change with every single
  measurement should be stored in a separate tree. Although the
  compression can handle redundant values, the processing time increase
  with every variable that has to be filled.
\item
  The function SetCacheSize(long) specifies the size of the cache for
  reading a TTree object from a file. The default value is 30MB. A
  manual increase may help in certain situations. Please note that the
  caching mechanism can cover only one TTree object per TFile object.
\item
  You can select the branches to be covered by the caching algorithm
  with AddBranchToCache and deactivate unneeded branches with
  SetBranchStatus. This mechanism can result in a significant speed-up
  for simple operations on trees with many branches.
\item
  You can measure the performance easily with TTreePerfStats. The ROOT
  documentation on this class also includes an introductory example. For
  example, TTreePerfStats can show you that it is beneficial to store
  meta data and payload data separately, i.e.~write the meta data tree
  in a bulk to a file at the end of your job instead of writing both
  trees interleaved.
\end{itemize}

    \begin{Verbatim}[commandchars=\\\{\}]
{\color{incolor}In [{\color{incolor} }]:} 
\end{Verbatim}


    % Add a bibliography block to the postdoc
    
    
    
    \end{document}
