
% Default to the notebook output style

    


% Inherit from the specified cell style.




    
\documentclass{article}

    
    
    \usepackage{graphicx} % Used to insert images
    \usepackage{adjustbox} % Used to constrain images to a maximum size 
    \usepackage{color} % Allow colors to be defined
    \usepackage{enumerate} % Needed for markdown enumerations to work
    \usepackage{geometry} % Used to adjust the document margins
    \usepackage{amsmath} % Equations
    \usepackage{amssymb} % Equations
    \usepackage{eurosym} % defines \euro
    \usepackage[mathletters]{ucs} % Extended unicode (utf-8) support
    \usepackage[utf8x]{inputenc} % Allow utf-8 characters in the tex document
    \usepackage{fancyvrb} % verbatim replacement that allows latex
    \usepackage{grffile} % extends the file name processing of package graphics 
                         % to support a larger range 
    % The hyperref package gives us a pdf with properly built
    % internal navigation ('pdf bookmarks' for the table of contents,
    % internal cross-reference links, web links for URLs, etc.)
    \usepackage{hyperref}
    \usepackage{longtable} % longtable support required by pandoc >1.10
    \usepackage{booktabs}  % table support for pandoc > 1.12.2
    \usepackage{ulem} % ulem is needed to support strikethroughs (\sout)
    

    
    
    \definecolor{orange}{cmyk}{0,0.4,0.8,0.2}
    \definecolor{darkorange}{rgb}{.71,0.21,0.01}
    \definecolor{darkgreen}{rgb}{.12,.54,.11}
    \definecolor{myteal}{rgb}{.26, .44, .56}
    \definecolor{gray}{gray}{0.45}
    \definecolor{lightgray}{gray}{.95}
    \definecolor{mediumgray}{gray}{.8}
    \definecolor{inputbackground}{rgb}{.95, .95, .85}
    \definecolor{outputbackground}{rgb}{.95, .95, .95}
    \definecolor{traceback}{rgb}{1, .95, .95}
    % ansi colors
    \definecolor{red}{rgb}{.6,0,0}
    \definecolor{green}{rgb}{0,.65,0}
    \definecolor{brown}{rgb}{0.6,0.6,0}
    \definecolor{blue}{rgb}{0,.145,.698}
    \definecolor{purple}{rgb}{.698,.145,.698}
    \definecolor{cyan}{rgb}{0,.698,.698}
    \definecolor{lightgray}{gray}{0.5}
    
    % bright ansi colors
    \definecolor{darkgray}{gray}{0.25}
    \definecolor{lightred}{rgb}{1.0,0.39,0.28}
    \definecolor{lightgreen}{rgb}{0.48,0.99,0.0}
    \definecolor{lightblue}{rgb}{0.53,0.81,0.92}
    \definecolor{lightpurple}{rgb}{0.87,0.63,0.87}
    \definecolor{lightcyan}{rgb}{0.5,1.0,0.83}
    
    % commands and environments needed by pandoc snippets
    % extracted from the output of `pandoc -s`
    \providecommand{\tightlist}{%
      \setlength{\itemsep}{0pt}\setlength{\parskip}{0pt}}
    \DefineVerbatimEnvironment{Highlighting}{Verbatim}{commandchars=\\\{\}}
    % Add ',fontsize=\small' for more characters per line
    \newenvironment{Shaded}{}{}
    \newcommand{\KeywordTok}[1]{\textcolor[rgb]{0.00,0.44,0.13}{\textbf{{#1}}}}
    \newcommand{\DataTypeTok}[1]{\textcolor[rgb]{0.56,0.13,0.00}{{#1}}}
    \newcommand{\DecValTok}[1]{\textcolor[rgb]{0.25,0.63,0.44}{{#1}}}
    \newcommand{\BaseNTok}[1]{\textcolor[rgb]{0.25,0.63,0.44}{{#1}}}
    \newcommand{\FloatTok}[1]{\textcolor[rgb]{0.25,0.63,0.44}{{#1}}}
    \newcommand{\CharTok}[1]{\textcolor[rgb]{0.25,0.44,0.63}{{#1}}}
    \newcommand{\StringTok}[1]{\textcolor[rgb]{0.25,0.44,0.63}{{#1}}}
    \newcommand{\CommentTok}[1]{\textcolor[rgb]{0.38,0.63,0.69}{\textit{{#1}}}}
    \newcommand{\OtherTok}[1]{\textcolor[rgb]{0.00,0.44,0.13}{{#1}}}
    \newcommand{\AlertTok}[1]{\textcolor[rgb]{1.00,0.00,0.00}{\textbf{{#1}}}}
    \newcommand{\FunctionTok}[1]{\textcolor[rgb]{0.02,0.16,0.49}{{#1}}}
    \newcommand{\RegionMarkerTok}[1]{{#1}}
    \newcommand{\ErrorTok}[1]{\textcolor[rgb]{1.00,0.00,0.00}{\textbf{{#1}}}}
    \newcommand{\NormalTok}[1]{{#1}}
    
    % Additional commands for more recent versions of Pandoc
    \newcommand{\ConstantTok}[1]{\textcolor[rgb]{0.53,0.00,0.00}{{#1}}}
    \newcommand{\SpecialCharTok}[1]{\textcolor[rgb]{0.25,0.44,0.63}{{#1}}}
    \newcommand{\VerbatimStringTok}[1]{\textcolor[rgb]{0.25,0.44,0.63}{{#1}}}
    \newcommand{\SpecialStringTok}[1]{\textcolor[rgb]{0.73,0.40,0.53}{{#1}}}
    \newcommand{\ImportTok}[1]{{#1}}
    \newcommand{\DocumentationTok}[1]{\textcolor[rgb]{0.73,0.13,0.13}{\textit{{#1}}}}
    \newcommand{\AnnotationTok}[1]{\textcolor[rgb]{0.38,0.63,0.69}{\textbf{\textit{{#1}}}}}
    \newcommand{\CommentVarTok}[1]{\textcolor[rgb]{0.38,0.63,0.69}{\textbf{\textit{{#1}}}}}
    \newcommand{\VariableTok}[1]{\textcolor[rgb]{0.10,0.09,0.49}{{#1}}}
    \newcommand{\ControlFlowTok}[1]{\textcolor[rgb]{0.00,0.44,0.13}{\textbf{{#1}}}}
    \newcommand{\OperatorTok}[1]{\textcolor[rgb]{0.40,0.40,0.40}{{#1}}}
    \newcommand{\BuiltInTok}[1]{{#1}}
    \newcommand{\ExtensionTok}[1]{{#1}}
    \newcommand{\PreprocessorTok}[1]{\textcolor[rgb]{0.74,0.48,0.00}{{#1}}}
    \newcommand{\AttributeTok}[1]{\textcolor[rgb]{0.49,0.56,0.16}{{#1}}}
    \newcommand{\InformationTok}[1]{\textcolor[rgb]{0.38,0.63,0.69}{\textbf{\textit{{#1}}}}}
    \newcommand{\WarningTok}[1]{\textcolor[rgb]{0.38,0.63,0.69}{\textbf{\textit{{#1}}}}}
    
    
    % Define a nice break command that doesn't care if a line doesn't already
    % exist.
    \def\br{\hspace*{\fill} \\* }
    % Math Jax compatability definitions
    \def\gt{>}
    \def\lt{<}
    % Document parameters
    \title{8-ROOT-in-Python}
    
    
    

    % Pygments definitions
    
\makeatletter
\def\PY@reset{\let\PY@it=\relax \let\PY@bf=\relax%
    \let\PY@ul=\relax \let\PY@tc=\relax%
    \let\PY@bc=\relax \let\PY@ff=\relax}
\def\PY@tok#1{\csname PY@tok@#1\endcsname}
\def\PY@toks#1+{\ifx\relax#1\empty\else%
    \PY@tok{#1}\expandafter\PY@toks\fi}
\def\PY@do#1{\PY@bc{\PY@tc{\PY@ul{%
    \PY@it{\PY@bf{\PY@ff{#1}}}}}}}
\def\PY#1#2{\PY@reset\PY@toks#1+\relax+\PY@do{#2}}

\expandafter\def\csname PY@tok@nd\endcsname{\def\PY@tc##1{\textcolor[rgb]{0.67,0.13,1.00}{##1}}}
\expandafter\def\csname PY@tok@mb\endcsname{\def\PY@tc##1{\textcolor[rgb]{0.40,0.40,0.40}{##1}}}
\expandafter\def\csname PY@tok@gs\endcsname{\let\PY@bf=\textbf}
\expandafter\def\csname PY@tok@nb\endcsname{\def\PY@tc##1{\textcolor[rgb]{0.00,0.50,0.00}{##1}}}
\expandafter\def\csname PY@tok@mf\endcsname{\def\PY@tc##1{\textcolor[rgb]{0.40,0.40,0.40}{##1}}}
\expandafter\def\csname PY@tok@bp\endcsname{\def\PY@tc##1{\textcolor[rgb]{0.00,0.50,0.00}{##1}}}
\expandafter\def\csname PY@tok@gh\endcsname{\let\PY@bf=\textbf\def\PY@tc##1{\textcolor[rgb]{0.00,0.00,0.50}{##1}}}
\expandafter\def\csname PY@tok@si\endcsname{\let\PY@bf=\textbf\def\PY@tc##1{\textcolor[rgb]{0.73,0.40,0.53}{##1}}}
\expandafter\def\csname PY@tok@gt\endcsname{\def\PY@tc##1{\textcolor[rgb]{0.00,0.27,0.87}{##1}}}
\expandafter\def\csname PY@tok@s\endcsname{\def\PY@tc##1{\textcolor[rgb]{0.73,0.13,0.13}{##1}}}
\expandafter\def\csname PY@tok@gu\endcsname{\let\PY@bf=\textbf\def\PY@tc##1{\textcolor[rgb]{0.50,0.00,0.50}{##1}}}
\expandafter\def\csname PY@tok@ge\endcsname{\let\PY@it=\textit}
\expandafter\def\csname PY@tok@nt\endcsname{\let\PY@bf=\textbf\def\PY@tc##1{\textcolor[rgb]{0.00,0.50,0.00}{##1}}}
\expandafter\def\csname PY@tok@kr\endcsname{\let\PY@bf=\textbf\def\PY@tc##1{\textcolor[rgb]{0.00,0.50,0.00}{##1}}}
\expandafter\def\csname PY@tok@cpf\endcsname{\let\PY@it=\textit\def\PY@tc##1{\textcolor[rgb]{0.25,0.50,0.50}{##1}}}
\expandafter\def\csname PY@tok@vi\endcsname{\def\PY@tc##1{\textcolor[rgb]{0.10,0.09,0.49}{##1}}}
\expandafter\def\csname PY@tok@sx\endcsname{\def\PY@tc##1{\textcolor[rgb]{0.00,0.50,0.00}{##1}}}
\expandafter\def\csname PY@tok@nc\endcsname{\let\PY@bf=\textbf\def\PY@tc##1{\textcolor[rgb]{0.00,0.00,1.00}{##1}}}
\expandafter\def\csname PY@tok@s1\endcsname{\def\PY@tc##1{\textcolor[rgb]{0.73,0.13,0.13}{##1}}}
\expandafter\def\csname PY@tok@sc\endcsname{\def\PY@tc##1{\textcolor[rgb]{0.73,0.13,0.13}{##1}}}
\expandafter\def\csname PY@tok@sr\endcsname{\def\PY@tc##1{\textcolor[rgb]{0.73,0.40,0.53}{##1}}}
\expandafter\def\csname PY@tok@nn\endcsname{\let\PY@bf=\textbf\def\PY@tc##1{\textcolor[rgb]{0.00,0.00,1.00}{##1}}}
\expandafter\def\csname PY@tok@gp\endcsname{\let\PY@bf=\textbf\def\PY@tc##1{\textcolor[rgb]{0.00,0.00,0.50}{##1}}}
\expandafter\def\csname PY@tok@cm\endcsname{\let\PY@it=\textit\def\PY@tc##1{\textcolor[rgb]{0.25,0.50,0.50}{##1}}}
\expandafter\def\csname PY@tok@kn\endcsname{\let\PY@bf=\textbf\def\PY@tc##1{\textcolor[rgb]{0.00,0.50,0.00}{##1}}}
\expandafter\def\csname PY@tok@kc\endcsname{\let\PY@bf=\textbf\def\PY@tc##1{\textcolor[rgb]{0.00,0.50,0.00}{##1}}}
\expandafter\def\csname PY@tok@mo\endcsname{\def\PY@tc##1{\textcolor[rgb]{0.40,0.40,0.40}{##1}}}
\expandafter\def\csname PY@tok@cs\endcsname{\let\PY@it=\textit\def\PY@tc##1{\textcolor[rgb]{0.25,0.50,0.50}{##1}}}
\expandafter\def\csname PY@tok@na\endcsname{\def\PY@tc##1{\textcolor[rgb]{0.49,0.56,0.16}{##1}}}
\expandafter\def\csname PY@tok@vc\endcsname{\def\PY@tc##1{\textcolor[rgb]{0.10,0.09,0.49}{##1}}}
\expandafter\def\csname PY@tok@nl\endcsname{\def\PY@tc##1{\textcolor[rgb]{0.63,0.63,0.00}{##1}}}
\expandafter\def\csname PY@tok@ow\endcsname{\let\PY@bf=\textbf\def\PY@tc##1{\textcolor[rgb]{0.67,0.13,1.00}{##1}}}
\expandafter\def\csname PY@tok@sd\endcsname{\let\PY@it=\textit\def\PY@tc##1{\textcolor[rgb]{0.73,0.13,0.13}{##1}}}
\expandafter\def\csname PY@tok@gd\endcsname{\def\PY@tc##1{\textcolor[rgb]{0.63,0.00,0.00}{##1}}}
\expandafter\def\csname PY@tok@c1\endcsname{\let\PY@it=\textit\def\PY@tc##1{\textcolor[rgb]{0.25,0.50,0.50}{##1}}}
\expandafter\def\csname PY@tok@kp\endcsname{\def\PY@tc##1{\textcolor[rgb]{0.00,0.50,0.00}{##1}}}
\expandafter\def\csname PY@tok@il\endcsname{\def\PY@tc##1{\textcolor[rgb]{0.40,0.40,0.40}{##1}}}
\expandafter\def\csname PY@tok@ni\endcsname{\let\PY@bf=\textbf\def\PY@tc##1{\textcolor[rgb]{0.60,0.60,0.60}{##1}}}
\expandafter\def\csname PY@tok@ss\endcsname{\def\PY@tc##1{\textcolor[rgb]{0.10,0.09,0.49}{##1}}}
\expandafter\def\csname PY@tok@c\endcsname{\let\PY@it=\textit\def\PY@tc##1{\textcolor[rgb]{0.25,0.50,0.50}{##1}}}
\expandafter\def\csname PY@tok@cp\endcsname{\def\PY@tc##1{\textcolor[rgb]{0.74,0.48,0.00}{##1}}}
\expandafter\def\csname PY@tok@o\endcsname{\def\PY@tc##1{\textcolor[rgb]{0.40,0.40,0.40}{##1}}}
\expandafter\def\csname PY@tok@kd\endcsname{\let\PY@bf=\textbf\def\PY@tc##1{\textcolor[rgb]{0.00,0.50,0.00}{##1}}}
\expandafter\def\csname PY@tok@go\endcsname{\def\PY@tc##1{\textcolor[rgb]{0.53,0.53,0.53}{##1}}}
\expandafter\def\csname PY@tok@kt\endcsname{\def\PY@tc##1{\textcolor[rgb]{0.69,0.00,0.25}{##1}}}
\expandafter\def\csname PY@tok@mi\endcsname{\def\PY@tc##1{\textcolor[rgb]{0.40,0.40,0.40}{##1}}}
\expandafter\def\csname PY@tok@no\endcsname{\def\PY@tc##1{\textcolor[rgb]{0.53,0.00,0.00}{##1}}}
\expandafter\def\csname PY@tok@ch\endcsname{\let\PY@it=\textit\def\PY@tc##1{\textcolor[rgb]{0.25,0.50,0.50}{##1}}}
\expandafter\def\csname PY@tok@ne\endcsname{\let\PY@bf=\textbf\def\PY@tc##1{\textcolor[rgb]{0.82,0.25,0.23}{##1}}}
\expandafter\def\csname PY@tok@gi\endcsname{\def\PY@tc##1{\textcolor[rgb]{0.00,0.63,0.00}{##1}}}
\expandafter\def\csname PY@tok@w\endcsname{\def\PY@tc##1{\textcolor[rgb]{0.73,0.73,0.73}{##1}}}
\expandafter\def\csname PY@tok@se\endcsname{\let\PY@bf=\textbf\def\PY@tc##1{\textcolor[rgb]{0.73,0.40,0.13}{##1}}}
\expandafter\def\csname PY@tok@s2\endcsname{\def\PY@tc##1{\textcolor[rgb]{0.73,0.13,0.13}{##1}}}
\expandafter\def\csname PY@tok@nv\endcsname{\def\PY@tc##1{\textcolor[rgb]{0.10,0.09,0.49}{##1}}}
\expandafter\def\csname PY@tok@m\endcsname{\def\PY@tc##1{\textcolor[rgb]{0.40,0.40,0.40}{##1}}}
\expandafter\def\csname PY@tok@k\endcsname{\let\PY@bf=\textbf\def\PY@tc##1{\textcolor[rgb]{0.00,0.50,0.00}{##1}}}
\expandafter\def\csname PY@tok@mh\endcsname{\def\PY@tc##1{\textcolor[rgb]{0.40,0.40,0.40}{##1}}}
\expandafter\def\csname PY@tok@gr\endcsname{\def\PY@tc##1{\textcolor[rgb]{1.00,0.00,0.00}{##1}}}
\expandafter\def\csname PY@tok@sb\endcsname{\def\PY@tc##1{\textcolor[rgb]{0.73,0.13,0.13}{##1}}}
\expandafter\def\csname PY@tok@sh\endcsname{\def\PY@tc##1{\textcolor[rgb]{0.73,0.13,0.13}{##1}}}
\expandafter\def\csname PY@tok@vg\endcsname{\def\PY@tc##1{\textcolor[rgb]{0.10,0.09,0.49}{##1}}}
\expandafter\def\csname PY@tok@nf\endcsname{\def\PY@tc##1{\textcolor[rgb]{0.00,0.00,1.00}{##1}}}
\expandafter\def\csname PY@tok@err\endcsname{\def\PY@bc##1{\setlength{\fboxsep}{0pt}\fcolorbox[rgb]{1.00,0.00,0.00}{1,1,1}{\strut ##1}}}

\def\PYZbs{\char`\\}
\def\PYZus{\char`\_}
\def\PYZob{\char`\{}
\def\PYZcb{\char`\}}
\def\PYZca{\char`\^}
\def\PYZam{\char`\&}
\def\PYZlt{\char`\<}
\def\PYZgt{\char`\>}
\def\PYZsh{\char`\#}
\def\PYZpc{\char`\%}
\def\PYZdl{\char`\$}
\def\PYZhy{\char`\-}
\def\PYZsq{\char`\'}
\def\PYZdq{\char`\"}
\def\PYZti{\char`\~}
% for compatibility with earlier versions
\def\PYZat{@}
\def\PYZlb{[}
\def\PYZrb{]}
\makeatother


    % Exact colors from NB
    \definecolor{incolor}{rgb}{0.0, 0.0, 0.5}
    \definecolor{outcolor}{rgb}{0.545, 0.0, 0.0}



    
    % Prevent overflowing lines due to hard-to-break entities
    \sloppy 
    % Setup hyperref package
    \hypersetup{
      breaklinks=true,  % so long urls are correctly broken across lines
      colorlinks=true,
      urlcolor=blue,
      linkcolor=darkorange,
      citecolor=darkgreen,
      }
    % Slightly bigger margins than the latex defaults
    
    \geometry{verbose,tmargin=1in,bmargin=1in,lmargin=1in,rmargin=1in}
    
    

    \begin{document}
    
    
    \maketitle
    
    

    
    ROOT offers the possibility to interface to Python via a set of bindings
called PyROOT. Python is used in a wide variety of application areas and
one of the most used scripting languages today. With the help of PyROOT
it becomes possible to combine the power of a scripting language with
ROOT tools. Introductory material to Python is available from many
sources on the web, see e. g. http://docs.python.org.

\subsection{8.1 PyROOT}\label{pyroot}

The access to ROOT classes and their methods in PyROOT is almost
identical to C++ macros, except for the special language features of
Python, most importantly dynamic type declaration at the time of
assignment. Coming back to our first example, simply plotting a function
in ROOT, the following C++ code:

\begin{verbatim}
TF1 *f1 = new TF1("f2","[0]*sin([1]*x)/x",0.,10.);
f1->SetParameter(0,1);
f1->SetParameter(1,1);
f1->Draw();
\end{verbatim}

in Python becomes:

    \begin{Verbatim}[commandchars=\\\{\}]
{\color{incolor}In [{\color{incolor}1}]:} \PY{k+kn}{import} \PY{n+nn}{ROOT}
\end{Verbatim}

    
    
    \begin{Verbatim}[commandchars=\\\{\}]
Welcome to JupyROOT 6.07/05
    \end{Verbatim}

    \begin{Verbatim}[commandchars=\\\{\}]
{\color{incolor}In [{\color{incolor}2}]:} \PY{c+c1}{\PYZsh{}from ROOT import gStyle, TCanvas, TGraphErrors}
        \PY{n}{canvas\PYZus{}8\PYZus{}1}\PY{o}{=}\PY{n}{ROOT}\PY{o}{.}\PY{n}{TCanvas}\PY{p}{(}\PY{l+s+s2}{\PYZdq{}}\PY{l+s+s2}{canvas\PYZus{}8\PYZus{}1}\PY{l+s+s2}{\PYZdq{}} \PY{p}{,}\PY{l+s+s2}{\PYZdq{}}\PY{l+s+s2}{Data}\PY{l+s+s2}{\PYZdq{}} \PY{p}{,}\PY{l+m+mi}{200} \PY{p}{,}\PY{l+m+mi}{10} \PY{p}{,}\PY{l+m+mi}{700} \PY{p}{,}\PY{l+m+mi}{500}\PY{p}{)}
        \PY{n}{f1} \PY{o}{=} \PY{n}{ROOT}\PY{o}{.}\PY{n}{TF1}\PY{p}{(}\PY{l+s+s2}{\PYZdq{}}\PY{l+s+s2}{f2}\PY{l+s+s2}{\PYZdq{}}\PY{p}{,}\PY{l+s+s2}{\PYZdq{}}\PY{l+s+s2}{[0]*sin([1]*x)/x}\PY{l+s+s2}{\PYZdq{}}\PY{p}{,}\PY{l+m+mf}{0.}\PY{p}{,}\PY{l+m+mf}{10.}\PY{p}{)}
        \PY{n}{f1}\PY{o}{.}\PY{n}{SetParameter}\PY{p}{(}\PY{l+m+mi}{0}\PY{p}{,}\PY{l+m+mi}{1}\PY{p}{)}\PY{p}{;}
        \PY{n}{f1}\PY{o}{.}\PY{n}{SetParameter}\PY{p}{(}\PY{l+m+mi}{1}\PY{p}{,}\PY{l+m+mi}{1}\PY{p}{)}\PY{p}{;}
        \PY{n}{f1}\PY{o}{.}\PY{n}{Draw}\PY{p}{(}\PY{p}{)}\PY{p}{;}
        \PY{n}{canvas\PYZus{}8\PYZus{}1}\PY{o}{.}\PY{n}{Draw}\PY{p}{(}\PY{p}{)}\PY{p}{;}
\end{Verbatim}

    \begin{center}
    \adjustimage{max size={0.9\linewidth}{0.9\paperheight}}{8-ROOT-in-Python_files/8-ROOT-in-Python_2_0.png}
    \end{center}
    { \hspace*{\fill} \\}
    
    A slightly more advanced example hands over data defined in the macro to
the ROOT class TGraphErrors. Note that a Python array can be used to
pass data between Python and ROOT. The first line in the Python script
allows it to be executed directly from the operating system, without the
need to start the script from python or the highly recommended powerful
interactive shell ipython. The last line in the python script is there
to allow you to have a look at the graphical output in the ROOT canvas
before it disappears upon termination of the script.

Here is the C++ version:

    \begin{Verbatim}[commandchars=\\\{\}]
{\color{incolor}In [{\color{incolor}3}]:} \PY{o}{\PYZpc{}\PYZpc{}}\PY{k}{cpp}
        //
        // Draw a graph with error bars and fit a function to it
        //
        gStyle\PYZhy{}\PYZgt{}SetOptFit(111) ; //superimpose fit results
        // make nice Canvas
        auto *c1 = new TCanvas(\PYZdq{}c1\PYZdq{} ,\PYZdq{}Daten\PYZdq{} ,200 ,10 ,700 ,500) ;
        c1\PYZhy{}\PYZgt{}SetGrid( ) ;
        //define some data points ...
        const Int\PYZus{}t n = 10;
        Float\PYZus{}t x[n] = \PYZob{}\PYZhy{}0.22, 0.1, 0.25, 0.35, 0.5, 0.61, 0.7, 0.85, 0.89, 1.1\PYZcb{};
        Float\PYZus{}t y[n] = \PYZob{}0.7, 2.9, 5.6, 7.4, 9., 9.6, 8.7, 6.3, 4.5, 1.1\PYZcb{};
        Float\PYZus{}t ey[n] = \PYZob{}.8 ,.7 ,.6 ,.5 ,.4 ,.4 ,.5 ,.6 ,.7 ,.8\PYZcb{};
        Float\PYZus{}t ex[n] = \PYZob{}.05 ,.1 ,.07 ,.07 ,.04 ,.05 ,.06 ,.07 ,.08 ,.05\PYZcb{};
        // and hand over to TGraphErros object
        TGraphErrors *gr = new TGraphErrors(n,x,y,ex,ey);
        gr\PYZhy{}\PYZgt{}SetTitle(\PYZdq{}TGraphErrors with Fit\PYZdq{}) ;
        gr\PYZhy{}\PYZgt{}DrawClone(\PYZdq{}AP\PYZdq{});
        // now perform a fit (with errors in x and y!)
        gr\PYZhy{}\PYZgt{}Fit(\PYZdq{}gaus\PYZdq{});
        c1\PYZhy{}\PYZgt{}Draw();
\end{Verbatim}

    \begin{Verbatim}[commandchars=\\\{\}]
FCN=1.53213 FROM MIGRAD    STATUS=CONVERGED      90 CALLS          91 TOTAL
                     EDM=2.58725e-08    STRATEGY= 1      ERROR MATRIX ACCURATE 
  EXT PARAMETER                                   STEP         FIRST   
  NO.   NAME      VALUE            ERROR          SIZE      DERIVATIVE 
   1  Constant     9.54303e+00   3.54707e-01   2.49013e-04   5.51076e-04
   2  Mean         5.66472e-01   2.69815e-02   2.04103e-05   4.18957e-03
   3  Sigma        2.85465e-01   3.04454e-02   2.72043e-05   1.85906e-03
    \end{Verbatim}

    \begin{center}
    \adjustimage{max size={0.9\linewidth}{0.9\paperheight}}{8-ROOT-in-Python_files/8-ROOT-in-Python_4_1.png}
    \end{center}
    { \hspace*{\fill} \\}
    
    In Python it looks like this:

    \begin{Verbatim}[commandchars=\\\{\}]
{\color{incolor}In [{\color{incolor}4}]:} \PY{c+c1}{\PYZsh{}}
        \PY{c+c1}{\PYZsh{} Draw a graph with error bars and fit a function to it}
        \PY{c+c1}{\PYZsh{}}
        \PY{k+kn}{from} \PY{n+nn}{ROOT} \PY{k+kn}{import} \PY{n}{gStyle}\PY{p}{,} \PY{n}{TCanvas}\PY{p}{,} \PY{n}{TGraphErrors}
        \PY{k+kn}{from} \PY{n+nn}{array} \PY{k+kn}{import} \PY{n}{array}
        \PY{n}{gStyle}\PY{o}{.}\PY{n}{SetOptFit} \PY{p}{(}\PY{l+m+mi}{111}\PY{p}{)} \PY{c+c1}{\PYZsh{} superimpose fit results}
        \PY{n}{canvas\PYZus{}8\PYZus{}1}\PY{o}{.}\PY{n}{SetGrid} \PY{p}{(}\PY{p}{)}
        \PY{c+c1}{\PYZsh{}define some data points . . .}
        \PY{n}{x} \PY{o}{=} \PY{n}{array}\PY{p}{(}\PY{l+s+s1}{\PYZsq{}}\PY{l+s+s1}{f}\PY{l+s+s1}{\PYZsq{}}\PY{p}{,} \PY{p}{(}\PY{o}{\PYZhy{}}\PY{l+m+mf}{0.22}\PY{p}{,} \PY{l+m+mf}{0.1}\PY{p}{,} \PY{l+m+mf}{0.25}\PY{p}{,} \PY{l+m+mf}{0.35}\PY{p}{,} \PY{l+m+mf}{0.5}\PY{p}{,} \PY{l+m+mf}{0.61}\PY{p}{,} \PY{l+m+mf}{0.7}\PY{p}{,} \PY{l+m+mf}{0.85}\PY{p}{,} \PY{l+m+mf}{0.89}\PY{p}{,} \PY{l+m+mf}{1.1}\PY{p}{)} \PY{p}{)}
        \PY{n}{y} \PY{o}{=} \PY{n}{array}\PY{p}{(}\PY{l+s+s1}{\PYZsq{}}\PY{l+s+s1}{f}\PY{l+s+s1}{\PYZsq{}}\PY{p}{,} \PY{p}{(}\PY{l+m+mf}{0.7}\PY{p}{,} \PY{l+m+mf}{2.9}\PY{p}{,} \PY{l+m+mf}{5.6}\PY{p}{,} \PY{l+m+mf}{7.4}\PY{p}{,} \PY{l+m+mf}{9.}\PY{p}{,} \PY{l+m+mf}{9.6}\PY{p}{,} \PY{l+m+mf}{8.7}\PY{p}{,} \PY{l+m+mf}{6.3}\PY{p}{,} \PY{l+m+mf}{4.5}\PY{p}{,} \PY{l+m+mf}{1.1}\PY{p}{)} \PY{p}{)}
        \PY{n}{ey} \PY{o}{=} \PY{n}{array}\PY{p}{(}\PY{l+s+s1}{\PYZsq{}}\PY{l+s+s1}{f}\PY{l+s+s1}{\PYZsq{}}\PY{p}{,} \PY{p}{(}\PY{o}{.}\PY{l+m+mi}{8} \PY{p}{,}\PY{o}{.}\PY{l+m+mi}{7} \PY{p}{,}\PY{o}{.}\PY{l+m+mi}{6} \PY{p}{,}\PY{o}{.}\PY{l+m+mi}{5} \PY{p}{,}\PY{o}{.}\PY{l+m+mi}{4} \PY{p}{,}\PY{o}{.}\PY{l+m+mi}{4} \PY{p}{,}\PY{o}{.}\PY{l+m+mi}{5} \PY{p}{,}\PY{o}{.}\PY{l+m+mi}{6} \PY{p}{,}\PY{o}{.}\PY{l+m+mi}{7} \PY{p}{,}\PY{o}{.}\PY{l+m+mi}{8}\PY{p}{)} \PY{p}{)}
        \PY{n}{ex} \PY{o}{=} \PY{n}{array}\PY{p}{(}\PY{l+s+s1}{\PYZsq{}}\PY{l+s+s1}{f}\PY{l+s+s1}{\PYZsq{}}\PY{p}{,} \PY{p}{(}\PY{o}{.}\PY{l+m+mo}{05} \PY{p}{,}\PY{o}{.}\PY{l+m+mi}{1} \PY{p}{,}\PY{o}{.}\PY{l+m+mo}{07} \PY{p}{,}\PY{o}{.}\PY{l+m+mo}{07} \PY{p}{,}\PY{o}{.}\PY{l+m+mo}{04} \PY{p}{,}\PY{o}{.}\PY{l+m+mo}{05} \PY{p}{,}\PY{o}{.}\PY{l+m+mo}{06} \PY{p}{,}\PY{o}{.}\PY{l+m+mo}{07} \PY{p}{,}\PY{o}{.}\PY{l+m+mi}{08} \PY{p}{,}\PY{o}{.}\PY{l+m+mo}{05}\PY{p}{)} \PY{p}{)}
        \PY{n}{nPoints}\PY{o}{=}\PY{n+nb}{len} \PY{p}{(} \PY{n}{x} \PY{p}{)}
        \PY{c+c1}{\PYZsh{} . . . and hand over to TGraphErros object}
        \PY{n}{gr}\PY{o}{=}\PY{n}{TGraphErrors} \PY{p}{(} \PY{n}{nPoints} \PY{p}{,} \PY{n}{x} \PY{p}{,} \PY{n}{y} \PY{p}{,} \PY{n}{ex} \PY{p}{,} \PY{n}{ey} \PY{p}{)}
        \PY{n}{gr}\PY{o}{.}\PY{n}{SetTitle}\PY{p}{(}\PY{l+s+s2}{\PYZdq{}}\PY{l+s+s2}{TGraphErrors with Fit}\PY{l+s+s2}{\PYZdq{}}\PY{p}{)}
        \PY{n}{gr}\PY{o}{.}\PY{n}{Draw} \PY{p}{(} \PY{l+s+s2}{\PYZdq{}}\PY{l+s+s2}{AP}\PY{l+s+s2}{\PYZdq{}} \PY{p}{)} 
        \PY{n}{gr}\PY{o}{.}\PY{n}{Fit}\PY{p}{(}\PY{l+s+s2}{\PYZdq{}}\PY{l+s+s2}{gaus}\PY{l+s+s2}{\PYZdq{}}\PY{p}{)} 
        \PY{n}{canvas\PYZus{}8\PYZus{}1}\PY{o}{.}\PY{n}{Update} \PY{p}{(}\PY{p}{)} 
        \PY{n}{canvas\PYZus{}8\PYZus{}1}\PY{o}{.}\PY{n}{Draw} \PY{p}{(}\PY{p}{)} 
\end{Verbatim}

    \begin{Verbatim}[commandchars=\\\{\}]
FCN=1.53213 FROM MIGRAD    STATUS=CONVERGED      88 CALLS          89 TOTAL
                     EDM=2.57894e-08    STRATEGY= 1      ERROR MATRIX ACCURATE 
  EXT PARAMETER                                   STEP         FIRST   
  NO.   NAME      VALUE            ERROR          SIZE      DERIVATIVE 
   1  Constant     9.54303e+00   3.54707e-01   2.49013e-04   5.50385e-04
   2  Mean         5.66472e-01   2.69815e-02   2.04103e-05   4.18001e-03
   3  Sigma        2.85465e-01   3.04454e-02   2.72043e-05   1.85849e-03
    \end{Verbatim}

    \begin{center}
    \adjustimage{max size={0.9\linewidth}{0.9\paperheight}}{8-ROOT-in-Python_files/8-ROOT-in-Python_6_1.png}
    \end{center}
    { \hspace*{\fill} \\}
    
    Comparing the C++ and Python versions in these two examples, it now
should be clear how easy it is to convert any ROOT Macro in C++ to a
Python version.

As another example, let us revisit macro3 from Chapter 4. A
straight-forward Python version relying on the ROOT class
\texttt{TMath}:

    \begin{Verbatim}[commandchars=\\\{\}]
{\color{incolor}In [{\color{incolor}5}]:} \PY{c+c1}{\PYZsh{} Builds a polar graph in a square Canvas.}
        
        \PY{k+kn}{from} \PY{n+nn}{ROOT} \PY{k+kn}{import} \PY{n}{TGraphPolar}\PY{p}{,} \PY{n}{TCanvas}\PY{p}{,} \PY{n}{TMath}
        \PY{k+kn}{from} \PY{n+nn}{array} \PY{k+kn}{import} \PY{n}{array}
        
        \PY{n}{canvas\PYZus{}8\PYZus{}1\PYZus{}sq} \PY{o}{=} \PY{n}{TCanvas}\PY{p}{(}\PY{l+s+s2}{\PYZdq{}}\PY{l+s+s2}{canvas\PYZus{}8\PYZus{}1\PYZus{}sq}\PY{l+s+s2}{\PYZdq{}}\PY{p}{,}\PY{l+s+s2}{\PYZdq{}}\PY{l+s+s2}{myCanvas}\PY{l+s+s2}{\PYZdq{}}\PY{p}{,}\PY{l+m+mi}{600}\PY{p}{,}\PY{l+m+mi}{600}\PY{p}{)}
        \PY{n}{rmin} \PY{o}{=} \PY{l+m+mf}{0.}
        \PY{n}{rmax} \PY{o}{=} \PY{n}{TMath}\PY{o}{.}\PY{n}{Pi}\PY{p}{(}\PY{p}{)}\PY{o}{*}\PY{l+m+mf}{6.}
        \PY{n}{npoints} \PY{o}{=} \PY{l+m+mi}{300}
        \PY{n}{r} \PY{o}{=} \PY{n}{array}\PY{p}{(}\PY{l+s+s1}{\PYZsq{}}\PY{l+s+s1}{d}\PY{l+s+s1}{\PYZsq{}}\PY{p}{,}\PY{p}{[}\PY{l+m+mi}{0}\PY{p}{]}\PY{o}{*}\PY{n}{npoints}\PY{p}{)}
        \PY{n}{theta}  \PY{o}{=} \PY{n}{array}\PY{p}{(}\PY{l+s+s1}{\PYZsq{}}\PY{l+s+s1}{d}\PY{l+s+s1}{\PYZsq{}}\PY{p}{,}\PY{p}{[}\PY{l+m+mi}{0}\PY{p}{]}\PY{o}{*}\PY{n}{npoints}\PY{p}{)}
        \PY{k}{for} \PY{n}{ipt} \PY{o+ow}{in} \PY{n+nb}{xrange}\PY{p}{(}\PY{l+m+mi}{0}\PY{p}{,}\PY{n}{npoints}\PY{p}{)}\PY{p}{:}
            \PY{n}{r}\PY{p}{[}\PY{n}{ipt}\PY{p}{]} \PY{o}{=} \PY{n}{ipt}\PY{o}{*}\PY{p}{(}\PY{n}{rmax}\PY{o}{\PYZhy{}}\PY{n}{rmin}\PY{p}{)}\PY{o}{/}\PY{n}{npoints}\PY{o}{+}\PY{n}{rmin}
            \PY{n}{theta}\PY{p}{[}\PY{n}{ipt}\PY{p}{]} \PY{o}{=} \PY{n}{TMath}\PY{o}{.}\PY{n}{Sin}\PY{p}{(}\PY{n}{r}\PY{p}{[}\PY{n}{ipt}\PY{p}{]}\PY{p}{)}
        
        \PY{n}{grP1} \PY{o}{=} \PY{n}{TGraphPolar}\PY{p}{(}\PY{n}{npoints}\PY{p}{,}\PY{n}{r}\PY{p}{,}\PY{n}{theta}\PY{p}{)}
        \PY{n}{grP1}\PY{o}{.}\PY{n}{SetTitle}\PY{p}{(}\PY{l+s+s2}{\PYZdq{}}\PY{l+s+s2}{A Fan}\PY{l+s+s2}{\PYZdq{}}\PY{p}{)}
        \PY{n}{grP1}\PY{o}{.}\PY{n}{SetLineWidth}\PY{p}{(}\PY{l+m+mi}{3}\PY{p}{)}
        \PY{n}{grP1}\PY{o}{.}\PY{n}{SetLineColor}\PY{p}{(}\PY{l+m+mi}{2}\PY{p}{)}
        \PY{n}{grP1}\PY{o}{.}\PY{n}{DrawClone}\PY{p}{(}\PY{l+s+s2}{\PYZdq{}}\PY{l+s+s2}{L}\PY{l+s+s2}{\PYZdq{}}\PY{p}{)}
        \PY{n}{grP1}\PY{o}{.}\PY{n}{Draw}\PY{p}{(}\PY{p}{)}
        \PY{n}{canvas\PYZus{}8\PYZus{}1\PYZus{}sq}\PY{o}{.}\PY{n}{Draw}\PY{p}{(}\PY{p}{)}
\end{Verbatim}

    \begin{center}
    \adjustimage{max size={0.9\linewidth}{0.9\paperheight}}{8-ROOT-in-Python_files/8-ROOT-in-Python_8_0.png}
    \end{center}
    { \hspace*{\fill} \\}
    
    \subsubsection{8.1.1 More Python- less C++}\label{more-python--less-c}

You may have noticed already that there are some Python modules
providing functionality similar to ROOT classes, which fit more
seamlessly into your Python code.

A more ``pythonic'' version of the above macro3 would use a replacement
of the ROOT class TMath for the provisoining of data to TGraphPolar.
With the math package, the part of the code becomes

\begin{verbatim}
import math
from array import array
from ROOT import TCanvas , TGraphPolar
...
ipt=range(0,npoints)
r=array('d',map(lambda x: x*(rmax-rmin)/(npoints-1.)+rmin,ipt))
theta=array('d',map(math.sin,r))
e=array('d',npoints*[0.])
...
\end{verbatim}

\paragraph{8.1.1.1 Customised Binning}\label{customised-binning}

This example combines comfortable handling of arrays in Python to define
variable bin sizes of a ROOT histogram. All we need to know is the
interface of the relevant ROOT class and its methods (from the ROOT
documentation):

\begin{verbatim}
TH1F(const char* name , const char* title , Int_t nbinsx , const Double_t* xbins)
\end{verbatim}

Here is the Python code:

    \begin{Verbatim}[commandchars=\\\{\}]
{\color{incolor}In [{\color{incolor}6}]:} \PY{k+kn}{import} \PY{n+nn}{ROOT}
        \PY{k+kn}{from} \PY{n+nn}{array} \PY{k+kn}{import} \PY{n}{array}
        \PY{n}{canvas\PYZus{}8\PYZus{}1\PYZus{}1} \PY{o}{=} \PY{n}{TCanvas}\PY{p}{(}\PY{l+s+s2}{\PYZdq{}}\PY{l+s+s2}{canvas\PYZus{}8\PYZus{}1\PYZus{}1}\PY{l+s+s2}{\PYZdq{}}\PY{p}{,}\PY{l+s+s2}{\PYZdq{}}\PY{l+s+s2}{myCanvas}\PY{l+s+s2}{\PYZdq{}}\PY{p}{)}
        \PY{n}{arrBins} \PY{o}{=} \PY{n}{array}\PY{p}{(}\PY{l+s+s1}{\PYZsq{}}\PY{l+s+s1}{d}\PY{l+s+s1}{\PYZsq{}} \PY{p}{,}\PY{p}{(}\PY{l+m+mi}{1} \PY{p}{,}\PY{l+m+mi}{4} \PY{p}{,}\PY{l+m+mi}{9} \PY{p}{,}\PY{l+m+mi}{16}\PY{p}{)} \PY{p}{)} \PY{c+c1}{\PYZsh{} array of bin edges}
        \PY{n}{histogram\PYZus{}8\PYZus{}1} \PY{o}{=} \PY{n}{ROOT}\PY{o}{.}\PY{n}{TH1F}\PY{p}{(}\PY{l+s+s2}{\PYZdq{}}\PY{l+s+s2}{histogram\PYZus{}8\PYZus{}1}\PY{l+s+s2}{\PYZdq{}}\PY{p}{,} \PY{l+s+s2}{\PYZdq{}}\PY{l+s+s2}{histogram\PYZus{}8\PYZus{}1}\PY{l+s+s2}{\PYZdq{}}\PY{p}{,} \PY{n+nb}{len}\PY{p}{(}\PY{n}{arrBins}\PY{p}{)}\PY{o}{\PYZhy{}}\PY{l+m+mi}{1}\PY{p}{,} \PY{n}{arrBins}\PY{p}{)}
        \PY{c+c1}{\PYZsh{} fill it with equally spaced numbers}
        \PY{k}{for} \PY{n}{i} \PY{o+ow}{in} \PY{n+nb}{range} \PY{p}{(}\PY{l+m+mi}{1} \PY{p}{,}\PY{l+m+mi}{16}\PY{p}{)} \PY{p}{:}
           \PY{n}{histogram\PYZus{}8\PYZus{}1}\PY{o}{.}\PY{n}{Fill}\PY{p}{(}\PY{n}{i}\PY{p}{)}
        \PY{n}{histogram\PYZus{}8\PYZus{}1}\PY{o}{.}\PY{n}{Draw} \PY{p}{(}\PY{p}{)}
        \PY{n}{canvas\PYZus{}8\PYZus{}1\PYZus{}1}\PY{o}{.}\PY{n}{Draw} \PY{p}{(}\PY{p}{)}
\end{Verbatim}

    \begin{center}
    \adjustimage{max size={0.9\linewidth}{0.9\paperheight}}{8-ROOT-in-Python_files/8-ROOT-in-Python_10_0.png}
    \end{center}
    { \hspace*{\fill} \\}
    
    \subsection{8.2 Custom code: from C++ to
Python}\label{custom-code-from-c-to-python}

The ROOT interpreter and type sytem offer interesting possibilities when
it comes to JITting of C++ code. Take for example this header file,
containing a class and a function.

\begin{verbatim}
##file cpp2pythonExample.h
#include "stdio.h"

class A{
public:
 A(int i):m_i(i){}
 int getI() const {return m_i;}
private:
 int m_i=0;
};

void printA(const A& a ){
  printf ("The value of A instance is %i.\n",a.getI());
}
\end{verbatim}

This example might seem trivial, but it shows a powerful ROOT feature.
C++ code can be JITted within PyROOT and the entities defined in C++ can
be transparently used in Python!

\begin{verbatim}
>>> import ROOT
>>> ROOT.gInterpreter.ProcessLine('#include "cpp2pythonExample.h"')
>>> a = ROOT.A(123)
>>> ROOT.printA(a)
The value of A instance is 123.
\end{verbatim}


    % Add a bibliography block to the postdoc
    
    
    
    \end{document}
